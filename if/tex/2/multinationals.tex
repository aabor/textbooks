\documentclass[international_finance_p2.tex]{subfiles}

\begin{document}
\setbeamercovered{transparent}
\section{Multinationals}

\subsection{Financial Control}
\begin{frame}{Financial Control}
\begin{itemize}[<+->]
\item
The multinational firms carry currency exchange risk and some other risks in which domestic companies may never have been involved.
\item
Typical control systems are based on setting standards with regard to sales, profits, inventory, or other specific variables and then examining financial statements and reports to evaluate the achievement of such goals.
\end{itemize}
\end{frame}

\subsection{Cash Management}
\begin{frame}{Cash Management}
\begin{itemize}[<+->]
\item
Cash management involves using the firm’s cash as efficiently as possible. 
\item
Cash is the most liquid asset. 
Since cash earns no interest, the firm has a strong incentive to minimize its holdings of cash.
\end{itemize}
\end{frame}
\begin{frame}{Multinational cash management}
\begin{itemize}[<+->]
\item
Centralization of cash management
\item
Netting
\end{itemize}
\end{frame}

\subsection{Letters of Credit}
\begin{frame}{Letters of Credit}
\begin{block}{A letter of credit (LOC)}
\quad is a written instrument issued by a bank at the request of an importer that obligates the bank to pay a specific amount of money to an exporter. 
\end{block}
Following condition are specified in the contract:
\begin{itemize}
\item
payment time;
\item
documents to be presented by the exporter prior to payment;
\item
a bill of lading (a detailed list of the content that is shipped, and can be used to identify missing or damaged items).
\end{itemize}

\end{frame}
\subsection{Intrafirm Transfers}
\begin{frame}{Intrafirm Transfers}
\begin{itemize}[<+->]
\item
The price that one subsidiary charges another subsidiary for internal goods transfers is called a transfer price. 
\item
The transfer prices can be a sensitive internal corporate issue because it helps to determine how total firm profits are allocated across divisions. 
\item
Governments are also interested in transfer pricing since the prices at which goods are transferred will determine tariff and tax revenues.
\item
The parent firm always has an incentive to minimize taxes by pricing transfers in order to keep profits low in high-tax countries and by shifting profits to subsidiaries in low-tax countries.
\end{itemize}
\end{frame}

\subsection{Capital Budgeting}
\begin{frame}{Capital Budgeting}
\begin{itemize}[<+->]
\item
Capital budgeting refers to the evaluation of prospective investment alternatives and the commitment of funds to preferred projects. 
\item
Long-term commitments of funds extending beyond one year are called capital expenditures
\item
Foreign projects involve foreign exchange risk, political risk, and foreign tax regulations.
\item
Several effects should be included in project valuation 
\begin{enumerate}
\item
depreciation charges; 
\item
credit terms extended to the subsidiary by a government or \item
official agency;
\item
deferred or reduced taxes given as incentive to undertake the expenditure.
\end{enumerate}
\end{itemize}
\end{frame}

\end{document}