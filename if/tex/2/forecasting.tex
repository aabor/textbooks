\documentclass[international_finance_p2.tex]{subfiles}

\begin{document}
\setbeamercovered{transparent}
\section{Forex forecasting}

\subsection{Types of Foreign Exchange Risk}
\begin{frame}{Types of Foreign Exchange Risk}
1. Translation exposure. 

2. Transaction exposure. 

3. Economic exposure. 
\end{frame}

\begin{frame}
\begin{block}{Translation exposure}
This is also known as accounting exposure. It is the difference between foreign currency denominated assets and foreign currency denominated liabilities.
\end{block}
\end{frame}

\begin{frame}
\begin{block}{Transaction exposure}
This is exposure resulting from the uncertain domestic currency value of a foreign currency denominated transaction to be completed at some future date.
\end{block}
\end{frame}

\begin{frame}
\begin{block}{Economic exposure}
This is exposure of the firm’s value to changes in exchange rates. If the value of the firm is measured as the present value of future after-tax cash flows, then economic exposure is concerned with the sensitivity of the real domestic currency value of long-term cash flows to exchange rate changes.
\end{block}
\end{frame}
\begin{frame}{Foreign exchange risk may be hedged or eliminated by the following strategies}
\begin{itemize}[<+->]
\item
Trading in forward, futures, or options markets. 
\item
Invoicing in the domestic currency.
\item
Speeding (slowing) payments of currencies expected to appreciate (depreciate).
\item
Speeding (slowing) collection of currencies expected to depreciate (appreciate).
\end{itemize}
\end{frame}


\subsection{Foreign Exchange Risk Premium}
\begin{frame}{Foreign Exchange Risk Premium}
\begin{itemize}[<+->]
\item
The forward exchange rate may serve as a predictor of future spot exchange rates. We may question whether the forward rate should be equal to the expected future spot rate.
\item
The effective return differential is equal to the percentage difference between the forward and expected future spot exchange rate.
\end{itemize}
\end{frame}

\begin{frame}{The effective return differential}
\begin{align}
i_{USD}-E_{t+1}^*-i_{EUR}=(F-E_{t+1}^*)/E_t,
\end{align}

where

$i_{USD}, \quad i_{EUR}$ are interest rates on US dollar and Euro respectively;

$E_{t+1}^*$ is the expected dollar price of EUR next period;

$E_t$ is the current spot exchange rate;

$F$ is the forward exchange rate.

$E_{t+1}^*$ may be interpreted as the price of future contract on the same currency pair that the forward exchange rate is considered.
\end{frame}

\begin{frame}{The risk premium in the forward exchange market}
\begin{align}
\frac{F-E_{t+1}^*}{E_t}
\end{align}
If the effective return differential is zero, then there would appear to be no risk premium. If the effective return differential is positive, then there is a positive risk premium on the domestic currency.
\end{frame}
\subsection{Foreign Exchange Forecasting}
\begin{frame}{Foreign Exchange Forecasting}
\begin{itemize}[<+->]
\item
A fundamental model forecasts exchange rates based on variables that are believed to be important determinants of exchange rates.
\item
A technical trading model uses the past history of exchange rates to predict future movements. Technical traders use charts or diagrams depicting the time path of an exchange rate which they believe will reverse or accelerate the trend. 
\end{itemize}
\end{frame}

\end{document}