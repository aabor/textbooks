\documentclass[international_finance_p2.tex]{subfiles}

\begin{document}
\setbeamercovered{transparent}
\section{International Parities}

\subsection{Interest Parity. Interest Rates and Inflation}
\begin{frame}{Interest Parity}{Interest Rates and Inflation}
\begin{itemize}[<+->]
\item
Interest rate parity is a no-arbitrage condition on the market under which investors will be indifferent to interest rates available on bank deposits in two different currencies. 
\item
No-arbitrage condition exists when the market prices do not allow for profitable arbitrage. 
\item
This condition does not always hold and this create potential opportunities for riskless profits from arbitrage deals. 
\item
Two assumptions central to interest rate parity are capital mobility and perfect substitutability of domestic and foreign assets.
\end{itemize}
\end{frame}
\begin{frame}{Two forms of interest rate parities}
\begin{itemize}[<+->]
\item
uncovered interest rate parity (UIRP) exists when exposure to foreign exchange risk  is uninhibited;
\item
covered interest rate parity (CIRP) exists when a forward contract has been used to cover  exchange rate risk
\end{itemize}
\end{frame}
\begin{frame}{Uncovered interest rate parity (UIRP)}
\begin{align}
1+i_{USD}=\frac{E_t S_{t+k}}{S_t}(1+i_{EUR}),
\end{align}
where
$E_t S_{t+k}$ is the expected future spot exchange rate at time $t + k$;

$k$ is the number of periods into the future from time $t$;

$S_t$ is the current spot exchange rate at time $t$;

$i_{USD},\quad i_{EUR}$  are the interest rates in the domestic and foreign currencies, for example USD and EUR respectively.

The dollar return on dollar deposits, $1+i_{USD}$, is shown to be equal to the dollar return on euro deposits, $\frac{E_t S_{t+k}}{S_t}(1+i_{EUR})$.
\end{frame}

\begin{frame}{Covered interest rate parity (CIRP)}
\begin{align}
1+i_{USD}&=\frac{F_t}{S_t}(1+i_{EUR}) \quad \nonumber or\\
i_{USD}-i_{EUR}&=\frac{F_t-S_t}{S_t}
\end{align}

where

$F_t$ is the forward exchange rate at time $t$;

The dollar return on dollar deposits, $1+i_{USD}$, is shown to be equal to the dollar return on euro deposits, $\frac{F_t}{S_t}(1+i_{EUR})$.

Covered interest arbitrage is an arbitrage trading strategy whereby an investor capitalizes on the interest rate differential between two countries by using a forward contract to cover exchange rate risk.

\end{frame}
\subsection{The relation between Exchange Rates, Interest Rates, and Inflation}
\begin{frame}{The relation between Exchange Rates, Interest Rates, and Inflation}
\begin{itemize}[<+->]
\item
The nominal interest rate is the rate actually observed in the market. 
\item
The real rate is a concept that measures the return after adjusting for inflation.
\end{itemize}
\end{frame}
\begin{frame}{The Fisher effect}
\begin{align}
i_{USD}&=r_{USD}+p^{USD}\\
i_{EUR}&=r_{EUR}+p^{EUR}
\end{align}

where 

$i$ is the nominal interest rate;

$r$ is the real interest rate;

$p^{USD}$ is the expected rate of inflation (in this case in US dollars or EUR).
\end{frame}

\begin{frame}{Real interest rate parity (RIRP)}
\begin{align}
UIRP:\Delta E_t S_{t+k}&=E_t S_{t+k}-S_t\nonumber \\
&=i_{USD}-i_{EUR},\\
E_t S_{t+k}&= \Delta E_t (p_{t+k}^{USD} ) - \Delta E_t (p_{t+k}^{EUR} ),
\end{align}

where

$p_{t+k}^{USD}, \quad p_{t+k}^{EUR}$ represent expected rate of inflation for both currencies respectively (dollar and Euro in this example).

If the above conditions hold, then they can be combined and rearranged as the following:
\begin{align}
RIRP:i_{USD} - \Delta E_t (p_{t+k}^{USD})=i_{EUR}- \Delta E_t (p_{t+k}^{EUR} ),
\end{align}
\end{frame}
\begin{frame}{The link between interest rates, inflation, and exchange rates}
\begin{align}
RIRP:i_{USD}-i_{EUR}&=p_{t+k}^{USD}-p_{t+k}^{EUR}\nonumber\\
&=\frac{F_t-S_t}{S_t}.
\end{align}
The interest differential is equal to expected rates of inflation differential and is also equal to the forward premium.

The parity condition suggests that real interest rates will equalize between countries and that capital mobility will result in capital flows that eliminate opportunities for arbitrage.
\end{frame}
\end{document}