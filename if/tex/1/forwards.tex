% !TeX program = lualatex -synctex=1 -interaction=nonstopmode --shell-escape %.tex

\documentclass[international_finance_p1.tex]{subfiles}

\begin{document}
\setbeamercovered{transparent}
\section{Forward-looking Market Instruments}

\subsection{The currency forwards}
\begin{frame}{The currency forwards}
\begin{block}{The currency forward }
   is a contract in the foreign exchange market that locks in the exchange rate for the purchase or sale of a currency on a future date.
\end{block}
\begin{itemize}[<+->]
\item
A currency forward represents \emph{a binding obligation}, which means that the contract buyer or seller cannot walk away if the ``locked in'' rate eventually proves to be adverse. 
\item
To compensate for the risk of non-delivery, financial institutions may require a \emph{deposit }from retail investors or smaller firms with whom they do not have a business relationship.
\end{itemize}
\end{frame}
\begin{frame}{Determining a currency forward rate}
\begin{align}
F=\frac{\text{Bid}+\text{Ask}}{2}\cdot \frac{1+i_1 \cdot \frac{n}{365}}{1+i_2 \cdot \frac{n}{365}}
\end{align}

where

$F$ – forward rate;

Bid, Ask – spot bid and ask rate respectively;

$i_1$ – annual interest rate for quoted currency;

$i_2$ – annual interest rate for base currency;

$n$ – forward contract period in days
\end{frame}
\setbeamercovered{invisible}
\begin{frame}
\begin{exampleblock}{Forward rate calculation}
assume a current spot rate for the Russian ruble of US\$1 = RUR76.45 or USD/RUR=76.45, a one-year interest rate for Russian rubles of 8\% (key rate held by the Central Bank of Russian Federation), and one-year interest rate for US dollars of 0.25\%. Assume that USD/RUR bid and ask are 75.00 and 76.0 respectively. So, 3-months (3M) forward interest rate would be 
\begin{align*}
F_{3M;USDRUB}&=\onslide<2->{ \frac{75+76}{2} \cdot \frac{1+0.08 \cdot \frac{90}{365}}{1+0.0025 \cdot \frac{90}{365}}\\}
\onslide<3->{F_{3M;USDRUB}&=76,9419}
\end{align*}
\end{exampleblock}
\end{frame}
\setbeamercovered{transparent}
\subsection{The foreign exchange swaps}
\begin{frame}
\begin{block}{A foreign exchange swap }
\quad is an arrangement where there is a simultaneous exchange of two currencies on a specific date at a rate agreed at the time of the contract, and a reverse exchange of the same two currencies at a date further in the future at a rate agreed at the time of the contract.
\end{block}
A foreign exchange swap consists of two legs:

1) a spot foreign exchange transaction, and;

2) a forward foreign exchange transaction.
\end{frame}
\begin{frame}{The forward points or swap points }
are quoted as the difference between forward and spot: $F - S$.
\begin{align}
F-S&=S \left(\frac{1+i_1 \cdot \frac{n}{base}}{1+i_2 \cdot \frac{n}{base}}-1\right)\nonumber\\
&=\frac{S(i_1-i_2)T}{1+i_2T}\nonumber\\
&\approx S(i_1-i_2)T
\end{align}
where
$n$ - forward contract term;

$T=\frac{n}{base}$;

$base=\text{365 or 366}$.
If $i_2T$ is small. Thus, the value of the swap points is roughly proportional to the interest rate differential.

The most common use of foreign exchange swaps is for institutions to fund their foreign exchange balances
\end{frame}
\setbeamercovered{invisible}
\begin{frame}
\begin{exampleblock}{Swap points calculation}
% Table generated by Excel2LaTeX from sheet 'Лист2'
\begin{table}[htbp]
  \centering
  \caption{Assume that we have such\\RUB and USD interest rates}
    \begin{tabular}{lr}
    \toprule
    \%RUB & 8\% \\
    \midrule
    \%USD & 0,25\% \\
    n     & 90 \\
    base  & 365 \\
    $SW_{USDRUB}$ & \onslide<2->{0,01911} \\
    \bottomrule
    \end{tabular}%
  \label{tab:addlabel}%
\end{table}%
\end{exampleblock}
\end{frame}
\setbeamercovered{transparent}
\subsection{The currency swaps}
\begin{frame}{}
\begin{itemize}[<+->]
\item
A currency swap is a contract in which two counterparties exchange streams of interest payments in different currencies for an agreed period of time and then exchange principal amounts in the respective currencies at an agreed exchange rate at maturity. 
\item
Currency swaps allow firms to obtain long-term foreign currency financing at lower cost than they can by borrowing directly
\end{itemize}
\end{frame}
\begin{frame}{}
% Table generated by Excel2LaTeX from sheet 'Лист3'
\begin{table}[htbp]
  \centering
  \caption{Loan rates for two firms in different currencies}
    \begin{tabular}{lrr}
    \toprule
          & USD interest rate, \% & EUR interest rate, \% \\
    \midrule
    Firm A & 10    & 7 \\
    Firm B & 9     & 8 \\
    \bottomrule
    \end{tabular}%
  \label{tab:addlabel}%
\end{table}%
Suppose, US firm has free access to loans from US banks, but can not have such favorable opportunities on Germany capital market. 

Similarly, the German firm can have good loan conditions in the homeland, but far less favorable ones in the USA.

By currency swap agreement both firms can use comparative advantages of each other to reduce the cost of loan.

\end{frame}

\begin{frame}[shrink=10]{Cash flows of the two borrowers}{in currency swap agreement}
\begin{figure}
	\centering
	\begin{overprint}
		\forloop{slideno}{1}{\value{slideno} < 2}{%
			\only<\value{slideno}>{
				\includegraphics[page=\value{slideno},
				scale=.7
				% trim={<left> <lower> <right> <upper>}				
				,trim={0cm 1cm 1.5cm 0cm},clip]
				{tikz/currency_swap}}}
	\end{overprint}
	\vspace*{-1.5em}
	\caption{\captionf{Currency swap}}
\end{figure}
\end{frame}

\begin{frame}
Thus, both firms obtain the loans in necessary foreign currency at a lower rate, than it would be in case of the request for the credit directly in foreign bank. As a whole the Firm A saves on interest payments 0,8 ml – 0,7 ml = 0,1 ml EUR, and the Firm B saves on interest payments 1,25 ml – 1,125 ml = 0,125 ml USD.
\end{frame}

\subsection{The foreign exchange futures}
\begin{frame}
Futures are similar to forward contracts. Each future contract has a fixed amount and pre-determined dates. The difference of futures from forwards consists in the following:

1. Futures trading is carried out on the open exchange market, and forwards – on interbank. Therefore dates of future contracts expiration are attached to certain dates. Futures contracts are standardized on expiration periods, volumes and terms of delivery. In case of forward contracts expiration periods and volumes of the transaction are determined by the mutual arrangement of the parties.

2. Futures are traded only on most liquid currency pairs. Forward contracts are traded almost on all currency pairs.
\end{frame}
\begin{frame}
3. The futures market is available not only to big investors, but also small. The forward market is limited by that the minimum sum for the transaction amounts 500 thousand US dollars.

4. For about 95\% futures trading volume come to an end with the conclusion of the opposite transaction, thus there is no real delivery of the currency. The parties receive only a difference between the initial price of the contract and the price existing in the day of the closing transaction. On the contrary, for about 95\% forwards transactions come to an end with currency delivery.

5. Futures transactions cost are cheaper because of standardization.

\end{frame}
\subsection{The foreign exchange options}
\begin{frame}
\begin{block}{Foreign currency options }
\quad are contracts that give a buyer the right to buy (call option) or sell (put option) currencies at a specified price within a specific period of time. The strike price is the price at which the owner of the contract has the right but not the obligation to transact.
\end{block}
\end{frame}
\begin{frame}{Comparison of currency call and put options I}
\begin{columns}
\begin{column}{0.5\textwidth}
\textbf{Currency call option}

1. In case the currency call option is executed, expenses of the currency buyer will make:
$$R_{oe}=R_o+P,$$

where $R_o$ – an exchange rate at which the currency will be acquired;

$P$ – premium for the option (buyer`s expenses).

\end{column}
\begin{column}{0.5\textwidth}  %%<--- here
\textbf{Currency put option}

1. In case the currency put option is executed, revenue of the currency seller will make:
$$R_{oe}=R_o-P,$$

where $R_o$ – an exchange rate at which the currency will be acquired;

$P$ – premium for the option (seller`s expenses anyway).
\end{column}
\end{columns}
\end{frame}

\begin{frame}{Comparison of currency call and put options II}
\begin{columns}
\begin{column}{0.5\textwidth}
\textbf{Currency call option}

2. In case the currency call option isn't executed, expenses of the currency buyer will make:
$$R_{te}=R_t+P,$$

where $R_t$ – the current market currency rate of exchange.
\end{column}
\begin{column}{0.5\textwidth}  %%<--- here
\textbf{Currency put option}

2. In case the currency put option isn't executed, revenue of the currency seller will make:
$$R_{te}=R_t-P,$$

where $R_t$ – the current market currency rate of exchange.
\end{column}
\end{columns}
\end{frame}

\begin{frame}{Comparison of currency call and put options III}
\begin{columns}
\begin{column}{0.5\textwidth}
\textbf{Currency call option}

3. As the buyer is interested in minimization of the expenses, a condition under which execution of this option will be favorable to him, will be:
\begin{align*}
 R_{oe} &< R_{te}\\ 
 &or \\
 R_o &< R_t.
\end{align*}
\end{column}
\begin{column}{0.5\textwidth}  %%<--- here
\textbf{Currency put option}

3. As the seller is interested in maximization of the revenue, a condition under which execution of this option will be favorable to him, will be:
\begin{align*}
 R_{oe} &> R_{te}\\ 
 &or \\
 R_o &> R_t.
\end{align*}
\end{column}
\end{columns}
\end{frame}


\begin{frame}{Comparison of currency call and put options IV}
\begin{columns}
\begin{column}{0.5\textwidth}
\textbf{Currency call option}

4. The buyer of the call option will make profit, if rate of exchange will rise so that it can cover option premium, i.e.
$$R_{oe} + P < R_{te}$$

If both sides of inequality are equal, the trade will be break-even.
\end{column}
\begin{column}{0.5\textwidth}  %%<--- here
\textbf{Currency put option}

4. The buyer of the put option will make profit, if rate of exchange will fall so that it can cover option premium, i.e.
$$R_{oe}-P>R_{te}$$

If both sides of inequality are equal, the trade will be break-even.
\end{column}
\end{columns}
\end{frame}


\begin{frame}[shrink=10]{}
\begin{exampleblock}{Example}
Consider the firm in Russia that needs \$1 million US dollars in 3 months. The firm decided to buy 3-months call option contract for USD/RUR
% Table generated by Excel2LaTeX from sheet 'optionprem'
\begin{table}[htbp]
  \centering
  \caption{Currency market conditions}
    \begin{tabular}{lrr}
    \toprule
          & Bid   & Ask \\
    \midrule
    USD/RUR spot rate & 32.5  & 33.2 \\
    USD/RUR 3-months forward premium & 0.6   & 0.8 \\
    USD/RUR 3-month option premium & 0.2   & 0.2 \\
    \bottomrule
    \end{tabular}%
  \label{tab:addlabel}%
\end{table}%
Should the firm execute the option if the Bid/Ask rate of exchange for USD/RUR makes, respectively, 32.80 and 33.40 in three months?
\end{exampleblock}
\end{frame}
\setbeamercovered{invisible}
\begin{frame}{}
\begin{exampleblock}{Solution}
If the firm execute the call option, its costs of dollar`s purchase will make 

\onslide<2->
33,20 + 0,80 + 0,20 = 34,20.

\onslide<3->
If the firm doesn't execute the option, its expenses will make:

\onslide<4->
33,40 + 0,20 = 33,60.

\onslide<5->
Thus, option execution will be unprofitable to firm since increases its costs of dollar`s purchase per 0,60 rubles for one dollar.
\end{exampleblock}
\end{frame}
\setbeamercovered{transparent}
\subsection{Pricing forex options}

\end{document}