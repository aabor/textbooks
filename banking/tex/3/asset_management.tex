% !TeX program = lualatex -synctex=1 -interaction=nonstopmode --shell-escape %.tex

\documentclass[_Banking_p3.tex]{subfiles}
\begin{document}

\setbeamercovered{invisible}

\subsection{Нормативные документы}
\begin{frame}[allowframebreaks]{Нормативные документы}
  \begin{thebibliography}{10}
  
  \beamertemplatearticlebibitems
    \bibitem{bb2}
Федеральный закон «О банках и банковской деятельности» от 02.12.1990 № 395-1 (ред. от 03.07.2016).


    \bibitem{bb2}
"Гражданский кодекс Российской Федерации (часть вторая)" от 26.01.1996 N 14-ФЗ (ред. от 28.03.2017). Глава 53. Доверительное управление имуществом

\pagebreak

    \bibitem{bb2}
"Гражданский кодекс Российской Федерации (часть третья)" от 26.11.2001 N 146-ФЗ (ред. от 28.03.2017). Статья 1173. Доверительное управление наследственным имуществом


    \bibitem{bb2}
"Гражданский кодекс Российской Федерации (часть первая)" от 30.11.1994 N 51-ФЗ (ред. от 29.07.2017) (с изм. и доп., вступ. в силу с 06.08.2017). Статья 38. Доверительное управление имуществом подопечного

\pagebreak

    \bibitem{bb2}
Федеральный закон от 13.07.2015 N 218-ФЗ (ред. от 29.07.2017) "О государственной регистрации недвижимости" (с изм. и доп., вступ. в силу с 11.08.2017). Статья 54. Особенности осуществления государственной регистрации прав при доверительном управлении и опеке, связанных с недвижимым имуществом


    \bibitem{bb2}
Федеральный закон от 29.11.2001 N 156-ФЗ (ред. от 01.05.2017) "Об инвестиционных фондах"

\pagebreak

    \bibitem{bb2}
"Положение о единых требованиях к правилам осуществления деятельности по управлению ценными бумагами, к порядку раскрытия управляющим информации, а также требованиях, направленных на исключение конфликта интересов управляющего" (утв. Банком России 03.08.2015 N 482-П) (Зарегистрировано в Минюсте России 04.12.2015 N 39968)

\pagebreak

    \bibitem{bb2}
Федеральный закон от 25.12.2008 N 273-ФЗ (ред. от 03.04.2017) "О противодействии коррупции" (с изм. и доп., вступ. в силу с 28.06.2017). Статья 12.3. Обязанность передачи ценных бумаг (долей участия, паев в уставных (складочных) капиталах организаций) в доверительное управление в целях предотвращения конфликта интересов

\pagebreak

    \bibitem{bb2}
"Налоговый кодекс Российской Федерации (часть вторая)" от 05.08.2000 N 117-ФЗ (ред. от 29.07.2017). Статья 276. Особенности определения налоговой базы участников договора доверительного управления имуществом

\pagebreak

    \bibitem{bb2}
Постановление Правительства РФ от 13.12.2006 N 761 (ред. от 12.07.2017) "Об установлении дополнительных ограничений на инвестирование средств пенсионных накоплений, переданных Пенсионным фондом Российской Федерации в доверительное управление управляющей компании, в депозиты в валюте Российской Федерации и иностранной валюте в кредитных организациях. 

  \end{thebibliography}
\end{frame}

\subsection{Определения}
\begin{frame} [ allowframebreaks]{Определения}
\begin{block}{Договор доверительного управления имуществом}
\quad
- по договору доверительного управления имуществом одна сторона (учредитель управления) передает другой стороне (доверительному управляющему) на определенный срок имущество в доверительное управление, а другая сторона обязуется осуществлять управление этим имуществом в интересах учредителя управления или указанного им лица (выгодоприобретателя).
\end{block}
Учредителем доверительного управления является собственник имущества или другие лица.
\pagebreak
Доверительным управляющим может быть кредитная организация, а в случаях, когда кредитная организация выступает учредителем доверительного управления имуществом, - индивидуальный предприниматель или коммерческая организация, за исключением унитарного предприятия.

Выгодоприобретатель - лицо, в интересах которого доверительный управляющий осуществляет управление имуществом (учредитель или третье лицо).

\end{frame}
\begin{frame}{Объекты доверительного управления }
\begin{itemize}
\item
предприятия и другие имущественные комплексы, 
\item
отдельные объекты, относящиеся к недвижимому имуществу, 
\item
ценные бумаги, 
\item
права, удостоверенные бездокументарными ценными бумагами, 
\item
исключительные права,
\item
другое имущество. 
\end{itemize}
Не могут быть самостоятельным объектом доверительного управления деньги, за исключением случаев, предусмотренных законом.
\end{frame}

\begin{frame}{Виды договоров доверительного управления}
- индивидуальный;

- договор, предусматривающий участие клиента в Общем фонде банковского управления. 
\end{frame}

\begin{frame}{Общие фонды банковского управления (ОФБУ)}
\begin{block}{Общий фонд банковского управления (ОФБУ)}
\quad
- это имущественный комплекс, состоящий из имущества, передаваемого в доверительное управление разными лицами и объединяемого на праве общей собственности, а также приобретаемого доверительным управляющим при осуществлении доверительного управления.
\end{block}
\end{frame}
\begin{frame}{Общие фонды банковского управления (ОФБУ)}
Создание новых ОФБУ допускается только при выполнении кредитной организацией следующих условий:
\begin{itemize}[<+->]
\item
с момента государственной регистрации кредитной организации прошло не менее одного года;
\item
размер собственных средств (капитала) кредитной организации должен составлять не менее 100 млн. рублей;
\item
кредитная организация на шесть последних отчетных дат перед датой принятия решения о создании ОФБУ должна быть отнесена к I категории по финансовому состоянию.

\end{itemize}
\end{frame}

\begin{frame}
\begin{block}{Сертификат долевого участия }
\quad
- документ, свидетельствующий факт передачи имущества в доверительное управление и размер доли учредителя в составе ОФБУ. Сертификат долевого участия не является имуществом и не может быть предметом договоров купли - продажи и иных сделок.
\end{block}
\end{frame}

\begin{frame}
\begin{block}{Инвестиционная декларация }
\quad
- документ, содержащий информацию о доле каждого вида ценных бумаг (акций, облигаций, векселей и т.д.), входящих в портфель инвестиций ОФБУ, доле средств, размещаемых в валютные ценности, об отраслевой диверсификации вложений (по видам отраслей эмитентов ценных бумаг).
\end{block}
\end{frame}

\begin{frame}{Конфликт интересов }
\begin{itemize}[<+->]
\item
при предоставлении кредита учредителю управления кредитной организацией - доверительным управляющим;
\item
при осуществлении кредитной организацией профессиональной деятельности на финансовых рынках по поручению учредителя управления;
\item
при размещении депозитов учредителем управления в кредитной организации - доверительном управляющем;
\item
при осуществлении других банковских операций по поручению учредителя управления.

\end{itemize}
\end{frame}

\subsection{ОФБУ}
\begin{frame}[shrink=10]{Общий фонд банковского управления (ОФБУ)}{Схема работы}
\begin{figure}
	\center
	\begin{overprint}
	\forloop{slideno}{1}{\value{slideno} < 9}{%
		\only<\value{slideno}>{
			\includegraphics[page=\value{slideno},
			scale=.9
			% trim={<left> <lower> <right> <upper>}				
			,trim={1cm 4.5cm 4cm 0cm},clip]
			{tikz/ofbu_scheme}}}
	\end{overprint}
	\vspace*{-1.5em}
	\caption{Механизм функционирования ОФБУ}
\end{figure}
\only<2>{
1) учредители, физические и/или юридические лица, заключают договор доверительного управления имуществом с банком-доверительным управляющим, а также подписывают инвестиционную декларацию;
}
\only<3>{
2) банк выдает учредителям сертификаты долевого участия;
}
\only<4>{
3) учредители передают в Общий фонд банковского управления деньги и ценные бумаги;
}
\only<5>{
4) банк управляет имуществом ОФБУ в соответствии с инвестиционной декларацией;
}
\only<6>{
5) банк направляет периодические отчеты о результатах управления имуществом фонда учредителям и отвечает на их письменные запросы;
}
\only<7>{
6) банк взимает комиссионное вознаграждение в размере 0.5-3\% от стоимости активов фонда и вознаграждение за успех в размере 15-20\% прибыли, если она имеется;
}
\only<8>{
7) стоимость долей может быть выплачена учредителям по их требованию за счет имущества ОФБУ в определенные договором интервалы времени.
}
\end{frame}

\begin{frame}{Направления инвестиций ОФБУ}
\begin{itemize}[<+->]
\item
в акции российских АО (котирующие и не котирующиеся на бирже). Доля акций российских эмитентов не должна составлять более 90\% стоимости активов ОФБУ;
\item
в ценные бумаги РФ и субъектов РФ. Доля таких бумаг должна составлять не менее 10\% стоимости активов ОФБУ;
\end{itemize}
\end{frame}
\begin{frame}{Направления инвестиций ОФБУ}
\begin{itemize}[<+->]
\item
в долговые обязательства коммерческих банков и предприятий, доля которых должна составлять не более 15\% стоимости активов ОФБУ;
\item
в производные финансовые инструменты: фьючерсы, форвардные и иные контракты на ценные бумаги и другие финансовые активы (не более 20\% стоимости активов ОФБУ).

\end{itemize}
\end{frame}

\begin{frame}{Отчет о деятельности управляющего}
- о размере доли учредителя управления в ОФБУ;

- о расходах, понесенных управляющим по доверительному управлению имуществом за отчетный период;

- о доходах (прибыли), полученных управляющим за отчетный период;

- о доходе, приходящемся на сертификат долевого участия учредителя управления;

- о составе портфеля инвестиций, сформированного в соответствии с инвестиционной декларацией.

\end{frame}

\subsection{Риски клиентов}
\begin{frame} {Риски, о которых банк должен предупредить клиентов}
\begin{itemize}[<+->]
\item
Страновой риск и риск неперевода средств

\item
Рыночный риск

\item
Процентный риск

\item
Риск потери ликвидности

\item
Операционный риск

\item
Правовой риск

\item
Иные существенные риски

\end{itemize}

\end{frame}

\subsection{Виды инвестиций}
\begin{frame}{Услуги банков по инвестициям в недвижимость}
\begin{itemize}[<+->]
\item
юридическую экспертизу возникновения прав собственности на недвижимость, анализ возможных рисков и обременений;

\item
комплекс организационно-правовых мероприятий по надлежащему оформлению и регистрации прав собственности на недвижимость;

\item
представление интересов клиентов в гражданских и арбитражных судах по наследственным, имущественным и иным спорам, связанным с недвижимостью;

\item
организация эксплуатации, профилактических работ в домах и на прилегающих земельных участках клиентов;

\end{itemize}

\end{frame}

\begin{frame}{Услуги банков по инвестициям в недвижимость}
\begin{itemize}[<+->]
\item
представление интересов клиентов для подрядных организаций, проведение контроля над процессом строительных работ подрядных организаций;

\item
разработка финансовых и организационных схем по операциям с недвижимостью;

\item
консультации и подготовка справочной информации о налоговых последствиях, вызванных операциями на рынке недвижимости.

\end{itemize}

\end{frame}


\begin{frame}{Инвестиции в драгоценные товары}
\begin{itemize}[<+->]
\item
Драгоценные металлы - золото, серебро, платина и металлы платиновой группы (палладий, иридий, родий, рутений и осмий).

\item
Драгоценные камни - природные алмазы, изумруды, рубины, сапфиры, александриты, а также природный жемчуг в сыром (естественно обработанном) виде. К драгоценным камням приравниваются уникальные янтарные образования.

\end{itemize}

\end{frame}

\begin{frame}{Инвестиции  в предметы искусства}
\begin{itemize}[<+->]
\item
консультации по приобретению предметов искусства и антиквариата для личных и корпоративных собраний;
\item
приобретение на западных аукционах предметов искусства и антиквариата по заказу клиента и доставка их по указанному клиентом адресу в России или за рубежом;
\item
решение всех таможенных формальностей по ввозу предметов искусства и антиквариата в РФ;
\end{itemize}
\end{frame}

\begin{frame}{Инвестиции  в предметы искусства}
\begin{itemize}[<+->]
\item
получение разрешительных документов на вывоз ранее ввезенных в РФ предметов искусства и антиквариата в Министерстве культуры РФ;
\item
проведение научной экспертизы и подтверждение подлинности предметов искусства и антиквариата, их реставрация;
\end{itemize}
\end{frame}

\begin{frame}{Инвестиции  в предметы искусства}
\begin{itemize}[<+->]
\item
предоставление услуг квалифицированных оценщиков предметов искусства и антиквариата, включенных в лист признаваемых оценщиков при перестраховании предметов искусства и антиквариата, например, со стороны, Lloyd`s и Дюссельдорфского перестраховочного общества;
\item
подписку на аукционные каталоги ведущих западных и американских аукционных домов по торговле предметами искусства и антиквариата.
\end{itemize}
\end{frame}

\begin{frame}{Инвестиции в винные коллекции}
\begin{itemize}[<+->]
\item
приобретение качественного вина известных марок для личного потребления;

\item
коллекционирование вин с целью их долгосрочного хранения и возможной последующей продажи.

\end{itemize}
\end{frame}

\begin{frame}{Инвестиции в винные коллекции}
Преимущества банковских предложений:

- неограниченный выбор любых марок вина (клиент может выбрать редкие сорта вин, торгующихся только на специализированных аукционах);

- привлекательности ценовой политики;

- оптимального подбора коллекции в соответствии со вкусами или инвестиционными предпочтениями клиентов.

\end{frame}


\begin{frame}{Наследование имущества}
\begin{itemize}[<+->]
\item
составление закрытого завещания;

\item
принятие наследства;

\item
наследование прав, связанных с участием в хозяйственных товариществах и обществах, производственных и потребительских кооперативах;

\item
наследование предприятия как имущественного комплекса.

\end{itemize}

\end{frame}

\subsection{Организация доверительного управления}
\begin{frame} [ allowframebreaks]{Доверительное управление}{Организационное обеспечение банковских операций}
Для ведения операций доверительного управления в банке должен быть создан отдел доверительного управления (ОДУ).

Общее руководство отделом должен осуществлять один из вице-президентов (заместителей председателя правления) банка, который в большинстве случаев курирует также инвестиционную деятельность и операции банка с ценными бумагами. 

\pagebreak
Начальник отдела осуществляет оперативное руководство отделом. Распределение работы, планирование, решение конфликтов, корректировка условий договоров доверительного управления. Может замещать вице-президента. Аудиторская проверка проводится ежегодно (без участия должностных лиц банка)
\end{frame}

\begin{frame}{Второй уровень ОДУ выполняет операции}
\begin{itemize}[<+->]
\item
индивидуального доверительного управления; (распоряжение и управление имуществом клиентов, участие в прибылях, представительство)
\item
развития операций; (реклама, личные контакты, планирование, расширение операций доверительного управления)
\item
обслуживания; (обеспечивают информацию по налогообложению, инвестициям, управлению имуществом, юридические услуги)
\end{itemize}
\end{frame}

\begin{frame}{Второй уровень ОДУ выполняет операции}
\begin{itemize}[<+->]
\item
учета; (учет, контроль, ведение архива, хранение ценностей, кассовое обслуживание)
\item
корпоративное доверительное управление(облигации, передача и учет акций, платежи);
\item
доверительное управление на уровне филиалов банка. 
\end{itemize}
\end{frame}


\begin{frame}{Третий уровень ОДУ – функциональные службы}
\begin{itemize}[<+->]
\item
ценных бумаг;
\item
денежных средств и материальных ценностей;
\item
взаимодействия с контролирующими органами;
\item
взаимодействия с бухгалтерией.
\end{itemize}
\end{frame}


\subsection{Контрольные вопросы}
\begin{frame}[ allowframebreaks ]{Контрольные вопросы}
1. Понятие договора управления имуществом, виды договоров доверительного управления.

2. Объекты доверительного управления.

3. Понятие общего фонда банковского управления. Сертификат долевого участия. Инвестиционная декларация.

\pagebreak
4. Схема работы общего фонда банковского управления. 

5. Направления инвестиций общих фондов банковского управления. Риски пайщиков фонда.

6. Организация отдела доверительного управления в банке.

\end{frame}

\setbeamercovered{transparent}
\end{document}
