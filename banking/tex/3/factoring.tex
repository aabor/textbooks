% !TeX program = lualatex -synctex=1 -interaction=nonstopmode --shell-escape %.tex

\documentclass[_Banking_p3.tex]{subfiles}
\begin{document}

\setbeamercovered{invisible}

\subsection{Нормативные документы}
\begin{frame}[allowframebreaks]{Нормативные документы}
  \begin{thebibliography}{10}
  
  \beamertemplatearticlebibitems
    \bibitem{bb4}
"Гражданский кодекс Российской Федерации (часть вторая)" от 26.01.1996 N 14-ФЗ (ред. от 28.03.2017) гл. 42 "Заем и кредит"

    \bibitem{bb4}
"Гражданский кодекс Российской Федерации (часть вторая)" от 26.01.1996 N 14-ФЗ (ред. от 28.03.2017) гл. 43 "Финансирование под уступку денежного требования"

    \bibitem{bb4}
Федеральный закон от 11.03.1997 N 48-ФЗ "О переводном и простом векселе"

\pagebreak

    \bibitem{bb4}
"Конвенция о Единообразном Законе о переводном и простом векселе" (Заключена в Женеве 07.06.1930)

    \bibitem{bb4}
Постановление ЦИК СССР и СНК СССР от 07.08.1937 N 104/1341 "О введении в действие Положения о переводном и простом векселе"

\pagebreak
    \bibitem{bb4}
"Конвенция, имеющая целью разрешение некоторых коллизий законов о переводных и простых векселях" (Заключена в Женеве 07.06.1930) (Вступила в силу для СССР 25.11.1936)

\pagebreak

    \bibitem{bb4}
"Конвенция УНИДРУА по международным факторинговым операциям" (Заключена в г. Оттаве 28.05.1988)
  
    \bibitem{bb4}
  Приказ Минтруда России от 19.03.2015 N 169н "Об утверждении профессионального стандарта "Специалист по факторинговым операциям" 
  
    \bibitem{bb4}
"Налоговый кодекс Российской Федерации (часть вторая)" от 05.08.2000 N 117-ФЗ (ред. от 29.07.2017). Статья 214.4. Особенности определения налоговой базы по операциям займа ценными бумагами

\pagebreak

    \bibitem{bb4}
"Кодекс Российской Федерации об административных правонарушениях" от 30.12.2001 N 195-ФЗ (ред. от 29.07.2017) (с изм. и доп., вступ. в силу с 10.08.2017). Статья 5.53. Незаконные действия по получению и (или) распространению информации, составляющей кредитную историю.

\pagebreak
    \bibitem{bb4}
"Уголовный кодекс Российской Федерации" от 13.06.1996 N 63-ФЗ (ред. от 29.07.2017). Статья 177. Злостное уклонение от погашения кредиторской задолженности


  \end{thebibliography}
\end{frame}

\subsection{Определения}
\begin{frame} [ allowframebreaks]{Определения}
\begin{block}{С правовой точки зрения факторинг }
\quad
- уступка прав требования.
\end{block}

\begin{block}{С экономической точки зрения факторинг  }
\quad
- это торговля долговыми обязательствами (разновидность учетных операций). Банковский факторинг - это покупка ими срочных платежных требований, возникающих из поставки товаров (выполнения работ, оказания услуг), в том числе прав требования возврата выданных кредитов.
\end{block}

\pagebreak

\begin{block}{Банковский факторинг }
\quad
- это покупка ими срочных платежных требований, возникающих из поставки товаров (выполнения работ, оказания услуг), в том числе прав требования возврата выданных кредитов.
\end{block}

\end{frame}

\begin{frame}{Финансирование под уступку денежного требования}{ГК РФ гл. 43}

Ст. 824. Одна сторона (финансовый агент) передает или обязуется передать другой стороне (клиенту) деньги в счет денежного требования клиента (кредитора) к третьему лицу (должнику), вытекающего из предоставления клиентом товаров, выполнения им работ или оказания услуг третьему лицу, а клиент уступает или обязуется уступить финансовому агенту это денежное требование.

\end{frame}

\begin{frame}{Финансовыми агентами могут быть }{ГК РФ ст. 825}
- банки и иные кредитные организации;

- другие коммерческие организации, имеющие разрешение (лицензию) на деятельность такого вида.

\end{frame}

\begin{frame}{Предмет уступки }{ГК РФ ст. 826-829}
Существующее требование – денежное требование, срок платежа по кото-рому уже наступил.

Будущее требование – право на получение денег, которое возникнет в бу-дущем.

\end{frame}


\begin{frame}{ФЗ "О банках и банковской деятельности"}
Ст. 5 "Банковские операции и другие сделки кредитной организации":

- выдача поручительств за третьих лиц, предусматривающих исполнение обязательств в денежной форме;

- приобретение права требования от третьих лиц исполнения обязательств в денежной форме.

\end{frame}

\subsection{Виды факторинга}
\begin{frame} {Виды факторинга}{По территориальному признаку:}
\begin{itemize}[<+->]
\item
внутренний – поставщик, его клиент и банк-фактор, находятся в одной стране;

\item
международный – поставщик находится в одной стране, а его клиент в другой.

\end{itemize}

\end{frame}
\begin{frame}[ allowframebreaks ] {Виды факторинга}{По видам операций:}
\begin{itemize}
\item
Взаимный (двухфакторный). При обслуживании экспортера банк передает определенный объем работ факторинговой компании, действующей в стране импортера. В свою очередь банк будет действовать в своей стране по поручению иностранного фактора. Экспортеру достаточно заключить соглашение только с банком, в котором он обслуживается. (высокая стоимость).

\pagebreak
\item
Полного обслуживания: гарантированный приток средств, управление кредитом, учет реализации, кредитование в форме предварительной оплаты либо оплаты долгов (за минусом издержек) к определенным датам.

\end{itemize}

\end{frame}

\begin{frame}[ allowframebreaks ] {Виды факторинга}{По осведомленности участников операции:}
\begin{itemize}
\item
Конвенционный (широкий, открытый). Поставщик указывает на своих счетах, что требование продано банку и финансовое, бухгалтерское, юридическое и прочее обслуживание ведет факторинговое подразделение банка.

\pagebreak
\item
Конфиденциальный. Контрагенты поставщика не осведомляются о кредитовании его продаж банком, который ограничивается выполнением только некоторых операций: покупкой права на получение денег от покупателей, оплатой долгов и т.д. Риск и стоимость конфиденциального факторинга выше, чем конвенционного, и он значительно дороже.

\end{itemize}


\end{frame}
\begin{frame} {Виды факторинга}{По моменту финансирования:}
\begin{itemize}[<+->]
\item
В форме предварительной оплаты (применяется в исключительных случаях). 

\item
К определенной дате. Банк незамедлительно платит поставщику сумму или часть суммы акцептованных плательщиком платежных требований за поставленные товары. Остальную часть суммы (за вычетом комиссионного вознаграждения) банк выплачивает после поступления средств от плательщика. Плательщик должен перечислить в пользу банка сумму долга и пеню за просрочку платежа.

\end{itemize}


\end{frame}
\begin{frame} [ allowframebreaks ] {Виды факторинга}{По праву регресса:}
\begin{itemize}
\item
с правом регресса. Банк имеет право продать поставщику любое неоплаченное долговое требование в случае отказа плательщика от платежа независимо от причин отказа, включая отсутствие у плательщика средств. В этом случае плательщик не оплачивает страхование кредитного риска, но должен тщательно отслеживать кредитоспособность своих контрагентов.

\pagebreak
\item
без права регресса. При обслуживании без права регресса банк берет на себя риск неплатежей плательщиками, состав которых он предварительно одобрил. Однако если долговое требование признано недействительным, то банк имеет право регресса к поставщику. 

\end{itemize}

\end{frame}
\begin{frame} {Виды факторинга}{По виду дебиторской задолженности клиента:}
\begin{itemize}[<+->]
\item
срочная;

\item
просроченная.
\end{itemize}

\end{frame}

\subsection{Организация факторинга}
\begin{frame}{Организация факторинговых операций банка}

Факторинг включает в себя следующие услуги: 
\begin{itemize}[<+->]
\item
кредитование сбыта, 
\item
принятие риска неплатежа, 
\item
бухгалтерский учет дебиторов, 
\item
контроль и индексацию задолженности, 
\item
статистику продаж, 
\item
др.

\end{itemize}
\end{frame}

\begin{frame}{Участники банковского факторинга}
\begin{itemize}[<+->]
\item
банк-фактор, покупатель неоплаченного должником требования;

\item
первоначальный кредитор (клиент банка, он же поставщик или продавец);

\item
должник, получивший от клиента (поставщика) отсрочку платежа.

\end{itemize}

\end{frame}
\begin{frame}{Договор факторинга}
\begin{itemize}[<+->]
\item
процент от суммы платежного требования, выплачиваемый фактором в пользу поставщика на следующий рабочий день после даты его переуступки;

\item
срок выплаты определенного процента от суммы платежного требования, отсчитываемый с даты его переуступки;

\item
остаток суммы платежного требования, выплачиваемый поставщику после получения средств от плательщика;

\end{itemize}
\end{frame}
\begin{frame}{Договор факторинга}
\begin{itemize}[<+->]
\item
срок кредитования;

\item
проценты за кредит;

\item
пени за просрочку;

\item
лимит кредитования (ежемесячный или на сумму конкретной сделки).
\end{itemize}

\end{frame}

\begin{frame}{Работы в рамках договора факторинга}
\begin{itemize}[<+->]
\item
расчет текущего сальдо оборотного фонда фактора;

\item
проверка соответствия типа факторинговой операции текущему состоянию оборотного фонда фактора и принятой стратегии формирования его портфеля;

\item
выбор подходящего вида факторинга;

\item
маркетинг отгруженных товаров;

\item
оценка возможных рисков;

\end{itemize}
\end{frame}
\begin{frame}{Работы в рамках договора факторинга}
\begin{itemize}[<+->]
\item
моделирование условий оплаты и вариантов расчетов с плательщиками;

\item
расчет лимитов кредитования;

\item
расчет комиссионных;

\item
анализ выбранной политики амортизации ссуды;

\end{itemize}
\end{frame}
\begin{frame}{Работы в рамках договора факторинга}
\begin{itemize}[<+->]
\item
моделирование введения векселя в схему расчетов с плательщиками;

\item
учет погашения кредита;

\item
учет векселя.


\end{itemize}

\end{frame}


\begin{frame} {Оценка доходности и эффективности факторинга}
\begin{itemize}[<+->]
\item
выбор условий кредитования;

\item
расчет аккумулированных платежей;

\item
бухгалтерский учет расчетов (платежей).

\end{itemize}

\end{frame}

\begin{frame}{Услуги банка-фактора плательщикам}
\begin{itemize}[<+->]
\item
оплата их платежных документов,

\item
проведения депозитных и кассовых операций, маркетинга,

\item
организация бухгалтерского учета.

\end{itemize}

\end{frame}
\begin{frame}{Вознаграждение банка-фактора}
\begin{itemize}[<+->]
\item
Комиссионные за обслуживание. Комиссия взимается за освобождение клиента от необходимости своими силами вести учет, страховаться от сомнительных долгов и рассчитывается как определенный процент от суммы счетов-фактур. 

\item
Плата за кредит. Ее размер определяется на основе ежедневного дебетового сальдо на счете поставщика за период между получением средств банка и датой поступления платежа от плательщика. 

\end{itemize}

\end{frame}
\begin{frame}{Доходы банка-фактора }
\begin{itemize}[<+->]
\item
процент за кредит;

\item
проценты от оборота (за 100\% риск неуплаты должником по требованиям);

\item
комиссии за пакет услуг (инкассо, ведение бухгалтерии, напоминания) в процентах от общей стоимости учтенных счетов-фактур. Размер комиссионных устанавливается в зависимости от оборота клиента, степени риска и объемом необходимых работ.

\end{itemize}

\end{frame}

\begin{frame} {Банк устанавливает следующие лимиты по операции}
\begin{itemize}[<+->]
\item
лимит кредитования. 

\item
лимит отгрузки. 

\item
лимит страхования отдельных сделок. 

\end{itemize}

\end{frame}
\begin{frame}{Достоинства факторинга для продавца}
\begin{itemize}[<+->]
\item
зачисление до 80\% стоимости отгруженной продукции на счет продавца на следующий банковский день после получения извещения от покупателя о приемке товара;

\item
обеспечение банком получения продавцом оставшейся части выручки в течении 2-3 банковских дней после истечения отсрочки платежа;

\item
увеличение рынка сбыта за счет покупателей, нуждающихся в отсрочке платежа.

\end{itemize}

\end{frame}
\begin{frame}{Достоинства факторинга для покупателя}
\begin{itemize}[<+->]
\item
получение товарного кредита с оплатой после реализации;

\item
факторинговая комиссия ниже ставок банковского кредита;

\item
сумма требуемого обеспечения обычно 20-40\% от общей стоимости кон-тракта на закупку товаров;

\item
различные виды обеспечения по факторинговым операциям (напр. Цен-ные бумаги).

\end{itemize}

\end{frame}

\begin{frame}{Заключение договора о факторинге}

\begin{itemize}[<+->]
\item
предприятие подает в банк заявку на факторинговое обслуживание;

\item
предприятие предоставляет список своих дебиторов, банк анализирует их платежеспособность.
\end{itemize}


Банк предлагает клиенту на выбор: открытый или конфиденциальный факторинг.

Поставщик может предъявить векселя в банк немедленно, погасить ими свои обязательства или дождаться срока их погашения.
\end{frame}

\subsection{Схемы факторинга}
\begin{frame}[shrink=15]{Схемы факторинга}
\begin{figure}
\center
\begin{overprint}
	\forloop{slideno}{1}{\value{slideno} < 10}{%
		\only<\value{slideno}>{
			\includegraphics[page=\value{slideno},
			scale=.9
			% trim={<left> <lower> <right> <upper>}				
			,trim={.5cm 4.5cm 4cm 0cm},clip]
			{tikz/factoring_1operation}}}
\end{overprint}
\vspace*{-2em}
\caption{Схема 1. Вариант однофакторинговой операции}
\end{figure}
\only<2>{
1) заключение договора с покупателем о факторинговом обслуживании, открытие кредитной линии;
}
\only<3>{
2) подписание договора с поставщиком о факторинговом обслуживании, включая гарантирование платежа и уступку права получения выручки в пользу банка;
}
\only<4>{
3) сообщение покупателю об уступке банку прав на выручку;
}
\only<5>{
4) отгрузка товара;
}
\only<6>{
5) подтверждение приемки товара;
}
\only<7>{
6) финансирование поставщика (например, до 80\% от суммы счета);
}
\only<8>{
7) платеж (100\%) плюс комиссия за обслуживание;
}
\only<9>{
8) выплата оставшейся суммы (за вычетом комиссии и стоимости финансирования).
}
\end{frame}

\begin{frame}[shrink=15]
\begin{figure}
\center
\begin{overprint}
	\forloop{slideno}{1}{\value{slideno} < 10}{%
		\only<\value{slideno}>{
			\includegraphics[page=\value{slideno},
			scale=.9
			% trim={<left> <lower> <right> <upper>}				
			,trim={.5cm 5.5cm 4cm 0cm},clip]
			{tikz/factoring_2operation}}}
\end{overprint}
\vspace*{-2.5em}
\caption{Схема 2. Вариант двухфакторинговой операции}
\end{figure}

\only<2>{
1) подписание договора о факторинговом обслуживании покупателя (включая обеспечение гарантии);
}
\only<3>{
2) подписание соглашения между факторами о гарантировании платежей и инкассации выручки после окончания отсрочки платежей;
}

\only<4>{
3) подписание договора о факторинговом обслуживании поставщика;
}

\only<5>{
4) уступка прав на получение выручки от продаж;
}

\only<6>{
5) отгрузка товара;
}

\only<7>{
6) финансирование поставщика (до 80\% суммы счета) после получения подтверждения приемки товаров покупателем;
}

\only<8>{
7) после окончания отсрочки – платеж от покупателя (100\% плюс плата за обязательство);
}

\only<9>{
8) выплата оставшейся суммы поставщику (за вычетом стоимости финансирования и комиссии).
}

\end{frame}


\begin{frame}[shrink=15] 
\begin{figure}
\center
\begin{overprint}
	\forloop{slideno}{1}{\value{slideno} < 9}{%
		\only<\value{slideno}>{
			\includegraphics[page=\value{slideno},
			scale=.9
			% trim={<left> <lower> <right> <upper>}				
			,trim={.5cm 4cm 3cm 0cm},clip]
			{tikz/factoring_export}}}
\end{overprint}
\vspace*{-2em}
\caption{Схема 3. Вариант факторинговых расчетов при экспорте товаров и услуг}
\end{figure}
\only<2>{
1) гарантия платежа в объеме закупочного лимита;
}
\only<3>{
2) поручение на получение экспортной выручки;
}

\only<4>{
3) отгрузка товара;
}

\only<5>{
4) финансирование экспорта после отгрузки (до 80\% от суммы счета на период отсрочки);
}

\only<6>{
5) 100\% оплата счетов в сроки платежа;
}

\only<7>{
6) перечисление экспортной выручки;
}

\only<8>{
7) выплата оставшейся суммы за вычетом факторинговой комиссии и стоимости финансирования.
}

\end{frame}


\begin{frame}[shrink=15]
\begin{figure}
\center
\begin{overprint}
	\forloop{slideno}{1}{\value{slideno} < 9}{%
		\only<\value{slideno}>{
			\includegraphics[page=\value{slideno},
			scale=.9
			% trim={<left> <lower> <right> <upper>}				
			,trim={.5cm 4cm 3cm 0cm},clip]
			{tikz/factoring_import}}}
\end{overprint}
\vspace*{-2em}
\caption{Схема 4. Вариант факторинговых расчетов при импорте товаров и услуг}
\end{figure}

\only<2>{
1) обеспечение гарантии платежа;
}
\only<3>{
2) гарантия платежа;
}

\only<4>{
3) уступка прав;
}

\only<5>{
4) отгрузка;
}

\only<6>{
5) финансирование экспортера после отгрузки (до 80\% от суммы счета на период отсрочки);
}

\only<7>{
6) 100\% оплата счетов в срок платежа;
}

\only<8>{
7) выплата оставшейся суммы (за вычетом комиссии и стоимости финансирования).
}

\end{frame}



\subsection{Форфейтинг}
\begin{frame}{Форфейтинг}{Определение и схемы проведения операций}

\begin{block}{Форфейтинг }
\quad
- это покупка банком долгов его клиентов, выраженных в оборотных ценных бумагах, получение права требовать у должников удовлетворения по таким ценным бумагам и реализация такого права.
\end{block}

Основными оборотными документами, используемыми при форфейтинге, являются векселя, однако, возможно использование и других ценных бумаг, важно, только, чтобы они содержали абстрактное обязательство.
\end{frame}

\begin{frame}{Механизм форфейтинга используется}
\begin{itemize}[<+->]
\item
В финансовых сделках – в целях быстрой реализации финансовых обяза-тельств долгосрочного характера.

\item
В экспортных сделках – для содействия более быстрому поступлению де-нег экспортеру, выдавшему иностранному покупателю товарный кредит на крупную сумму и с длительной рассрочкой платежа.
\end{itemize}
\end{frame}

\begin{frame}{Преимущества форфейтинга для экспортера}
\begin{itemize}[<+->]

\item
Фиксированая процентная ставка на весь период сделки.

\item
Большая часть финансовых рисков перенесена на форфейтера.

\item
Финансирование сразу после поставки товаров (услуг) по основному договору.

\item
Быстрое согласование условий сделки.

\item
Конфиденциальность.

\item
Экспортер не отвечает за управление дебиторской задолженностью.

\end{itemize}
\end{frame}

\begin{frame}{}
Для импортера форфейтирование позволяет получить средне и долгосрочный кредит и быстро оформить сделку.

Банк получает более высокий доход, чем от кредитования и может реализовать приобретенные долговые обязательства на вторичном рынке.
\end{frame}


\begin{frame}{Недостатки форфейтинга}
\begin{itemize}[<+->]

\item
Для экспортера – высокая стоимость. 


\item
Импортер, выдавший безусловное обязательство, несет риск получения некачественной продукции.

\item
Банк несет большую часть рисков по сделке, поскольку у него отсутствует право регресса в случае неуплаты импортером долга.
\end{itemize}
\end{frame}

\begin{frame}{Схема форфейтинговой операции}
1. Инициатор предоставляет форфейтеру (банку) краткое описание будущей сделки, данные о нужном объеме финансирования, валюте сделки, ее сроках.

2. Банк проводит кредитный анализ, оценивает предполагаемые риски.

3. Форфейтер посылает  предложения инициатору сделки. Последний должен письменно подтвердить свое согласие и получить согласие контрагента.

\end{frame}

\begin{frame}{Схема форфейтинговой операции}
4. Экспортер и импортер заключают договор о поставках.

5. Форфейтер и его клиент подписывают договор финансирования под уступку денежного требования. Существенные условия – вид форфейтируемых обязательств, тип банковской гарантии, механизм погашения обязательств, валюта платежа, размер дисконтирования.
\end{frame}

\begin{frame}{Причины использования векселей при форфейтинге}
\begin{itemize}[<+->]

\item
значительный опыт вексельного обращения в мире;

\item
на международном уровне согласована большая часть вопросов правового регулирования 
\item
применения векселей.

\item
существует вторичный рынок векселей.

\item
на вексель может быть получено дополнительное обеспечение в виде банковской гарантии и аваля.

\end{itemize}
\end{frame}

\subsection{Контрольные вопросы}
\begin{frame}[ allowframebreaks ]{Контрольные вопросы}
1. Понятие факторинга с правовой и экономической точки зрения. Банковский факторинг.

2. Классификация видов факторинга по различным признакам.

3. Организация факторинговых операций в банке. Договор факторинга.

\pagebreak
4. Достоинства факторинга для продавца и для покупателя.

5. Схемы факторинга: однофакторный, двухфакторинговый, экспортный факторинг, импортный факторинг.

6. Понятие форфейтинга, схема форфейтинговой операции.

\end{frame}

\setbeamercovered{transparent}
\end{document}
