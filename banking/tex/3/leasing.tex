% !TeX program = lualatex -synctex=1 -interaction=nonstopmode %.tex

\documentclass[_Banking_p3.tex]{subfiles}

\begin{document}

\setbeamercovered{invisible}

\subsection{Нормативные документы}
\begin{frame}[allowframebreaks]{Лизинговые сделки}{Нормативные документы}
  \begin{thebibliography}{10}
  
  \beamertemplatearticlebibitems

    \bibitem{bb2}
"Гражданский кодекс Российской Федерации (часть вторая)" от 26.01.1996 N 14-ФЗ (ред. от 28.03.2017). § 6. Финансовая аренда (лизинг)

    \bibitem{bb2}
Федеральный закон от 29.10.1998 N 164-ФЗ (ред. от 26.07.2017) "О финансовой аренде (лизинге)"

\pagebreak

    \bibitem{bb2}
Постановление Правительства РФ от 27.06.1996 N 752 (ред. от 06.06.2002) "О государственной поддержке развития лизинговой деятельности в Российской Федерации"


    \bibitem{bb2}
Постановление Правительства РФ от 30.12.2011 N 1212 (ред. от 24.05.2017) "Об утверждении Правил предоставления субсидий из федерального бюджета на возмещение российским авиакомпаниям...

\pagebreak

    \bibitem{bb2}
Постановление Правительства РФ от 20.07.2016 N 702 "О применении базовых индикаторов при расчете параметров субсидирования процентной ставки за счет средств федерального бюджета...


    \bibitem{bb2}
Постановление Правительства РФ от 03.09.1998 N 1020 (ред. от 06.06.2002) "Об утверждении Порядка предоставления государственных гарантий на осуществление финансовой аренды (лизинга)"

\pagebreak

    \bibitem{bb2}
Постановление Правительства РФ от 26.09.1994 N 1085 (ред. от 06.06.2002) "Об организации обеспечения агропромышленного комплекса продукцией племенного животноводства на основе финансовой аренды (лизинга)"

    \bibitem{bb2}
Приказ Минфина РФ от 17.02.1997 N 15 (ред. от 23.01.2001) "Об отражении в бухгалтерском учете операций по договору лизинга"

\pagebreak

    \bibitem{bb2}
"Методические рекомендации по расчету лизинговых платежей" (утв. Минэкономики РФ 16.04.1996)

    \bibitem{bb2}
Постановление Пленума ВАС РФ от 14.03.2014 N 17 "Об отдельных вопросах, связанных с договором выкупного лизинга"

  \end{thebibliography}

\end{frame}


\subsection{Определения}
\begin{frame}{История возникновения лизинга}
\begin{block}
\quad
Богатство состоит в пользовании, а не в праве собственности

Аристотель
\end{block}
\begin{itemize}
\item
Развитие лизинга произошло в Америке в начале 50-х гг. XX.
\item
Лизинг интенсивно развивался в США, Канаде и Европе в 60-80 гг.
\end{itemize}
\end{frame}

\begin{frame}{}
В США из общего количества переданного в лизинг имущества (оборудования) составляют:

\begin{itemize}
\item
33,2\% средства вычислительной техники, 
\item
9,5\% - оргтехника, 
\item
21,9\% - промышленное оборудование, 
\item
4,9\% - транспортные средства, 
\item
1,3\% - строительные машины, 
\item
10,1\% - торговое оборудование.
\end{itemize}
\end{frame}

\begin{frame}{Соотношение отдельных видов лизинга в США}
\begin{itemize}
\item
сделки финансовой аренды (лизинга) - 65\% от общей стоимости заключенных контрактов;
\item
сделки аренды – 10\%;
\item
сделки ``леведж''-лизинга - 25\%.
\end{itemize}
\end{frame}

\begin{frame}[shrink=25]
% Table generated by Excel2LaTeX from sheet 'Лист2'
\begin{table}[htbp]
  \centering
  \caption{Рейтинг лизинговых компаний\\ по сумме лизингового портфеля на 1.08.2017г.}
  	\begin{tabularx}{\linewidth}[b]{@{}>{\raggedright\arraybackslash}Xr@{}}
    \toprule
    Компания & Портфель, млрд рублей. \\
    \midrule
    ВЭБ-лизинг & 515,8 \\
    ВТБ Лизинг & 338,8 \\
    Сбербанк Лизинг & 211,9 \\
    ГТЛК & 90,0 \\
    ТрансФин-М & 89,0 \\
    Газпромбанк Лизинг & 71,9 \\
    Альфа-Лизинг & 59,0 \\
    Ильюшин Финанс Ко & 46,5 \\
    Бизнес Альянс & 34,4 \\
    Европлан & 33,5 \\
    \bottomrule
    \end{tabularx}%
  \label{tab:addlabel}%
\end{table}%
Источник: Информационный портал All-Leasing.Ru, 2017. \url{http://www.all-leasing.ru}
\end{frame}

\begin{frame}[shrink=25]
% Table generated by Excel2LaTeX from sheet 'Лист3'
\begin{table}[htbp]
  \centering
  \caption{Рейтинги кредитоспособности лизинговых компаний\\ по состоянию на 24.08.2017}
  	\begin{tabularx}{\linewidth}[b]{@{}>{\raggedright\arraybackslash}Xccc@{}}
    \toprule
    Компания & Рейтинг & Дата  & Прогноз \\
    \midrule
    ЮниКредит Лизинг & ruAA  & 29.05.2017 & Стабильный \\
    РЕСО-Лизинг & ruA+  & 27.06.2017 & Стабильный \\
    ЛК "Европлан" & ruA   & 10.07.2017 & Развивающийся \\
    Элемент Лизинг & ruA-  & 08.06.2017 & Стабильный \\
    ТрансФин-М & ruA-  & 23.06.2017 & Стабильный \\
    ЛК «Дельта» & ruBBB+ & 16.06.2017 & Стабильный \\
    Объединенная ЛК & ruBBB+ & 15.06.2017 & Стабильный \\
    СТОУН-ХХI & ruBBB & 06.06.2017 & Стабильный \\
    МКБ-лизинг & ruBBB- & 10.07.2017 & Стабильный \\
    \bottomrule
    \end{tabularx}%
  \label{tab:addlabel}%
\end{table}%
Источник: Рейтинговое агентство Эксперт РА (RAEX), 2017. \url{https://raexpert.ru/}.
\end{frame}

\begin{frame} [ allowframebreaks]{Определения}

\begin{block}{Лизинг}
\quad
 является своеобразной инвестицией капитала, поскольку предполагает вложение средств в материальное имущество с целью получения доходов, 
\end{block}

\begin{block}{Лизинг}
\quad
сохраняет черты кредита (предоставляется на началах платности, срочности, возвратности).
\end{block}

\pagebreak
\begin{block}{Лизинг} 
\quad
представляет собой особый вид предпринимательской деятельности, направленной на инвестирование временно свободных или привлеченных финансовых средств, когда арендодатель (лизингодатель) обязуется приобрести в собственность обусловленное договором имущество у определенного продавца и предоставить это имущество арендатору (лизингополучателю) за плату во временное пользование для предпринимательских целей.
\end{block}
\end{frame}

\begin{frame}
\begin{block}{Финансовая аренда (лизинг)}
\quad
Ст. 665 ГК РФ: По договору финансовой аренды (договору лизинга) арендодатель обязуется приобрести в собственность указанное арендатором имущество у определенного им продавца и предоставить арендатору это имущество за плату во временное владение и пользование. 

Арендодатель в этом случае не несет ответственности за выбор предмета аренды и продавца.
\end{block}
\end{frame}

\begin{frame}[ allowframebreaks ]{Конвенция УНИДРУА}{Международный институт по унификации частного права (фр.)}
Сделка финансового лизинга - это сделка, включающая следующие характеристики:

а) арендатор определяет оборудование и выбирает поставщика, не полагаясь в первую очередь на опыт и суждение арендодателя;

\pagebreak
б) оборудование приобретается арендодателем в связи с договором лизинга, который, и поставщик осведомлен об этом, заключен или должен быть заключен между арендодателем и арендатором; и

в) периодические платежи, подлежащие выплате по договору лизинга, рассчитываются, в частности, с учетом амортизации всей или существенной части стоимости оборудования.
\end{frame}

\begin{frame}[ allowframebreaks]{ФЗ ``О финансовой аренде (лизинге)''}
\begin{block}{Лизинг}
\quad
- совокупность экономических и правовых отношений, возникающих в связи с реализацией договора лизинга, в том числе приобретением предмета лизинга
\end{block}

\pagebreak
\begin{block}{Договор лизинга }
\quad
- договор, в соответствии с которым арендодатель (далее - лизингодатель) обязуется приобрести в собственность указанное арендатором (далее - лизингополучатель) имущество у определенного им продавца и предоставить лизингополучателю это имущество за плату во временное владение и пользование.
\end{block}

\pagebreak
Договором лизинга может быть предусмотрено, что выбор продавца и приобретаемого имущества осуществляется лизингодателем.
\end{frame}

\begin{frame}[ allowframebreaks]{Субъекты арендных операций}

\begin{block}{Лизингодатель }
\quad
— это ф./ю. л., которое за счет привлеченных и (или) собственных средств приобретает в ходе реализации договора лизинга в собственность имущество и предоставляет его в качестве предмета лизинга лизингополучателю за определенную плату, на определенный срок и на определенных условиях во временное владение и в пользование с переходом или без перехода к лизингополучателю права собственности на предмет лизинга. 
\end{block}

\pagebreak
\begin{block}{Лизингополучатель }
\quad
— ф./ю. л., которое в соответствии с договором лизинга обязано принять предмет лизинга в соответствии с договором лизинга. 
\end{block}

\begin{block}{Продавец}
\quad
— ф./ю. л., которое в соответствии с договором купли-продажи с лизингодателем продает лизингодателю в обусловленный срок имущество, являющееся предметом лизинга. 
\end{block}
\pagebreak
Косвенными участниками лизинговой сделки выступают коммерческие и инвестиционные банки, кредитующие лизингодателя и выступающие гарантами сделок, страховые компании, посредники, лизинговые брокеры.
\end{frame}
\begin{frame}[shrink=25]
\begin{figure}
\center
\begin{overprint}
	\forloop{slideno}{1}{\value{slideno} < 9}{%
		\only<\value{slideno}>{
			\includegraphics[page=\value{slideno},
			scale=1.4
			% trim={<left> <lower> <right> <upper>}				
			%,trim={.5cm 6.5cm 4cm 0cm}
			,clip]
			{tikz/leasing_scheme}}}
\end{overprint}
\vspace*{-2em}
\caption{Классическая схема лизинга}
\end{figure}
\only<2>{
1) потенциальный лизингополучатель определяет продавца необходимого имущества (выбор продавца может быть осуществлен и лизинговой компанией); 
}
\only<3>{
2) лизингополучатель заключает договор с лизинговой компанией;
}
\only<4>{
3) банк выдает лизингодателю кредит на покупку предмета лизинга;
}
\only<5>{
4) лизингодатель приобретает в собственность имущество, выбранное лизингополучателем, одновременно, заключая с банком договор о залоге предмета лизинга;
}
\only<6>{
5) лизингодатель страхует приобретенное имущество и делает периодические страховые взносы за счет лизинговых платежей лизингополучателя;
}
\only<7>{
6) лизингодатель сдает имущество в аренду лизингополучателю;
}
\only<8>{
7) лизингополучатель производит лизинговые платежи лизингодателю.
}
\end{frame}

\begin{frame}{Специфика лизингового бизнеса }
Арендатору (лизингополучателю) передается новое имущество, специально выбранное и приобретенное по желанию арендатора (лизингополучателя). 

Данное условие отражено в ГК РФ.
\end{frame}

\begin{frame}[allowframebreaks]{Объекты аренды}

Любые непотребляемые вещи, в том числе предприятия и другие имущественные комплексы, здания, сооружения, оборудование, транспортные средства и другое движимое н недвижимое имущество, которое может использоваться для предпринимательской деятельности. 

\pagebreak
Предметом договора лизинга не могут быть земельные участки и другие природные объекты, а также имущество, которое федеральными законами запрещено для свободного обращения или для которого установлен особый порядок обращения.
\end{frame}

\begin{frame}[allowframebreaks]{Договор лизинга}
\begin{itemize}
\item
Письменная форма договора (п. 1 ст. 15 Федерального закона «О финансовой аренде (лизинге)»)

\item
Подлежит регистрации, если в качестве объекта лизинга выступает недвижимость или предприятие.

\pagebreak
\item
Обязанность участников договора по осуществлению ремонта имущества устанавливается договором.

\item
Возможность сдачи арендованного имущества в субаренду допускается с письменного согласия лизингодателя. 

\item
Порядок расторжения договора: общепринятый (глава 29 ГК РФ).
\end{itemize}
\end{frame}

\begin{frame}[allowframebreaks]{Конвенция УНИДРУА}
- лизингополучатель несет риски порчи и потери имущества, являющегося объектом сделки финансового лизинга.

- освобождение лизингодателя – собственника оборудования от ответственности за ущерб или убытки, причиненные этим оборудованием лизингополучателю или третьим лицам. 

\pagebreak
- право пользователя непосредственно обращаться с претензиями по оборудованию к поставщику (при том, что поставщик не несет ответственности одновременно перед лизингодателем и пользователем за один и тот же ущерб).

\end{frame}

\begin{frame}[ allowframebreaks]{Российская практика лизинга}
\begin{itemize}
\item
К предмету лизинга предъявляется требование об использовании его исключительно для предпринимательских целей

\item
Земельные участки и другие природные объекты не могут быть пред-мегом договора лизинга 

\pagebreak
\item
Лизинг – это прежде всего инвестиционная деятельность (Ст. 2 ФЗ). При отсутствии инвестиций лизинговая сделка может быть признана притворной.

\item
Лицензирование лизинговой деятельности не требуется. Сделка признается лизинговой в каждом отдельном случае налоговым органом.
\pagebreak
\item
Лизинговые компании имеют право привлекать средства юридических и (или) физических лиц (резидентов Российской Федерации и нерезидентов Российской Федерации) (Ст. 5 ФЗ).
\item
ФЗ «О финансовой аренде (лизинге)»: отменено деление лизинга на оперативный и финансовый.

\pagebreak
\item
Ст. 7 ФЗ «О финансовой аренде (лизинге): Основными формами лизинга являются внутренний лизинг (между резидентами) и международный лизинг (лизингодатель или лизингополучатель является нерезидентом Российской Федерации).
\end{itemize}
\end{frame}


\subsection{Лизинг, кредит, аренда}
\begin{frame}[shrink=25]{Сравнение лизинга, кредита и аренды}
% Table generated by Excel2LaTeX from sheet 'Сравнение'
\begin{table}[htbp]
  \centering
  \caption{Сравнение лизнга и кредита}
	\begin{tabularx}{\linewidth}[b]{@{}>{\raggedright\arraybackslash}XX@{}}
    \toprule
    Кредит & Лизинг \\
    \midrule
    Инвестиции направляются на любую предпринимательскую деятельность & Инвестиции направляются на производственную деятельность \\
    Контроль за целевым расходованием средств затруднен из-за отсутствия действенных инструментов & Гарантирован контроль за целевым использованием средств \\
    Необходима 100\%-ная гарантия возврата кредита и процентов за его использование & Лизинговое имущество (оборудование, машины, суда и др.) само является гарантией \\
    \bottomrule
    \end{tabularx}%
  \label{tab:addlabel}%
\end{table}%
\end{frame}

\begin{frame}[shrink=25]{Сравнение лизинга, кредита и аренды}
% Table generated by Excel2LaTeX from sheet 'Сравнение'
\begin{table}[htbp]
  \centering
  \caption{Сравнение лизнга и кредита}
	\begin{tabularx}{\linewidth}[b]{@{}>{\raggedright\arraybackslash}XX@{}}
    \toprule
    Кредит & Лизинг \\
    \midrule
    Приобретенное имущество отражается на балансе предприятия, на него начисляется амортизация & Имущество отражается на балансе лизингодателя или предприятия-лизингополучателя; начисляется ускоренная амортизация (с коэффициентом 3) \\
    Плата за кредит покрывается за счет полученных предприятием доходов, на которые начисляются все предусмотренные налоги & Лизинговые платежи (включаются в себестоимость продукции) снижают налог на прибыль \\
    \bottomrule
    \end{tabularx}%
  \label{tab:addlabel}%
\end{table}%
\end{frame}

\begin{frame}[shrink=25]{Лизинг. Аренда}
% Table generated by Excel2LaTeX from sheet 'Отличия'
\begin{table}[htbp]
  \centering
  \caption{Отличия лизинга и аренды}
	\begin{tabularx}{\linewidth}[b]{@{}>{\raggedright\arraybackslash}XXX@{}}
    \toprule
    Признак & Аренда & Лизинг \\
    \midrule
    Субъекты отношений & Арендодатель и арендатор & Поставщик (изготовитель имущества), лизингодатель и лизингополучатель \\
    Объекты отношений & Любое имущество, разрешенное в обороте, включая природные объекты & Имущество, используемое для предпринимательской деятельности, исключая природные объекты \\
    Правовое отношение сторон & Имущественное двустороннее правоотношение & Коммерческое имущественное трехстороннее правоотношение \\
    \bottomrule
    \end{tabularx}%
  \label{tab:addlabel}%
\end{table}%
\end{frame}

\begin{frame}[shrink=25]{Лизинг. Аренда}
% Table generated by Excel2LaTeX from sheet 'Отличия'
\begin{table}[htbp]
  \centering
  \caption{Отличия лизинга и аренды}
	\begin{tabularx}{\linewidth}[b]{@{}>{\raggedright\arraybackslash}XXX@{}}
    \toprule
    Признак & Аренда & Лизинг \\
    \midrule
    Отношения с продавцом имущества & Арендодатель (продавец), арендатор (покупатель) & Лизингодатель и лизингополучатель выступают солидарными покупателями имущества у продавца \\
    Ответственность наймодателя за качество имущества & За качество имущества отвечает - арендодатель & Лизингодатель не отвечает за качество имущества, кроме случаев, когда он сам выбирает продавца \\
    Обязанности наймодателя & Как у собственника имущества & Инвестирование лизинговой сделки \\
    \bottomrule
    \end{tabularx}%
  \label{tab:addlabel}%
\end{table}%
\end{frame}


\begin{frame}[shrink=25]{Лизинг. Аренда}
% Table generated by Excel2LaTeX from sheet 'Отличия'
\begin{table}[htbp]
  \centering
  \caption{Отличия лизинга и аренды}
	\begin{tabularx}{\linewidth}[b]{@{}>{\raggedright\arraybackslash}XXX@{}}
    \toprule
    Признак & Аренда & Лизинг \\
    \midrule
    Уведомление продавца о цели приобретения имущества & Не производится & Лизингодатель указывает цель передачи имущества в лизинг конкретному лизингополучателю \\
    Право собственности на имущество после возмещения его стоимости & Если предусматривается,  то в форме купли-продажи & Обычно предполагается опцион \\
    Риск случайной гибели имущества & Несет арендодатель & Несет лизингополучатель \\
    \bottomrule
    \end{tabularx}%
  \label{tab:addlabel}%
\end{table}%
\end{frame}


\begin{frame}[shrink=25]{Лизинг. Аренда}
% Table generated by Excel2LaTeX from sheet 'Отличия'
\begin{table}[htbp]
  \centering
  \caption{Отличия лизинга и аренды}
	\begin{tabularx}{\linewidth}[b]{@{}>{\raggedright\arraybackslash}XXX@{}}
    \toprule
    Признак & Аренда & Лизинг \\
    \midrule
    Страхование имущества & Имущество страхует арендодатель & Имущество страхует лизингополучатель \\
    Расторжение договора по вине пользователя & Прекращение арендных платежей, кроме выплаты неустойки & Не освобождает лизингополучателя от полного погашения долга за весь договорной период \\
    Спрос и предложения на имущество & Учитывает расчет платежей за пользование имуществом & Цену имущества, процентную ставку, срок договора, его остаточную стоимость и др. \\
    \bottomrule
    \end{tabularx}%
  \label{tab:addlabel}%
\end{table}%
\end{frame}


\subsection{Виды лизинга}
\begin{frame}[shrink=15]{Виды лизинга}
\begin{figure}
\center
\begin{overprint}
	\forloop{slideno}{1}{\value{slideno} < 6}{%
		\only<\value{slideno}>{
			\includegraphics[page=\value{slideno},
			scale=1.4
			% trim={<left> <lower> <right> <upper>}				
			%,trim={.5cm 6.5cm 4cm 0cm}
			,clip]
			{tikz/transit_leasing}}}
\end{overprint}
\vspace*{-2em}
\caption{Транзитный международный лизинг}
\end{figure}
\only<2>{
1) лизингодатель и лизингополучатель, находящиеся в разных странах, заключает между собой договор лизинга;}

\only<3>{
2) лизингодатель в стране А берет кредит и приобретает имущество в стране С;}

\only<4>{
3) производитель имущества поставляет его лизингополучателю, находящемуся в стране В;}

\only<5>{
4) лизингополучатель из страны В производит лизинговые платежи лизингодателю в страну А.}

\end{frame}

\begin{frame}{Транзитный международный лизинг}{Преимущества для лизингодателя}
Доступ к финансированию в стране лизингополучателя;

- снижение валютного риска;

- расширение номенклатуры сдаваемого в лизинг имущества;

- снижение барьеров на перевод лизинговых платежей за границу;

- снятие ограничений на деятельность иностранных партнеров-лизингодателей;

- упрощение регистрации имущества на имя иностранных владельцев.
\end{frame}

\begin{frame}[shrink=15]{Виды лизинга}
\begin{figure}
\center
\begin{overprint}
	\forloop{slideno}{1}{\value{slideno} < 6}{%
		\only<\value{slideno}>{
			\includegraphics[page=\value{slideno},
			scale=.9
			% trim={<left> <lower> <right> <upper>}				
			%,trim={.5cm 6.5cm 4cm 0cm}
			,clip]
			{tikz/direct_leasing}}}
\end{overprint}
\vspace*{-2em}
\caption{Прямой лизинг}
\end{figure}

\only<2>{
1) лизингодатель (производитель имущества) создает в своей структуре специальное подразделение, входящее в службу маркетинга;
}

\only<3>{
2) лизингодатель заключает прямой договор с лизингополучателем;
}

\only<4>{
3) лизингодатель поставляет имущество лизингополучателю;
}

\only<5>{
4) лизингополучатель производит лизинговые платежи лизингодателю.}

\end{frame}

\begin{frame}{Виды лизинга}
\begin{figure}
\center
\begin{overprint}
	\forloop{slideno}{1}{\value{slideno} < 6}{%
		\only<\value{slideno}>{
			\includegraphics[page=\value{slideno},
			scale=.9
			% trim={<left> <lower> <right> <upper>}				
			%,trim={.5cm 6.5cm 4cm 0cm}
			,clip]
			{tikz/lease_back}}}
\end{overprint}
\vspace*{-2em}
\caption{Возвратный лизинг}
\end{figure}
\only<2>{
1) лизингополучатель (производитель) продает лизингодателю свое оборудование или предприятие в целом;}
\only<3>{
2) и одновременно берет его в лизинг, сохраняя при этом право владения и пользования им;}
\only<4>{
3) деньги, полученные за проданное имущество, лизингополучатель может использовать для производственных или инвестиционных целей;}
\only<5>{
4) лизингополучатель осуществляют лизинговые платежи в пользу лизингодателя по договору лизинга.}
\end{frame}

\begin{frame}{Виды лизинга}
\begin{figure}
\center
\begin{overprint}
	\forloop{slideno}{1}{\value{slideno} < 6}{%
		\only<\value{slideno}>{
			\includegraphics[page=\value{slideno},
			scale=1.3
			% trim={<left> <lower> <right> <upper>}				
			%,trim={.5cm 6.5cm 4cm 0cm}
			,clip]
			{tikz/indirect_leasing}}}
\end{overprint}
\vspace*{-2em}
\caption{Косвенный лизинг}
\end{figure}
\only<2>{
1) лизингодатель заключает договор лизинга с лизингополучателем;
}
\only<3>{
2) посредник (лизингодатель) покупает имущество у производителя;
}
\only<4>{
3) лизингодатель поставляет имущество лизингополучателю;
}
\only<5>{
4) лизингополучатель осуществляют лизинговые платежи в пользу лизингодателя по договору лизинга.
}
\end{frame}


\begin{frame}[shrink=20]{Виды лизинга}
\begin{figure}
\center
\begin{overprint}
	\forloop{slideno}{1}{\value{slideno} < 9}{%
		\only<\value{slideno}>{
			\includegraphics[page=\value{slideno},
			scale=1.2
			% trim={<left> <lower> <right> <upper>}				
			%,trim={.5cm 6.5cm 4cm 0cm}
			,clip]
			{tikz/full_leasing}}}
\end{overprint}
\vspace*{-2em}
\caption{Полный лизинг}
\end{figure}
\only<2>{
1) лизингодатель заключает договор лизинга с лизингополучателем;
}

\only<3>{
2) лизингодатель покупает имущество у производителя;
}

\only<4>{
3) лизингодатель поставляет имущество лизингополучателю;
}
\only<5>{
4) лизингодатель выполняет техническое обслуживание, ремонт имущества, подготовку персонала, и маркетинговую деятельность;
}
\only<6>{
5) лизингодатель осуществляет периодические платежи по налогу на имущество;
}
\only<7>{
6) лизингодатель страхует имущество, переданное в лизинг, и делает периодические страховые взносы;
}
\only<8>{
7) лизингополучатель осуществляют лизинговые платежи в пользу лизингодателя по договору лизинга.
}
\end{frame}


\subsection{Лизинговые платежи}
\subsubsection{Минэкономразвития}
\begin{frame}[allowframebreaks]{Расчет лизинговых платежей}
Расчет общей суммы лизинговых выплат может выполняться по формуле (согласно Методическим рекомендациям):

\begin{align}
\text{ЛП} = \text{АО} + \text{ПК} + \text{KB}+ \text{ДУ} + \text{НДС},
\end{align}
где  АО — величина амортизационных отчислений, причитающихся лизингодателю в текущем году; 

\pagebreak
ПК — плата за использованные лизингодателем кредитные ресурсы; 

KB — комиссионное вознаграждение лизингодателю; 

ДУ — плата лизингодателю за дополнительные услуги; 

НДС — налог на добавленную стоимость, уплачиваемый лизингополучателем по услугам лизингодателя (в том случае, если лизингодатель не освобожден от уплаты НДС).
\end{frame}
\begin{frame}
Плата за используемые арендодателем кредитные ресурсы на приобретение имущества (ПК) рассчитывается по формуле:
\begin{align}
\text{ПК}=\text{КР}\times \frac{\text{СТ}_\text{К}}{100\%}
\end{align}
где $\text{СТ}_\text{К}$ — ставка процента за кредит;

КР — величина кредитных ресурсов, используемых лизингодателем для приобретения объекта лизинга.
\end{frame}

\begin{frame}[allowframebreaks]
Расчет комиссионного вознаграждения (KB):

1-ый способ: от балансовой стоимости имущества:
\begin{align}
\text{KB} = p * \text{БС};
\end{align}

\pagebreak
2-ой способ: от среднегодовой остаточной стоимости:
\begin{align}
\text{KB}=\frac{\text{ОС}_\text{н}+\text{ОС}_\text{к}}{2} \times p
\end{align}
где	p — ставка комиссионного вознаграждения (в долях);

БС — балансовая стоимость имущества у арендатора (лизингополучателя);

$\text{ОС}_\text{н}$ и $\text{ОС}_\text{к}$ — остаточная стоимость арендованного имущества соответственно на начало и на конец отчетного периода.

\end{frame}

\begin{frame}[ allowframebreaks]{Расчет лизинговых платежей}{согласно Методическим рекомендациям}

Организацией (лизингополучателем) в январе 2017 г. заключен договор лизинга, содержащий ряд следующих условий:

1. Стоимость имущества, приобретаемого лизингодателем, составляет 1180 тыс. руб.\\ (в том числе НДС — 180 тыс. руб.).

\pagebreak
2. Нормативный срок службы имущества — 15 лет; в условиях лизинга используется коэффициент ускоренной амортизации 3, поэтому срок списания объекта уменьшается до пяти лет.

3. Срок действия договора совпадает со сроком амортизации имущества.

\pagebreak
4. Для приобретения имущества лизингодатель использует кредит, взятый на срок договора лизинга. Процентная ставка составляет 10\%. Для упрощения расчетов возврат кредита осуществляется равными частями в течение пяти лет в конце года.

\pagebreak
5. Договором лизинга оговорен порядок расчета комиссионного вознаграждения в размере 1,5\% от балансовой стоимости имущества на начало года.

6. Для упрощения рассчитываем налог на имущество как произведение его ставки (2\%) на балансовую стоимость имущества на начало года.

\end{frame}

\begin{frame}{Решение}
\begin{table}[htbp]
  \centering
  \caption{Зададим начальные условия}
	\begin{tabularx}{\linewidth}[b]{@{}>{\raggedright\arraybackslash}Xr@{}}
    \toprule
    Стоимость имущества & 1180 \\
    \midrule
    Срок службы имущества & 15 \\
    Ускоренный коэф. аморт. & 3 \\
    Норма аморт. & 20,00\% \\
    Срок кредита (лизинга), лет & 5 \\
    \bottomrule
    \end{tabularx}%
  \label{tab:addlabel}%
\end{table}%
\end{frame}

\begin{frame}{Налог на имущество и комиссионное вознаграждение}
% Table generated by Excel2LaTeX from sheet 'Лист1'
\begin{table}[htbp]
	\centering
	\scriptsize
	\caption{Расчет, тыс. руб.}
	\begin{tabularx}{\linewidth}[b]{@{}>{\raggedright\arraybackslash}Xrrrr@{}}
		\toprule
		 Год  &  Снг & Ам & НалИм & Ком \\ \midrule
		2017  & 1000 &    &       &     \\
		2018  &      &    &       &     \\
		2019  &      &    &       &     \\
		2020  &      &    &       &     \\
		2021  &      &    &       &     \\ \midrule
		Итого &      &    &       &     \\ \bottomrule
	\end{tabularx}%
	\label{tab:addlabel}%
\end{table}%
\end{frame}

\begin{frame}
Снг - стоимость на начало года; Ам - амортизация; НалИм - сумма налога на имущество; Ком - сумма комиссионного вознаграждения.
\end{frame}

\iftagged{professor}
{
	\begin{frame}{Налог на имущество и комиссионное вознаграждение}
	% Table generated by Excel2LaTeX from sheet 'Лист1'
	\begin{table}[htbp]
		\centering
		\scriptsize
		\caption{Расчет, тыс. руб.}
	\begin{tabularx}{\linewidth}[b]{@{}>{\raggedright\arraybackslash}Xrrrr@{}}
			\toprule
			Год  &                Снг &                  Ам &             НалИм &               Ком \\ \midrule
			2017  & \onslide<2->{1000} &   \onslide<7->{200} & \onslide<13->{20} & \onslide<19->{15} \\
			2018  & \onslide<3->{ 800} &   \onslide<8->{200} & \onslide<14->{16} & \onslide<20->{12} \\
			2019  & \onslide<4->{ 600} &   \onslide<9->{200} & \onslide<15->{12} & \onslide<21->{ 9} \\
			2020  & \onslide<5->{ 400} &  \onslide<10->{200} & \onslide<16->{ 8} & \onslide<22->{ 6} \\
			2021  & \onslide<6->{ 200} &  \onslide<11->{200} & \onslide<17->{ 4} & \onslide<23->{ 3} \\ \midrule
			Итого &                    & \onslide<12->{1000} & \onslide<18->{60} & \onslide<24->{45} \\ \bottomrule
		\end{tabularx}%
		\label{tab:addlabel}%
	\end{table}%
\end{frame}
}

\begin{frame}{Расчет процентов по кредиту}
% Table generated by Excel2LaTeX from sheet 'Лист2'
\begin{table}[htbp]
	\centering
	\scriptsize
	\caption{Проценты по кредиту, тыс. руб.}
	\begin{tabularx}{\linewidth}[b]{@{}>{\raggedright\arraybackslash}Xrrr@{}}
		\toprule
		Год  & КрПл & Пр &  Днг \\ \midrule
		2014  &      &    & 1180 \\
		2015  &      &    &      \\
		2016  &      &    &      \\
		2017  &      &    &      \\
		2018  &      &    &      \\ \midrule
		Итого & 1180 &    &      \\ \bottomrule
	\end{tabularx}%
	\label{tab:addlabel}%
\end{table}%
\end{frame}

\begin{frame}
КрПл - ежегодный платеж по кредиту; Пр - сумма процентов по кредиту за год; Днг - долг на начало года.
\end{frame}

\iftagged{professor}{
\begin{frame}{Расчет процентов по кредиту}
% Table generated by Excel2LaTeX from sheet 'Лист2'
\begin{table}[htbp]
	\centering
	\scriptsize
	\caption{Проценты по кредиту, тыс. руб.}
	\begin{tabularx}{\linewidth}[b]{@{}>{\raggedright\arraybackslash}Xrrr@{}}
		\toprule
		 Год  &              КрПл &                  Пр &              Днг \\ \midrule
		2014  & \onslide<2->{236} &  \onslide<7->{ 118} &             1180 \\
		2015  & \onslide<3->{236} &  \onslide<8->{94,4} & \onslide<13->{944} \\
		2016  & \onslide<4->{236} &  \onslide<9->{70,8} & \onslide<14->{708} \\
		2017  & \onslide<5->{236} & \onslide<10->{47,2} & \onslide<15->{472} \\
		2018  & \onslide<6->{236} & \onslide<11->{23,6} & \onslide<16->{236} \\ \midrule
		Итого &              1180 & \onslide<12->{ 354} & \onslide<17->{  0} \\ \bottomrule
	\end{tabularx}%
	\label{tab:addlabel}%
\end{table}%
\end{frame}
}

\begin{frame}{Расчет лизинговых платежей}
% Table generated by Excel2LaTeX from sheet 'Лист3'
\begin{table}[htbp]
  \centering
  \scriptsize
  \caption{Лизинговые платежи, тыс. руб.}
	\begin{tabularx}{\linewidth}[b]{@{}>{\raggedright\arraybackslash}Xrrrrrrr@{}}
    	\toprule
    	Год   & ПрКр & Ам & НалИм & Ком & $\sum$ & НДС & Пл \\ \midrule
    	1     &    2 &  3 &     4 &   5 &      6 &   7 &  8 \\ \midrule
    	2014  &      &    &       &     &        &     &    \\
    	2015  &      &    &       &     &        &     &    \\
    	2016  &      &    &       &     &        &     &    \\
    	2017  &      &    &       &     &        &     &    \\
    	2018  &      &    &       &     &        &     &    \\ \midrule
    	Итого &      &    &       &     &        &     &    \\ \bottomrule
    \end{tabularx}%
  \label{tab:addlabel}%
\end{table}%
\end{frame}

\begin{frame}
Пр - сумма процентов по кредиту за год;	Ам - амортизация; НалИм - сумма налога на имущество; Ком - сумма комиссионного вознаграждения; НДС - налог на добавленную стоимость; Пл - лизинговый платеж.
\end{frame}

\iftagged{professor}{
\begin{frame}{Расчет лизинговых платежей}
% Table generated by Excel2LaTeX from sheet 'Лист3'
\begin{table}[htbp]
	\centering
	\scriptsize
	\caption{Лизинговые платежи, тыс. руб.}
	\begin{tabularx}{\linewidth}[b]{@{}>{\raggedright\arraybackslash}Xrrrrrrr@{}}
		\toprule
		Год   &             ПрКр &                Ам &           НалИм &             Ком &          $\sum$ &                 НДС &                   Пл \\ \midrule
		1     &                2 &                 3 &               4 &               5 &                  6 &                   7 &                    8 \\ \midrule
		2014  & \onslide<2->{ 118} &  \onslide<8->{ 200} & \onslide<14->{20} & \onslide<20->{15} & \onslide<26->{  353} & \onslide<32->{ 63,54} & \onslide<38->{ 416,54} \\
		2015  & \onslide<3->{94,4} &  \onslide<9->{ 200} & \onslide<15->{16} & \onslide<21->{12} & \onslide<27->{322,4} & \onslide<33->{58,032} & \onslide<39->{380,432} \\
		2016  & \onslide<4->{70,8} & \onslide<10->{ 200} & \onslide<16->{12} & \onslide<22->{ 9} & \onslide<28->{291,8} & \onslide<34->{52,524} & \onslide<40->{344,324} \\
		2017  & \onslide<5->{47,2} & \onslide<11->{ 200} & \onslide<17->{ 8} & \onslide<23->{ 6} & \onslide<29->{261,2} & \onslide<35->{47,016} & \onslide<41->{308,216} \\
		2018  & \onslide<6->{23,6} & \onslide<12->{ 200} & \onslide<18->{ 4} & \onslide<24->{ 3} & \onslide<30->{230,6} & \onslide<36->{41,508} & \onslide<42->{272,108} \\ \midrule
		Итого & \onslide<7->{ 354} & \onslide<13->{1000} & \onslide<19->{60} & \onslide<25->{45} & \onslide<31->{ 1459} & \onslide<37->{262,62} & \onslide<43->{1721,62} \\ \bottomrule
	\end{tabularx}%
	\label{tab:addlabel}%
\end{table}%
\end{frame}
}

\subsubsection{Аннуитеты}
\begin{frame}[shrink=10]{Использование аннуитетов при расчете лизинговых платежей}
Для потока постнумерандо (поступления в конце периодов):
\begin{align}
\label{eq:postnumerando}
PV_{pst}=R \times \sum_{k=1}^n \frac{1}{(1+r)^k}=R \times \frac{1-(1+r)^{-n}}{r}
\end{align}
Для потока пренумерандо (поступления в начале периодов):
\begin{align}
PV_{pre}&= R \times (1+r) \times \sum_{k=1}^n \frac{1}{(1+r)^k} \nonumber \\
&= R \times (1+r) \times \frac{1-(1+r)^{-n}}{r}
\end{align}
\end{frame}

\begin{frame}[ allowframebreaks] {Расчет лизинговых платежей методом аннуитетов}
Организацией (лизингополучателем) 1 января 2017 г. заключен договор лизинга, содержащий ряд следующих условий:

1.	Первоначальная стоимость имущества у лизингодателя на момент заключения договора — 100 тыс. руб.

2.	Нормативный срок службы имущества — 5 лет, т. е. амортизация начисляется лизингополучателем линейным способом по годовой норме 20\%.

\pagebreak
3.	Срок действия договора — 4 года; по истечении данного времени лизингополучатель выкупит имущество по остаточной стоимости.

\pagebreak
4.	Лизингодатель использует собственные средства при приобретении объекта лизинга (плата за кредитные ресурсы лизингодателю отсутствует).

5.	Ежегодно (31 декабря) лизингополучатель уплачивает лизингодателю 2 тыс. руб. — плата за техническое обслуживание имущества в течение года.

\pagebreak
6.	Договором лизинга оговорен порядок расчета комиссионного вознаграждения в размере 20\% от среднегодовой остаточной стоимости имущества.

\end{frame}

\begin{frame}{Решение}
Расчет ежегодного лизингового платежа осуществляется по формуле для аннуитетов постнумерандо (\ref{eq:postnumerando}), ставка дисконтирования 20\%, стоимость имущества 100 тыс. руб.
\end{frame}

\begin{frame}[shrink=20]
% Table generated by Excel2LaTeX from sheet 'Аннуитеты'
\begin{table}[htbp]
  \centering
  \caption{Зададим начальные условия}
	\begin{tabularx}{\linewidth}[b]{@{}>{\raggedright\arraybackslash}Xr@{}}
    \toprule
    Показатель & Значение \\
    \midrule
    Стоимость имущества, руб. & 100000 \\
    Срок службы имущества & 5 \\
    Срок действия договора & 4 \\
    Плата за техническое обслуживание, руб. & 2000 \\
    Комиссия лизингодателя \linebreak от остаточной стоимости имущества & 0,2 \\
    Выкупная цена имущества, руб. & 20000 \\
    Современная стоимость выкупной цены, руб. & ?? \\
    Дисконтированная стоимость ежегодных лизинговых платежей & 90355 \\
    Ежегодный лизинговый платеж  \linebreak (по формуле аннуитета) & ??\\
    \bottomrule
    \end{tabularx}%
  \label{tab:addlabel}%
\end{table}%

\end{frame}

\begin{frame}{Порядок распределения лизинговых платежей}
% Table generated by Excel2LaTeX from sheet 'Аннуитеты'
\begin{table}[htbp]
  \centering
  \scriptsize
  \caption{Сторона лизингополучателя}
    \begin{tabular}{crrrrr}
    \toprule
    \multicolumn{1}{c}{\multirow{2}[0]{*}{Дата платежа}} & \multicolumn{4}{c}{Платежи}   & \multicolumn{1}{c}{\multirow{2}[0]{*}{ОстД}} \\\cmidrule{2-5}
    \multicolumn{1}{c}{} & \multicolumn{1}{c}{всего} & \multicolumn{1}{c}{ТекРас} & \multicolumn{1}{c}{Ком} & \multicolumn{1}{c}{ПогОб} & \multicolumn{1}{c}{} \\
    \midrule
          &       &       &       &       & 100000 \\
    \midrule
    2017  &       &       &       &       &  \\
    2018  &       &       &       &       &  \\
    2019  &       &       &       &       &  \\
    2020  &       &       &       &       &  \\
    2021 (выкуп) &       &       &       &       &  \\
    \midrule
    Итого &       &       &       &       &  \\
    \bottomrule
    \end{tabular}%
  \label{tab:addlabel}%
\end{table}%
\end{frame}

\begin{frame}
ТекРас - текущие расходы; Ком - комиссионные расходы; 

ПогОб	погашение обязательств; ОстД - остаток долга.

\end{frame}

\iftagged{professor}{
\begin{frame}{Порядок распределения лизинговых платежей}
% Table generated by Excel2LaTeX from sheet 'Аннуитеты'
\begin{table}[htbp]
	\centering
	\scriptsize
	\caption{Сторона лизингополучателя}
	\begin{tabular}{crrrrr}
		\toprule
\multicolumn{1}{c}{\multirow{2}[0]{*}{Дата платежа}} &                                         \multicolumn{4}{c}{Платежи}                                          & \multicolumn{1}{c}{\multirow{2}[0]{*}{ОстД}} \\
\cmidrule{2-5}
\multicolumn{1}{c}{}         & \multicolumn{1}{c}{всего} & \multicolumn{1}{c}{ТекРас} & \multicolumn{1}{c}{Ком} & \multicolumn{1}{c}{ПогОб} &                         \multicolumn{1}{c}{} \\ \midrule
		             &         &    &       &       &  1000 \\ \midrule
		    2017     & 369 & 20 &   200 & 149,0 & 850,9 \\
		    2018     & 369 & 20 & 170,1 & 178,8 & 672,1 \\
		    2019     & 369 & 20 & 134,4 & 214,6 & 457,5 \\
		    2020     & 369 & 20 &  91,5 & 257,5 &   200 \\
		2021 (выкуп) &     200 &    &       &   200 &     0 \\ \midrule
		   Итого     & 1676 & 80 & 596,1 &  1000 &       \\ \bottomrule
	\end{tabular}%
	\label{tab:addlabel}%
\end{table}%
\end{frame}
}

\begin{frame}{Распределение лизинговых платежей для целей бухгалтерского учета}
% Table generated by Excel2LaTeX from sheet 'Аннуитеты'
\begin{table}[htbp]
  \centering
  \scriptsize
  \caption{Распределение лизинговых платежей}
	\begin{tabularx}{\linewidth}[b]{@{}>{\raggedright\arraybackslash}Xrrrrrrr@{}}
		\toprule
		Год   & ПогСт & НДС & КомТек & НДС & ЛП & НДС & Всего \\ \midrule
		2017  &       &     &        &     &    &     &       \\
		2018  &       &     &        &     &    &     &       \\
		2019  &       &     &        &     &    &     &       \\
		2020  &       &     &        &     &    &     &       \\ \midrule
		Выкуп &       &     &        &     &    &     &       \\ \midrule
		Итого &       &     &        &     &    &     &       \\ \bottomrule
	\end{tabularx}%
  \label{tab:addlabel}%
\end{table}%
\end{frame}

\begin{frame}
ПогСт - сумма в возмещение стоимости; КомТек - комиссия и текущие расходы; ЛП - лизинговый платеж.
\end{frame}

\iftagged{professor}{
\begin{frame}{Распределение лизинговых платежей для целей бухгалтерского учета}
% Table generated by Excel2LaTeX from sheet 'Аннуитеты'
\begin{table}[htbp]
	\centering
	\scriptsize
	\caption{Распределение лизинговых платежей}
	\begin{tabularx}{\linewidth}[b]{@{}>{\raggedright\arraybackslash}Xrrrrrrr@{}}
		\toprule
		Год   & ПогСт  & НДС   & КомТек & НДС   & ЛП     & НДС   & Всего  \\ \midrule
		2017  & 149,0  & 26,8  & 220,0  & 39,6  & 369,0  & 66,4  & 435,5  \\
		2018  & 178,8  & 32,2  & 190,2  & 34,2  & 369,0  & 66,4  & 435,5  \\
		2019  & 214,6  & 38,6  & 154,4  & 27,8  & 369,0  & 66,4  & 435,5  \\
		2020  & 257,5  & 46,4  & 111,5  & 20,1  & 369,0  & 66,4  & 435,5  \\ \midrule
		Выкуп & 200,0  & 36,0  & 0,0    &       & 200,0  & 36,0  & 236,0  \\ \midrule
		Итого & 1000,0 & 180,0 & 676,1  & 121,7 & 1676,1 & 301,7 & 1977,8 \\ \bottomrule
	\end{tabularx}%
	\label{tab:addlabel}%
\end{table}%
\end{frame}
}

\subsection{Эффективность лизинга}
\begin{frame}{Анализ эффективности лизинговых операций}{Преимущества лизинга}
\begin{itemize}

\item
100\% кредитование сделки;

\item
гибкий график лизинговых платежей;

\item
техническая поддержка оборудования лизингодателем;

\item
объект лизинга не на балансе лизингополучателя, что избавляет его от налога на имущество;
\end{itemize}
\end{frame}
\begin{frame}{Анализ эффективности лизинговых операций}{Преимущества лизинга}
\begin{itemize}
\item
улучшаются финансовые показатели лизингополучателя
\item
лизингополучателю по окончании договора в собственность достается имущество с нулевой остаточной стоимостью;
\item
амортизационные и налоговые льготы;
\item
дополнительные возможности сбыта для производителей имущества.
\end{itemize}
\end{frame}
\begin{frame}{Сравнение лизинга, кредита и простой покупки имущества}
% Table generated by Excel2LaTeX from sheet 'Лист1'
\begin{table}[htbp]
  \centering
  \caption{Затраты}
	\begin{tabularx}{\linewidth}[b]{@{}>{\raggedright\arraybackslash}XXX@{}}
    \toprule
    Лизинг & Кредит & Покупка \\
    \midrule
    Лизинговые платежи, включая НДС & Расходы на погашение кредита, в том числе проценты по кредиту & Расходы на закупку оборудования, включая НДС \\
          & Налог на имущество & Налог на имущество \\
    \bottomrule
    \end{tabularx}%
  \label{tab:addlabel}%
\end{table}%
\end{frame}

\begin{frame}[shrink=15]{Сравнение лизинга, кредита и простой покупки имущества}
% Table generated by Excel2LaTeX from sheet 'Лист1'
\begin{table}[htbp]
  \centering
  \caption{Налоговые вычеты}
	\begin{tabularx}{\linewidth}[b]{@{}>{\raggedright\arraybackslash}XXX@{}}
    \toprule
    Лизинг & Кредит & Покупка \\
    \midrule
Возврат НДС по лизин­говым платежам & Возврат НДС по приобретенному имуществу & Возврат НДС по приобретенному имуществу \\
    Налоговая экономия в результате списания лизинговых платежей на себестоимость & Налоговая экономия на амортизации, процентах по кредиту и налоге на имущество & Налоговая экономия на амортизации и налоге на имущество \\
    \bottomrule
    \end{tabularx}%
  \label{tab:addlabel}%
\end{table}%
\end{frame}

\begin{frame}[ allowframebreaks]{Определение эффективности лизинга}
Стоимость приобретаемого имущества составляет 1180 тыс. руб. (в том числе НДС в размере 180 тыс. руб.).

Нормативный срок службы имущества — 15 лет.

\pagebreak
В условиях лизинга используется коэффициент ускоренной амортизации (k = 3), поэтому срок списания объекта уменьшается до пяти лет. Срок действия договора лизинга совпадает со сроком амортизации имущества. Для приобретения имущества лизингодатель использует кредит, взятый на срок договора лизинга. 

\pagebreak

Процентная ставка составляет 10\% годовых. 

Для упрощения расчетов считается, что возврат кредита осуществляется равными частями в конце очередного года в течение пяти лет.

\pagebreak
Договором лизинга оговорен порядок расчета комиссионного вознаграждения в размере 1,5\% от балансовой стоимости имущества на начало года.

\pagebreak
Приобретатель имеет возможность использовать банковский кредит на приобретение имущества. Сроки, процентные ставки, а также иные условия кредитования в целях сопоставимости анализа совпадают с условиями кредита, взятого лизингодателем для покупки имущества и сдаче в лизинг.

\pagebreak
Ставка налога на прибыль 20\%. Налог на добавленную стоимость в полном объеме предъявляется бюджету.
\end{frame}

\begin{frame}{Приобретение имущества по лизингу}
% Table generated by Excel2LaTeX from sheet 'Срав1'
\begin{table}[htbp]
  \centering
  \scriptsize
  \caption{Расчет совокупных затрат}
	\begin{tabularx}{\linewidth}[b]{@{}>{\raggedright\arraybackslash}Xrrrrr@{}}
    	\toprule
    	 Год  & ЛП & НДС & ЧЛП & НЭ & ЧДП \\ \midrule
    	2017  &    &     &     &    &     \\
    	2018  &    &     &     &    &     \\
    	2019  &    &     &     &    &     \\
    	2020  &    &     &     &    &     \\
    	2021  &    &     &     &    &     \\ \midrule
    	Итого &    &     &     &    &     \\ \bottomrule
    \end{tabularx}%
  \label{tab:addlabel}%
\end{table}%
\end{frame}

\begin{frame}
ЛП - лизинговый платеж; НДС - налог на добавленную стоимость; ЧЛП - лизинговый платеж без НДС; НЭ - налоговая экономия; ЧДП - чистый денежный поток.

Сумму лизинговых платежей берем из примера по расчету лизинговых платежей согласно Методическим рекомендациям Минэкономразвития.
\end{frame}

\iftagged{professor}{
\begin{frame}{Приобретение имущества по лизингу}
\begin{table}[htbp]
	\centering
	\scriptsize
	\caption{Расчет совокупных затрат}
	\begin{tabularx}{\linewidth}[b]{@{}>{\raggedright\arraybackslash}Xrrrrr@{}}	
		\toprule
		 Год  &     ЛП &   НДС &    ЧЛП &    НЭ &    ЧДП \\ \midrule
		2017  &  416,5 &  75,0 &  341,6 &  68,3 &  273,3 \\
		2018  &  380,4 &  68,5 &  312,0 &  62,4 &  249,6 \\
		2019  &  344,3 &  62,0 &  282,3 &  56,5 &  225,9 \\
		2020  &  308,2 &  55,5 &  252,7 &  50,5 &  202,2 \\
		2021  &  272,1 &  49,0 &  223,1 &  44,6 &  178,5 \\ \midrule
		Итого & 1721,6 & 309,9 & 1411,7 & 282,3 & 1129,4 \\ \bottomrule
	\end{tabularx}%
	\label{tab:addlabel}%
\end{table}%
\end{frame}
}

\begin{frame}{Вариант покупки в кредит}
% Table generated by Excel2LaTeX from sheet 'Срав1'
\begin{table}[htbp]
  \centering
  \scriptsize
  \caption{Расчет налоговой экономии}
  	\begin{tabularx}{\linewidth}[b]{@{}>{\raggedright\arraybackslash}Xrrrrrrr@{}}	
    	\toprule
    	 Год  & Снг & Ам & ОД & НалИм & ПрКр & Себ & НЭ \\ \midrule
    	2017  &     &    &    &       &      &     &    \\
    	2018  &     &    &    &       &      &     &    \\
    	2019  &     &    &    &       &      &     &    \\
    	2020  &     &    &    &       &      &     &    \\
    	2021  &     &    &    &       &      &     &    \\ \midrule
    	Итого &     &    &    &       &      &     &    \\ \bottomrule
    \end{tabularx}%
  \label{tab:addlabel}%
\end{table}%
\end{frame}

\begin{frame}
Снг -Стоимость имущества на начало года; Ам - амортизация; ОД - остаток долга; НалИм - налог на имущество; ПрКр - проценты за кредит; Себ - себестоимость; НЭ - налоговая экономия.
\end{frame}

\iftagged{professor}{
\begin{frame}{Вариант покупки в кредит}
% Table generated by Excel2LaTeX from sheet 'Срав1'
\begin{table}[htbp]
	\centering
	\scriptsize
	\caption{Расчет налоговой экономии}
	\begin{tabularx}{\linewidth}[b]{@{}>{\raggedright\arraybackslash}Xrrrrrrr@{}}	
		\toprule
		 Год  &    Снг &    Ам &     ОД & НалИм &  ПрКр &   Себ &    НЭ \\ \midrule
		2017  & 1000,0 &  66,7 & 1180,0 &  20,0 & 118,0 & 204,7 &  40,9 \\
		2018  &  933,3 &  66,7 &  944,0 &  18,7 &  94,4 & 179,7 &  35,9 \\
		2019  &  866,7 &  66,7 &  708,0 &  17,3 &  70,8 & 154,8 &  31,0 \\
		2020  &  800,0 &  66,7 &  472,0 &  16,0 &  47,2 & 129,9 &  26,0 \\
		2021  &  733,3 &  66,7 &  236,0 &  14,7 &  23,6 & 104,9 &  21,0 \\ \midrule
		Итого &        & 333,3 &    0,0 &  86,7 & 354,0 & 774,0 & 154,8 \\ \bottomrule
	\end{tabularx}%
	\label{tab:addlabel}%
\end{table}%
\end{frame}	
}

\begin{frame}{Покупка имущества в кредит}
% Table generated by Excel2LaTeX from sheet 'Срав1'
\begin{table}[htbp]
  \centering
  \scriptsize
  \caption{Расчет совокупных затрат в случае покупки имущества с помощью кредита}
  	\begin{tabularx}{\linewidth}[b]{@{}>{\raggedright\arraybackslash}Xrrrrrr@{}}	
    	\toprule
    	 Год  & ПлКр & НДС & НалИм & ПрКр & НЭ & ЧДП \\ \midrule
    	2017  &      &     &       &      &    &     \\
    	2018  &      &     &       &      &    &     \\
    	2019  &      &     &       &      &    &     \\
    	2020  &      &     &       &      &    &     \\
    	2021  &      &     &       &      &    &     \\ \midrule
    	Итого &      &     &       &      &    &     \\ \bottomrule
    \end{tabularx}%
  \label{tab:addlabel}%
\end{table}%
\end{frame}

\begin{frame}
	ПлКр - платеж за кредит; НДС - налог на добавленную стоимость; НалИм - налог на имущество; ПрКр - проценты за кредит; НЭ - налоговая экономия; ЧДП - чистый денежный поток.
\end{frame}

\iftagged{professor}{
\begin{frame}{Покупка имущества в кредит}
% Table generated by Excel2LaTeX from sheet 'Срав1'
\begin{table}[htbp]
	\centering
	\scriptsize
	\caption{Расчет совокупных затрат}
	\begin{tabularx}{\linewidth}[b]{@{}>{\raggedright\arraybackslash}Xrrrrrr@{}}	
		\toprule
		 Год  &   ПлКр &   НДС & НалИм &  ПрКр &    НЭ &    ЧДП \\ \midrule
		2017  &  236,0 &  36,0 &  20,0 & 118,0 &  40,9 &  297,1 \\
		2018  &  236,0 &  36,0 &  18,7 &  94,4 &  35,9 &  277,1 \\
		2019  &  236,0 &  36,0 &  17,3 &  70,8 &  31,0 &  257,2 \\
		2020  &  236,0 &  36,0 &  16,0 &  47,2 &  26,0 &  237,2 \\
		2021  &  236,0 &  36,0 &  14,7 &  23,6 &  21,0 &  217,3 \\ \midrule
		Итого & 1180,0 & 180,0 &  86,7 & 354,0 & 154,8 & 1285,9 \\ \bottomrule
	\end{tabularx}%
	\label{tab:addlabel}%
\end{table}%
\end{frame}	
}

\begin{frame}{Вариант использования собственных средств для финансирования приобретения имущества}
% Table generated by Excel2LaTeX from sheet 'Срав1'
\begin{table}[htbp]
  \centering
  \scriptsize
  \caption{Расчет налоговой экономии}
	\begin{tabularx}{\linewidth}[b]{@{}>{\raggedright\arraybackslash}Xrrrr@{}}	
  		\toprule
    	 Год  & Ам & НалИм & Себ & НЭ \\ \midrule
    	2017  &    &       &     &    \\
    	2018  &    &       &     &    \\
    	2019  &    &       &     &    \\
    	2020  &    &       &     &    \\
    	2021  &    &       &     &    \\ \midrule
    	Итого &    &       &     &    \\ \bottomrule
    \end{tabularx}%
  \label{tab:addlabel}%
\end{table}%
\end{frame}
\begin{frame}
Ам - амортизация; НалИм - налог на имущество; Себ - себестоимость; НЭ - налоговая экономия.
\end{frame}

\iftagged{professor}{
\begin{frame}{Вариант использования собственных средств для финансирования приобретения имущества}
% Table generated by Excel2LaTeX from sheet 'Срав1'
\begin{table}[htbp]
	\centering
	\scriptsize
	\caption{Расчет налоговой экономии}
	\begin{tabularx}{\linewidth}[b]{@{}>{\raggedright\arraybackslash}Xrrrr@{}}	
		\toprule
		 Год  &    Ам & НалИм &   Себ &   НЭ \\ \midrule
		2017  &  66,7 &  20,0 &  86,7 & 17,3 \\
		2018  &  66,7 &  18,7 &  85,3 & 17,1 \\
		2019  &  66,7 &  17,3 &  84,0 & 16,8 \\
		2020  &  66,7 &  16,0 &  82,7 & 16,5 \\
		2021  &  66,7 &  14,7 &  81,3 & 16,3 \\ \midrule
		Итого & 333,3 &  86,7 & 420,0 & 84,0 \\ \bottomrule
	\end{tabularx}%
	\label{tab:addlabel}%
\end{table}%
\end{frame}	
}


\begin{frame}{Приобретение имущества за счет собственных средств}
% Table generated by Excel2LaTeX from sheet 'Срав1'
\begin{table}[htbp]
  \centering
  \scriptsize
  \caption{Расчет совокупных затрат}
	\begin{tabularx}{\linewidth}[b]{@{}>{\raggedright\arraybackslash}Xrrrrrr@{}}	
    	\toprule
    	 Год  & Расход & НДС & НалИм & РасхПок & НЭ & ЧДП \\ \midrule
    	2017  &        &     &       &         &    &     \\
    	2018  &        &     &       &         &    &     \\
    	2019  &        &     &       &         &    &     \\
    	2020  &        &     &       &         &    &     \\
    	2021  &        &     &       &         &    &     \\ \midrule
    	Итого &        &     &       &         &    &     \\ \bottomrule
    \end{tabularx}%
  \label{tab:addlabel}%
\end{table}%
\end{frame}

\begin{frame}
	НДС - налог на добавленную стоимость; НалИм - налог на имущество; РасхПок - потери при покупке; НЭ - налоговая экономия; ЧДП - чистый денежный поток.
	
\end{frame}

\iftagged{professor}{
\begin{frame}{Приобретение имущества за счет собственных средств}
% Table generated by Excel2LaTeX from sheet 'Срав1'
\begin{table}[htbp]
	\centering
	\scriptsize
	\caption{Расчет совокупных затрат}
	\begin{tabularx}{\linewidth}[b]{@{}>{\raggedright\arraybackslash}Xrrrrrr@{}}	
		\toprule
		 Год  & Расход &   НДС & НалИм & РасхПок &   НЭ &    ЧДП \\ \midrule
		2017  & 1180,0 &  36,0 &  20,0 &   295,0 & 17,3 & 1441,7 \\
		2018  &        &  36,0 &  18,7 &         & 17,1 &  -34,4 \\
		2019  &        &  36,0 &  17,3 &         & 16,8 &  -35,5 \\
		2020  &        &  36,0 &  16,0 &         & 16,5 &  -36,5 \\
		2021  &        &  36,0 &  14,7 &         & 16,3 &  -37,6 \\ \midrule
		Итого & 1180,0 & 180,0 &  86,7 &   295,0 & 84,0 & 1297,7 \\ \bottomrule
	\end{tabularx}%
	\label{tab:addlabel}%
\end{table}%
\end{frame}

\begin{frame}{Эффективность лизинга}
% Table generated by Excel2LaTeX from sheet 'Срав1'
\begin{table}[htbp]
	\centering
	\scriptsize
	\caption{Покупка за наличные, кредит, лизинг: сравнение эффективности}
	\begin{tabularx}{\linewidth}[b]{@{}>{\raggedright\arraybackslash}Xrrr}
		\toprule
		&        & \multicolumn{2}{c}{Разница в пользу лизинга}   \\
		\cmidrule{3-4}
		Затраты при покупке         &        & в руб. & в проц. \\ \midrule
		за счет собственных средств & 1297,7 & 168,3  & 14,90\% \\
		в кредит                    & 1285,9 & 156,5  & 13,86\% \\
		в лизинг                    & 1129,4 &        &         \\ \bottomrule
	\end{tabularx}%
	\label{tab:addlabel}%
\end{table}%
\end{frame}
}

\subsection{Контрольные вопросы}
\begin{frame}[ allowframebreaks ]{Контрольные вопросы}
1. История возникновения лизинга.

2. Понятие лизинга.

3. Понятие финансовой аренды.

4. Договор лизинга. Специфика лизингового бизнеса.

\pagebreak
5. Классическая схема лизинга.

6. Российская практика лизинга.

7. Сравнение лизинга, кредита и аренды.

8. Схема транзитного международного лизинга.

\pagebreak
9. Схема прямого лизинга.

10. Схема возвратного лизинга.

11. Схема косвенного лизинга.

12. Схема полного лизинга.

\pagebreak
13. Порядок расчета лизинговых платежей.

14. Использование аннуитетов при расчете лизинговых платежей.

15. Сравнение лизинга, кредита и простой покупки имущества по эффективности.
\end{frame}
\setbeamercovered{transparent}

\end{document}
