\documentclass[_Banking_p3.tex]{subfiles}
\begin{document}

\setbeamercovered{transparent}

\subsection{Нормативные документы}
\begin{frame}[allowframebreaks]{Нормативные документы}
  \begin{thebibliography}{10}
  
  \beamertemplatearticlebibitems
    \bibitem{bb2}
Федеральный закон «О банках и банковской деятельности» от 02.12.1990 № 395-1 (ред. от 03.07.2016).

    \bibitem{bb2}
Федеральный закон от 27.06.2011 N 161-ФЗ (ред. от 18.07.2017) "О национальной платежной системе"

\pagebreak

    \bibitem{bb2}
Федеральный закон от 10.07.2002 N 86-ФЗ (ред. от 18.07.2017) "О Центральном банке Российской Федерации (Банке России)"

    \bibitem{bb2}
"Гражданский кодекс Российской Федерации (часть вторая)" от 26.01.1996 N 14-ФЗ (ред. от 28.03.2017). Статья 849. Сроки операций по счету


\pagebreak

    \bibitem{bb7}
Инструкция Банка России от 30.05.2014 № 153-И "Об открытии и закрытии банковских счетов, счетов по вкладам (депозитам), депозитных счетов".

    \bibitem{bb3}
"Положение о правилах осуществления перевода денежных средств" (утв. Банком России 19.06.2012 N 383-П) (ред. от 06.11.2015).

\pagebreak

    \bibitem{bb8}
Положение Банка России от 24.12.2004 № 266-П "Об эмиссии платежных карт и об операциях, совершаемых с их использованием".


    \bibitem{bb4}
"Положение о порядке ведения кассовых операций и правилах хранения, перевозки и инкассации банкнот и монеты Банка России в кредитных организациях на территории Российской Федерации" (утв. Банком России 24.04.2008 № 318-П) (ред. от 16.02.2015).


\pagebreak

    \bibitem{bb2}
"Методические рекомендации о повышении внимания кредитных организаций к операциям клиентов - юридических лиц и индивидуальных предпринимателей, получающих наличные денежные средства с использованием корпоративных карт" (утв. Банком России 21.07.2017 N 19-МР)

    \bibitem{bb2}
"Положение об эмиссии платежных карт и об операциях, совершаемых с их использованием" (утв. Банком России 24.12.2004 N 266-П) (ред. от 14.01.2015) (Зарегистрировано в Минюсте России 25.03.2005 N 6431)

\pagebreak
    \bibitem{bb2}
"Уголовный кодекс Российской Федерации" от 13.06.1996 N 63-ФЗ (ред. от 29.07.2017). Статья 187. Неправомерный оборот средств платежей

    \bibitem{bb2}
Федеральный закон от 03.06.2009 N 103-ФЗ (ред. от 03.07.2016) "О деятельности по приему платежей физических лиц, осуществляемой платежными агентами"

\pagebreak

    \bibitem{bb2}
Федеральный закон от 22.05.2003 N 54-ФЗ (ред. от 03.07.2016) "О применении контрольно-кассовой техники при осуществлении наличных денежных расчетов и (или) расчетов с использованием электронных средств платежа"


    \bibitem{bb2}
"Основы законодательства Российской Федерации о нотариате" (утв. ВС РФ 11.02.1993 N 4462-1) (ред. от 03.07.2016) (с изм. и доп., вступ. в силу с 01.07.2017). Статья 96. Предъявление чека к платежу и удостоверение неоплаты чека


\pagebreak

    \bibitem{bb2}
Федеральный закон от 21.07.2014 N 213-ФЗ (ред. от 29.12.2015) "Об открытии банковских счетов и аккредитивов, о заключении договоров банковского вклада, договора на ведение реестра владельцев ценных бумаг хозяйственными обществами, имеющими стратегическое значение для оборонно-промышленного комплекса и безопасности Российской Федерации"

\pagebreak

    \bibitem{bb2}
Указание Банка России от 05.07.2007 N 1853-У (ред. от 15.06.2017) "Об особенностях осуществления кредитной организацией расчетных операций после отзыва лицензии на осуществление банковских операций и о счетах, используемых конкурсным управляющим (ликвидатором, ликвидационной комиссией)" (Зарегистрировано в Минюсте России 23.07.2007 N 9875)

    \bibitem{bb2}
Информационное письмо Президиума ВАС РФ от 15.01.1999 N 39 "Обзор практики рассмотрения споров, связанных с использованием аккредитивной и инкассовой форм расчетов"

\pagebreak

    \bibitem{bb2}
Постановление Правительства РФ от 06.05.2008 N 359 (ред. от 15.04.2014) "О порядке осуществления наличных денежных расчетов и (или) расчетов с использованием платежных карт без применения контрольно-кассовой техники"


    \bibitem{bb2}
Постановление Правительства РФ от 15.11.2010 N 920 (ред. от 30.06.2012) "Об утверждении перечня товаров (работ, услуг), в оплату которых платежный агент не вправе принимать платежи физических лиц"

  \end{thebibliography}
\end{frame}
\subsection{Контрольные вопросы}
\begin{frame}[ allowframebreaks ]{Контрольные вопросы}
1. Характеристика платежной системы России. Организации, выполняющие денежные переводы в России.

2. Схемы безналичных расчетов платежными поручениями и платежными требованиями-поручениями.

3. Схемы безналичных расчетов по инкассо и аккредитивная форма расчетов.

\pagebreak
4. Схемы безналичных расчетов простыми и переводными векселями.

5. Кассовое обслуживание клиентов  банками.

6. Межбанковские корреспондентские отношения. Расчеты через подразделения расчетной сети Банка России.

\pagebreak
7. Банковские операции с пластиковыми картами.

8. Дистанционное банковское обслуживание. Системы "Клиент-банк".

9. Банковское обслуживание электронной коммерции.
\end{frame}
\setbeamercovered{transparent}
\end{document}
