% !TeX program = lualatex -synctex=1 -interaction=nonstopmode --shell-escape %.tex

\documentclass[_Banking_p3.tex]{subfiles}
\begin{document}

\setbeamercovered{invisible}

\subsection{Нормативные документы}
\begin{frame}[allowframebreaks]{Нормативные документы}
  \begin{thebibliography}{10}
  
  \beamertemplatearticlebibitems

    \bibitem{bb4}
"Гражданский кодекс Российской Федерации (часть первая)" от 30.11.1994 № 51-ФЗ (ред. от 29.07.2017) (с изм. и доп., вступ. в силу с 06.08.2017). § 6. Независимая гарантия.

    \bibitem{bb4}
"Налоговый кодекс Российской Федерации (часть первая)" от 31.07.1998 № 146-ФЗ (ред. от 18.07.2017) Статья 74.1. Банковская гарантия.

\pagebreak

    \bibitem{bb4}
Федеральный закон от 05.04.2013 № 44-ФЗ (ред. от 29.07.2017) "О контрактной системе в сфере закупок товаров, работ, услуг для обеспечения государственных и муниципальных нужд". Статья 45. Условия банковской гарантии. Реестры банковских гарантий

    \bibitem{bb4}
Федеральный закон от 22.04.1996 № 39-ФЗ (ред. от 18.07.2017) "О рынке ценных бумаг". Статья 27.5. Облигации, обеспеченные банковской гарантией, государственной или муниципальной гарантией

\pagebreak

    \bibitem{bb4}
Об утверждении Порядка осуществления деятельности по управлению ценными бумагами: Приказ ФСФР РФ от 03.04.2007 № 07-37/пз-н

    \bibitem{bb4}
Федеральный закон от 27.11.2010 № 311-ФЗ (ред. от 18.06.2017) "О таможенном регулировании в Российской Федерации". Статья 141. Банковская гарантия

\pagebreak

    \bibitem{bb4}
Постановление Пленума Верховного Суда РФ № 13, Пленума ВАС РФ № 14 от 08.10.1998 (ред. от 24.03.2016) "О практике применения положений Гражданского кодекса Российской Федерации о процентах за пользование чужими денежными средствами". Пункт 19.

    \bibitem{bb4}
Постановление Пленума ВАС РФ от 23.07.2009 № 63 (ред. от 06.06.2014) "О текущих платежах по денежным обязательствам в деле о банкротстве". Пункт 7.

\pagebreak

    \bibitem{bb4}
Постановление Пленума ВАС РФ от 23.03.2012 № 14 "Об отдельных вопросах практики разрешения споров, связанных с оспариванием банковских гарантий"

  \end{thebibliography}
\end{frame}

\subsection{Залог}
\begin{frame} [ allowframebreaks]{Залог и залоговый механизм}
\begin{block}{Залог имущества }
\quad
- это форма обеспечения возвратности банковского кредита. Оформляется договором о залоге, подписанным двумя сторонами и подтверждающим право кредитора при неисполнении платежного обязательства заемщиком получить преимущественное удовлетворение претензии из стоимости заложенного имущества.
\end{block}
\pagebreak

\begin{block}{ Залоговый механизм }
\quad
- это процесс подготовки, заключения и исполнения договора о залоге.
\end{block}
- право собственности на заложенное имущество принадлежит заемщику;

- владение заемщиком заложенным имуществом может быть непосредственное и опосредованное;

- залог может сопровождаться правом пользования предметами залога в соответствии с его назначением.
\end{frame}

\begin{frame}[shrink=15]
\begin{figure}
\center
\begin{overprint}
	\forloop{slideno}{1}{\value{slideno} < 5}{%
		\only<\value{slideno}>{
			\includegraphics[page=\value{slideno},
			scale=.9
			% trim={<left> <lower> <right> <upper>}				
			%,trim={.5cm 6.5cm 4cm 0cm}
			,clip]
			{tikz/pledge_scheme}}}
\end{overprint}
\vspace*{-2em}
\caption{Схема залога}
\end{figure}
\only<2>{
1) заключение кредитного договора;
}
\only<3>{
2) заключение договора о залоге;
}
\only<4>{
3) передача (без передачи) залогового имущества.
}

\end{frame}

\begin{frame}{Предметы залога}{Залог имущества клиента}
\begin{itemize}

\item
залог товарно-материальных ценностей:
\begin{itemize}
\item
залог сырья, материалов, полуфабрикатов;

\item
залог товаров и готовой продукции;

\item
залог валютных ценностей (наличной валюты), золотых изделий, украшений, предметов искусства и антиквариата,

\item
залог прочих товарно-материальных ценностей;
\end{itemize}
\item
залог ценных бумаг, включая векселя;
\item
залог депозитов, находящихся в том же банке;
\item
ипотека (залог недвижимости). 

\end{itemize}
\end{frame}

\begin{frame}{Предметы залога}{Залог имущественных прав}
\begin{itemize}

\item
залог права арендатора;

\item
залог права автора на вознаграждение;

\item
залог права заказчика по договору подряда;

\item
залог права комиссионера по договору комиссии.

\end{itemize}
\end{frame}
\begin{frame}{Требования к залоговому имуществу}
\begin{itemize}[<+->]

\item
Приемлемость. 
\item
Достаточность.

\end{itemize}
\end{frame}

\subsection{Цессия}
\begin{frame} {Уступка требований (цессия) и передача права собственности}
\begin{block}{Уступка (цессия) }
\quad
- это документ заемщика (цедента), в котором он уступает свое требование (дебиторскую задолженность) кредитору (банку) в качестве обеспечения возврата кредита.
\end{block}

Договор о цессии предусматривает переход к банку права получения денежных средств по уступленному требованию.
\end{frame}

\begin{frame}[shrink=15]
\begin{figure}
\center
\begin{overprint}
	\forloop{slideno}{1}{\value{slideno} < 6}{%
		\only<\value{slideno}>{
			\includegraphics[page=\value{slideno},
			scale=.9
			% trim={<left> <lower> <right> <upper>}				
			%,trim={.5cm 5.5cm 3cm 0cm}
			,clip]
			{tikz/assignment_scheme}}}
\end{overprint}
\vspace*{-2em}
\caption{Схема цессии}
\end{figure}
\only<2>{
1) заемщик (цедент) имеет требование к дебитору, например, своему покупателю;
}
\only<3>{
2) заемщик заключает с банком кредитный договор;
}
\only<4>{
3) заемщик передает банку право требования по своей дебиторской задолженности;
}

\only<5>{
3) банк предъявляет требование по дебиторской задолженности дебитору.
}
\end{frame}

\begin{frame}[shrink=15]{Разновидность цессии}{Передача собственности кредитору}
\begin{figure}
\center
\begin{overprint}
	\forloop{slideno}{1}{\value{slideno} < 5}{%
		\only<\value{slideno}>{
			\includegraphics[page=\value{slideno},
			scale=1
			% trim={<left> <lower> <right> <upper>}				
			%,trim={.5cm 7cm 2cm 0cm}
			,clip]
			{tikz/property_transfer}}}
\end{overprint}
\vspace*{-2em}
\caption{Схема передачи собственности кредитору}
\end{figure}
\only<2>{
1) заемщик заключает с кредитором договор о передаче права собственности на движимое имущество;
}
\only<3>{
2) движимое имущество остается для использования у заемщика без права собственности;
}
\only<4>{
3) банк приобретает право собственности на движимое имущество без права пользования.
}

\end{frame}


\subsection{Банковская гарантия}
\begin{frame} [ allowframebreaks]{Банковская гарантия}
\begin{block}{ }
\quad
В силу банковской гарантии банк, или иное кредитное учреждение, или страховая организация (гарант) дают по просьбе другого лица (принципала) письменное обязательство уплатить кредитору принципала (бенефициару) в соответствии с условиями даваемого гарантом обязательства денежную сумму по представлении бенефициаром письменного требования о ее уплате.
\end{block}
\end{frame}

\begin{frame}[ allowframebreaks ]{Условия банковской гарантии}
\begin{itemize}
\item
Банковская гарантия обеспечивает надлежащее исполнение принципалом его обязательств перед бенефициаром (основного обязательства).

\item
За выдачу банковской гарантии принципал уплачивает гаранту вознаграждение.

\pagebreak
\item
Обязательство гаранта перед бенефициаром не зависит в отношениях между ними от того основного обязательства, в обеспечение исполнения которого она выдана, даже если в гарантии содержится ссылка на это обязательство.
\end{itemize}
\end{frame}

\begin{frame}{Условия банковской гарантии}
\begin{itemize}[<+->]
\item
Банковская гарантия не может быть отозвана гарантом, если в ней не предусмотрено иное.

\item
Право требования бенефициара к гаранту не может быть передано другому лицу.

\item
Банковская гарантия вступает в силу со дня ее выдачи, если в гарантии не предусмотрено иное.
\end{itemize}
\end{frame}

\begin{frame}[shrink=15]
\begin{figure}
\center
\begin{overprint}
	\forloop{slideno}{1}{\value{slideno} < 8}{%
		\only<\value{slideno}>{
			\includegraphics[page=\value{slideno},
			scale=.9
			% trim={<left> <lower> <right> <upper>}				
			,trim={.5cm 5cm 3cm 0cm},clip]
			{tikz/bank1_garantee}}}
\end{overprint}
\vspace*{-2em}
\caption{Схема банковской гарантии с участием одного банка}
\end{figure}
\only<2>{
1) договор поставки; 
}

\only<3>{
2) договор банковской гарантии (оговорено обеспечение) или кредитный договор с залогом; 
}

\only<4>{
3) комиссионные; 
}

\only<5>{
4) требование; 
}

\only<6>{
5) платеж гарантийной суммы; 
}

\only<7>{
6) возмещение гарантийной суммы банку (за счет обеспечения).
}
\end{frame}

\begin{frame}[shrink=15]
\begin{figure}
\center
\begin{overprint}
	\forloop{slideno}{1}{\value{slideno} < 8}{%
		\only<\value{slideno}>{
			\includegraphics[page=\value{slideno},
			scale=.9
			% trim={<left> <lower> <right> <upper>}				
			,trim={.5cm 5cm 3cm 0cm},clip]
			{tikz/bank2_garantee}}}
\end{overprint}
\vspace*{-2em}
\caption{Схема банковской гарантии\\ с участием двух банков}
\end{figure}
\only<2>{
1) кредитный договор; 
}

\only<3>{
2) договор банковской гарантии (с обеспечением); 
}

\only<4>{
3) комиссионные; 
}

\only<5>{
4) требование; 
}

\only<6>{
5) платеж гарантийной суммы; 
}

\only<7>{
6) возмещение суммы гаранту (за счет обеспечения).
}
\end{frame}

\begin{frame}{Виды банковских гарантий}
\begin{itemize}[<+->]
\item
Срочная безотзывная и отзывная
\item
Безусловная и условная
\item
Обеспеченная и необеспеченная
\item
Ограниченная и неограниченная по сумме
\end{itemize}
\end{frame}

\begin{frame}{Виды банковских гарантий}
\begin{itemize}[<+->]
\item
Простая и синдицированная
\item
Прямая гарантия и контргарантия.
\item
Используемые в качестве средства обеспечения выполнения обязательств или используемые в качестве средства обеспечения платежей
\end{itemize}
\end{frame}

\begin{frame}{Схема работы банка с гарантиями за клиентов}
Банк может:

\begin{itemize}[<+->]
\item
выдавать, 

\item
получать (банк сам обращается в другие банки за гарантией),

\item
принимать гарантии (гарантию другого банка представляет клиент).

\end{itemize}
\end{frame}

\subsection{Поручительство}
\begin{frame} [ allowframebreaks]{Поручительство}
\begin{block}{Поручительство}
\quad
- это форма обеспечения возвратности кредита. Поручитель обязуется перед кредитором отвечать за исполнение должником его обязательств на всю сумму кредитов и процентов по нему или на их часть (например, только на сумму процентов).
\end{block}

\pagebreak
Применяется как при взаимоотношениях банка с юридическими, так и с физическими лицами и всегда оформляется письменным договором.

По договору поручительства возникает солидарная ответственность по обязательствам должника (заемщика) перед банком: банк не может предъявить свои требования к поручителю до момента, пока последний не обратится за взысканием непосредственно к заемщику и не получит отказа погасить кредит. 


\end{frame}


\subsection{Контрольные вопросы}
\begin{frame}[ allowframebreaks ]{Контрольные вопросы}
1. Залог и залоговый механизм

2. Цессия

3. Уступка требований (цессия) и передача права собственности

4. Понятие банковской гарантии и её виды.

\pagebreak
5. Схемы банковских гарантий: с участием одного банка, с участием двух банков.

6. Договор поручительства.
\end{frame}

\setbeamercovered{transparent}
\end{document}
