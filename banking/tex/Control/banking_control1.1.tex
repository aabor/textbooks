% !TeX program = lualatex -synctex=1 -interaction=nonstopmode --shell-escape %.tex 

\documentclass[12pt, table]{exam}
\usepackage[rus]{borochkin}

\usepackage{borochkin_exam}

%%%%%%%%%%%%%%%%%%%%%%%%%%%%%%%%%%%%%%%%%

\professor
\iftagged{professor}{ \printanswers }
%%%%%%%%%%%%%%%%%%%%%%%%%%%%%%%%%%%%%%%%%


\begin{document}
\setcounter{section}{0\relax}%
\noindent
% Контр/р № #1, Вариант #2, Предмет #3
\studentpersonalinfo{1}{1}{БД}
\normalsize

\begin{questions}
\question[15] На основе приведенных в таблице макроэкономических показателей рассчитайте размер чистых иностранных активов в банковской системе, размер требований банковской системы к внутреннему сектору, объем широкой денежной массы. Составьте консолидированный баланс банковской системы, так чтобы сальдо входящих статей было равно нулю. Справочно: консолидированный баланс банковской системы включает в себя баланс Центрального Банка РФ и консолидированный баланс кредитных организаций, за вычетом взаимных требований между ними.

\small
\begin{tabularx}{\linewidth}[b]{@{}>{\raggedright\arraybackslash}Xr@{}}
	                                                   & млрд. руб. \\ \toprule
	Прочие статьи (нетто)                              & 9 154      \\
	Срочные депозиты в рублях                          & 18 059     \\
	Чистые требования к государству                    & -   6 590  \\
	Обязательства перед нерезидентами                  & 10 661     \\
	Наличная валюта вне банковской системы             & 7 055      \\
	Переводные депозиты в рублях                       & 8 861      \\
	Депозиты в банках нерезидентах                     & 7 076      \\
	Требования к другим секторам                       & 53 373     \\
	Депозитные и сберегательные сертификаты            & 571        \\
	Наличная иностранная валюта                        & 1 384      \\
	Долговые ценные бумаги, размещенные у нерезидентов & 25 031     \\
	Акции и другие формы участия в капитале            & 21 712     \\
	Срочные депозиты в иностранной валюте              & 16 295     \\
	Прочие требования к нерезидентам                   & 7 510      \\
	Монетарное золото и СДР                            & 4 583      \\
	Сальдо (всегда равно нулю)                         & -          \\ \bottomrule
\end{tabularx}%
\normalsize

\pagebreak
\begin{solution}[6em] В результате перегруппировки балансовых статей получаем:
	
\small
\begin{tabularx}{\linewidth}[b]{@{}>{\raggedright\arraybackslash}Xr@{}}
	                                                  & млрд. руб. \\ \toprule
	Чистые иностранные активы                         & 34 924     \\ \midrule
	Требования к нерезидентам                         & 45 585     \\
	Монетарное золото и СДР                           & 4 583      \\
	Наличная иностранная валюта                       & 1 384      \\
	Депозиты в банках нерезидентах                    & 7 076      \\
	Долговые ценные бумаги размещенные у нерезидентов & 25 031     \\
	Прочие требования к нерезидентам                  & 7 510      \\
	Обязательства перед нерезидентами                 & 10 661     \\ \midrule
	Внутренние требования                             & 46 783     \\
	Чистые требования к государству                   & -   6 590  \\
	Требования к другим секторам                      & 53 373     \\ \midrule
	Широкая денежная масса                            & 50 841     \\
	Наличная валюта вне банковской системы            & 7 055      \\
	Переводные депозиты в рублях                      & 8 861      \\
	Срочные депозиты в рублях                         & 18 059     \\
	Срочные депозиты в иностранной валюте             & 16 295     \\
	Депозитные и сберегательные сертификаты           & 571        \\ \midrule
	Акции и другие формы участия в капитале           & 21 712     \\ \midrule
	Прочие статьи (нетто)                             & 9 154      \\ \midrule
	Сальдо (всегда равно нулю)                        & -          \\ \bottomrule
\end{tabularx}%
\normalsize
	
\end{solution}

\question[20] Гражданин Иванов И.И. обращается в банк «Бета» за кредитом на открытие мойки для легковых автомобилей. Сумма кредита составляет 300~тыс.~руб., срок кредита – один год, банк будет брать годовую процентную ставку 22.00\%. Основной долг и проценты должны быть выплачены в течение года. Если заемщик обанкротится, банк вернет только десятую часть долга и причитающихся процентов. Стоимость денег для банка «Бета» составляет 17.00\% годовых.  В результате оценки кредитоспособности Иванова И. И. банк определил 95\%-ную вероятность возврата кредита.
\noaddpoints

\begin{subparts}
\subpart[10] Следует ли выдать гражданину Иванову И.И. кредит?

\begin{solution}[6em]
Пусть $L$ – сумма кредита; $r_D$ – ставка по привлеченным средствам банка;  $r_L$ – ставка по кредиту заемщику; $k$ – доля долга и процентов, получаемых банком в случае банкротства заемщика; $p$ – вероятность возврата кредита.

Банку следует выдать кредит, если ожидаемая сумма возврата превысит сумму, уплаченную вкладчикам банка.

Рассматриваются два события: возврат кредита и банкротство заемщика. В сумме вероятности этих событий дают единицу. 

$L\cdot (1+r_D )$ – сумма, которую банк должен вернуть своим вкладчикам;

$L \cdot p \cdot (1+r_L )$ – сумма, возвращаемая в банк, если заемщик не обанкротится;

$L \cdot k \cdot (1-p) \cdot (1+r_L )$ - сумма, получаемая банком, если заемщик обанкротится.

Общее неравенство, которое должно быть выполнено для одобрения банком кредита следующее:
\begin{align*}
L\cdot (1+r_D )<L \cdot p \cdot (1+r_L ) + L \cdot k \cdot (1-p) \cdot (1+r_L )
\end{align*}
В результате преобразований определяется требуемая минимальная вероятность возврата кредита:
\begin{align*}
p<\frac{\frac{1+r_D}{1+r_L}-k}{1-k}
\end{align*}	
Расчет по приведенной формуле дает пограничную вероятность 95.45\%, следовательно, заемщику в кредите должно быть отказано.
\end{solution}

\subpart[10] Какова должна быть вероятность возврата кредита, чтобы банк мог ожидать 4\%-ную прибыль от данной кредитной операции?

\begin{solution}[6em]
Аналогичным образом составляется уравнение для определения требуемой вероятности возврата кредита, которая бы обеспечила необходимую прибыль банку.

Пусть $\pi$ - норматив прибыли по кредитной операции, тогда ожидаемый доход от кредитной операции должен быть равен сумме прибыли по нормативу:

\begin{align*}
L \cdot p \cdot (1+r_L ) + L \cdot k \cdot (1-p) \cdot (1+r_L ) - L \cdot (1+r_D ) = \pi \cdot L
\end{align*}
Преобразование дает:
\begin{align*}
p = \frac{\frac{1+r_D+\pi}{1+r_L} - k}{1 - k}
\end{align*}
Согласно расчетам по этой формуле, для обеспечения банку 4\%-ной прибыли по этой операции, вероятность возврата кредита должна составлять не менее 99.09\%.
\end{solution}

\end{subparts}
\addpoints

\question[20] Экономисты банка рассчитали, что общая потребность в наличных денежных средствах на ближайший год для банка составляет 250~млн.~руб. Банк придерживается политики управления ликвидными средствами по модели Баумоля. 

Максимальный размер резерва наличности установлен в сумме 5~млн.~руб. Расходование ликвидного резерва происходит равномерно. При достижении резервом наличности нулевой величины, банк продает часть менее ликвидных активов и снова пополняет резерв до установленного максимального значения. При этом расходы банка за каждую такую операцию составляют 2000~руб.

Экономисты банка посчитали, что средства,  хранимые в кассе, будучи размещенными в качестве кредитов, могли бы приносить 7\% годовых.
\noaddpoints

\pagebreak
\begin{subparts}
	\subpart[8] Определите совокупные расходы банка (включая упущенную выгоду) на поддержание установленного резерва наличности в течение года.
	
	\begin{solution}[6em]
		
		\raggedright
		Совокупные расходы банка равны:
		\begin{align}
		TC&=\frac{C}{2}\cdot K + \frac{T}{C} \cdot F \\
		&= 275~000~р.\nonumber
		\end{align}	
		где 
		
		$T$, $C$, $F$, $K$ - соответственно, общие потребности в наличности для банка на ближайший год, максимальный размер резерва наличности в течение года, стоимость одной транзакции, стоимость денег для банка;
		
		$\frac{C}{2}\cdot K$ - стоимость упущенной выгоды за год (недополученные проценты по ссудам);
		
		$\frac{T}{C}$ - количество транзакций на пополнение резерва наличности;
		
		$\frac{T}{C} \cdot F$ - стоимость транзакций на пополнение резерва наличности.
	\end{solution}
	
	\subpart[10] 	Определите такую величину максимального размера резерва наличности, при которой расходы банка на его поддержание будут минимальными. 
	
	\begin{solution}[6em]
		Расходы банка будут минимальными при минимальном значении функции совокупных расходов, $TC$. Это условие выполняется в том случае, если её производная, $TC'$, равна нулю.
		\begin{align}
		TC'&=\frac{1}{2} \cdot K - \frac{T}{C^2} \cdot F = 0\\
		C^*&=\sqrt{\frac{2 \cdot F \cdot T}{K}}\\
		&=3.78~\text{млн. руб.}\nonumber
		\end{align}
		
	\end{solution}
	
	\subpart[2] Как следует банку скорректировать величину максимального размера резерва наличности с целью минимизации расходов?
	
	\begin{solution}[6em]
		Банку необходимо уменьшить максимальный размер резерва наличности до оптимальной величины $C^*$.
	\end{solution}
	
\end{subparts}
\addpoints

\pagebreak
\question[20] Российская фирма покупает у французской компании комплект оборудования стоимостью 1.5~млн.~Евро. Кредит предоставляется на три года, выплаты в погашение основной суммы долга одинаковые и производятся один раз в год.
Продавец выписывает комплект переводных векселей в которых прописана сумма, равная стоимости отгружаемого товара и процентов за кредит, и направляет эти векселя покупателю для акцепта. Продавец использует простую ставку 2.5\% годовых для определения суммы процентов по векселю. 
Продавец, сражу же после получения портфеля акцептованных векселей, учитывает его в банке без права оборота на себя (форфейтинг), получая деньги в самом начале сделки. По форфейтной операции банк применяет простую учетную ставку в размере 1.25\% годовых.

\noaddpoints

\begin{subparts}
	\subpart[5] Определите номинальные суммы переводных векселей, выписанных экспортером и общую сумму денег, которую должен будет заплатить импортер. Результаты представьте в таблице.
	
	\begin{solution}[6em]
		
		Сумма векселя, погашаемого в момент времени $t$, равна:
		\begin{align}
		V_t&=\frac{P}{n} \cdot \left(1+(n-t+1) \cdot i \right).
		\end{align}
		Общая сумма векселей равна:
		\begin{align}
		\sum_t V_t = P \cdot \left(1+ \frac{n+1}{2} \cdot i \right).
		\end{align}
		
	\end{solution}
	
	\subpart[5] Определите суммы, которые получит экспортер по каждому векселю, учтенному в банке и общий размер своей выручки. Результаты представьте в таблице. 
	
	\begin{solution}[6em]
		
		Экспортер получит в банке по каждому учтенному векселю следующую сумму:
		\begin{align}
		PV_t=V_t \cdot (1-t \cdot d)
		\end{align}
		и всего:
		\begin{align}
		A&= P \cdot \left( 1+ \frac{n+1}{2} \cdot \left(i - d - i \cdot d \cdot \frac{n+2}{3} \right) \right)\\
		&=1.536~млн.~Евро\nonumber
		\end{align}
		\begin{tabularx}{\linewidth}[b]{@{}>{\raggedright\arraybackslash}cXrrrr@{}}	
			\toprule
			Период & & $\frac{P}{n}$ & $V_t$ & $PV_t$ \\
			\midrule			
			1    & & 500   & 537,5 & 530,8 \\
			2    & & 500   & 525,0 & 511,9 \\
			3    & & 500   & 512,5 & 493,3 \\
			\midrule			
			Итого & & 1500  & 1575,0 & 1535,9 \\
			\bottomrule
		\end{tabularx}%
	\end{solution}
		
	\subpart[10] Покроет ли выручка от учета векселей в банке стоимость отгруженной экспортером продукции? 
	
	Определите барьерную (минимальную) процентную ставку, которую должен назначить экспортер, чтобы получить сумму больше, чем первоначальная цена продукции.
	
	\begin{solution}[6em]
		
		Экспортер получит сумму, меньшую первоначальной цены. 
		
		Продавец не будет иметь потери, если процентные ставки $i$ и $d$ находятся в следующем соотношении (вывод из предыдущего уравнения):
		\begin{align}
		i-d=i \cdot d \cdot \frac{n+2}{3}.
		\end{align}
		
		Поэтому барьерная процентная ставка равна:
		\begin{align}
		i^*&=\frac{d}{1-\frac{n+2}{3} \cdot d}\\
		&=1.28\%.\nonumber
		\end{align}
		
	\end{solution}
\end{subparts}
\addpoints

\pagebreak
\question[20] Вы являетесь финансовым менеджером коммерческого банка ``Сириус``. 
Показатели деятельности КБ ``Сириус`` представлены в таблице.

\begin{tabularx}{\linewidth}[b]{@{}>{\raggedright\arraybackslash}Xr@{}}	
	& млн. руб. \\
	\midrule
    Уставный фонд & 5000 \\
	Общие резервы & 700 \\
	Прибыль прошлых лет & 2500 \\
	Итого собственные средства & 8200 \\
	\midrule
	Срочные депозиты физических лиц & 8000 \\
	Депозиты до востребования юридических лиц & 2500 \\
	Срочные депозиты юридических лиц & 1000 \\
	Итого привлеченные средства & 11500 \\
	Пассивы, всего & 19700 \\
	\midrule
	Активные операции &  \\
	\midrule
	Доли в общем объеме &  \\
	кредитование юридических лиц & 85\% \\
	вложения в ценные бумаги & 15\% \\
	\midrule
	Доходность &  \\
	кредитование юридических лиц & 9\% \\
	вложения в ценные бумаги & 15\% \\
	\midrule
	Ставки привлечения средств &  \\
	Срочные депозиты юридических лиц & 4,0\% \\
	Депозиты до востребования юридических лиц & 0,50\% \\
	Срочные депозиты физических лиц & 6,0\% \\
	\midrule
	Доли использования ресурсов банка в активных операциях &  \\
	собственные средства & 80\% \\
	депозиты до востребования & 10\% \\
	срочные депозиты & 100\% \\
	\midrule
	Норматив обязательных резервов по депозитам & 4,50\% \\
	Отчисления в фонд страхования вкладов физических лиц & 2\% \\
	Ставка налога на прибыль & 35\% \\
	Коэффициент выплат дивидендов (от прибыли после налогообложения) & 50\% \\
	Норматив отношения собственных и заемных средств, не менее & 15\% \\
	\bottomrule
\end{tabularx}%
\noaddpoints

\pagebreak
\begin{subparts}
	\subpart[5] Определите текущую доходность акций банка (Доходность акций рассчитывается как отношение фонда дивидендов к собственным средствам банка). 
	
	\begin{solution}[6em]
		
		Текущий общий объем средств банка, направляемых на активные операции равен:
		
		СобСр * ДоляИспользованияСобСр + СрДепФизЛиц * (1 - НормРезДепоз) + СрДепЮрЛиц * (1- НормРезДепоз)  + ДоляИспольованияДепДовострЮрЛиц * ДепДовострЮрЛиц * (1- НормРезДепоз) = 15394
		
		
		ДоходностьАктивов = ДоляКредитовЮрЛицам * ДоходностьКредитовЮрЛицам + ДоляЦенныхБумаг * ДоходностьЦенныхБумаг = 9.9\%
		
		Доходы = Активы * ДоходностьАктивов = 1950
		
		Расходы = ПроцСрДепФизЛиц + ПроцСрДепЮрЛиц + ПроцДепДовострФизЛиц + 
			СрДепФизЛиц * (1+ НормОтчФондСтрахВкл)= 553
		
		ЧистаяПрибыль = (1- НалПриб) * (Доходы - Расходы) = 909
				
		ДоходностьАкций = КоэфВыплДив * ЧистаяПриб / СобСр = 5.540\%
				
	\end{solution}
	
	\subpart[5] Предположим, что банку требуется повысить уровень доходности своих акций до 13\% годовых. Определите, будет ли достигнута эта цель, если банк направит всю чистую прибыль на выплату дивидендов?
	
	\begin{solution}[6em]
		
		Если банк направит всю чистую прибыль на выплату дивидендов, то доходность акций составит:
		
		ЧистаяПриб / СобСр = 11.080\%.
		
		Целевая доходность акций не достигается.
	\end{solution}

	\subpart[10] Определите, какой объем срочных депозитов юридических лиц должен привлечь банк, чтобы достигнуть целевого уровня доходности своих акций. Будет ли банк в этом случае выполнять норматив по соотношению собственных и привлеченных средств?
	
	\begin{solution}[6em]
		
		Требуемый размер валовой прибыли для увеличения доходности акций до целевого уровня составит:
		
		СобСр * ТребуемаяДоходность / [КоэффициентВыплатДивидендов * (1 - СтавкаНалогаПрибыль)] = 3280
		
		Таким образом, валовая прибыль должна быть увеличена на 2371. 
		
		Процентная маржа, получаемая банком, за счет депозитов юридических лиц равна:
		
		ПроцМаржаСрочДепозЮрЛиц = ДоходностьАктивов * (1 - НормаОбязательныхРезервов) -
		СтСрочДепозЮрЛиц = 5.45\%
		
		ПриростСрочДепозЮрЛиц =  ПриростПрибыли / ПроцентнаяМаржа = 43477
				
		Соотношение собственных и привлеченных средств банка при этом составит:
		
		СобСр / (ПривлечСр + ПриростСрочДепозЮрЛиц) = 14.92\%.
		
		Таким образом, требование регулятора о соотношении собственных и заемных средств банка не выполняются. 		
	\end{solution}
\end{subparts}
\addpoints

\end{questions}

\end{document}
