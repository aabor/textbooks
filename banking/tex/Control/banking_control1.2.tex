% !TeX program = lualatex -synctex=1 -interaction=nonstopmode --shell-escape %.tex

\documentclass[12pt, table]{exam}
\usepackage[rus]{borochkin}

\usepackage{borochkin_exam}

%%%%%%%%%%%%%%%%%%%%%%%%%%%%%%%%%%%%%%%%%

\professor
\iftagged{professor}{ \printanswers }

%%%%%%%%%%%%%%%%%%%%%%%%%%%%%%%%%%%%%%%%%


\begin{document}
\setcounter{section}{0\relax}%
\noindent
% Контр/р № #1, Вариант #2, Предмет #3
\studentpersonalinfo{1}{2}{БД}
\normalsize

\begin{questions}
\question[15] На основе приведенных в таблице макроэкономических показателей рассчитайте размер чистых иностранных активов в банковской системе, размер требований банковской системы к внутреннему сектору, объем широкой денежной массы. Составьте консолидированный баланс банковской системы, так чтобы сальдо входящих статей было равно нулю. Справочно: консолидированный баланс банковской системы включает в себя баланс Центрального Банка РФ и консолидированный баланс кредитных организаций, за вычетом взаимных требований между ними.

\small
\begin{tabularx}{\linewidth}[b]{@{}>{\raggedright\arraybackslash}Xr@{}}				& млрд. руб.\\
	\toprule
    Срочные депозиты в иностранной валюте &    12 214    \\
	Наличная иностранная валюта &      1 585    \\
	Переводные депозиты в рублях &    10 024    \\
	Обязательства перед нерезидентами &      7 973    \\
	Монетарное золото и СДР &      4 575    \\
	Прочие статьи (нетто) &    10 213    \\
	Депозиты в банках нерезидентах &      7 611    \\
	Наличная валюта вне банковской системы &      8 033    \\
	Акции и другие формы участия в капитале &    18 345    \\
	Прочие требования к нерезидентам &      5 638    \\
	Чистые требования к государству & -   4 310    \\
	Требования к другим секторам &    54 666    \\
	Срочные депозиты в рублях &    21 218    \\
	Депозитные и сберегательные сертификаты &         447    \\
	Долговые ценные бумаги, размещенные у нерезидентов &    18 702    \\
	Сальдо (всегда равно нулю) &           -      \\
	\bottomrule
\end{tabularx}%
\normalsize

\begin{solution}[6em] В результате перегруппировки балансовых статей получаем:
	
	\small
	\begin{tabularx}{\linewidth}[b]{@{}>{\raggedright\arraybackslash}Xr@{}}				& млрд. руб.\\
		\toprule
		Чистые иностранные активы &    30 137    \\
		\midrule
		Требования к нерезидентам &    38 110    \\
		Монетарное золото и СДР &      4 575    \\
		Наличная иностранная валюта &      1 585    \\
		Депозиты в банках нерезидентах &      7 611    \\
		Долговые ценные бумаги размещенные у нерезидентов &    18 702    \\
		Прочие требования к нерезидентам &      5 638    \\
		Обязательства перед нерезидентами &      7 973    \\
		\midrule
		Внутренние требования &    50 356    \\
		Чистые требования к государству & -   4 310    \\
		Требования к другим секторам &    54 666    \\
		\midrule
		Широкая денежная масса &    51 936    \\
		Наличная валюта вне банковской системы &      8 033    \\
		Переводные депозиты в рублях &    10 024    \\
		Срочные депозиты в рублях &    21 218    \\
		Срочные депозиты в иностранной валюте &    12 214    \\
		Депозитные и сберегательные сертификаты &         447    \\
		\midrule
		Акции и другие формы участия в капитале &    18 345    \\
		\midrule
		Прочие статьи (нетто) &    10 213    \\
		\midrule
		Сальдо (всегда равно нулю) &           -      \\
		\bottomrule
	\end{tabularx}%
	\normalsize
	
\end{solution}

\pagebreak
\question[20] Гражданин Сидоров С.С. обращается в банк «Гамма» за кредитом на открытие парикмахерской. Сумма кредита составляет 0.5~млн.~руб., срок кредита – один год, банк будет брать годовую процентную ставку 15.48\%. Основной долг и проценты должны быть выплачены в течение года. Если заемщик обанкротится, банк вернет только десятую часть долга и причитающихся процентов. Стоимость денег для банка «Гамма» составляет 5.24\% годовых.  В результате оценки кредитоспособности Сидорова С.С. банк определил 95\%-ную вероятность возврата кредита.
\noaddpoints
\begin{subparts}
\subpart[10] Следует ли выдать гражданину Сидорову С.С. кредит?

\begin{solution}[6em]
	Пусть $L$ – сумма кредита; $r_D$ – ставка по привлеченным средствам банка;  $r_L$ – ставка по кредиту заемщику; $k$ – доля долга и процентов, получаемых банком в случае банкротства заемщика; $p$ – вероятность возврата кредита.
	
	Банку следует выдать кредит, если ожидаемая сумма возврата превысит сумму, уплаченную вкладчикам банка.
	
	Рассматриваются два события: возврат кредита и банкротство заемщика. В сумме вероятности этих событий дают единицу. 
	
	$L\cdot (1+r_D )$ – сумма, которую банк должен вернуть своим вкладчикам;
	
	$L \cdot p \cdot (1+r_L )$ – сумма, возвращаемая в банк, если заемщик не обанкротится;
	
	$L \cdot k \cdot (1-p) \cdot (1+r_L )$ - сумма, получаемая банком, если заемщик обанкротится.
	
	Общее неравенство, которое должно быть выполнено для одобрения банком кредита следующее:
	\begin{align*}
	L\cdot (1+r_D )<L \cdot p \cdot (1+r_L ) + L \cdot k \cdot (1-p) \cdot (1+r_L )
	\end{align*}
	В результате преобразований определяется требуемая минимальная вероятность возврата кредита:
	\begin{align*}
	p<\frac{\frac{1+r_D}{1+r_L}-k}{1-k}
	\end{align*}	
	Расчет по приведенной формуле дает пограничную вероятность 90.15\%, следовательно, заемщику можно выдать кредит.
\end{solution}

\subpart[10] Какова должна быть вероятность возврата кредита, чтобы банк мог ожидать 3\%-ную прибыль от данной кредитной операции?

\begin{solution}[6em]
	Аналогичным образом составляется уравнение для определения требуемой вероятности возврата кредита, которая бы обеспечила необходимую прибыль банку.
	
	Пусть $\pi$ - норматив прибыли по кредитной операции, тогда ожидаемый доход от кредитной операции должен быть равен сумме прибыли по нормативу:
	
	\begin{align*}
	L \cdot p \cdot (1+r_L ) + L \cdot k \cdot (1-p) \cdot (1+r_L ) - L \cdot (1+r_D ) = \pi \cdot L
	\end{align*}
	Преобразование дает:
	\begin{align*}
	p = \frac{\frac{1+r_D+\pi}{1+r_L} - k}{1 - k}
	\end{align*}
	Согласно расчетам по этой формуле, для обеспечения банку 3\%-ной прибыли по этой операции, вероятность возврата кредита должна составлять не менее 93.03\%.
\end{solution}

\end{subparts}
\addpoints

\question[20] БД.К1.З3.В2. Ежедневные остатки наличности в кассах банка представлены в таблице. Стоимость одной транзакции для банка составляет 15~руб., а стоимость денег~13\%. Банк придерживается политики управления ликвидными средствами по модели Миллера-Орра. Экономисты банка рассчитали, что минимальный объем средств в кассах должен составлять 1.2~млн.~руб. ежедневно.
\begin{table}[htbp]
	\centering
	\begin{tabular}{lr}
		& руб.\\
		\toprule
	    День 1  &      2 903 000    \\
		День 2  &      3 150 000    \\
		День 3  &      2 260 000    \\
		День 4  &      3 593 000    \\
		День 5  &      4 424 000    \\
		День 6  &      2 764 000    \\
		День 7  &      1 795 000    \\
		День 8  &      3 239 000    \\
		\bottomrule
	\end{tabular}%
	\label{tab:addlabel}%
\end{table}%

\noaddpoints

\begin{subparts}
	\subpart[5] Определите стандартное отклонение ежедневных остатков наличности в банке.
	
	\begin{solution}[6em]
		$\sigma=0.803\text{~млн.~руб.}$
	\end{solution}
	
	\subpart[5] Рассчитайте целевой ориентир баланса наличности по модели Миллера-Орра.	
	\begin{solution}[6em]
		
		\begin{align}
		Z^*&=\sqrt[3]{\frac{3 \cdot F \cdot \sigma^2}{4 \cdot (K/365)}}+L\\
		&=1.439\text{~млн.~руб.},\nonumber
		\end{align}
		где
		
		$Z^*$ - целевой ориентир баланса наличности;
		
		$F$ - Стоимость трансакции;
		
		$\sigma$ - стандартное отклонение дневного потока наличности в банке;
		
		$K$ - возможная стоимость денег для банка;
		
		$L$ - нижний предел уровня наличности.
		
	\end{solution}
	
	\subpart[5] Определите верхний предел уровня наличности, при котором банк будет покупать высоколиквидные ценные бумаги, согласно модели Миллера-Орра.
	\begin{solution}[6em] 
		Верхний предел уровня наличности равен:
		\begin{align}
		H^*&=3 \cdot Z^* - 2 \cdot L\\
		&=1.916\text{~млн.~руб.}\nonumber
		\end{align}
	\end{solution}
	
	\subpart[5] Рассчитайте средний баланс наличности, который будет наблюдаться в кассах банка, при стратегии управления наличными по модели Миллера-Орра.
	\begin{solution}[6em]
		Средний баланс наличности равен:
		\begin{align}
		ABC&=\frac{4 \cdot Z^*-L}{3}\\
		&=1.518\text{~млн.~руб.}\nonumber
		\end{align}	
	\end{solution}
	
\end{subparts}
\addpoints

\question[20] Российская фирма покупает у французской компании комплект оборудования стоимостью 6~млн.~Евро. Кредит предоставляется на три года, выплаты в погашение основной суммы долга одинаковые и производятся один раз в год.
Продавец выписывает комплект переводных векселей в которых прописана сумма, равная стоимости отгружаемого товара и процентов за кредит, и направляет эти векселя покупателю для акцепта. Продавец использует простую ставку 12.3\% годовых для определения суммы процентов по векселю. 
Продавец, сражу же после получения портфеля акцептованных векселей, учитывает его в банке без права оборота на себя (форфейтинг), получая деньги в самом начале сделки. По форфейтной операции банк применяет простую учетную ставку в размере 7.5\% годовых.

\noaddpoints

\begin{subparts}
	\subpart[5] Определите номинальные суммы переводных векселей, выписанных экспортером и общую сумму денег, которую должен будет заплатить импортер. Результаты представьте в таблице.
	
	\begin{solution}[6em]
		
		Сумма векселя, погашаемого в момент времени $t$, равна:
		\begin{align}
		V_t&=\frac{P}{n} \cdot \left(1+(n-t+1) \cdot i \right).
		\end{align}
		Общая сумма векселей равна:
		\begin{align}
		\sum_t V_t = P \cdot \left(1+ \frac{n+1}{2} \cdot i \right).
		\end{align}
		
	\end{solution}
	
	\subpart[5] Определите суммы, которые получит экспортер по каждому векселю, учтенному в банке и общий размер своей выручки. Результаты представьте в таблице. 
	
	\begin{solution}[6em]
		
		Экспортер получит в банке по каждому учтенному векселю следующую сумму:
		\begin{align}
		PV_t=V_t \cdot (1-t \cdot d)
		\end{align}
		и всего:
		\begin{align}
		A&= P \cdot \left( 1+ \frac{n+1}{2} \cdot \left(i - d - i \cdot d \cdot \frac{n+2}{3} \right) \right)\\
		&=6.392~млн.~Евро\nonumber
		\end{align}
		\begin{tabularx}{\linewidth}[b]{@{}>{\raggedright\arraybackslash}cXrrrr@{}}	
			\toprule
			Период & & $\frac{P}{n}$ & $V_t$ & $PV_t$ \\
			\midrule			
			1     & & 2000  & 2738,0 & 2532,7 \\
			2     & & 2000  & 2492,0 & 2118,2 \\
			3     & & 2000  & 2246,0 & 1740,7 \\
			\midrule			
			Итого & & 6000  & 7476,0 & 6391,5 \\
			\bottomrule
		\end{tabularx}%
	\end{solution}
	
	\subpart[10] Покроет ли выручка от учета векселей в банке стоимость отгруженной экспортером продукции? 
	
	Определите барьерную (минимальную) процентную ставку, которую должен назначить экспортер, чтобы получить сумму больше, чем первоначальная цена продукции.
	
	\begin{solution}[6em]
		
		Экспортер получит сумму, меньшую первоначальной цены. 
		
		Продавец не будет иметь потери, если процентные ставки $i$ и $d$ находятся в следующем соотношении (вывод из предыдущего уравнения):
		\begin{align}
		i-d=i \cdot d \cdot \frac{n+2}{3}.
		\end{align}
		
		Поэтому барьерная процентная ставка равна:
		\begin{align}
		i^*&=\frac{d}{1-\frac{n+2}{3} \cdot d}\\
		&=8.57\%.\nonumber
		\end{align}
		
	\end{solution}
\end{subparts}
\addpoints

\pagebreak
\question[20] Вы являетесь финансовым менеджером коммерческого банка ``Импэкс``. 
Показатели деятельности КБ ``Импэкс`` представлены в таблице.

\begin{tabularx}{\linewidth}[b]{@{}>{\raggedright\arraybackslash}Xr@{}}	
	& млн. руб. \\
	\midrule
    & млн. руб. \\
	Уставный фонд & 4500 \\
	Общие резервы & 500 \\
	Прибыль прошлых лет & 2000 \\
	Итого собственные средства & 7000 \\
	Срочные депозиты физических лиц & 5000 \\
	Депозиты до востребования юридических лиц & 1500 \\
	Срочные депозиты юридических лиц & 500 \\
	Итого привлеченные средства & 7000 \\
	Пассивы, всего & 14000 \\
	Активные операции &  \\
	Доли в общем объеме &  \\
	кредитование юридических лиц & 90\% \\
	вложения в ценные бумаги & 10\% \\
	Доходность &  \\
	кредитование юридических лиц & 18\% \\
	вложения в ценные бумаги & 12\% \\
	Ставки привлечения средств &  \\
	Срочные депозиты юридических лиц & 10\% \\
	Депозиты до востребования юридических лиц & 1,00\% \\
	Срочные депозиты физических лиц & 11\% \\
	Доли использования ресурсов банка в активных операциях &  \\
	собственные средства & 65\% \\
	депозиты до востребования & 10\% \\
	срочные депозиты & 100\% \\
	Норматив обязательных резервов по депозитам & 4,50\% \\
	Отчисления в фонд страхования вкладов физических лиц & 2\% \\
	Ставка налога на прибыль & 35\% \\
	Коэффициент выплат дивидендов (от прибыли после налогообложения) & 60\% \\
	Норматив отношения собственных и заемных средств, не менее & 15\% \\
	\bottomrule
\end{tabularx}%
\noaddpoints

\begin{subparts}
	\subpart[5] Определите текущую доходность акций банка (Доходность акций рассчитывается как отношение фонда дивидендов к собственным средствам банка). 
	
	\begin{solution}[6em]
		
		Текущий общий объем средств банка, направляемых на активные операции равен:
		
		СобСр * ДоляИспользованияСобСр + СрДепФизЛиц * (1 - НормРезДепоз) + СрДепЮрЛиц * (1- НормРезДепоз)  + ДоляИспольованияДепДовострЮрЛиц * ДепДовострЮрЛиц * (1- НормРезДепоз) = 9946
						
		
		ДоходностьАктивов = ДоляКредитовЮрЛицам * ДоходностьКредитовЮрЛицам + ДоляЦенныхБумаг * ДоходностьЦенныхБумаг = 17.4\%
		
		Доходы = Активы * ДоходностьАктивов = 2436
		
		Расходы = ПроцСрДепФизЛиц + ПроцСрДепЮрЛиц + ПроцДепДовострФизЛиц + 
		СрДепФизЛиц * (1+ НормОтчФондСтрахВкл)= 670
				
		ЧистаяПрибыль = (1- НалПриб) * (Доходы - Расходы) = 1148
				
		ДоходностьАкций = КоэфВыплДив * ЧистаяПриб / СобСр = 9.839\%
		
		
	\end{solution}
	
	\subpart[5] Предположим, что банку требуется повысить уровень доходности своих акций до 20\% годовых. Определите, будет ли достигнута эта цель, если банк направит всю чистую прибыль на выплату дивидендов?
	
	\begin{solution}[6em]
		
		Если банк направит всю чистую прибыль на выплату дивидендов, то доходность акций составит:
		
		ЧистаяПриб / СобСр = 16.399\%.
		
		Целевая доходность акций не достигается.
	\end{solution}
	
	\subpart[10] Определите, какой объем срочных депозитов юридических лиц должен привлечь банк, чтобы достигнуть целевого уровня доходности своих акций. Будет ли банк в этом случае выполнять норматив по соотношению собственных и привлеченных средств?
	
	\begin{solution}[6em]
		
		Требуемый размер валовой прибыли для увеличения доходности акций до целевого уровня составит:
		
		СобСр * ТребуемаяДоходность / [КоэффициентВыплатДивидендов * (1 - СтавкаНалогаПрибыль)] = 3590
		
		Таким образом, валовая прибыль должна быть увеличена на 2442. 
		
		Процентная маржа, получаемая банком, за счет депозитов юридических лиц равна:
		
		ПроцМаржаСрочДепозЮрЛиц = ДоходностьАктивов * (1 - НормаОбязательныхРезервов) -
		СтСрочДепозЮрЛиц = 6.62\%
		
		ПриростСрочДепозЮрЛиц =  ПриростПрибыли / ПроцентнаяМаржа = 36903
				
		Соотношение собственных и привлеченных средств банка при этом составит:
		
		СобСр / (ПривлечСр + ПриростСрочДепозЮрЛиц) = 15.94\%.
		
		Таким образом, требование регулятора о соотношении собственных и заемных средств банка выполняются. 		
	\end{solution}
\end{subparts}
\addpoints

\end{questions}

\end{document}
