\documentclass[_Banking_p1.tex]{subfiles}

\begin{document}
\setbeamercovered{invisible}

\subsection{Нормативные документы}

\begin{frame}[allowframebreaks]{Создание банка}{Нормативные документы}
  \begin{thebibliography}{10}
  
  \beamertemplatearticlebibitems
    \bibitem{bb1}

Федеральный закон от 10.07.2002 N 86-ФЗ (ред. от 18.07.2017) "О Центральном банке Российской Федерации (Банке России)"

    \bibitem{bb2}
Федеральный закон от 02.12.1990 N 395-1 (ред. от 26.07.2017) "О банках и банковской деятельности"

\pagebreak

    \bibitem{bb5}
Инструкция ЦБ РФ «О порядке принятия Банком России решения о государственной регистрации кредитных организаций и выдаче лицензий на осуществление банковских операций» от 02.04.2010 г. № 135-И  (ред. от 21.03.2016).

    \bibitem{bb6}
Федеральный закон от 26.10.2002 N 127-ФЗ (ред. от 29.07.2017) "О несостоятельности (банкротстве)"


  \end{thebibliography}
\end{frame}

\subsection{Подготовка}
\begin{frame}{Подготовительный этап создания банка}{}
\begin{itemize}[<+->]
\item
миссия и стратегия будущей деятельности банка;
\item
размеры и структура уставного капитала, требования к учредителям и другим участникам;
\item
цели деятельности, ее сферы и преимущественные направления, клиентская база;
\item
организационная структура, органы управления, их полномочия, основные требования к организации управления.
\end{itemize}
\end{frame}

\begin{frame}[allowframebreaks]{Лицензия на осуществление банковских операций}
\begin{itemize}
\item
Инструкция Банка России от 02.04.2010 N 135-И (ред. от 25.11.2014) "О порядке принятия Банком России решения о государственной регистрации кредитных организаций и выдаче лицензий на осуществление банковских операций".
\item
Лицензия на осуществление банковских операций подписывается Председателем Банка России или его заместителем, возглавляющим Комитет банковского надзора Банка России, или лицами, их замещающими. 
\pagebreak
\item
В лицензии на осуществление банковских операций указываются банковские операции, на осуществление которых данная кредитная организация имеет право, а также валюта, в которой эти банковские операции могут осуществляться.
\item
Лицензия на осуществление банковских операций выдается без ограничения сроков ее действия.
\end{itemize}
\end{frame}

\begin{frame}{Минимальный размер уставного капитала вновь регистрируемых:}
\begin{itemize}
\item
банка 300 миллионов рублей;
\item
банка (генеральная лицензия) 900 миллионов рублей.
небанковской кредитной организации (для получения права на осуществление расчетов по поручению юридических лиц) 90 миллионов рублей;
\item
небанковской кредитной организации (для получения прав на осуществление переводов денежных средств без открытия банковских счетов) 18 миллионов рублей.
\end{itemize}
\end{frame}

\begin{frame}
\begin{block}{Генеральная лицензия}
\quad
предоставляет кредитной организации право осуществлять банковские операции со средствами в рублях и иностранной валюте, привлекать во вклады денежные средства физических и юридических лиц в рублях и иностранной валюте.
\end{block}
\end{frame}
\begin{frame}[shrink=15]{Учредители кредитной организации}
\begin{itemize}
\item
Учредителями кредитной организации могут быть юридические и (или) физические лица, участие которых в кредитной организации не запрещено федеральными законами.
\item
Учредители банка не имеют права выходить из состава участников банка в течение первых трех лет со дня его государственной регистрации.
\item
Учредитель - юридическое лицо должен иметь устойчивое финансовое положение, достаточно собственных средств для внесения в уставный капитал кредитной организации, осуществлять деятельность в течение не менее трех лет и выполнять обязательства перед бюджетами всех уровней за последние три года.
\end{itemize}
\end{frame}

\subsection{Регистрация}
\begin{frame}[allowframebreaks]{Регистрация банка в качестве юридического лица}{}
1) заявление с ходатайством о государственной регистрации кредитной организации и выдаче лицензии на осуществление банковских операций;

2) учредительный договор, если его подписание предусмотрено федеральным законом;

3) устав;

\pagebreak

4) бизнес-план, утвержденный собранием учредителей (участников) кредитной организации, протокол собрания учредителей (участников), содержащий решения об утверждении устава кредитной организации, а также кандидатур для назначения на должности руководителя кредитной организации и главного бухгалтера кредитной организации;

\pagebreak
5) документы об уплате государственной пошлины;

6) аудиторские заключения о бухгалтерской (финансовой) отчетности учредителей - юридических лиц;

7) документы, подтверждающие источники происхождения средств, вносимых учредителями - физическими лицами в уставный капитал кредитной организации;

\pagebreak
8) анкеты кандидатов на должности руководителя кредитной организации, главного бухгалтера их заместителей. Сведения:

- о наличии у этих лиц высшего юридического или экономического образования и опыта руководства отделом или иным подразделением кредитной организации, связанным с осуществлением банковских операций, не менее одного года, а при отсутствии специального образования - опыта руководства таким подразделением не менее двух лет;

- о наличии (об отсутствии) судимости;

\pagebreak
9) анкеты кандидатов на должности единоличного исполнительного органа и главного бухгалтера небанковской кредитной организации, имеющей право на осуществление переводов денежных средств без открытия банковских счетов и связанных с ними иных банковских операций. 

Сведения:

- о наличии у этих лиц высшего образования;

- о наличии (об отсутствии) судимости;

\pagebreak
10) документы, необходимые для оценки деловой репутации учредителей (участников) кредитных организаций, кандидатов в члены совета директоров (наблюдательного совета) кредитной организации, лица, осуществляющего функции единоличного исполнительного органа юридического лица - учредителя (участника) кредитной организации, приобретающего более 10 процентов акций (долей) кредитной организации.

\end{frame}
\begin{frame}
Центральный банк Российской Федерации \textbf{самостоятельно }запрашивает в органах власти:

- сведения о государственной регистрации учредителей юридических лиц;

- сведения о выполнении учредителями - юридическими лицами обязательств перед федеральным бюджетом, бюджетами субъектов Российской Федерации и местными бюджетами за последние три года. 
\end{frame}
\begin{frame}[allowframebreaks]{\setfontsize{16pt} Порядок государственной регистрации кредитной организации}

- Принятие решения о государственной регистрации банка в срок 6 месяцев.

- Небанковской кредитной организации в срок 3 месяцев.

- Банк России после принятия решения о государственной регистрации кредитной организации направляет в уполномоченный регистрирующий орган сведения и документы, необходимые для регистрации кредитной организации как юридического лица.
\pagebreak
- После регистрации учредители кредитной организации обязаны в течение месяца оплатить 100% уставного капитала.

- Для оплаты уставного капитала Банк России открывает зарегистрированному банку или  небанковской кредитной организации корреспондентский счет в Банке России. 

- При предъявлении документов, подтверждающих оплату 100 процентов объявленного уставного капитала кредитной организации, Банк России в трехдневный срок выдает кредитной организации лицензию на осуществление банковских операций.
\end{frame}

\subsection{Отказ в госрегистрации}
\begin{frame}[allowframebreaks]{\setfontsize{14pt} Основания для отказа в государственной регистрации кредитной организации}
1) несоответствие кандидата, предлагаемого на должность руководителя кредитной организации, главного бухгалтера кредитной организации или его заместителя (далее также - кандидат) квалификационным требованиям и требованиям к деловой репутации.

2) неудовлетворительное финансовое положение учредителей кредитной организации или неисполнение ими своих обязательств перед федеральным бюджетом, бюджетами субъектов Российской Федерации и местными бюджетами за последние три года;

\pagebreak
3) несоответствие документов, поданных в Банк России для государственной регистрации кредитной организации и получения лицензии на осуществление банковских операций, требованиям федеральных законов и принимаемых в соответствии с ними нормативных актов Банка России;

4) несоответствие кандидата на должность члена совета директоров (наблюдательного совета) кредитной организации требованиям к деловой репутации;

\pagebreak
5) неудовлетворительная деловая репутация учредителя (участника) кредитной организации, приобретающего более 10 процентов акций (долей) кредитной организации. 

6) несоответствие лица, осуществляющего функции единоличного исполнительного органа юридического лица - учредителя (участника) кредитной организации, приобретающего более 10 процентов акций (долей) кредитной организации, требованиям к деловой репутации.

\pagebreak
Решение об отказе в государственной регистрации кредитной организации и выдаче ей лицензии на осуществление банковских операций сообщается учредителям кредитной организации в письменной форме и должно быть мотивировано.

Решение об отказе в государственной регистрации кредитной может быть обжаловано в арбитражном суде. 

\end{frame}

\subsection{Отзыв лицензии}
\begin{frame}[allowframebreaks]{Основания для отзыва \\ банковской лицензии}{Банк России \textbf{может} отозвать лицензию}
1) установления недостоверности сведений, на основании которых выдана указанная лицензия;

2) задержки начала осуществления банковских операций, предусмотренных этой лицензией, более чем на один год со дня ее выдачи;

3) установления фактов существенной недостоверности отчетных данных;

4) задержки более чем на 15 дней представления ежемесячной отчетности (отчетной документации);

\pagebreak
5) осуществления, в том числе однократного, банковских операций, не предусмотренных указанной лицензией;

6) неисполнения федеральных законов, регулирующих банковскую деятельность.

7) неоднократного в течение одного года виновного неисполнения содержащихся в исполнительных документах судов, арбитражных судов требований о взыскании денежных средств со счетов (с вкладов) клиентов кредитной организации при наличии денежных средств на счетах (во вкладах) указанных лиц;

8) наличия ходатайства временной администрации;
\pagebreak

9) неоднократного непредставления сведений для внесения изменений в единый государственный реестр юридических лиц;

10) неисполнение кредитной организацией, являющейся управляющим ипотечным покрытием, требований Федерального закона "Об ипотечных ценных бумагах";

11) неоднократного нарушения в течение одного года требований Федерального закона "О противодействии неправомерному использованию инсайдерской информации и манипулированию рынком и о внесении изменений в отдельные законодательные акты Российской Федерации".

\end{frame}

\begin{frame}[allowframebreaks]{Основания для отзыва\\ банковской лицензии}{Банк России \textbf{обязан} отозвать лицензию}
1) если достаточность капитала кредитной организации становится ниже 2 процентов.

2) если размер собственных средств (капитала) кредитной организации ниже минимального значения уставного капитала, установленного на дату государственной регистрации кредитной организации. Указанное основание для отзыва лицензии на осуществление банковских операций не применяется к кредитным организациям в течение первых двух лет со дня выдачи лицензии на осуществление банковских операций;

\pagebreak
3) если кредитная организация не исполняет в срок, установленный Федеральным законом "О несостоятельности (банкротстве) кредитных организаций", требования Банка России о приведении в соответствие величины уставного капитала и размера собственных средств (капитала);

4) если кредитная организация не способна удовлетворить требования кредиторов по денежным обязательствам и (или) исполнить обязанность по уплате обязательных платежей в течение 14 дней с наступления даты их удовлетворения и (или) исполнения на сумму не менее 1000-кратного размера минимального размера оплаты труда.
\end{frame}
\subsection{Контрольные вопросы}
\begin{frame}{Контрольные вопросы}
1. Этапы создания банка и получения лицензии на осуществление банковских операций. Перечень необходимых документов для получения банковской лицензии.

2. Основания для отказа в государственной регистрации банка.

3. Основания для отзыва лицензии банка.

4. Виды лицензий, выдаваемых Банком России: лицензия банка, генеральная лицензия банка, лицензия небанковской кредитной организации.
\end{frame}

\setbeamercovered{transparent}
\end{document}