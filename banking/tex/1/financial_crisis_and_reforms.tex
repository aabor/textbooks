% !TeX program = lualatex -synctex=1 -interaction=nonstopmode %.tex

\documentclass[_Banking_p1.tex]{subfiles}
\begin{document}

\setbeamercovered{invisible}

\subsection{Нормативные документы}
\begin{frame}[allowframebreaks]{\setfontsize{16pt} Финансовые кризисы и реформирование мировой финансовой системы}{Нормативные документы}
  \begin{thebibliography}{10}
  
  \beamertemplatearticlebibitems

    \bibitem{bb6}
Инструкция ЦБ РФ «Об обязательных нормативах банков» от 16.01.2004 № 110-И (ред. от 29.06.2016).

    \bibitem{bb6}
Федеральный закон от 26.10.2002 N 127-ФЗ (ред. от 29.07.2017) "О несостоятельности (банкротстве)"

\pagebreak

    \bibitem{bb2}
Федеральный закон от 07.08.2001 N 115-ФЗ (ред. от 29.07.2017) "О противодействии легализации (отмыванию) доходов, полученных преступным путем, и финансированию терроризма"

    \bibitem{bb6}
Федеральный закон от 18.07.2009 N 181-ФЗ (ред. от 22.12.2014) "Об использовании государственных ценных бумаг Российской Федерации для повышения капитализации банков"

  \end{thebibliography}

\end{frame}

\subsection{Определения}
\begin{frame} [ allowframebreaks]{Определения}
\begin{block}{Финансовый кризис }
\quad
- это событие, когда держатели краткосрочных долговых обязательств, выпущенных финансовыми посредниками, в массовом порядке изымают свои деньги или отказываются возобновить выданные кредиты. Финансовым кризисом считается также неминуемая угроза подобного массового бегства, предотвращенная прямым или косвенным вмешательством правительства или ожиданием такого вмешательства.
\end{block}
\pagebreak

\textit{Неспособность платить, возникшая в результате неожиданных и непредвиденных обстоятельств, не может быть сама по себе доказательством несостоятельности или банкротства.}

\end{frame}


\subsection{Модель банка}
\begin{frame} [ allowframebreaks]{\setfontsize{12pt}Изменение традиционной модели ведения банковского бизнеса}
\begin{block}{Правило 3-6-3}
\quad
- платить 3\% по депозитам, ссужать деньги под 6\% и начинать игру в гольф в 3 часа после обеда.
\end{block}
\pagebreak
Традиционные банки имели заинтересованность саморегулироваться, осваивать управление риском. 

Страхование банковских депозитов не создавало проблем с мотивацией банков к осмотрительному ведению бизнеса. 

\pagebreak
Однако конкуренция со стороны бросовых облигаций и фондов денежного рынка, дерегулирование рынка финансовых посредников и определение потолков процентных ставок привело к снижению ценности банковской лицензии, что в свою очередь заставило банки увеличивать риск и снижать размер капитала.

\end{frame}

\begin{frame}{Фонд денежного рынка}
Обычно инвестирует в высококачественные, почти безрисковые ценные бумаги, такие как государственные облигации, банковские депозитные сертификаты, коммерческие ценные бумаги. 

Фонды денежного рынка по своей природе похожи на банки. Инвесторы (вкладчики) фондов денежного рынка имеют долю в фонде, которой могут распоряжаться посредством денежных переводов. Т.е. вклад в фонд денежного рынка во многом функционирует как банковский депозит, по которому начисляются проценты.
\end{frame}

\begin{frame}[ allowframebreaks ]{Операции РЕПО}
В операциях РЕПО пенсионный фонд, например, размещает у финансового посредника 100 млн. долл. овернайт. Пенсионный фонд получит процент по своему депозиту в размере, скажем, 3\%. 

Для того, чтобы обезопасить этот депозит, финансовый посредник предоставляет пенсионному фонду залог, который переходит к нему на хранение. 

\pagebreak
Залог, которым может быть обеспеченная активами ценная бумага, приносит доход, скажем 6\% для финансового посредника. Таким образом, финансовый посредник занимает под 3\% и получает 6\%, также как и традиционный банк, только в новой форме.
\end{frame}

\begin{frame}{Нужны ли финансовые кризисы?}
Экономисты Romain Ranciere, Aaron Tornell, and Frank Westermann (2008) обнаружили, что страны, которые иногда испытывали финансовые кризисы имеют тенденцию расти быстрее, чем те страны, которые не испытывали финансовых кризисов. Их анализ был проведен как по развитым, так и по развивающимся странам.

Если перед политиками, преследующими цель увеличения ВВП, встает вопрос, можно ли допустить периодические кризисы или нет, то им следует выбрать наличие кризисов, а не слабый рост экономики.

\end{frame}

\begin{frame}[ allowframebreaks ]{\setfontsize{12pt}Взаимосвязь между банковским капиталом и банкротствами}
Практически нет доказательств связи между банкротством банков  и размером их капитала. 

Банки, обанкротившиеся за недавнее время, в среднем имели больше капитала, чем того требуют правила Базель III.

\pagebreak
Улучшение методологии принятия решений о выдаче кредитов, управление страхованием и риском, информационные технологии, совершенствование банковского регулирования, привело к тому, что банки постепенно уменьшали размер капитала, который им необходим для совершения операций.

\end{frame}

\subsection{Финансовый кризис 2008-2009 гг.}
\begin{frame}{\setfontsize{12pt} Причины мирового финансового кризиса 2008-2009 гг.}
\begin{itemize}[<+->]
\item
Достигнут предел расширения рынков сбыта и «долларовой зоны». 

\item
Высокие темпы экономического роста в мире с начала 2000-х годов на фоне глубоких дисбалансов в сфере  сбережений и накопления

\item
Отрицательные реальные процентные ставки в развитых странах в течение большей части 2000-х годов

\item
Огромный дефицит счета текущих операций и бюджета в США.
\end{itemize}

\end{frame}
\begin{frame}{\setfontsize{12pt}Причины мирового финансового кризиса 2008-2009 гг.}
\begin{itemize}[<+->]
\item
Общая тенденция к дерегулированию и либерализации финансового.

\item
Обеспечение высокой прибыли инвестиционных банков и фондов за счет использования инновационных производных финансовых инструментов.

\item
Глобализация как основная тенденция развития современной мировой экономики. 

\end{itemize}

\end{frame}

\begin{frame}[shrink=25]{}
% Table generated by Excel2LaTeX from sheet 'Лист1'
\begin{table}[htbp]
	\caption{Антикризисные расходы\\ федерального правительства России, млрд. руб.}
  	\centering
\begin{tabularx}{\linewidth}[b]{@{}>{\raggedright\arraybackslash}Xr@{}}    \toprule
    \multicolumn{1}{c}{Направление} & Сумма \\
    \midrule
    Беззалоговые кредиты банкам* & 1924 \\
    Размещение средств федерального бюджета на счетах банков** & 687 \\
	Поддержка фондового рынка через ВЭБ*** & 175 \\
    Поддержка сельского хозяйства & 300 \\
	Закупка транспорта для государственных нужд & 100 \\
	Рефинансирование долгов компаний через ВЭБ**** & 11,6 \\
	Субсидии реальному сектору на проценты по кредитам & 18,4 \\
	Поддержка малого бизнеса & 6,2 \\
	Поддержка предприятий оборонной промышленности & 70 \\
    \bottomrule
    \end{tabularx}%
  \label{tab:addlabel}%
  
  \raggedright
\scriptsize{
	*  - не погашен 61~млрд.~руб.;
	
	** - полностью возвращены c доходом 19,2 млрд. руб.;
	
	*** - возвращены; 
	
	**** - возвращено 3,9 млрд. руб.
	\par}
\end{table}%

\end{frame}

\subsection{Реформы}
\subsubsection{Финансового регулирования}
\begin{frame}{\setfontsize{12pt} Основные направления реформ в области финансового регулирования}
\begin{itemize}[<+->]
\item
Прозрачная финансовая система.

\item
Снижение уровня задолженности и повышение устойчивости финансовых институтов.

\item
Соблюдение баланса между глобализацией и возможным «эффектом домино» между элементами финансовой системы.
\end{itemize}

\end{frame}
\begin{frame}{\setfontsize{12pt} Основные направления реформ в области финансового регулирования}
\begin{itemize}[<+->]
\item
Размер резервного капитала и резервов ликвидности должны соответствовать уровню системного риска финансовых институтов.

\item
Особый план действий в отношении системно значимых институтов, испытывающих финансовые проблемы.
\end{itemize}

\end{frame}

\begin{frame}{\setfontsize{14pt} Международные организации, занимающиеся реформами финансовой системы}
\begin{itemize}[<+->]
\item
Совет по финансовой стабильности (СФС (англ. Financial Stability Board) – международная организация, созданная странами Большой индустриальной двадцатки на Лондонском саммите в апреле 2009 года.

\item
Базельский комитет по банковскому надзору (БКБН) основан при Банке международных расчётов (англ. Committee on Banking Supervision of the Bank for international Settlements) в г. Базель, Швейцария в 1974 году президентами центральных банков стран «группы десяти» (G10).
\end{itemize}

\end{frame}

\begin{frame}{\setfontsize{12pt} Международные организации, занимающиеся реформами финансовой системы}
\begin{itemize}[<+->]
\item
Офшорная группа банковских надзорных органов (англ. Offshore Group of Banking Supervisors (OGBS)) создана в октябре 1980 года по инициативе Базельского комитета по банковскому надзору – ассоциация государственных органов банковского надзора стран, на территории которых действуют офшорные финансовые центры.
\end{itemize}

\end{frame}
\begin{frame}{\setfontsize{14pt} Международные организации, занимающиеся реформами финансовой системы}
\begin{itemize}[<+->]

\item
Международный валютный фонд, МВФ (англ. International Monetary Fund, IMF) – специализированное учреждение ООН, со штаб-квартирой в Вашингтоне, США, создан на Бреттон-Вудской конференции ООН по валютно-финансовым вопросам 22 июля 1944 года.


\end{itemize}

\end{frame}

\subsubsection{Банки}
\begin{frame}{\setfontsize{12pt}Реформы в области регулирования банковской сферы}
\begin{itemize}[<+->]

\item
Стандарты банковского капитала (соглашение Базель III).

\item
Стандарты достаточности банковского капитала для покрытия рисков (соглашение Базель III).

\item
Дополнительные отчисления в резервы для глобальных системно значимых финансовых банков.

\item
Требования к уровню ликвидности (соглашение Базель III).
\end{itemize}

\end{frame}

\begin{frame}{\setfontsize{12pt}Реформы в области регулирования банковской сферы}
\begin{itemize}[<+->]
\item
Соотношение собственных и заемных средств (соглашение Базель III).

\item
Нормативы СФС для финансовых институтов по определению уровня заработной платы и бонусов.

\item
Принципы корпоративного управления, разработанные БКБН.
\end{itemize}

\end{frame}
\begin{frame}{Реформы в области регулирования банковской сферы}
\begin{itemize}[<+->]
\item
Правило Волкера (Закон Додда – Франка, США).

\item
Доклад Виккерса (Великобритания).
\end{itemize}

\end{frame}


\begin{frame} {\setfontsize{12pt} Изменения в стандартах достаточности банковского капитала}
\begin{itemize}[<+->]

\item
Улучшенная процедура оценки рисков.

\item
Показатель изменения риска портфеля при изменении его структуры.

\item
Более высокий уровень отчислений в резервы при осуществлении сделок с деривативами и операций РЕПО.

\end{itemize}

\end{frame}
\begin{frame} {\setfontsize{12pt} Изменения в стандартах достаточности банковского капитала}
\begin{itemize}[<+->]
\item
Дополнительный резервный капитал (консервационный и контрциклический резервы).

\item
Дополнительные отчисления в резервы для глобальных системно значимых финансовых институтов.

\item
Отчисления в резервы, уровень которых определяется степенью риска дефолта основного контрагента.

\end{itemize}

\end{frame}

\begin{frame} [ allowframebreaks]{\setfontsize{12pt}Требования к уровню ликвидности (соглашение Базель III)}
\begin{itemize}[<+->]
\item
Показатель краткосрочной ликвидности: необходимо наличие ликвидных активов высокого качества, достаточных для функционирования в течение 30 дней в случае оттока средств клиентов.

\item
Показатель чистого стабильного финансирования: требование к повышению уровня соответствия срочности активов и обязательств.

\end{itemize}

\end{frame}


\begin{frame}{\setfontsize{12pt} Соотношение собственных и заемных средств (соглашение Базель III)}
пороговое значение для отношения суммы по открытым позициям (независимо от степени риска) к капиталу.
\end{frame}

\begin{frame}[allowframebreaks]{\setfontsize{12pt} Нормативы СФС для финансовых институтов по определению уровня заработной платы и бонусов}
\begin{itemize}
\item
Устанавливает ответственность правления за политику в области определения уровня зарплаты и бонусов.

\item
Уровень вознаграждения должен определяться с учетом рисков и горизонта планирования.

\pagebreak
\item
Надзорным органам следует осуществлять мониторинг порядка определения уровня заработной платы и бонусов.

\end{itemize}

\end{frame}
\begin{frame}[allowframebreaks]{Прочие реформы в области банковской сферы}
\begin{itemize}
\item
Принципы корпоративного управления, разработанные БКБН (приоритет ответственного корпоративного управления, обеспечивающего устойчивое развитие банка).

\item
Правило Волкера (Закон Додда – Франка, США) (ограничивает участие депозитных финансовых институтов в торговой деятельности, владении частным капиталом, создании собственных хедж-фондов)

\pagebreak
\item
Доклад Виккерса (Великобритания) (запрет на инвестиционную деятельность и более строгие стандарты достаточности капитала для банков, обслуживающих население).

\end{itemize}

\end{frame}

\subsubsection{Финансовые рынки}
\begin{frame} {Реформы в области регулирования финансовых рынков}{Внебиржевые рынки деривативов}
\begin{itemize}[<+->]

\item
Стандартизация контрактов. 

\item
Осуществление расчетов по стандартизованным контрактам через центрального контрагента.

\item
Торговля стандартизованными контрактами на биржах или торговых платформах.
\end{itemize}

\end{frame}

\begin{frame} {Реформы в области регулирования финансовых рынков}{Внебиржевые рынки деривативов}
\begin{itemize}[<+->]
\item
Фиксация контракта в репозитарии.

\item
Более высокие стандарты достаточности капитала и залога при торговле производными финансовыми инструментами, клиринг которых осуществляется не через центрального контрагента.
\end{itemize}

\end{frame}


\subsubsection{Прочие}
\begin{frame}{\setfontsize{16pt} Реформы в области регулирования небанковских финансовых институтов}
\begin{itemize}[<+->]
\item
Надзор в сфере теневой банковской системы и оценка связанных с ней рисков.

\item
Регистрация хедж-фондов, повышенные стандарты для секьюритизации.

\item
Улучшение системы косвенного регулирования (ограничения взаимодействия банковских и небанковских институтов), разработка более строгих правил оценки и определения уровня ликвидности фондов денежного рынка, стандартов для сделок РЕПО и кредитования под залог ценных бумаг.

\end{itemize}

\end{frame}

\begin{frame}{Реформы рейтинговых агентств}
\begin{itemize}[<+->]
\item
Регистрация и регулирование кредитно-рейтинговых агентств; регулирование включает более полное раскрытие информации о применяемых методах рейтинговой оценки и предоставление исходных данных.

\item
Сокращение зависимости регулирующих органов от использования рейтингов; так, в США эта инициатива привела к исключению ссылок на кредитные рейтинги в законах и нормативах.

\end{itemize}

\end{frame}

\subsection{Финансовые соглашения}
\begin{frame}{Региональные финансовые соглашения}
\begin{itemize}[<+->]

\item
соглашения о свободной торговле (со 2 Мировой Войны)

\item
Глобальные и региональные объединений по обеспечению стабильного функционирования финансовой системы (с июня 2012)

\end{itemize}
Проблема – отсутствие кредитора последней инстанции, действующего на глобальном уровне.

\end{frame}

\begin{frame}{Преимущества региональных финансовых соглашений}
\begin{itemize}[<+->]

\item
более дёшевы за счет наличия своп-соглашений между странами или более высокого кредитного рейтинга у регионального союза.

\item
Маленькие страны получают доступ к финансовой помощи при возникновении трудностей.

\item
Большие страны имеют возможность предотвратить «эффект домино» и укрепить финансовую стабильность во всем регионе.
\end{itemize}

\end{frame}

\begin{frame}{Наиболее значимые региональные финансовые механизмы}
\begin{itemize}[<+->]

\item
Европейский стабилизационный механизм, ЕСМ (ESM, European Stability Mechanism);

\item
Многосторонняя инициатива Чианг Май, МИЧМ (ChiangMai Initiative Multilateralisation);

\item
Антикризисный фонд Евразийского экономического сообщества, АКФ;

\item
Латиноамериканский резервный фонд, ЛРФ (FLAR, Latin American Reserve Fund);

\item
Арабский валютный фонд, АВФ (Arab Monetary Fund).
\end{itemize}

\end{frame}

\begin{frame}
\begin{block}{Евротройка}
Европейская комиссия (ЕК)

Европейский центральный банк (ЕЦБ) 

Международный валютный фонд (МВФ).
\end{block}
\end{frame}

\subsection{Контрольные вопросы}
\begin{frame}{Контрольные вопросы}
1. Понятие финансового кризиса, определяемое экономистами.

2. Традиционная модель ведения банковского бизнеса и её изменение в связи с распространением новых видов ценных бумаг на финансовых рынках.

3. Причины и история Мирового финансового кризиса 2008-нач.2009 гг.

4. Новые механизмы финансового регулирования, принятые на мировом уровне в ответ на Мировой финансовый кризис 2008-нач.2009 гг.
\end{frame}
\setbeamercovered{transparent}

\end{document}
