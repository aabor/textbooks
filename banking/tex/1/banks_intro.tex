\documentclass[_Banking_p1.tex]{subfiles}

\begin{document}

\setbeamercovered{transparent}

\subsection{Происхождение банков}
\begin{frame}{Происхождение банков}{}
\begin{itemize}[<+->]
\item
Первые банки возникли в условиях ману­фактурной стадии капитализма и появились прежде всего в отдельных итальянских городах (Венеции, Генуе) в XIV и XV вв. 
\item
Банки возникли в более ран­ний период - при феодализме. 
\end{itemize}
\end{frame}

\begin{frame}{Происхождение слова банк}
``banco'' (ит.) ``стол''

Менялы: 

в Древнем Риме – менсарии (от латинского слова ``mensa'', означающего стол);

в Древней Греции – трапезидами  (от греческого ``трапеза'', означающего ``стол'').

\end{frame}
\begin{frame}
В современном виде банковское дело появилось в богатых итальянских городах в эпоху Возрождения (начиная с 14 века): Флоренции, Сьене, Венеции и Генуе.

Современные банковские приемы ведения дел, включая частичное обеспечение депозитов и выпуск банкнот возникли в 17-18 веках. К этому моменту относится возникновение центральных банков в Европе.

Первым банком, начавшим выпускать банкноты на постоянной основе был Банк Англии с 1695 г.

\end{frame}

\subsection{Характеристика банка как предприятия}
\begin{frame}{Характеристика банка как предприятия}{}
\begin{block}{\textbf{С экономической точки зрения банк }}
\quad
– это предприятие, производящее специфический продукт – платежные средства.
\end{block}

\begin{block}{\textbf{Банк как специфическое предприятие }}
\quad
производит продукт, существенно отличающийся от продукта сферы материального производства, он производит не просто товар, а товар особого рода в виде денег, платежных средств.
\end{block}

\end{frame}

\begin{frame}{Особенность банковского кредита}
Особенность банковского кредита состоит в том, что он предоставляется не как некая сумма денег, а как капитал.

Под словом «капитал» понимается богатство, которое может быть использовано для получения еще большего богатства.
\end{frame}

\begin{frame}{Отличие банковского кредита от ростовщического}
банк работает в основном на чужих деньгах, привлекаемых на возвратной основе. Ростовщик же ссужает собственные деньги.
\end{frame}

\begin{frame}{Определение процента за кредит}
Определение точки безубыточности по кредиту на сумму \textit{L }с вероятностью возврата \textit{p}: 
\begin{align}
L=L\times (1+r_L)\times p + 0 \times (1-p)
\end{align}
\end{frame}

\begin{frame}{Вычисление кредитного риска}
Пусть портфель состоит из \textit{n} идентичных и независимых кредитов, каждый из которых получает ставку процента $r_L$.
Если $\sigma_1$ – это стандартное отклонение единичного проекта инвестиций и $\sigma_n$ - это стандартное отклонение \textit{n} проектов инвестиций, тогда:
\begin{align}
\sigma_n=\frac{\sigma_1}{\sqrt{n}}
\end{align}
\end{frame}

\begin{frame}[allowframebreaks]{Совокупный ожидаемый доход для банка по кредитному портфелю}
Процентный доход от каждого платящего равен $r_L \times \frac{L}{n}$. Существует вероятность \textit{p}, что каждый из \textit{n} кредитов не будет выплачен. Если кредит не выплачен, вся сумма кредита $\frac{L}{n}$ потеряна и становится стоимостью потерь от кредитов за период. 

\pagebreak
При этих допусках, ожидаемый доход от каждого из кредитов описывается следующим выражением:
\begin{align}
E\left(\frac{\text{доход}}{\text{кредит}}\right)=\frac{L}{n}\times \left((1+r_L)\times p -1\right)
\end{align}

Поскольку существуют \textit{n} таких кредитов, совокупный ожидаемый доход для банка равен:
\begin{align}
E(\text{доход}) = L \times \left((1 + r_L) \times p - 1)\right)
\end{align}
\end{frame}

\begin{frame}{Вычисление рисковой надбавки}
Риск по портфелю кредитов измеряется стандартным отклонением ожидаемых доходов $\sigma_n$. Для его вычисления необходимо рассчитать риск, связанный с приведением всех банковских активов в один кредит:
\begin{align}
\sigma_1&=L\cdot (1+r_L) \cdot \sqrt{p \cdot(1-p)}\\
\sigma_n&=\frac{\sigma_1}{\sqrt{n}}=L\cdot \frac{1+r_L}{\sqrt{n}} \cdot \sqrt{p \cdot(1-p)}
\end{align}
\end{frame}
\subsection{Основные определения}
\begin{frame}[allowframebreaks]{Основные определения}{}
\begin{block}{Кредитная организация }
\quad
- юридическое лицо, которое для извлечения прибыли как основной цели своей деятельности на основании специального разрешения (лицензии) Центрального банка Российской Федерации (Банка России) имеет право осуществлять банковские операции, предусмотренные законом о банках и банковской деятельности. Кредитная организация образуется на основе любой формы собственности как хозяйственное общество.
\end{block}
\pagebreak

\begin{block}{Банк }
\quad
- кредитная организация, которая имеет исключительное право осуществлять в совокупности следующие банковские операции: привлечение во вклады денежных средств физических и юридических лиц, размещение указанных средств от своего имени и за свой счет на условиях возвратности, платности, срочности, открытие и ведение банковских счетов физических и юридических лиц.

\end{block}
\pagebreak

\begin{block}{Небанковская кредитная организация}
\quad
- кредитная организация, имеющая право осуществлять отдельные банковские операции, предусмотренные настоящим Федеральным законом. Допустимые сочетания банковских операций для небанковских кредитных организаций устанавливаются Банком России.
\end{block}
\pagebreak

\begin{block}{Иностранный банк}
\quad
- банк, признанный таковым по законодательству иностранного государства, на территории которого он зарегистрирован
\end{block}
\pagebreak

\begin{block}{Банковская система Российской Федерации}
\quad
включает в себя Банк России, кредитные организации, а также представительства иностранных банков.

\end{block}
\pagebreak
\begin{block}{Филиалом кредитной организации}
\quad
 является ее обособленное подразделение, расположенное вне места нахождения кредитной организации и осуществляющее от ее имени все или часть банковских операций, предусмотренных лицензией Банка России, выданной кредитной организации.
\end{block}

\pagebreak
\begin{block}{Представительством кредитной организации}
\quad
 является ее обособленное подразделение, расположенное вне места нахождения кредитной организации, представляющее ее интересы и осуществляющее их защиту. Представительство кредитной организации не имеет права осуществлять банковские операции.
\end{block}

\pagebreak
Филиалы и представительства кредитной организации не являются юридическими лицами и осуществляют свою деятельность на основании положений, утверждаемых создавшей их кредитной организацией.

\pagebreak
Кредитные организации могут создавать союзы и ассоциации, не преследующие цели извлечения прибыли, для защиты и представления интересов своих членов, координации их деятельности, развития межрегиональных и международных связей, удовлетворения научных, информационных и профессиональных интересов, выработки рекомендаций по осуществлению банковской деятельности и решению иных совместных задач кредитных организаций.

\pagebreak
\begin{block}{Банковской группой}
\quad
 признается не являющееся юридическим лицом объединение кредитных организаций, в котором одна (головная) кредитная организация оказывает прямо или косвенно (через третье лицо) существенное влияние на решения, принимаемые органами управления другой (других) кредитной организации (кредитных организаций).
\end{block}
\pagebreak
\begin{block}{Банковским холдингом}
\quad
 признается не являющееся юридическим лицом объединение юридических лиц с участием кредитной организации (кредитных организаций), в котором юридическое лицо, не являющееся кредитной организацией (головная организация банковского холдинга), имеет возможность прямо или косвенно (через третье лицо) оказывать существенное влияние на решения, принимаемые органами управления кредитной организации (кредитных организаций).
\end{block}

\end{frame}

\subsection{Банковские операции и другие сделки кредитной организации}
\begin{frame}[allowframebreaks]{Банковские операции и другие сделки кредитной организации}{Банковские операции}
1) привлечение денежных средств физических и юридических лиц во вклады (до востребования и на определенный срок);

2) размещение привлеченных средств от своего имени и за свой счет;

3) открытие и ведение банковских счетов физических и юридических лиц;

4) осуществление переводов денежных средств по поручению физических и юридических лиц, в том числе банков-корреспондентов, по их банковским счетам;

\pagebreak
5) инкассация денежных средств, векселей, платежных и расчетных документов и кассовое обслуживание физических и юридических лиц;

6) купля-продажа иностранной валюты в наличной и безналичной формах;

7) привлечение во вклады и размещение драгоценных металлов;

8) выдача банковских гарантий;

9) осуществление переводов денежных средств без открытия банковских счетов, в том числе электронных денежных средств (за исключением почтовых переводов).
\end{frame}

\begin{frame}[allowframebreaks]{Банковские операции и другие сделки кредитной организации}{Другие сделки кредитных организаций}
1) выдачу поручительств за третьих лиц, предусматривающих исполнение обязательств в денежной форме;

2) приобретение права требования от третьих лиц исполнения обязательств в денежной форме;

3) доверительное управление денежными средствами и иным имуществом по договору с физическими и юридическими лицами;

4) осуществление операций с драгоценными металлами и драгоценными камнями в соответствии с законодательством Российской Федерации;

\pagebreak
5) предоставление в аренду физическим и юридическим лицам специальных помещений или находящихся в них сейфов для хранения документов и ценностей;

6) лизинговые операции;

7) оказание консультационных и информационных услуг.



\textbf{Кредитной организации запрещается заниматься производственной, торговой и страховой деятельностью.}

\end{frame}

\begin{frame}{Операции банков с ценными бумагами}
Банк вправе осуществлять выпуск, покупку, продажу, учет, хранение и иные операции с ценными бумагами, выполняющими функции платежного документа, с ценными бумагами, подтверждающими привлечение денежных средств во вклады и на банковские счета, с иными ценными бумагами, осуществление операций с которыми не требует получения специальной лицензии в соответствии с федеральными законами, а также вправе осуществлять доверительное управление указанными ценными бумагами по договору с физическими и юридическими лицами.

\end{frame}
\subsection{Контрольные вопросы}
\begin{frame}{Контрольные вопросы}
1. Происхождение банков.

2. Характеристика банка как предприятия.

3. Юридические определения кредитной организации, банка, иностранного банка, небанковской кредитной организации.

4. Понятие филиалов и представительств кредитной организации. Банковские группы и холдинги.

5. Перечень разрешенных в России банковских операций и других сделок кредитной организации. Операции банков с ценными бумагами.

\end{frame}

\setbeamercovered{transparent}
\end{document}