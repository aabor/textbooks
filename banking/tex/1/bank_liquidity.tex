\documentclass[_Banking_p2.tex]{subfiles}
\begin{document}

\setbeamercovered{invisible}

\subsection{Нормативные документы}

\begin{frame}[allowframebreaks]{Нормативные документы}
  \begin{thebibliography}{10}
  
%  \beamertemplatearticlebibitems
    \bibitem{bb1}

Федеральный закон от 10.07.2002 N 86-ФЗ (ред. от 18.07.2017) "О Центральном банке Российской Федерации (Банке России)"

    \bibitem{bb2}
Федеральный закон от 02.12.1990 N 395-1 (ред. от 26.07.2017) "О банках и банковской деятельности"

\pagebreak

    \bibitem{bb3}
Инструкция ЦБ РФ «Об обязательных нормативах банков» от 16.01.2004 № 110-И (ред. от 29.06.2016).

    \bibitem{bbBaselIII}
"Положение о методике определения величины собственных средств (капитала) кредитных организаций ("Базель III")" (утв. Банком России 28.12.2012 N 395-П) (ред. от 04.08.2016)

    \bibitem{bb4}
Федеральный закон от 26.10.2002 N 127-ФЗ (ред. от 29.07.2017) "О несостоятельности (банкротстве)"

\pagebreak

    \bibitem{bb5}
Федеральный закон от 07.08.2001 N 115-ФЗ (ред. от 29.07.2017) "О противодействии легализации (отмыванию) доходов, полученных преступным путем, и финансированию терроризма"

    \bibitem{bb6}
Федеральный закон от 18.07.2009 N 181-ФЗ (ред. от 22.12.2014) "Об использовании государственных ценных бумаг Российской Федерации для повышения капитализации банков"

  \end{thebibliography}
\end{frame}


\subsection{Понятие ликвидности и платежеспособности}
\begin{frame}[shrink=10]{Понятие ликвидности и платежеспособности}{}
\begin{block}{Ликвидность банка }
\quad
– способность своевременно и без потерь выполнять свои обязательства перед вкладчиками и кредиторами.
\end{block}
\end{frame}

\begin{frame}[shrink=15]{Обязательства банка}
\begin{itemize}[<+->]
\item
\textbf{Реальные обязательства банка }– это депозиты до востребования, срочные депозиты, привлеченные межбанковские ре­сурсы, различные займы. 
\item
\textbf{Потенциальные обязательства }– гарантии и поручительства выданные банком (пассивные забалансовые операции банка), а также неиспользованные кредитные линии, выставленные аккредитивы и др. (актив­ные забалансовые операции). 
\item
Помимо этого банку необходимо иметь некоторую величину ликвидных средств для оперативной выдачи ссуд клиентам без открытия кредитных линий.
\end{itemize}
\end{frame}
\begin{frame}{Источники средств для выполнения обязательств банка}
\begin{itemize}[<+->]
\item
денежная наличность банка, т.е. деньги в кассе и на корреспондентских счетах (в ЦБ РФ и других коммерческих банках);
\item
активы, которые можно быстро превратить в наличность (например, ценные бумаги первоклассных эмитентов);
\item
межбанковские  кредиты, получаемые на межбанковском рынке либо в ЦБ РФ.
\end{itemize}
\end{frame}

\begin{frame}
\begin{block}{Платежеспособность банка}
\quad
рассматривается в аспекте выполнения им на конкретную дату всех обязательств, в том числе финансовых, например, перед бюджетом по налогам, перед работниками по заработной плате и т.д. 
\end{block}
\end{frame}
\begin{frame}{Критерии ликвидности и \\платежеспособности банка}
\textbf{Критерием ликвидности банка} является сопря­женность всех его активов и пассивов по срокам и суммам, при возникновении несоответствия - способность обеспечить себя ликвид­ными активами. 

\textbf{Критерием платежеспособности} выступает достаточ­ность на определенную дату средств на корреспондентском счете для выполнения платежей, в том числе из прибыли банка.
\end{frame}

\subsection{Факторы ликвидности банка}
\begin{frame}{Факторы ликвидности банка}{Внутренние факторы}
\begin{itemize}[<+->]
\item
крепкая капитальная база банка;
\item
качество активов банка; 
\item
качество депозитов;
\item
умеренная зависимость от внешних источников;
\item
сопряженность активов и пассивов по срокам;
\item
грамотный менеджмент;
\item
первоклассный имидж банка.
\end{itemize}
\end{frame}

\begin{frame}{Факторы ликвидности банка}{Внешние факторы}
\begin{itemize}[<+->]
\item
Общая политическая и экономическая обстановка в стране.
\item
Развитие рынка ценных бумаг и межбанковского рынка.
\item
Развитие системы рефинансирования банком России коммерческих банков.
\end{itemize}
\end{frame}

\subsection{Обязательные нормативы банков}
\begin{frame}[ allowframebreaks]{Обязательные нормативы банков}{Установленные инструкцией ЦБ РФ \\«Об обязательных нормативах банков»}
\begin{itemize}
\item
достаточности собственных средств (капитала);
\item
ликвидности;
\item
максимального размера риска на одного заемщика или группу связанных заемщиков;
\item
максимального размера крупных кредитных рисков;
\pagebreak
\item
максимального размера кредитов, банковских гарантий и поручительств, предоставленных банком своим участникам (акционерам);
\item
совокупной величины риска по инсайдерам банка;
\item
использования собственных средств (капитала) банков для приобретения акций (долей) других юридических лиц.
\end{itemize}
\end{frame}

\begin{frame}[allowframebreaks]{Обязательные нормативы банков}{Установленные другими документами ЦБ РФ}
\begin{itemize}
\item
минимальный размер уставного капитала для создаваемых кредитных организаций; 
\item
размер собственных средств (капитала) для действующих кредитных организаций в качестве условия создания их дочерних организаций и (или) открытия их филиалов на территории иностранного государства;
\item
получения небанковской кредитной организацией статуса банка;
\pagebreak
\item
получения кредитной организацией статуса дочернего банка;
\item
предельный размер имущественных (неденежных) вкладов в уставный капитал кредитной организации;
\pagebreak
\item
перечень видов имущества в неденежной форме, которое может быть внесено в оплату уставного капитала; 
\item
минимальный размер резервов, создаваемых под риски; 
\item
\pagebreak
размеры валютного и процентного рисков; 
\item
обязательные нормативы для банковских групп и небанковских кредитных организаций.
\end{itemize}
\end{frame}

\begin{frame}[allowframebreaks]{Нормативы достаточности собственных средств (капитала)}
\begin{block}{Норматив достаточности собственных средств}
\quad
 - это отношение величины базового капитала банка, величины основного капитала банка и величины собственных средств (капитала) банка к сумме рискованных активов.
\end{block}
\pagebreak

Сумма рискованных активов банка включает в себя:

- кредитного риска по активам, отраженным на балансовых счетах бухгалтерского учета (активы за вычетом сформированных резервов на возможные потери и резервов на возможные потери по ссудам, ссудной и приравненной к ней задолженности, взвешенные по уровню риска);

- кредитного риска по условным обязательствам кредитного характера;

\pagebreak
- кредитного риска по производным финансовым инструментам;

- величине риска изменения стоимости кредитного требования в результате ухудшения кредитного качества контрагента;

- операционного риска;

- рыночного риска.

\pagebreak
Величина капитала банка определяется согласно документу "Положение о методике определения величины собственных средств (капитала) кредитных организаций ("Базель III")" (утв. Банком России 28.12.2012 N 395-П) (ред. от 04.08.2016).
\end{frame}

\begin{frame}[shrink=15]{Нормативы ликвидности}{Норматив мгновенной ликвидности банка}
\begin{block}
\quad
регулирует (ограничивает) риск потери банком ликвидности в течение одного операционного дня.
\end{block}
\begin{align}
\text{Н2}=\frac{\text{Л}_\text{ам}}{\text{О}_\text{вм}-0.5\times \text{О}_\text{вм*}}\times 100\%\geq15\%
\end{align}

$\text{Л}_\text{ам}$ - высоколиквидные активы, то есть финансовые активы, которые должны быть получены в течение ближайшего календарного дня.

$\text{О}_\text{вм}$ - обязательства (пассивы) по счетам до востребования, по которым вкладчиком и (или) кредитором может быть предъявлено требование об их незамедлительном погашении.

$\text{О}_\text{вм*}$ - минимальный совокупный остаток средств по счетам до востребования.
\end{frame}


\begin{frame}[shrink=15]{Нормативы ликвидности}{Норматив текущей ликвидности банка}
\begin{block}
\quad
регулирует (ограничивает) риск потери банком ликвидности в течение 30 календарных дней.
\end{block}
\begin{align}
\text{Н3}=\frac{\text{Л}_\text{ат}}{\text{О}_\text{вт}-0.5\times \text{О}_\text{вт*}}\times 100\%\geq 50\%
\end{align}

$\text{Л}_\text{ат}$ - ликвидные активы, то есть финансовые активы, которые должны быть получены банком, и (или) могут быть востребованы в течение ближайших 30 календарных дней.

$\text{О}_\text{вт}$ - обязательства (пассивы) по счетам до востребования и обязательства банка перед кредиторами (вкладчиками) сроком исполнения в ближайшие 30 календарных дней.

$\text{О}_\text{вт*}$ - минимальный совокупный остаток средств по счетам обязательств сроком до 30 дней.
\end{frame}

\begin{frame}[shrink=15]{Нормативы ликвидности}{Норматив долгосрочной ликвидности банка}
\begin{block}
\quad
регулирует (ограничивает) риск потери банком ликвидности в результате размещения средств в долгосрочные активы.
\end{block}
\begin{align}
\text{Н4}=\frac{\text{К}_\text{рд}}{\text{К}+\text{ОД}+	0.5\times \text{О*}}\times 100\%\leq 120\%
\end{align}

$\text{К}$ - капитал банка.

$\text{К}_\text{рд}$ - кредитные требования с оставшимся сроком до даты погашения свыше года, а также пролонгированные свыше года, за вычетом сформированного резерва на возможные потери.

$\text{ОД}$ - обязательства банка с оставшимся сроком погашения свыше года, за вычетом субординированных кредитов.

$\text{О*}$ - величина минимального совокупного остатка средств по счетам со сроком исполнения обязательств до года.
\end{frame}

\begin{frame}[shrink=15]{\setfontsize{14pt} Норматив максимального размера риска на одного или группу связанных заемщиков}
\begin{block}
\quad
регулирует (ограничивает) кредитный риск банка в отношении одного заемщика или группы связанных заемщиков.
\end{block}
\begin{align}
\text{Н6}=\frac{\text{К}_\text{рз}}{\text{К}_0}\times 100\%\leq 25\%
\end{align}

$\text{К}$ - капитал банка.

$\text{К}_\text{рз}$ - совокупная сумма кредитных требований банка к заемщику или группе связанных заемщиков за вычетом сформированного резерва на возможные потери.
\end{frame}

\begin{frame}{Норматив максимального размера крупных кредитных рисков}
\begin{block}
\quad
регулирует (ограничивает) совокупную величину крупных кредитных рисков банка. 
\end{block}
\begin{align}
\text{Н7}=\frac{\sum \text{К}_\text{скрi}}{\text{К}_0}\times 100\%\leq 800\%\end{align}

$\text{К}_0$ - капитал банка.

$\text{К}_\text{скрi}$ - \textit{i}-й крупный кредитный риск за вычетом сформированного резерва на возможные потери.
\end{frame}

\begin{frame}[shrink=15]{\setfontsize{14pt} Норматив максимального размера кредитов, гарантий и поручительств, участникам банка}
\begin{block}
\quad
регулирует (ограничивает) кредитный риск банка в отношении участников (акционеров) банка. 
\end{block}
\begin{align}
\text{Н9.1}=\frac{\sum \text{К}_\text{раi}}{\text{К}_0}\times 100\%\leq 50\%\end{align}

$\text{К}_0$ - капитал банка.

$\text{К}_\text{раi}$ - величина \textit{i}-го кредитного требования банка, а также кредитного риска по условным обязательствам кредитного характера, производным финансовым инструментам в отношении участников (акционеров), которые имеют право распоряжаться 5 и более процентами долей (голосующих акций) банка, за вычетом сформированного резерва на возможные потери.
\end{frame}

\begin{frame}[shrink=15]{Норматив совокупной величины риска по инсайдерам банка}
\begin{block}
\quad
определяет максимальное отношение совокупной суммы кредитных требований к инсайдерам к собственным средствам (капиталу) банка. 
\end{block}
\begin{align}
\text{Н10.1}=\frac{\sum \text{К}_\text{рсиi}}{\text{К}_0}\times 100\%\leq 3\%\end{align}

$\text{К}_0$ - капитал банка.

$\text{К}_\text{рсиi}$ - величина \textit{i}-го кредитного требования к инсайдеру банка, кредитного риска по условным обязательствам кредитного характера, производным финансовым инструментам, заключенным с инсайдером, за вычетом сформированного резерва на возможные потери .
\end{frame}

\begin{frame}[shrink=15]{\setfontsize{14pt} Норматив использования капитала банка для приобретения акций других юридических лиц}
\begin{block}
\quad
регулирует (ограничивает) совокупный риск вложений банка в акции (доли) других юридических лиц. 
\end{block}
\begin{align}
\text{Н12}=\frac{\sum \text{К}_\text{инi}}{\text{К}_0}\times 100\%\leq 25\%\end{align}

$\text{К}_0$ - капитал банка.

$\text{К}_\text{инi}$ - величина \textit{i}-й инвестиции банка в акции (доли) других юридических лиц за вычетом сформированного резерва на возможные потери.
\end{frame}
\subsection{Контрольные вопросы}
\begin{frame}{Контрольные вопросы}
1. Понятие ликвидности и платежеспособности банка.

2. Факторы ликвидности банка.

3. Обязательные нормативы банков, установленные инструкцией Банка России «Об обязательных нормативах банков»

4. Обязательные нормативы банков, установленные в документах Банка России, помимо инструкции «Об обязательных нормативах банков».
\end{frame}

\setbeamercovered{transparent}
\end{document}
