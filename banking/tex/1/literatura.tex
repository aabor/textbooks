\documentclass[_Banking_p1.tex]{subfiles}

\begin{document}

\subsection*{Учебники}



\begin{frame}[allowframebreaks]{Основная литература}
    
  \begin{thebibliography}{10}
    
  \beamertemplatebookbibitems
  % Start with overview books.

  \bibitem{AlexeevTavasiev2015}
Н.К.~Алексеев, А.М.~Тавасиев  
\newblock{\em Банковское дело: словарь официальных терминов с комментариями} 
\newblock М.: Издательско-торговая корпорация "Дашков и К", 2015. – 656 с.

\footnotesize{\url{http://znanium.com/catalog.php?bookinfo=513901}}

  \bibitem{Markova2016}
О.М.~Маркова 
\newblock{\em Организация деятельности коммерческого банка: Учебник.} 
\newblock М.: ИД ФОРУМ, НИЦ ИНФРА-М, 2016. - 496 с. 

\footnotesize{\url{http://znanium.com/catalog.php?bookinfo=522039}}

 
  \bibitem{Starodub2015}
  Е.Б.~Стародубцева
\newblock{\em Основы банковского дела: Учебник.} 
\newblock М.: ИД ФОРУМ: НИЦ ИНФРА-М, 2015. - 288 с.

\footnotesize{\url{http://znanium.com/bookread2.php?book=473186 }}
  \pagebreak  

  \bibitem{Ageeva2014}
Н.А.~Агеева
\newblock{\em Основы банковского дела: Учебное пособие.}
\newblock М.: ИЦ РИОР: НИЦ ИНФРА-М, 2014. - 274 с.

\footnotesize{\url{http://znanium.com/bookread2.php?book=437140}}

  \bibitem{Ageeva2014}
С.Ю.~Хасянова
\newblock{\em Кредитный анализ в коммерческом банке : учеб. пособие.}
\newblock М. : ИНФРА-М, 2017. - 196 с.

\footnotesize{\url{http://znanium.com/catalog.php?bookinfo=635227}}

  \end{thebibliography}
\end{frame}

\begin{frame}[ allowframebreaks]{Дополнительная литература}
  \begin{thebibliography}{10}
    
  \beamertemplatebookbibitems
  % Start with overview books.

  \bibitem{Kovalev2013}

\newblock{\em Банковский риск-менеджмент: Учебное пособие.} 
\newblock М.: КУРС: НИЦ ИНФРА-М, 2013. - 320 с.

\footnotesize{\url{http://znanium.com/bookread2.php?book=411068}}

  \bibitem{Smulov2014}
А.М.~Смулов (ред.)
\newblock{\em Управление проблемной банковской задолженностью: Учебник.} 
\newblock М.: НИЦ ИНФРА-М, 2014. - 352 с.

\footnotesize{\url{http://znanium.com/bookread2.php?book=424141}}

\pagebreak

  \bibitem{Gracheva2016}
Е.Ю.~Грачева
\newblock{\em Банковское право Российской Федерации: Учебное пособие.}
\newblock М.: Норма: НИЦ Инфра-М, 2016. - 400 с.

\footnotesize{\url{http://znanium.com/catalog.php?bookinfo=549829}} 

  \bibitem{Revenkov2016}
П.~Ревенков    
\newblock{\em Финансовый мониторинг в условиях интернет-платежей.} 
\newblock М.: КноРус, ЦИПСиР, 2016. - 64 с.

\footnotesize{\url{http://znanium.com/bookread2.php?book=542583}} 

  \end{thebibliography}
\end{frame}

\subsection*{Научные статьи}

\begin{frame}{Научные статьи}
  \begin{thebibliography}{10}
  \beamertemplatearticlebibitems
  % Followed by interesting articles. Keep the list short. 
  \bibitem{Dwyer2015}
	\newblock Dwyer G. P. 
	\newblock The economics of Bitcoin and similar private digital 		currencies {\em Journal of Financial Stability}. 2015. Т.~17. С.~81-91.
  \bibitem{Teng2014}
	\newblock Teng F. et al. 
	\newblock Money and relationships: When and why thinking about money leads people to approach others {\em Organizational Behavior and Human Decision Processes.} 2016. Т.~137. С.~58-70.
  \end{thebibliography}
\end{frame}

\subsection*{Нормативные документы}

\begin{frame}[allowframebreaks]{Нормативные документы}
  \begin{thebibliography}{10}
  
  \beamertemplatearticlebibitems
    \bibitem{bb1}

Федеральный закон от 10.07.2002 N 86-ФЗ (ред. от 18.07.2017) "О Центральном банке Российской Федерации (Банке России)"

    \bibitem{bb2}
Федеральный закон от 02.12.1990 N 395-1 (ред. от 26.07.2017) "О банках и банковской деятельности"

    \bibitem{bb6}
Федеральный закон от 26.10.2002 N 127-ФЗ (ред. от 29.07.2017) "О несостоятельности (банкротстве)"

\pagebreak
    \bibitem{bb2}
Федеральный закон от 23.12.2003 N 177-ФЗ (ред. от 29.07.2017) "О страховании вкладов физических лиц в банках Российской Федерации"

    \bibitem{bb2}
Федеральный закон от 07.08.2001 N 115-ФЗ (ред. от 29.07.2017) "О противодействии легализации (отмыванию) доходов, полученных преступным путем, и финансированию терроризма"

  \end{thebibliography}
\end{frame}

\subsection*{Интернет ресурсы}
\begin{frame}[allowframebreaks]{Интернет ресурсы}
    
  \begin{thebibliography}{10}
  
  \setbeamertemplate{bibliography item}[online]

  \bibitem{micex}
    Московская биржа.
    \newblock \url{http://rts.micex.ru/}.
  \bibitem{cbr}
    Центральный банк Российской федерации.
    \newblock \url{http://www.cbr.ru/}.
  \bibitem{gks}
    Росстат.
    \newblock \url{http://www.gks.ru}.
  \bibitem{raex}
   Рейтинговое агентство Эксперт РА (RAEX)
    \newblock \url{https://raexpert.ru/}
  \bibitem{frs}
 	Federal Reserve Bank of St. Louis
	\newblock https://fred.stlouisfed.org/    
  \end{thebibliography}
\end{frame}

\subsection*{Научные журналы}

\begin{frame}{Журналы}
    
  \begin{thebibliography}{10}
  
  \beamertemplatearticlebibitems
  \bibitem{}
  {\em Деньги и кредит}
  \bibitem{}
  	{\em Финансы и кредит}
  \bibitem{}
  	{\em Экономический анализ: теория и практика}
  \bibitem{}
  	{\em Банковское дело}
  \bibitem{}
  	{\em Эксперт}
  \end{thebibliography}
\end{frame}

\subsection*{Газеты}

\begin{frame}{Газеты}
    
  \begin{thebibliography}{10}
  
  \beamertemplatearticlebibitems
  \bibitem{}
  	{\em Коммерсантъ}
  \bibitem{}
  	{\em Ведомости}
  \end{thebibliography}
\end{frame}

\subsection*{Материалы лекций}
\begin{frame}
Электронные презентации лекций размещены в личном профиле на сайте www.academia.edu:
  \begin{thebibliography}{10}
  
  \setbeamertemplate{bibliography item}[online]

  \bibitem{acad}
  Academia
    
    \footnotesize{\url{https://unn-ru.academia.edu/AlexanderBorochkin}}
  \end{thebibliography}
\end{frame}

\end{document}
