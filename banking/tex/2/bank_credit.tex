% !TeX program = lualatex -synctex=1 -interaction=nonstopmode --shell-escape %.tex
\documentclass[_Banking_p2.tex]{subfiles}
\begin{document}

\setbeamercovered{invisible}

\subsection{Нормативные документы}

\begin{frame}[allowframebreaks]{Нормативные документы}
  \begin{thebibliography}{10}
  
  \beamertemplatearticlebibitems
    \bibitem{bb1}

Федеральный закон от 10.07.2002 N 86-ФЗ (ред. от 18.07.2017) "О Центральном банке Российской Федерации (Банке России)"

    \bibitem{bb2}
Федеральный закон от 02.12.1990 N 395-1 (ред. от 26.07.2017) "О банках и банковской деятельности"

    \bibitem{bb3}
Письмо Банка России от 17.09.2013 N 183-Т
”О предложении кредитными организациями
расчетных (дебетовых) карт с овердрафтом и
кредитных карт клиентам”

    \bibitem{bb4}
Федеральный закон от 30.12.2004 N 218-ФЗ
(ред. от 03.07.2016) ”О кредитных историях”

    \bibitem{bb5}
Федеральный закон от 21.12.2013 N 353-ФЗ
(ред. от 03.07.2016) ”О потребительском
кредите (займе)”

    \bibitem{bb6}
Федеральный закон от 02.07.2010 N 151-ФЗ
(ред. от 01.05.2017) ”О микрофинансовой
деятельности и микрофинансовых
организациях”
  \end{thebibliography}
\end{frame}

\subsection{Процесс}
\begin{frame}
\begin{itemize}
	\item большой объем;
	\item высокая доходность;
	\item высокая рискованность.
\end{itemize}	
\end{frame}

\begin{frame}[ allowframebreaks ]{Процесс кредитования}
I. Программирование

1. Выбор кредитной политики банка.

2. Разработка внутренних нормативных
документов банка.

\vspace{1em}
II. Предоставление банковской ссуды

1. Переговоры с заемщиком и первичный анализ
заявки на кредит.

2. Анализ кредитоспособности заемщика.

3. Принятие решения о предоставлении кредита и
заключение кредитного договора.

\pagebreak
\textbf{III. Текущий мониторинг кредитов}

1. Оценка качества кредита.

2. Оценка соблюдения условий кредитного договора.

3. Оценка состояния обеспечения кредита.

\vspace{1em}
\textbf{IV. Погашение кредита}

1. Добровольное погашение кредита.

2. Работа с проблемными ссудами.

\end{frame}

\subsection{Кредитная политика}
\begin{frame}
\begin{block}{Кредитная политика}
	\quad	– устанавливает философские основы
	кредитной деятельности банка. Кредитная
	политика утверждается Советом Директоров и
	оформляется меморандумом о кредитной
	политике.
\end{block}
\end{frame}
\begin{frame}[ allowframebreaks ]{Содержание кредитной политики}
1. Введение – формулирует общую философию
кредитной деятельности банка.

2. Цели кредитной политики. Здесь установлены
цели и ориентиры, которых банк намерен
достигнуть.

3. Соответствие законам и правилам:
констатирует намерение банка работать в полном
соответствии с местными и федеральными
законами и правилами соответствующих
государственных органов.

\pagebreak
4. Администрирование кредитной политики:
выделяет процедуры обновления, интерпретации
и реализации кредитной политики.

5. Полномочия по выдаче ссуд. В банке за
выдачу кредитов отвечает совет директоров,
однако он делегирует свои полномочия
конкретным работникам или группам работников.

\end{frame}

\begin{frame}{Общие критерии одобрения
	кредитов}
\begin{itemize}
	\item рыночный район;
	\item назначение;
	\item источник погашения;
	\item программа погашения ссуды, не имеющей определенного источника возмещения.
\end{itemize}
\end{frame}
\begin{frame}[ allowframebreaks]{Обязательные условия одобрения
	кредитов}
\begin{itemize}
	\item кредит должен быть обеспечен ликвидационной стоимостью залога с
	разумным излишком;
	\item банк получает периодические отчеты о финансовом состоянии заемщика и о состоянии залога;
	\item проводятся периодические проверки состояния залога в натуре;
	
	\pagebreak 
	\item цена кредита включает дополнительные расходы, связанные с контролем за финансовым состоянием заемщика и наличием залога;
	\item у персонала должен быть достаточный опыт администрирования таких кредитов.
\end{itemize}
\end{frame}

\begin{frame}[allowframebreaks]{Другие аспекты кредитной
	политики}
\begin{itemize}
	\item полномочия персонала по установлению максимальных сроков кредитования по различным типам ссуд;
	\item кредитная информация: финансовая отчетность, отчеты кредитных бюро, материалы прямых проверок и т.д.
	\item взаимосвязь кредитных и депозитных отношений с клиентом;
	
	\pagebreak
	\item концентрация ссудной задолженности;
	\item долевое кредитование;
	\item структура кредитов;
	\item цена кредита;
	\item желательные и нежелательные кредиты.
\end{itemize}
\end{frame}

\begin{frame}[ allowframebreaks ]{Администрирование кредитов}
\begin{itemize}
	\item процесс одобрения кредитов;
	\item кредитные дела;
	\item обеспечение;
	\item гарантии и обязательства;
	\item мониторинг;
	\item ревизия кредитов;
	
	\pagebreak
	\item идентификация проблемных ссуд и их администрирование;
	\item обращение взыскания на залог;
	\item политика в области отказа от получения процентов;
	\item списание в убыток;
	\item пересмотр резерва на покрытие убытков по ссудам.
\end{itemize}
\end{frame}

\begin{frame}{Внутренние нормативные документы банка}
\begin{itemize}
	\item инструкция о порядке выдачи кредитов;
	\item положение о кредитном комитете банка (филиала);
	\item положение о внутреннем контроле.
\end{itemize}
\end{frame}
\subsection{Заявка на кредит}
\begin{frame}{Рассмотрение кредитной заявки}{Содержание кредитной заявки}
\begin{itemize}
	\item цель кредита;
	\item размер кредита;
	\item срок кредита;
	\item предполагаемое обеспечение;
	\item источники погашения кредита;
	\item краткая характеристика заемщика, информация о видах его деятельности и деловых партнерах.
\end{itemize}
\end{frame}

\begin{frame}{Переговоры с заявителем}
\begin{itemize}
	\item общие сведения об организации и ее деятельности;
	\item характеристика кредита (назначение, конкретные параметры и условия, предлагаемое обеспечение и т.д.);
	\item отношения с другими банками.
\end{itemize}
\end{frame}

\begin{frame}[allowframebreaks]{Документы для оценки кредитоспособности}

\textbf{1. Заполненная и подписанная клиентом
анкета (сведения о клиенте и его руководящих лицах).}

\textbf{2. Юридические документы:}

а) учредительные и регистрационные
документы; в зависимости от вида юридического лица - устав либо учредительный договор и устав, Свидетельство (решение) о государственной
регистрации;

б) карточка образцов подписей и печати, заверенная нотариально;

\pagebreak
в) документ, подтверждающий полномочия
лица на ведение переговоров и подписание
кредитных договоров.


\textbf{3. Финансовая отчетность (3 отчетные даты):}

а) бухгалтерский баланс, отчет о финансовых результатах, отчет о движении денежных средств и приложение к бухгалтерскому балансу за последний год, заверенные налоговым органом по месту регистрации;

\pagebreak
б) расшифровка структуры дебиторской и кредиторской задолженности;

в) справки:

- о полученных кредитах и займах;

- о выданных поручительствах;

г) копии выписок из расчетных и текущих валютных счетов заемщика, заверенные обслуживающим банком (банками).

\pagebreak
\textbf{4. Бизнес-план:}

а) основные виды деятельности предприятия и размер уплачиваемых налогов;

б) цель кредита, с указанием конкретных направлений использования заемных средств;

в) предполагаемые сроки и сумма выпуска или приобретения продукции или оказываемых услуг с указанием расценок за единицу продукции или услуги;

г) планируемый рынок сбыта;

д) сумма ожидаемой прибыли от реализации (после уплаты платежей в бюджет).

\pagebreak
\textbf{5. Документы, подтверждающие бизнес-план:}

а) договоры о покупке необходимых ценностей (с указанием состава приобретаемых ценностей, условий поставки, форм расчетов);

б) документация, разрешающая проведение сделок, предусмотренных бизнес-планом (лицензии, сертификаты качества, соответствия, гигиенические сертификаты и т.д.);

в) договоры о реализации конечной продукции;

г) другие договоры, необходимые для выполнения бизнес-плана (например, с
транспортными организациями).

\pagebreak
\textbf{6. Дополнительная информация о заемщике:}

а) аудиторское заключение;

б) рекламные проспекты и публикации в СМИ;

в) сведения от деловых партнеров заемщика.

\textbf{7. В случае принятия в залог имущества от Залогодателя требуются соответствующие документы по залоговому имуществу.}
\end{frame}
\subsection{Определения}
\begin{frame}{Определения}
\begin{block}{Платежеспособность клиента}
	\quad — это его способность своевременно погасить все виды обязательств и задолженности.	
\end{block}

\begin{block}{Кредитоспособность}
\quad характеризует лишь возможность предприятия
погасить ссудную задолженность.

Источники ее погашения:

- гарантия (поручительство) другого банка или
иного лица (физического или юридического);

- страховые возмещения.
\end{block}
\end{frame}

\begin{frame}{Оценка кредитоспособности}
\begin{itemize}
	\item Оценка финансового положения заемщика.
	\item Оценка его деловых качеств (можно ли ему доверять).	
\end{itemize}
\end{frame}

\subsection{Выдача кредита}
\begin{frame}{Решение о предоставлении кредита}
\begin{itemize}
	\item Кредитный эксперт составляет письменное заключение о возможности или невозможности предоставления кредита и определяет группу риска заемщика.
	\item Кредитный комитет банка принимает решение об одобрении кредита.
\end{itemize}
\end{frame}

\begin{frame}[ allowframebreaks ]{Юридическое оформление кредитной сделки}
Обязательно оформляется:
\begin{itemize}
	\item кредитный договор;
	\item срочное обязательство (распоряжение заемщика о своевременном списании средств в пользу банка-кредитора со ссудного счета, открываемого ему в банке).
\end{itemize}

При необходимости также:
\begin{itemize}
	\item договор залога;
	\item другие договоры (о переуступке прав, о блокированном счете, договор поручительства);
	\item другая документация.
\end{itemize}

\pagebreak
Дополнительно банк истребует от заемщика:
\begin{itemize}
	\item справку из налогового органа об уведомлении данного органа о намерении налогоплательщика открыть ссудный счет;
	\item справку об уведомлении Пенсионного фонда;
	\item др. документы.
\end{itemize}
\end{frame}
\subsection{Кредитный мониторинг}
\begin{frame}[allowframebreaks]{Кредитный мониторинг}{Возможные признаки ухудшения финансового положения заемщика}
\begin{itemize}
	\item снижение объема продаж;
	\item снижение доли денежной составляющей в составе выручки от реализации;
	\item резкое увеличение дебиторской и кредиторской задолженности (общей суммы и по отдельным видам) и замедление ее оборачиваемости;
	\item рост убытков или снижение прибыли;
	\item рост отношения заемные средства / оборотные активы;
	
	\pagebreak
	\item непропорциональный по сравнению с дебиторской рост краткосрочной
	задолженности;
	\item рост просроченных долгов;
	\item имеются требования третьих лиц в отношении заемщика в арбитражном процессе;
	\item имеется ли задолженность заемщика перед бюджетом, взыскание которой сделает погашение кредита проблематичным.
\end{itemize}
\end{frame}
\begin{frame}[allowframebreaks]{Факты, выявляемые службой безопасности банка}
\begin{itemize}
	\item представления банку недостоверной и
	фальсифицированной отчетности и других
	данных;
	\item невыполнения обязательств перед банками и
	контрагентами;
	\item ведущихся судебных разбирательств;
	
	\pagebreak
	\item резких изменений в планах деятельности
	клиента;
	\item ожидаемых радикальных изменений в составе
	руководства компании или неблагоприятных
	тенденций на рынке заемщика;
	\item данных о личности руководителей.
\end{itemize}
\end{frame}
\subsection{Погашение кредита}
\begin{frame}{Работа по погашению просроченной задолженности}
\begin{itemize}
	\item разработка совместно с заемщиком плана
	мероприятий и, восстановления
	стабильности компании (продажа активов,
	сокращение накладных расходов, изменение
	рыночной стратегии, смена руководства);
	\item продажа залога;
	\item иск поручителю (гаранту);
	\item объявление неплательщика банкротом.
\end{itemize}
\end{frame}

\subsection{Кредитование граждан}
\begin{frame}{Кредитование физических лиц}
\begin{itemize}
	\item Клиент получает необходимую информацию
	по условиям кредитования, обеспечения и
	возврата кредита.
	\item Кредитный эксперт ведет переговоры с
	клиентом для выяснения цели кредита;
	\item Кредитный эксперт разъясняет клиенту
	условия и порядок предоставления кредита,
	знакомит с перечнем документов,
	необходимых для получения кредита.
\end{itemize}
\end{frame}

\begin{frame}{Понятие кредитоспособности физического лица}
\begin{block}{Кредитоспособность физического лица}
	\quad в юридическом смысле: правоспособен ли
	клиент заключить кредитный договор.
	
	\quad с экономической точки зрения — имеет ли
	клиент необходимые доходы, имущество для
	полного и своевременного выполнения условий
	кредитного договора.
\end{block}
\end{frame}
\begin{frame}{Оценка кредитоспособности физических лиц}
1. Системы оценки кредитоспособности клиентов,
основанные на экспертных оценках и прогнозах
результатов экономической деятельности с
использованием предоставленного кредита.

2. Балльные системы оценки кредитоспособности
клиентов.
\end{frame}
\subsection{Контрольные вопросы}
\begin{frame}[ allowframebreaks ]{Контрольные вопросы}
1. Кредитная политика банка.

2. Понятие кредитоспособности и платежеспособности предприятия.

3. Организация банковского кредитования предприятий. Документы, предоставляемые в банк для получения кредита. Порядок выдачи и погашения кредита.

\pagebreak
4. Мониторинг и ревизия выданных кредитов банком.

5. Работа банка с проблемными кредитами.

6. Вексельные кредиты коммерческих банков: учет векселей, ссуды под залог векселей, векселедательский и акцептный кредит.

7. Организация банковского кредитования физических лиц. Методики оценки кредитоспособности физических лиц.

\end{frame}

\setbeamercovered{transparent}
\end{document}