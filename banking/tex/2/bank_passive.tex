% !TeX program = lualatex -synctex=1 -interaction=nonstopmode --shell-escape %.tex
\documentclass[_Banking_p2.tex]{subfiles}
\begin{document}

\setbeamercovered{invisible}

\subsection{Нормативные документы}

\begin{frame}[allowframebreaks]{Нормативные документы}
  \begin{thebibliography}{10}
  
  \beamertemplatearticlebibitems
    \bibitem{bb1}

Федеральный закон от 10.07.2002 N 86-ФЗ (ред. от 18.07.2017) "О Центральном банке Российской Федерации (Банке России)"

    \bibitem{bb2}
Федеральный закон от 02.12.1990 N 395-1 (ред. от 26.07.2017) "О банках и банковской деятельности"

\pagebreak

    \bibitem{bb2}
Федеральный закон от 23.12.2003 N 177-ФЗ (ред. от 29.07.2017) "О страховании вкладов физических лиц в банках Российской Федерации"

    \bibitem{bb6}
Федеральный закон от 29.07.2004 N 96-ФЗ (ред. от 22.12.2014) "О выплатах Банка России по вкладам физических лиц в признанных банкротами банках, не участвующих в системе обязательного страхования вкладов физических лиц в банках Российской Федерации"


  \end{thebibliography}
\end{frame}




\subsection{Капитал}
\begin{frame}{Капитал банка}{}
\begin{itemize}[<+->]
\item
О методике определения величины собственных средств (капитала) кредитных организаций ("Базель III"): Положение Банка России от 28.12.2012 N 395-П (ред. от 25.11.2014).
\item
Величина собственных средств (капитала) кредитных организаций определяется как сумма \textbf{основного капитала }и дополнительного капитала за вычетом показателей, предусмотренных положением ЦБ РФ.
\item
\textbf{Основной капитал }равен сумме базового капитала и добавочного капитала.

\end{itemize}
\end{frame}
\begin{frame}[allowframebreaks]{Базовый капитал кредитной организации}
+ уставный капитал обыкновенные и привилегированные акции,  выпущенные до 1.03.2013;

+ эмиссионный доход от вышеперечисленных акций;

+ резервный фонд, сформированный за счет прибыли, подтвержденной аудиторами;

+ прибыль текущего года и предшествующих лет, подтвержденная аудиторами;

- нематериальные активы, за вычетом начисленной амортизации, деловая репутация, а также вложения в создание (изготовление) и приобретение нематериальных активов;

\pagebreak
- сумма налога на прибыль, подлежащая возмещению в будущих отчетных периодах в отношении перенесенных на будущее убытков;

- сумма налога на прибыль, подлежащая возмещению в будущих отчетных периодах в отношении вычитаемых временных разниц;
вложения в собственные обыкновенные и привилегированные акции;

- убытки предшествующих лет и текущего года;

- вложения кредитной организации в обыкновенные акции (доли) финансовых организаций;

- отрицательная величина добавочного капитала.


\end{frame}

\begin{frame}{Источники добавочного капитала}
+ субординированные кредиты (депозиты, займы, облигационные займы) без ограничения срока привлечения (или сроком свыше 50 лет);

- вложения в собственные акции;

- вложения в акции финансовых организаций;

- предоставленные другим финансовым организациям - субординированные кредиты (депозиты, займы, облигационные займы) для пополнения добавочного капитала;

- отрицательная величина дополнительного капитала;

- др.

\end{frame}

\begin{frame}{Источники дополнительного капитала}
+ выпущенные привилегированные акции;

+ эмиссионный доход;

+ прирост стоимости имущества при переоценке;

+ резервный фонд кредитной организации в части, сформированной за счет отчислений из прибыли текущего и предыдущего года, неподтвержденной аудиторами;

+ прибыль текущего года, не подтвержденная аудиторской организацией;

+ прибыль предшествующих лет до аудиторского подтверждения.

\end{frame}

\begin{frame}{Функции собственного капитала банка}
\begin{itemize}
\item
Защитная;
\item
Оперативная;
\item
Регулирующая.
\end{itemize}
\end{frame}

\subsection{Привлеченные средства}
\begin{frame}{Привлеченные средства банка}{}
\begin{itemize}[<+->]
\item
депозиты (вклады);
\item
средства, привлеченные путем выпуска и распространения собственных долговых ценных бумаг (облигаций, векселей, сертификатов);
\item
межбанковские кредиты (в том числе кредиты центрального и иностранных банков).
\end{itemize}
\end{frame}

\begin{frame}
\begin{block}{Депозиты}
\quad
- это денежные средства, переданные их собственником в банк для хранения на определенных условиях (от лат. depositum - вещь, отданная на хранение).
\end{block}
В зависимости от категории вкладчиков существуют депозиты юридических лиц (предприятий, организаций, других банков) и депозиты физических лиц, а от срока изъятия — срочные и до востребования.
\end{frame}

\begin{frame}{Банковские вклады физических лиц}
\begin{block}{Вклад}
денежные средства в валюте Российской Федерации или иностранной валюте, размещаемые физическими лицами в целях хранения и получения дохода.
\end{block}
Доход по вкладу выплачивается в денежной форме в виде процентов. 

Вклад возвращается вкладчику по его первому требованию в порядке, предусмотренном для вклада данного вида федеральным законом и соответствующим договором.
\end{frame}

\begin{frame}
\begin{block}{Проценты}
\begin{itemize}
\item
Простые;
\item
сложные.
\end{itemize}
\end{block}

\begin{block}{Процентная ставка }
\begin{itemize}
\item
фиксированная 
\item
переменная.
\end{itemize}
\end{block}
\end{frame}

\begin{frame}{Собственные долговые обязательства}
\begin{itemize}
\item
банковские сертификаты;
\item
векселя;
\item
облигации.
\end{itemize}
\end{frame}

\begin{frame}
\begin{block}{Банковский сертификат}
- это ценная бумага, удостоверяющая внесение в банк срочного вклада и дающая вкладчику право по истечении установленного срока получить обратно сумму вклада и проценты по нему. Юридическим лицам выдают депозитный сертификат, а физическим - сберегательный.
\end{block}
\end{frame}

\begin{frame}
\begin{block}{Вексель}
- письменное денежное обязательство, оформленное по строго установленной форме, дающее владельцу векселя (векселедержателю) право на получение от должника по векселю определённой в нём суммы в конкретном месте.
\end{block}
\end{frame}

\begin{frame}
\begin{block}{Облигация банка }
- это долговая ценная бумага, свидетельствующая о предоставлении ее держателем денежных средств банку на определенный срок и дающая держателю право по истечении этого срока получить обратно предоставленные денежные средства с каким-либо доходом.
\end{block}
\end{frame}

\subsection{Банковский счет}
\begin{frame}{Банковский счет. Виды банковских счетов}{}
\begin{itemize}[<+->]
\item
текущие счета; 
\item
расчетные счета; 
\item
бюджетные счета; 
\item
корреспондентские счета; 
\item
корреспондентские субсчета; 
\item
счета доверительного управления; 
\item
специальные банковские счета; 
\item
депозитные счета судов, подразделений службы судебных приставов, правоохранительных органов, нотариусов; 
\item
счета по вкладам (депозитам).
\end{itemize}
\end{frame}

\begin{frame}[allowframebreaks]{Виды банковских счетов}

\textbf{Текущие счета }открываются физическим лицам для совершения операций, не связанных с предпринимательской деятельностью или частной практикой.

\textbf{Расчетные счета} открываются юридическим лицам, не являющимся кредитными организациями, а также индивидуальным предпринимателям или физическим лицам, занимающимся частной практикой, для совершения операций, связанных с предпринимательской деятельностью или частной практикой. Расчетные счета открываются представительствам кредитных организаций, а также некоммерческим организациям.

\pagebreak
\textbf{Бюджетные счета }открываются в случаях, установленных законодательством Российской Федерации, юридическим лицам, осуществляющим операции со средствами бюджетов бюджетной системы Российской Федерации.

\textbf{Корреспондентские счета }открываются кредитным организациям, а также иным организациям в соответствии с законодательством Российской Федерации или международным договором. Банку России открываются корреспондентские счета в иностранных валютах.

\textbf{Корреспондентские субсчета} открываются филиалам кредитных организаций.

\pagebreak
\textbf{Счета доверительного управления} открываются доверительному управляющему для осуществления операций, связанных с деятельностью по доверительному управлению.

\textbf{Специальные банковские счета}: банковского платежного агента, банковского платежного субагента, платежного агента, поставщика, торговый банковский счет, клиринговый банковский счет, счет гарантийного фонда платежной системы, номинальный счет, счет эскроу, залоговый счет, специальный банковский счет должника, открываются в соответствии с законодательством Российской Федерации.

\pagebreak
\textbf{Депозитные счета судов, подразделений службы судебных приставов, правоохранительных органов, нотариусов} открываются для зачисления денежных средств, поступающих в их временное распоряжение.

\textbf{Счета по вкладам (депозитам)} открываются соответственно физическим и юридическим лицам для учета денежных средств, размещаемых в банках с целью получения доходов в виде процентов, начисляемых на сумму размещенных денежных средств.
\end{frame}

\begin{frame}{Открытие текущего счета физическому лицу}
а) документ, удостоверяющий личность физического лица;

б) карточка подписей;

в) документы, подтверждающие полномочия лиц, указанных в карточке, на распоряжение денежными средствами;

г) свидетельство о постановке на учет в налоговом органе (при наличии).

Иностранные граждане или лица без гражданства предоставляют дополнительно миграционную карту и (или) документ, подтверждающий право на пребывание (проживание) в Российской Федерации.
\end{frame}

\begin{frame}[allowframebreaks]{Открытие банковских счетов юридическому лицу}
а) свидетельство о государственной регистрации юридического лица;

б) учредительные документы юридического лица;

в) выданные юридическому лицу лицензии (разрешения), если данные лицензии (разрешения) имеют непосредственное отношение к правоспособности клиента заключать договор, на основании которого открывается счет;

г) карточка  подписей;

\pagebreak
д) документы, подтверждающие полномочия лиц, указанных в карточке, на распоряжение денежными средствами;

е) документы, подтверждающие полномочия единоличного исполнительного органа юридического лица;

ж) свидетельство о постановке на учет в налоговом органе.

Иностранные юридические лица вместо п. а), б) предоставляют документы, подтверждающие правовой статус юридического лица по законодательству страны, на территории которой создано это юридическое лицо.

\end{frame}


\subsection{Контрольные вопросы}
\begin{frame}[ allowframebreaks ]{Контрольные вопросы}
1. Капитал банка. Основной и дополнительный капитал. Функции банковского капитала. Оценка достаточности собственного капитала.

2. Порядок выпуска и регистрации акций банками.

3. Порядок выпуска и регистрации облигаций банками.

4. Виды банковских счетов.

5. Порядок открытия банковского счета для юридических и физических лиц.

\pagebreak
6. Характеристика привлеченных средств банка: депозиты до востребования и срочные депозиты.

7. Порядок выпуска и обращения банками депозитных и сберегательных сертификатов.

8. Порядок выпуска и обращения банками собственных векселей.

9. Межбанковские кредиты: их виды и порядок предоставления. Рынок межбанковских кредитов.
\end{frame}

\setbeamercovered{transparent}
\end{document}