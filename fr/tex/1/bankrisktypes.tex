\documentclass[financial_risks_lectures.tex]{subfiles}
\begin{document}

\setbeamercovered{transparent}
\subsection{Рыночный риск}
\begin{frame}{Понятие рыночного риска}
\begin{block}{Рыночный риск }
\quad — это риск возникновения у банка убытков вследствие неблагоприятного изменения рыночной стоимости финансовых инструментов торгового портфеля и производных финансовых инструментов кредитной организации, а также курсов иностранных валют и (или) драгоценных металлов.
\end{block}
\end{frame}
\begin{frame}{Виды рыночного риска}
\begin{itemize}[<+->]
\item
\textbf{ценовой риск }— включает в себя фондовый риск (риск снижения стоимости ценных бумаг) и товарный риск (риск изменения цен товаров);
\item
\textbf{валютный риск }— риск возникновения потерь, связанных с неблагоприятным изменением курсов иностранных валют и драгоценных металлов по открытым позициям в иностранных валютах и драгоценных металлах;
\item
\textbf{процентный риск }— риск возможных потерь в результате неблагоприятного изменения процентных ставок по активам, пассивам и внебалансовым инструментам банка.
\end{itemize}
\end{frame}

\begin{frame}{Управление рыночными рисками}
Ограничение величины возможных потерь по открытым позициям.

Методы оценки рыночного риска:
\begin{itemize}
\item
\textit{VaR }(Value-at-Risk);
\item
Expected Shortfall;
\item
Аналитические подходы (например, дельта-гамма подход);
\item
Стресс тестирование;
\item
Положение ЦБ РФ «О порядке расчета кредитными организациями величины рыночного риска».
\end{itemize}
\end{frame}

\subsection{Риск ликвидности}
\begin{frame}[shrink=15]{Понятие риска ликвидности}
\begin{block}{Риск ликвидности}
\quad — это риск потерь, вызванный несоответствием сроков погашения обязательств по активам и пассивам.
\end{block}

\begin{itemize}[<+->]
\item
\textbf{Риск рыночной ликвидности }связан с потерями, которые может понести участник из-за недостаточной ликвидности рынка, когда реальная цена сделки сильно отличается от рыночной цены в худшую сторону.
\item
\textbf{Риск балансовой ликвидности или неплатежеспособности }заключается в том, что кредитная организация может оказаться неплатежеспособной и не сможет выполнить свои обязательства перед кредиторами, вкладчиками и прочими контрагентами в силу нехватки наличных средств или других высоколиквидных активов.
\end{itemize}
\end{frame}

\begin{frame}{Цель управления ликвидностью}
\begin{itemize}[<+->]
\item
обеспечивать своевременное выполнение обязательств;
\item
удовлетворять спрос клиентов на кредитные ресурсы;
\item
поддерживать репутацию банка как надежного финансового института.
\end{itemize}
\end{frame}

\begin{frame}{Управление риском ликвидности в банке}
\begin{itemize}[<+->]
\item
анализ разрыва в сроках погашения требований и обязательств, сгруппированных по срокам (до востребования, от 1 до 30 дней, от 31 до 90 дней и т.д.);
\item
установление лимитов и ставок привлечения/размещения средств по отдельным инструментам и (или) срокам;
разработка сценариев и мероприятий по восстановлению ликвидности банка;
\item
контроль за выполнением установленных процедур по управлению ликвидностью.
\end{itemize}
\end{frame}

\begin{frame}{Управление риском ликвидности в банке}{Дополнительные подходы}
\begin{itemize}[<+->]
\item
Механизмы обязательного резервирования (усреднения обязательных резервов ) и рефинансирование кредитных организаций со стороны ЦБ РФ.
\item
При определении балансовой ликвидности рассчитываются показатели мгновенной, текущей и долгосрочной ликвидности в порядке, предусмотренном Инструкцией Банка России от 16.01.2004 г. №110-И «Об обязательных нормативах банков».
\item
ГЭП-анализ (методом анализа разрыва в сроках погашения требований и обязательств), Письмо ЦБ РФ от 27.07.2000 г. №139-Т «О рекомендациях по анализу ликвидности кредитных организаций».
\end{itemize}
\end{frame}

\begin{frame}{Сценарное моделирование рисков}
\begin{itemize}[<+->]
\item
стандартный сценарий без кризисных ситуаций;
\item
сценарий «кризис в банке»: усиление оттока клиентских средств, закрытие ряда источников покупной ликвидности;
\item
сценарий «кризис рынка»: падение рыночных цен на финансовые инструменты, прекращение торгов, неликвидный рынок ценных бумаг, большой отток клиентских средств.
\end{itemize}
\end{frame}

\begin{frame}{Измерение риска рыночной ликвидности}
\begin{block}{Реализованный спред}
Величину реализованного спреда можно вычислить как разность между средневзвешенными ценами сделок за определенный период времени, совершенных по цене спроса, и сделок, совершенных по цене предложения. 
\end{block}
\end{frame}
\begin{frame}[shrink=10]{Формула для определения \textit{VaR }с учетом риска ликвидности}
\begin{align}
VAR_p=P_p \sigma_p z_{\alpha} u,
\end{align}
где

$VAR_p$ - \textit{VaR }порфеля;

$P_p$ - стоимость портфеля;

$\sigma_p$ - стандартное отклонение доходности портфеля, соответствующее времени, для которого рассчитывается \textit{VaR};

$z_{\alpha}$ - количество стандартных отклонений, соответствующих уровню доверительной вероятности~$\alpha$;

$u$ - спред ($u=\log \frac{Ask}{Bid}$, \textit{Bid }- цена предложения, \textit{Ask }- цена спроса).

\end{frame}

\begin{frame}{Методы снижения риска ликвидности}
\begin{itemize}[<+->]
\item
резервы текущей ликвидности;
\item
лимиты для ограничения разрывов в структуре активов и пассивов по срокам и по валютам (предельные суммы, допустимые сроки превышения предельных сумм, сроки возвращения в рамки допустимых интервалов).
\end{itemize}
\end{frame}


\subsection{Операционный риск}
\begin{frame}{Понятие операционного риска}
\begin{block}{Операционный риск}
\quad — это опастность возникновения убытков в результате недостатков или ошибок во внутренних процессах банка, в действиях сотрудников и иных лиц, в работе информационных систем, либо вследствие внешнего воздействия.
\end{block}
\end{frame}

\begin{frame}{Источники операционного риска}
\begin{itemize}[<+->]
\item
персонал (намеренные действия сотрудников компании, которые могут нанести ущерб ее деятельности);
\item
процессы (ошибки и некорректное исполнение операций в ходе осуществления бизнес-процессов либо исполнения должностных обязанностей);
\item
системы (нарушение текущей деятельности в результате сбоя информационных систем или недоступности сервиса);
\item
внешняя среда (атаки либо иные угрозы, исходящие из внешней среды, которые не могут управляться компанией и выходят за рамки ее непосредственного контроля).
\end{itemize}
\end{frame}

\begin{frame}{Прямые операционные потери}
\begin{itemize}[<+->]
\item
снижение стоимости активов;
\item
досрочное списания (выбытия) материальных активов;
денежных выплат в виде судебных издержек, взысканий по решению суда, штрафных санкций надзорных органов и т.д.;
\item
денежные выплат компенсаций клиентам и контрагентам;
\item
затраты на восстановление хозяйственной деятельности и устранение последствий ошибок, аварий, стихийных бедствий и других аналогичных обстоятельств;
\item
прочие затраты.
\end{itemize}
\end{frame}

\begin{frame}{Косвенные операционные потери}
\begin{itemize}[<+->]
\item
потеря деловой репутации;
\item
недополученные запланированные доходы;
\item
приостановка деятельности в результате неблагоприятного события (технологический сбой);
\item
отток клиентов и т.д.
\end{itemize}
\end{frame}

\begin{frame}{Управление операционным риском}
\begin{itemize}[<+->]
\item
сложно создать универсальный перечень причин возникновения данного вида риска;
\item
операционный риск очень сложно оценить количественным методом.
\end{itemize}
\end{frame}

\begin{frame}{Подходы к оценке операционного риска}
\begin{itemize}[<+->]
\item
Подход на основе базового индикатора основан на прямой зависимости уровня операционного риска (коэффициента резервирования) от масштабов деятельности организации (валового дохода). Рекомендуется 15\% от средней величины валового дохода за три последних года.
\item
Стандартный подход определяет размер резервируемого капитала от валового дохода в разрезе стандартных видов деятельности банка. Рекомендуется 18\%-12\% от валового дохода по видам деятельности.
\end{itemize}
\end{frame}

\begin{frame}{«Об организации управления операционным риском в кредитных организациях»}{Письмо Банка России от 24.05.2005 г. №76-Т} 
\begin{itemize}[<+->]
\item
\textbf{Статистический анализ распределения фактических убытков}
\item
\textbf{Балльно-весовой метод (метод оценочных карт)} - экспертным анализом выбираются коэффициенты, и определяется их относительная значимость, затем, обобщающий показатель определяется как их средневзвешенная величина.
\item
\textbf{Моделирование }– определяются сценарии возникновения событий, приводящих к операционным убыткам, и разрабатывается модель распределения частоты возникновения и размеров убытков.
\end{itemize}
\end{frame}



\subsection{Правовой риск}
\begin{frame}{Понятие правового риска}
\begin{block}{Правовой риск}
\quad — это опастность возникновения у кредитной организации убытков вследствие того, что условия договора окажется невозможным выполнить по действующему законодательству или же что договор окажется ненадлежащим образом оформлен (в т.ч. потери из-за «пробелов» или нарушения юридических требований действующего законодательства).
\end{block}
\textit{Письмо Банка России от 30.06.2005 г. №92-Т «Об организации управления правовым риском и риском потери деловой репутации»}
\end{frame}

\begin{frame}{Цель и метод управления правовыми рисками}
\begin{itemize}[<+->]
\item
\textbf{Цель управления правовыми рисками }— минимизация возможных потерь банка с учетом стоимости их контроля.
\item
\textbf{Метод управления правовым риском }— метод статистического анализа распределения фактических убытков.
\end{itemize}
\end{frame}


\begin{frame}[shrink=15]{Внутренние факторы правового риска}
\begin{itemize}[<+->]
\item
несоблюдение кредитной организацией требований нормативных правовых актов, заключенных договоров, учредительных и внутренних документов;
\item
несоответствие внутренних документов банка действующему законодательству РФ;
\item
неспособность банка своевременно приводить свою деятельность и внутренние документы в соответствие с произошедшими изменениями законодательства;
\item
неэффективная организация правовой работы, приводящая к правовым ошибкам в деятельности банка вследствие действий работников или органов управления кредитной организации;
\item
нарушение банком условий договоров.
\end{itemize}
\end{frame}

\begin{frame}{Внешние факторы правового риска}
\begin{itemize}[<+->]
\item
несовершенство правовой системы;
\item
нарушения клиентами и (или) контрагентами банка нормативных правовых актов, а также условий договоров;
\item
нахождение банка, его филиалов, дочерних и зависимых организаций, клиентов и контрагентов под юрисдикцией различных государств.
\end{itemize}
\end{frame}

\begin{frame}{Виды правовых рисков}
\begin{itemize}[<+->]
\item
снижения стоимости активов;
\item
досрочного списания (выбытия) материальных активов;
\item
денежных выплат на основании постановлений (решений) судов;
\item
выплат денежных компенсаций клиентам и контрагентам во внесудебном порядке.
\end{itemize}
\end{frame}
%\setbeamercovered{invisible}
\begin{frame}{Принципы управления правовым риском I}
\begin{itemize}[<+->]
\item
Принцип обеспечения правомерности совершаемых банковских операций;
\item
Принцип «знай своего клиента»;
\item
Принцип ответственности руководителе подразделений банка за управление правовым риском, присущего функциям, которые выполняются в подразделениях;
\item
Принцип управления правовым риском на основе экономической целесообразности;
\end{itemize}
\end{frame}
\begin{frame}{Принципы управления правовым риском II}
\begin{itemize}[<+->]
\item
Принцип комплексности и непрерывности (взаимодействие всех подразделений банка);
\item
Принцип использования количественной и качественной оценки правовых рисков (1) заключение кредитного договора с отсутствием в нем права банка на досрочное истребование кредита в случае нарушения заемщиком существенных условий договора; 2) заключение кредитного договора руководителями акционерного общества без одобрения акционерами крупной сделки — потери будут равняться как минимум сумме кредита)
\end{itemize}
\end{frame}

\begin{frame}{Инструменты минимизации правового риска I}
\begin{itemize}[<+->]
\item
разграничение прав доступа к информации;
\item
разработка защиты от несанкционированного входа в информационную систему;
\item
разработка защиты от выполнения несанкционированных операций средствами информационной системы;
\item
организация системы контроля до исполнения документов;
настройка и подключение автоматических проверочных процедур для диагностики ошибочных действий;
\end{itemize}
\end{frame}

\begin{frame}{Инструменты минимизации правового риска II}
\begin{itemize}[<+->]
\item
автоматическое выполнение рутинных повторяющихся действий;
\item
разработка адекватной организационной структуры;
\item
разграничение полномочий и ответственности при совершении действий;
\item
формализация банковских операций и сделок (определяются наиболее значимые типы и виды сделок, для которых разработаны стандартные формы сопровождающих документов).
\end{itemize}
\end{frame}



\subsection{Репутационный риск}
\begin{frame}{Понятие репутационного риска}
\begin{block}{Репутационный риск}
\quad — это опастность возникновения у кредитной организации убытков вследствие неблагоприятного восприятия имиджа банка клиентами, контрагентами, акционерами (участниками), деловыми партнерами, регулирующими органами и прочее.
\end{block}
\end{frame}

\begin{frame}[shrink=15]{Причины репутационого риска I}
\begin{itemize}[<+->]
\item
несоблюдение кредитной организацией законодательства Российской Федерации, учредительных и внутренних документов кредитной организации, обычаев делового оборота, принципов профессиональной этики; 
\item
неисполнение договорных обязательств перед кредиторами, вкладчиками и иными клиентами и контрагентами;
отсутствие механизмов, позволяющих регулировать конфликты интересов;
\item
неспособность кредитной организации противодействовать легализации (отмыванию) доходов, полученных преступным путем, и финансированию терроризма;
\item
недостатки в управлении банковскими рисками кредитной организации, приводящие к возможности нанесения ущерба деловой репутации;
\end{itemize}
\end{frame}

\begin{frame}[shrink=15]{Причины репутационого риска II}
\begin{itemize}[<+->]
\item
осуществление кредитной организацией рискованной кредитной, инвестиционной и рыночной политики, высокий уровень операционного риска;
\item
недостатки кадровой политики;
\item
возникновение у кредитной организации конфликта интересов с учредителями (участниками), клиентами и контрагентами, а также другими заинтересованными лицами;
\item
опубликование негативной информации о кредитной организации или ее служащих, учредителях (участниках), членах органов управления и аффилированных лицах в средствах массовой информации.
\end{itemize}
\end{frame}


\begin{frame}{Управление репутационным риском}
\begin{itemize}[<+->]
\item
Основная цель управления репутационным риском — уменьшение возможных убытков, сохранение и поддержание деловой репутации банка.
\item
Принципы и методы управления репутационным риском такие же, как и при управлении правовым риском.
\end{itemize}
\end{frame}

\begin{frame}[shrink=10]{Параметры, принимаемые в расчет при управлении репутационным риском}
\begin{itemize}[<+->]
\item
изменение финансового состояния банка;
\item
сокращение клиентской базы, изменение структуры пассивов;
возрастание (сокращение) количества жалоб и претензий к банку;
\item
негативные и позитивные отзывы и сообщения о банке и связанных с ним лицах в средствах массовой информации по сравнению с другими банками за определенный период времени;
\item
динамика доли требований к аффилированным лицам в общем объеме активов банка;
\item
своевременность расчетов по поручению клиентов и контрагентов банка;
\item
осуществление банком рискованной кредитной и рыночной политики.
\end{itemize}
\end{frame}

\begin{frame}[shrink=10]{Способы снижения риска}
\begin{itemize}[<+->]
\item
создание внутренней нормативной базы для исключения конфликта интересов между работниками банка и клиентами, контрагентами, между работниками кредитной организации и самим банком;
\item
контроль за соблюдением законодательства РФ против легализации (отмывания) доходов;
\item
своевременность расчетов по поручению клиентов и контрагентов банка;
\item
контроль за достоверностью финансовой отчетности и иной публикуемой информации;
\item
обеспечение постоянного повышения квалификации работников банка.
\end{itemize}
\end{frame}

\subsection{Кредитный риск}
\begin{frame}{Понятие кредитного риска}
\begin{block}{Кредитный риск}
\quad — это опастность возникновения у кредитной организации убытков вследствие неисполнения, несвоевременного либо неполного исполнения должником финансовых обязательств перед кредитной организацией в соответствии с условиями договора.
\end{block}
Виды кредитного риска: страновой риск, региональный риск, отраслевой риск, риски клиента, производственный риск, платежный риск, риски проекта, риски обеспечения.
\end{frame}

\begin{frame}[shrink=10]{Уровни управления кредитным риском}
\begin{itemize}[<+->]
\item
\textbf{Индивидуальный уровень }подразумевает анализ, оценку и разумное снижение рисков по конкретной сделке, осуществляется, как правило, для сделок, не подпадающих под агрегированный уровень.
\item
\textbf{Агрегированный уровень }подразумевает разработку программ и выработку критериев, которым должна соответствовать сделка, что позволяет ограничивать величину принимаемых банком рисков, применяется для для типовых сделок с объемом кредитного риска, не превышающим установленной величины.
\item
\textbf{Портфельный уровень} подразумевает под собой оценку совокупного кредитного риска, его концентрации, динамики и т.п., а также выработку предложений по установлению лимитов и управленческих решений в целях снижения риска.
\end{itemize}
\end{frame}

\begin{frame}[shrink=10]{Факторы, повышающие кредитный риск}
\begin{itemize}[<+->]
\item
концентрация кредитного риска (на одном заемщике, отрасли, регионе);
\item
большой удельный вес кредитов и других банковских контрактов, приходящихся на проблемных клиентов;
\item
концентрация деятельности банка в мало изученных, новых нетрадиционных сферах;
\item
внесение частых или существенных изменений в политику банка по предоставлению кредитов;
\item
большой удельный вес новых и недавно привлеченных клиентов, о которых банк располагает недостаточной информацией;
\item
неосмотрительная кредитная политика банка;
\item
не обеспеченные ссуды или принятие в залог низколиквидного обеспечения.
\end{itemize}
\end{frame}

\begin{frame}{Методы оценки кредитного риска}
\begin{itemize}[<+->]
\item
\textbf{Российские банки}: совокупный показатель риска (кредитный рейтинг), или консолидированный уровень риска, путем соединения оценок отдельных коэффициентов;
\item
\textbf{Современный подход }основан на концепции VaR и ее развитии;
\item
\textbf{Базельский комитет }предлагает рассчитывать коэффициент риска путем использования банком внешних кредитных рейтингов (Moody's, Standard \& Poor's и т.д.) или собственной (внутренней) системы рейтингов активов и забалансовых статей (для расчета требований к капиталу).
\end{itemize}
\end{frame}
\setbeamercovered{invisible}
\end{document}