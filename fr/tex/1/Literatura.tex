\documentclass[financial_risks_lectures.tex]{subfiles}

\begin{document}

\section*{Литература}
\subsection*{Учебники}

\begin{frame}[shrink=15]
  \frametitle<presentation>{Учебники}
    
  \begin{thebibliography}{10}
    
  \beamertemplatebookbibitems
  % Start with overview books.

  \bibitem{Kovalev2013}
	П. П.~Ковалев. 
    \newblock {\em Банковский риск-менеджмент: Учебное пособие}.
    \newblock М.: КУРС: НИЦ ИНФРА-М, 2013.
  \bibitem{Krichevski2013}
	М. Л.~Кричевский. 
    \newblock {\em Финансовые риски}.
    \newblock М.: КноРус, 2013.
  \bibitem{Larionova2014}
	И.В.~Ларионова. 
    \newblock {\em Риск-менеджмент в коммерческом банке}.
    \newblock М.: КноРус, 2014.
  \pagebreak
  \bibitem{Shapkin2013}
	А.С.~Шапкин, В.А.~Шапкин 
    \newblock {\em Экономические и финансовые риски. Оценка, управление, портфель инвестиций}.
    \newblock М.: Дашков и К, 2013. 
  \bibitem{Chetirkin2015}
	Е.М.~Четыркин. 
    \newblock {\em Финансовые риски: науч.-практич. пособие}.
    \newblock М.: Изд. дом «Дело» РАНХиГС, 2015.
  
  \end{thebibliography}
\end{frame}

\subsection*{Научные статьи}

\begin{frame}[shrink=15]
  \frametitle<presentation>{Научные статьи}
  \begin{thebibliography}{10}
  \beamertemplatearticlebibitems
  % Followed by interesting articles. Keep the list short. 
  \bibitem{Aven2010}
    \newblock T.~Aven 
    \newblock On how to define, understand and describe risk. {\em Reliability Engineering \& System Safety}. 2010. T.~95, №~6. C.~623-631.
  \bibitem{Aven2012}
    \newblock T.~Aven 
    \newblock The risk concept—historical and recent development trends.{\em Reliability Engineering \& System Safety}. 2012. T.~99. C.~33-44.
  \bibitem{Vankovich2014}
  	\newblock И.~М.~Ванькович
    \newblock Финансовые риски: теоретические и практические аспекты. {\em Российское предпринимательство}, 2014. T.~15, №~13.
 \bibitem{Bespalova2014}
    \newblock И.~Беспалова, Н.~Яшина.
	\newblock Финансовая модель управления рисками российских банков. {\em Современные проблемы науки и образования}, 2014. №~2.
  \end{thebibliography}
\end{frame}

\subsection*{Нормативные документы}

\begin{frame}[allowframebreaks]
  \frametitle<presentation>{Нормативные документы}
	\textbf{Управление рисками в банке}
  \begin{thebibliography}{10}
  
  \beamertemplatearticlebibitems
    \bibitem{bb1}
	Инструкция Банка России от 03.12.2012 №~139-И (ред. от 01.09.2015) "Об обязательных нормативах банков".
    \bibitem{bb2}
	Письмо Банка России от 02.10.2007 №~15-1-3-6/3995 «О международных подходах (стандартах) организации управления процентным риском».
    \bibitem{bb3}
	Письмо Банка России от 23.03.2007 №~26-Т «Методические рекомендации по проведению проверки системы управления банковскими рисками в кредитной организации (ее филиале)».
    \bibitem{bb4}
	Письмо Банка России от 23.06.2004 №~70-Т «О типичных банковских рисках».
    \bibitem{bb5}
	Письмо Банка России от 24.05.2005 №~76-Т «Об организации управления операционным риском в кредитных организациях».
    \bibitem{bb6}
	Положение о порядке расчета кредитными организациями величины рыночного риска (утв. Банком России 28.09.2012 №~387-П) (ред.~от~01.09.2015).
\pagebreak

	\textbf{Управление рисками профессиональных участников рынка ценных бумаг}
  \bibitem{ss1}
    Методика определения риск-параметров валютного рынка и рынка драгоценных металлов ОАО Московская Биржа
    \newblock (утв. решением Правления ЗАО АКБ «Национальный Клиринговый Центр» 8~октября~2013~г.).
  \bibitem{ss2}
	Методика определения риск-параметров рынка ценных бумаг    \newblock (утв. решением Правления ЗАО АКБ «Национальный Клиринговый Центр» 29~мая~2013~г.).
  \bibitem{ss3}
 	Приказ ФСФР от 11.10.2012 (ред. от 11.04.2013) №~12-87/пз-н  «Об утверждении положения о требованиях к клиринговой деятельности».
  \bibitem{ss4}
	Приказ ФСФР от 16.04.2013 №~13-30/пз-н  «Об утверждении требований к составу собственных средств организаторов торговли, а также к порядку и срокам их расчета».
  \bibitem{ss5}
 	Приказ ФСФР от 25.06.2013 № 13-53/пз-н  «Об утверждении требований к деятельности организатора торговли в части организации системы управления рисками и порядка осуществления внутреннего контроля, а также к отдельным внутренним документам организатора торговли».
  \bibitem{ss6}
	Указание Банка России от 21.07.2014 № 3329-У «О требованиях к собственным средствам профессиональных участников рынка ценных бумаг и управляющих компаний инвестиционных фондов, паевых инвестиционных фондов и негосударственных пенсионных фондов».

\pagebreak

	\textbf{Управление рисками страховщиков}
	
  \bibitem{sss1}
	Об организации страхового дела в Российской Федерации: Закон РФ от 27.11.1992 №~4015-1 (ред.~от~13.07.2015).
  \bibitem{sss2}
	Указание Банка России от 28.07.2015 №~3743-У "О порядке расчета страховой организацией нормативного соотношения собственных средств (капитала) и принятых обязательств".
  \end{thebibliography}
\end{frame}

\subsection*{Интернет ресурсы}
\begin{frame}[allowframebreaks]
  \frametitle<presentation>{Интернет ресурсы}
    
  \begin{thebibliography}{10}
  
  \setbeamertemplate{bibliography item}[online]

  \bibitem{micex}
    Московская биржа.
    \newblock http://rts.micex.ru/.
  \bibitem{dcc}
    Депозитарно-клиринговая компания.
    \newblock http://www.dcc.ru.
  \bibitem{cbr}
    Центральный банк Российской федерации.
    \newblock http://www.cbr.ru/ .
  \bibitem{gks}
    Росстат.
    \newblock http://www.gks.ru.
  \end{thebibliography}
\end{frame}

\subsection*{Научные журналы}

\begin{frame}
  \frametitle<presentation>{Журналы}
    
  \begin{thebibliography}{10}
  
  \beamertemplatearticlebibitems
  \bibitem{}
  {\em Деньги и кредит}
  \bibitem{}
  	{\em Финансы и кредит}
  \bibitem{}
  	{\em Экономический анализ: теория и практика}
  \bibitem{}
  	{\em Банковское дело}
  \bibitem{}
  	{\em Эксперт}
  
  \end{thebibliography}
\end{frame}

\subsection*{Газеты}

\begin{frame}
  \frametitle<presentation>{Газеты}
    
  \begin{thebibliography}{10}
  
  \beamertemplatearticlebibitems
  \bibitem{}
  	\newblock{\em Коммерсантъ}
  \bibitem{}
  	\newblock{\em Ведомости}
 
  \end{thebibliography}
\end{frame}


\end{document}