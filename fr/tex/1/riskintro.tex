\documentclass[financial_risks_lectures.tex]{subfiles}
\begin{document}

\subsection{Риск: определение}
\setbeamercovered{transparent}
\begin{frame}[shrink=15]{История возникновения понятия риска}
\begin{itemize}[<+->]
\item
\textbf{Французский \textit{\foreignlanguage{french}{risqué}}} - опастность или неудобство, как предсказуемое так и нет (1578 г., 1690 г. официальный термин).
\item
\textbf{Итальянский \textit{risco}} (1-ая половина XIV века) - возможность вреда, неприятных последствий и т.п.

\item
\textbf{Средневековая латынь \textit{resicum, risicum}} (сер. XII века) - слово использовалось в торговле в смысле ``угроза, опасность'', в морском деле - ``подводная скала, риф''.

\item
\textbf{Средневековый французский, испанский диалекты \textit{resicq, risicq, rezegue, risc, riesgo}} - возможность потери или повреждения товаров.

\item
\textbf{Арабский \textit{rizq }}- слово использовалось в разных смыслах ``надел данный Богом каждому человеку'', ``благодать, благославение (от Бога)'', ``богатство, собственность, доход'', ``доля, судьба, удача, шанс''.

\end{itemize}

\end{frame}

\begin{frame}{Основные теории происхождения слова риск, выводы}
\begin{itemize}
\item
В европейских языках слово ``риск'' изначально имело значение ``нечто могущее порезать'', иными словами, ``скала, риф, утес'', что имело отношение к морской торговле и в целом, к морскому делу. Как глагол использовалось в смысле ``хождение по морю в неизвестных водах или слишком близко от рифов''.
\item
Арабское и греческое происхождение слова ``риск'' указывает на связь с понятиями ``богатсво, доля, судьба, шанс''.

\end{itemize}
\end{frame}
\begin{frame}[shrink=10]{Повседневное использование слова ``риск''}
\begin{itemize}
\item
Подверженность возможности потерь, ущербу, повреждению или другим враждебным или неблагоприятным обстоятельствам; шанс или ситуация, заключающая в себе такую возможность.
\item
Опасное путешествие, предприятие, или образ действий, авантюра.
\item
Человек или вещь, способные произвести хороший или плохой результат в определенном смысле; также человек или вещь, рассматриваемые как угроза или источник ущерба.
\end{itemize}
Понятие ``риск'' может использоваться как в положительном, так и в отрицательном смысле, а также может быть как глаголом так и существительным.
\end{frame}
\begin{frame}[allowframebreaks]{Классификация определений понятия ``риск''}
1. Риск=Ожидаемая прибыль/убыток: риск это произведение вероятности убытка на размер потерь.

2. Риск=Вероятность ущерба или убытка.

3. Риск=Объективная неопределенность: риск это измеряемая неопределенность, т.е. неопределенность, для которой известно распределение исходов (статистически или на основе прошлого опыта).

4. Риск=Неопределенность: относительно издержек, потерь или ущерба.

5. Риск=Возможность/потенциал убытков: риск это возможность неприятных последствий от события.

\pagebreak

6. Риск=Вероятность и сценарии/Последствия/тяжесть последствий: риск это комбинация угроз и вероятности их наступления, риск это единица измерения одновременно вероятности и тяжести негативных последствий.

7. Риск=Событие или последствие: риск это ситуация, когда есть угроза для имущества или жизни людей, и результат является неопределенным.

\pagebreak

8. Риск=Последствия/Ущерб/и их тяжесть + неопределенность: риск это комбинация последствий от действий по отношению к каким-либо вещам, ценным для человека. Отклонения от целей, ожиданий.

9. Риск это влияние неопределенности на цели (ISO).

10. Риск это триада: события (сценарии), их вероятности, а также их последствия.

\end{frame}
\begin{frame}{Понятие риска}
\begin{block}{Риск}
\quad – это деятельность, связанная с преодолением неопределенности в ситуации неизбежного выбора, в процессе которой имеется возможность количественно и качественно оценить вероятность достижения предполагаемого результата, неудачи и отклонения от цели.
\end{block}
\end{frame}
\begin{frame}{Риск и неопределенность}
  \begin{itemize}
  \item
Различают задачи принятия решений при риске и соответственно в условиях неопределенности. Если существует возможность качественно и количественно определить степень вероятности того или иного варианта, то это и будет ситуация риска.
  \item
Разница между риском и неопределенностью относится к способу задания информации и определяется наличием (в случае риска) или отсутствием (при неопределенности) вероятностных характеристик неконтролируемых переменных. 
  \end{itemize}
\end{frame}

\begin{frame}{}
  \begin{block}{Ситуация риска (рискованная ситуация)}
  это разновидность неопределенности, когда наступление событий вероятно и может быть определено, т.е. в этом случае объективно существует возможность оценить вероятность событий, возникающих в результате совместной деятельности партнеров по производству, контрдействий конкурентов или противников, влияние природной среды на развитие экономики, внедрение достижений науки в народное хозяйство и т.д.
  \end{block}
\end{frame}

\begin{frame}{Характеристики рисковой ситуации}
\setbeamertemplate{itemize items}[circle]
  	\begin{itemize}[<+->]
	  \item
		случайный характер события, который определяет, какой из возможных исходов реализуется на практике (наличие неопределенности);
	  \item
		наличие альтернативных решений;
	  \item
		известны или можно определить вероятности исходов и ожидаемые результаты;
	  \item
		вероятность возникновения убытков;
	  \item
		вероятность получения дополнительной прибыли.
  \end{itemize}
\end{frame}

\begin{frame}[allowframebreaks]{Риск и его особенности}{}
  \begin{itemize}
  \item
  \textbf{Субъективная и объективная стороны риска.} Риск всегда связан с выбором определенных альтернатив (субъективная сторона) и расчетом вероятности их результата (объективная сторона).
  \item
  \textbf{Особенности предпринимательского риска.} Предприниматель проявляет готовность идти на риск в условиях неопределенности, поскольку наряду с риском потерь существует возможность дополнительных доходов. 
  
  Экономическое поведение предпринимателя при рыночных отношениях основано на выбираемой, на свой риск реализуемой индивидуальной программе предпринимательской деятельности в рамках возможностей, которые вытекают из законодательных актов. Каждый участник рыночных отношений изначально лишен заранее известных, однозначно заданных параметров, гарантий успеха: обеспеченной доли участия в рынке, доступности к производственным ресурсам по фиксированным ценам, устойчивости покупательной способности денежных единиц, неизменности норм и нормативов и других инструментов экономического управления.
  \item
  \textbf{Конструктивная форма предпринимательского риска.} Риск предпринимателя, как правило, ориентирован на получение значимых результатов нетрадиционными методами. Тем самым он позволяет преодолеть консерватизм, догматизм, косность, психологические барьеры, препятствующие перспективным нововведениям. Способность рисковать — один из путей успешной деятельности предпринимателя. В этом проявляется конструктивная форма регулятивной функции предпринимательского риска.
  \item
  \textbf{Риск и авантюризм.} Вместе с тем риск может стать проявлением авантюризма, если решение принимается в условиях неполной информации, без должного учета закономерностей развития явления. В этом случае риск выступает в качестве дестабилизирующего фактора.
  \item
  \textbf{Риск и инновации.} Большинство фирм, компаний добиваются успеха, становятся конкурентоспособными на основе инновационной экономической деятельности, связанной с риском.
	\item
\textbf{Риск и неудачи. }Инициативным, предприимчивым хозяйственникам нужны правовые, политические и экономические гарантии, исключающие в случае неудачи наказание и стимулирующие оправданный риск. Предприниматель должен быть уверен, что возможная ошибка (риск) не может скомпрометировать ни его дело, ни его имидж, так как она произошла вследствие не оправдавшего себя, хотя и рассчитанного риска.
  \end{itemize}
\end{frame}

\begin{frame}{Управление риском}{Выбор оптимального решения в условиях риска}
  \begin{itemize}
  \item
	Основной задачей предпринимателя является не отказ от риска вообще, а выборы решений, связанных с риском на основе объективных критериев, а именно: до каких пределов может действовать предприниматель, идя на риск.
  \item
	Использование специальных методов анализа в сложных ситуациях.
  \end{itemize}
\end{frame}
\subsection{Классификация рисков}

% You can reveal the parts of a slide one at a time
% with the \pause command:
\begin{frame}[shrink=10]{Виды рисков по характеру потерь:}{Чистые риски}
\setbeamercovered{transparent}
	\textbf{Чистые риски.} Несут в себе только потери для предпринимательской деятельности. Их причинами могут быть стихийные бедствия, несчастные случаи, недееспособность руководителей фирм и др.:
	\setbeamertemplate{itemize items}[circle]
	\begin{itemize}
		\item<1-> {природно-естественные риски – это риски связанные с проявлением стихийных сил природы;
		}
		\item<2->{
экологические риски связаны с наступлением гражданской ответственности за нанесение ущерба окружающей среде;
}
  		\item<3-> {
политические риски – это возможность возникновения убытков или сокращения размеров прибыли, являющихся следствием государственной политики;
}
  		\item<4-> {
транспортные риски связаны с перевозками грузов различными видами транспорта.
}
	\end{itemize}  
\end{frame}

\begin{frame}{Виды рисков по характеру потерь:}{Спекулятивные риски}
\begin{block}{\textbf{Спекулятивные риски.}}
Несут в себе либо потери, либо дополнительную прибыль для предпринимателя. Их причинами могут быть изменение курсов валют, изменение конъюнктуры рынка, изменение условий инвестиций и др.
\end{block}
\end{frame}

\begin{frame}[shrink=15]{Виды рисков по сфере возникновения}
\begin{block}{Коммерческий риск}
\quad– это риск потерь в процессе финансово-хозяйственной деятельности; его причинами могут быть снижение объемов реализации, непредвиденное снижение объемов закупок, повышение закупочной цены товара, повышение издержек обращения, потери товара в процессе обращения и др.
\end{block}
Виды коммерческих рисков
\begin{itemize}
\item
\textbf{имущественные риски }– это риски от потери имущества предпринимателя по причинам от него не зависящим;
\item
\textbf{имущественные риски }зависят от убытков по причине задержки платежей, не поставки товара, отказа от платежа и т.п;
\item
\textbf{риск упущенной выгоды }заключается в том, что возникает финансовый ущерб в результате неосуществления некоторого мероприятия
\end{itemize}
\end{frame}

\begin{frame}{Финансовый риск}
\begin{block}{\textbf{Финансовый риск}}
\quad - это возможность потерь в результате изменения стоимости финансовых активов или других изменений в финансовом секторе экономики. Причинами здесь могут быть изменение покупательной способности денег, неосуществление платежей, изменение валютных курсов и т.п.
\end{block}
\end{frame}

\begin{frame}[allowframebreaks]{Виды финансовых рисков}
\begin{itemize}
\item
	инфляционные риски, которые обусловлены обесцениванием реальной покупательной способности денег, при этом предприниматель несет реальные потери;
\item
	дефляционный риск связан с тем, что при росте дефляции падает уровень цен и, следовательно, снижаются доходы;
\item
	валютные риски связаны с изменением валютных курсов;
\item
	риск ликвидности связан с потерями при реализации ценных бумаг или других товаров из-за изменения оценки их качества и потребительской стоимости;
\item
	процентный риск, возникающий в результате превышения процентных ставок, выплачиваемых по привлеченным средствам, над ставками по предоставленным кредитам;
\item
	кредитный риск, возникающий в случае неуплаты заемщиком основного долга и процентов, причитающихся кредитору;
\pagebreak
\item
	биржевые риски представляют собой опасность потерь от биржевых сделок;
\item
	инвестиционные риски возникают из-за неправильного выбора направления вложения капиталов, вида ценных бумаг для инвестирования;
\item
	риск банкротства связан с полной потерей предпринимателем собственного капитала из-за его неправильного вложения.
\end{itemize}
\end{frame}


\subsection{Неопределенность}
\begin{frame}{Понятие неопределенности}
\begin{block}{\textbf{Неопределенность}}
\quad – это неполное или неточное представление о значениях различных параметров в будущем, порождаемых различными причинами и, прежде всего, неполнотой или неточностью информации об условиях реализации решения, в том числе связанных с ними затратах и результатах. 
\end{block}

\end{frame}



\begin{frame}{Связь неопределенности и риска}
\begin{block}{Неопределенность и риск}
Неопределенность, связанная с возможностью возникновения в ходе реализации решения неблагоприятных ситуаций и последствий, характеризуется понятием риск.
\end{block}
\textbf{Ключевые моменты риска:}
\begin{itemize}
\item
 наличие неопределенного исхода события, вероятностные характеристики которого известны;
 \item
 возможность потерь, при неблагоприятном развитии событий.
\end{itemize}
\end{frame}

\begin{frame}{Неопределенность как причина возникновения экономического риска}
\textbf{Рискованная ситуация} связана со статистическими процессами и ей сопутствуют три сосуществующих условия: \begin{inparaenum}[\itshape a\upshape)]
\item наличие неопределенности; \item необходимость выбора альтернативы и \item возможность качественной и количественной оценки вероятности осуществления того или иного варианта\end{inparaenum}.

\end{frame}

\begin{frame}{Неопределенности по факторам возникновения }{Экономические (коммерческие) и политические}
\begin{itemize}
\item
\textbf{Экономические неопределенности }обусловлены неблагоприятными изменениями в экономике предприятия или в экономике страны, к ним относятся: неопределенность рыночного спроса, слабая предсказуемость рыночных цен, неопределенность рыночного предложения, недостаточность информации о действиях конкурентов и т.д. 
\item
\textbf{Политические неопределенности }обусловлены изменением политической обстановки, влияющей на предпринимательскую деятельность. 

\end{itemize}
\end{frame}

\begin{frame}[allowframebreaks]{Неопределенность в зависимости от вероятности выпадения событий}
\begin{itemize}
\item
\textbf{Полная неопределенность }характеризуется близкой к нулю прогнозируемостью наступления события.
\item
\textbf{Полная определенность} - это близкая к единице прогнозированность событий.
Это возможно, прежде всего, в тех случаях, когда при решении задачи в условиях неопределенности определяется с какой-то вероятностью оптимальное решение и с заранее известной вероятностью (обычно равной 0,9 – 0,99) находится доверительный прогнозируемый интервал, позволяющий прогнозировать не только свою стратегию на рынке, а и его собственное поведение, тенденции развития и т.п.
\item
\textbf{Частичная неопределенность }отвечает таким событиям, прогнозируемость которых лежит в пределах от 0 до 1.
\end{itemize}
\end{frame}

\begin{frame}[allowframebreaks]{Прочие виды неопределенностей}
\begin{itemize}
\item
\textbf{Природная неопределенность }описывается совокупностью факторов, среди которых могут быть: климатические, погодные условия, различного рода помехи (атмосферные, электромагнитные и др.).
\item
\textbf{Неопределенность внешней среды.} Внутренняя среда предприятия включает факторы, обусловленные деятельностью самого предпринимателя и его контактами. Внешняя среда представлена факторами, которые не связаны непосредственно с деятельностью предпринимателя и имеют более широкий социальный, демографический, политический и иной характер.
\item
\textbf{Конфликтные ситуации}, в качестве которых могут быть: стратегия и тактика лиц, участвующих в том или ином конкурсе, действия конкурентов, ценовая политика олигополистов и т.п. Проблемы несовпадающих интересов и многокритериального выбора оптимальных решений в условиях неопределенности
\end{itemize}
\end{frame}

\begin{frame}{Неопределенность и принятие экономических решений}
На практике, при проведении экономического анализа, во многих случаях игнорируются обстоятельства, вызванные фактором неопределенности и действуют (принимают решение) на основе детерминированных моделей. Иначе говоря, предполагается, что факторы, влияющие на принимаемые решения, известны точно. К сожалению, действительность часто не соответствует таким представлениям. Поэтому политика выбора эффективных решений без учета неконтролируемых факторов во многих случаях приводит к значительным потерям экономического, социального и иного содержания.
\end{frame}

\subsection{Управление риском}
\begin{frame}{Пределы риска}
\begin{block}{Допустимый предел риска}
- это уровень риска в пределах его среднего уровня, то есть среднего по отношению к другим видам деятельности и другим хозяйственным субъектам. 
\end{block}
\end{frame}

\begin{frame}[shrink=15]{Уровни риска}
\begin{itemize}
\item
\textbf{Допустимый риск}
\begin{align}
R_D<R,
\end{align}
где $R$ – средний уровень риска в экономике\\
$R_D$ - допустимый риск.
\item
\textbf{Критический риск $R_{critical}$ }понимается риск, уровень которого выше среднего, но в пределах максимально допустимых значений риска $R_{max}$, принятых в данной экономической системе для определенных видов деятельности, т.е.
\begin{align}
R_D<R_{critical}<R_{max}.
\end{align}
\item
\textbf{Катастрофический риск $R_{catastrophic}$} – это такой риск, который превышает максимальную границу риска $R_{max}$, сложившуюся в данной экономической системе, и для которого выполняется условие
\begin{align}
R_{catastrophic}>R_{max}.
\end{align} 
\end{itemize}
\end{frame}

\begin{frame}{Задачи системы управления риском}
\begin{itemize}[<+->]
\itemобеспечить максимальную сохранность собственных средств; 
\item
минимизировать отрицательное воздействие внешних и внутренних факторов;
\item
повысить ответственность перед клиентами, контрагентами и инвесторами.
\end{itemize}
\end{frame}

\begin{frame}[ allowframebreaks ]{Принципы управления рисками}
\begin{itemize}
\item
не рисковать, если есть такая возможность;
\item
не рисковать больше, чем это может позволить собственный капитал банка;
\item
думать о последствиях риска и не рисковать многим ради малого;
\item
не создавать рисковых ситуаций ради получения сверхприбыли;
\item
держать риски под контролем;
\item
снижать риски, распределяя их среди клиентов и участников по видам деятельности;
\pagebreak
\item
создавать необходимые резервы для покрытия рисков;
\item
устанавливать постоянное наблюдение за изменением рисков;
\item
количественно измерять уровень принимаемых рисков;
\item
определять новые источники и критические зоны риска и (или) групп операций с повышенным уровнем риска.
\end{itemize}
\end{frame}

\subsection{Анализ и контроль риска}
\begin{frame}[shrink=10]{Схема анализа риска}
\begin{center}

\tikzstyle{block} = [draw, fill=white, rectangle, 
    minimum height=1.5em, minimum width=6em]
\begin{tikzpicture}[auto, node distance=0.3cm,>=latex']

    \node [block] (ranalys) {Анализ риска}; 
	\node [block, below left=of ranalys, node distance=0.3cm] (identification) {Выявление};       
	\node [block, below right=of ranalys, node distance=0.3cm] (evaluation) {Оценка};       
	\node [block, anchor=center, below=1cm of ranalys, text width=4.5cm,align=center] (methods) {Выбор методов\\воздействия на риск\\при сравнении их\\эффективности};       
	\node [block, below=0.2cm of methods, node distance=0.3cm] (solution) {Принятие решения};       
	\node [block, below=0.2cm of solution, node distance=0.3cm] (influence) {Воздействие на риск};       
	\node [block, below=0.2cm of influence, node distance=0.3cm] (save) {Сохранение};       
	\node [block, below right=of influence, node distance=0.3cm] (reduce) {Уменьшение};       
	\node [block, below left=of influence, node distance=0.3cm] (convey) {Передача};       
	\node [block, below=0.2cm of save, node distance=0.3cm] (control) {Контроль результатов};       

    \draw [-] (ranalys) -- (identification);
    \draw [-] (ranalys) -- (evaluation);
    \draw [-] (evaluation) -- (methods);
    \draw [-] (identification) -- (methods);
    \draw [-] (methods) -- (solution);
    \draw [-] (solution) -- (influence);
    \draw [-] (influence) -- (save);
    \draw [-] (influence) -- (reduce.north);
    \draw [-] (influence) -- (convey.north);
    \draw [-] (control) -- (reduce.south);
    \draw [-] (control) -- (convey.south);
    \draw [-] (save) -- (control);

\end{tikzpicture}
\end{center}
\end{frame}

\begin{frame}[shrink=10]{Этапы анализа риска}
\begin{itemize}
\item
Анализ риска, предполагающий численное определение отдельных рисков и риска проекта (решения) в целом и  допустимого в уровень риска данного проекта.
\item
Результат: картина возможных рисковых событий, вероятность их наступления и последствий. 
\item
Стратегия управления риском, предполагающая меры предотвращения и уменьшения риска.
\end{itemize}
\begin{block}{Величина банковского риска }
\quad — это стоимостная оценка подверженности риску, которая может выражаться как максимальная сумма, которую можно потерять в результате изменения факторов риска за определенный период времени.
\end{block}
\end{frame}

\begin{frame}{Подходы к анализу риска}
\begin{itemize}[<+->]
\item
\textbf{Качественный анализ. }Словесное описание уровня риска (например, «проекту присвоен высокий уровень риска») путем выявления негативной информации, на основании которой взвешиваются (оцениваются) негативные факторы, влияющие на величину риска.
\item
\textbf{Количественный анализ.} Величина риска в относительном выражении (например, «максимальные потери по проекту составят 50\% от суммы кредита»). Относительное выражение риска в виде установления допустимого уровня при совершении различных операций применяется при выработке политики банка.
\end{itemize}
\end{frame}

\begin{frame}{Контроль риска }
\quad включает в себя все меры, направленные на своевременное выявление риска с целью его снижения или исключения. 

\textbf{Способы контроля риска:}

\begin{itemize}[<+->]
\item
Внутренний аудит;
\item
Внешний аудит;
\item
Внутренний контроль.
\end{itemize}
\end{frame}

\begin{frame}[shrink=5]{Внутренний аудит}
\begin{block}{Внутренний аудит}
\quad осуществляется внутренним структурным подразделением коммерческой организации: ревизионные комиссии, внутренние аудиторы в рамках существующей в банке системы внутреннего контроля.
\end{block}

\begin{itemize}[<+->]
\item
обеспечение эффективности финансово-хозяйственной деятельности;
\item
обеспечение достоверности, полноты, объективности и своевременности составления и представления отчетности, а также информационной безопасности;
\item
соблюдение законодательства;
\item
борьба с легализацией (отмыванием) доходов.
\end{itemize}
\end{frame}

\begin{frame}{Внешний аудит}
Надзор за деятельностью кредитных организаций осуществляется центральным аппаратом и территориальными учреждениями Банка России (надзорные органы).

Ежегодный обязательный аудит в соответствии со статьей 5 Федерального закона от 30.12.2008 г. №307-ФЗ «Об аудиторской деятельности».
\end{frame}

\begin{frame}{Внутренний контроль}
\begin{itemize}[<+->]
\item
мониторинга и оценки эффективности политики управления рисками;
\item
расследования причин возникновения убытков, фактов наступления событий или обстоятельств, приводящих к убыткам (в т.ч. в ходе проведения мониторинга системы внутреннего контроля);
\item
участия в разработке предложений и мероприятий по оптимизации бизнес-процессов и внутрибанковских процедур с целью минимизации рисков;
\item
контроля за рисками новых продуктов и совершением рисковых сделок.
\end{itemize}
\end{frame}
\subsection{Способы снижения риска}
\begin{frame}{Способы снижения риска}
\begin{itemize}
\item
отказ от риска;
\item
снижение риска (резервирование, диверсификация, лимитирование, минимизация);
\item
передача риска третьему лицу (страхование, хеджирование, распределение).
\end{itemize}
\end{frame}

\begin{frame}[ allowframebreaks ]{Меры по снижению риска}
\begin{itemize}
\item
\textbf{Резервирование }является одним из основных способов управления банковским риском. Банк имеет собственные средства (капитал), а также формирует обязательные резервы на возможные потери по ссудам и прочим активам, относимые на расходы кредитной организации.
\item
\textbf{Диверсификация }— это процесс распределения инвестируемых средств между различными объектами вложения капитала, которые непосредственно не связаны между собой, с целью снижения степени риска. 
\item
\textbf{Лимитирование }- это установление предельных сумм расходов по различным банковским операциям (например, кредиты , не превышающие 100 млрд. руб.). Процесс установления лимитов допустимой величины риска должен быть гибким, основываеться на изучении рынка, суждении и опыте банковских аналитиков.
\item
\textbf{Минимизация (нивелирование) риска} снижение вероятности наступления событий или обстоятельств, приводящих к убыткам и сокращение потенциальных убытков. Мера реализуется механизмами внутреннего контроля, эффективна до возникновения реальных убытков.
\end{itemize}
\end{frame}

\begin{frame}[ allowframebreaks ]{Варианты передачи риска третьему лицу}
\begin{itemize}
\item
\textbf{Страхование. }При страховании обеспечения от утраты или повреждения, риски передаются страховщику или гаранту. Данный метод наиболее подходит для снижения кредитных рисков. Страхование требует отвлечения дополнительных ресурсов в виде уплаты страховой премии страховщику, при этом страхователем выступает заемщик, а выгодоприобретателем – банк.
\item
\textbf{Хеджирование} - это передача риска участникам финансового рынка путем заключения сделок с использованием производных финансовых инструментов (форварды, фьючерсы, опционы, свопы и т.д.). Как и при страховании, хеджирование требует отвлечения дополнительных ресурсов в виде уплаты опционной премии или внесения депозита, применяется для снижения рыночных рисков.

\item
\textbf{Распределение } риска между участниками кредитной сделки в виде его включения в стоимость услуг: в процентную ставку (рисковая надбавка), комиссию, штрафные санкции и т.д. 
\end{itemize}
\end{frame}

\subsection{Финансовые риски}
\begin{frame}[shrink=10]{Особенности финансовых рисков}
\setbeamercovered{transparent}
\begin{block}{Эффект финансового рычага (ЭФР)}
\quad есть приращение к рентабельности собственных средств, получаемое благодаря использованию кредита, несмотря на платность последнего.
\end{block}
\begin{itemize}[<+->]
\item
С помощью ЭФР акционеры компаний повышают коэффициенты доходности использования своего капитала (ROE), но при этом повышается чувствительность этого коэффициента к колебаниям эффективности в производственной деятельности фирмы, которая измеряется коэффициентом доходности активов (ROA). 
\item
Используя ЭФР, акционеры фирмы подвергаются как финансовому, так и производственному (операционному) риску.

\end{itemize}
\setbeamercovered{invisible}
\end{frame}

\begin{frame}{Модель фирмы Дюпон}
\begin{align}
\label{dupon}
GPM\times AT \times FL= ROA\times FL=ROE
\end{align}

\begin{tabular}{l}
GPM (Gross Profit Margin) - валовая маржа;\\
AT (Asset Turnover) - оборачиваемость активов;\\
ROA (Return On Assets) - рентабельность активов;\\
FL (Financial Leverage) - финансовый рычаг;\\
ROE (Return On Equity) - рентабельность капитала.\\
\end{tabular}
\end{frame}

\begin{frame}[shrink=30]{Модель фирмы Дюпон}
\begin{exampleblock}{Пример}
% Table generated by Excel2LaTeX from sheet 'Лист1'
\begin{table}[htbp]
  \centering
  %\caption{Add caption}
    \begin{tabular}{p{2cm}ccr}
    \toprule
    \multicolumn{1}{c}{Коэффициент} & Формула & Компания* & \multicolumn{1}{c}{Знач.} \\
    \midrule
    \multicolumn{1}{l}{\multirow{2}[4]{*}{GPM}} & \multirow{2}[4]{*}{ЧП/Вр} & А     & 0,03 \\
    \multicolumn{1}{l}{} &       & Б     & 0,3 \\
    \multicolumn{1}{l}{\multirow{2}[4]{*}{AT}} & \multirow{2}[4]{*}{Вр/ВБ} & А     & 5 \\
    \multicolumn{1}{l}{} &       & Б     & 0,5 \\
    \multicolumn{1}{l}{\multirow{2}[4]{*}{\textbf{ROA}}} & \multirow{2}[4]{*}{\textbf{$\times$}} & А     & \textbf{0,15} \\
    \multicolumn{1}{l}{} &       & Б     & \textbf{0,15} \\
    \multicolumn{1}{l}{\multirow{2}[4]{*}{L}} & \multirow{2}[4]{*}{ВБ/СК} & А     & 3 \\
    \multicolumn{1}{l}{} &       & Б     & 3 \\
    \multicolumn{1}{l}{\multirow{2}[4]{*}{\textbf{ROE}}} & \multirow{2}[4]{*}{\textbf{$\times$}} & А     & \textbf{4,5} \\
    \multicolumn{1}{l}{} &       & Б     & \textbf{4,5} \\
    \bottomrule
    \multicolumn{4}{l}{\begin{tabular}[c]{@{}l@{}}*А – сеть супермаркетов; Б – предприятие\\ коммунальных услуг. \end{tabular} }\\ 
    \multicolumn{4}{l}{ЧП – чистая прибыль;}\\ 
    \multicolumn{4}{l}{Вр – выручка;}\\ 
    \multicolumn{4}{l}{ВБ – валюта баланса;}\\ 
    \multicolumn{4}{l}{СК – собственный капитал.}\\ 
   
    \end{tabular}%
  \label{tab:addlabel}%
\end{table}%
\end{exampleblock}
\end{frame}

\begin{frame}
\begin{itemize}
\item
Сеть супермаркетов имеет низкий коэффициент доходности продаж (3\%) и достигает 15\% показателя доходности активов, оборачивая свои активы пять раз в год. Для капиталоемкой сферы коммунальных услуг коэффициент оборачиваемости составляет всего 0,5 раз в год и 15\%-ый коэффициент доходности активов достигается благодаря тому, что ее коэффициент доходности продаж составляет 30\%.
\item
Низкие коэффициенты доходности продаж или оборачиваемости активов вовсе не означают, что у фирмы существуют финансовые проблемы (риски).

\end{itemize}
\end{frame}

\begin{frame}[shrink=12]{Уровень операционного рычага}{DOL, Degree of Operating Leverage}
\setbeamercovered{transparent}
\begin{align}
\label{degree_of_operating_leverage}
\text{DOL} & \nonumber
=\frac{\text{Вр} - \text{ПерЗ}}
{\text{Вр} - \text{ПерЗ} - \text{ПостЗ}}\\ \nonumber  & 
\onslide<2->{=
\frac{\text{Вр} - \text{ПерЗ}}{\text{ВП}}
\\  &}
\onslide<3->{=
\frac{\text{ВП} + \text{ПостЗ}}{\text{ВП}}
}\end{align}
	Вр - выручка;\\
	ПерЗ - переменные затраты;\\
	ПостЗ - постоянные затраты;\\
    ВП  - Валовая прибыль или Прибыль до уплаты процентов и налогов (EBIT, Earnings before interest and taxes);
\end{frame}

\begin{frame}[shrink=15]{Эффект операционного (производственного, хозяйственного) рычага (ЭОР)}
\begin{itemize}
\item
Любое изменение выручки от реализации всегда порождает более сильное изменение прибыли:
\begin{align}
\label{degree_of_financial_leverage1}
\text{ЭОР} &
= \text{Предельный доход} \times \frac{\text{Выручка}-\text{ПерЗ}}{\text{ВП}}
\end{align}
\item
Операционный риск, связанный с конкретным бизнесом в его рыночной нише, обусловлен неустойчивостью спроса и цен на готовую продукцию, а также цен сырья и энергии, недостаточной себестоимостью для получения приемлемой прибыли и т.п.
\end{itemize}
\end{frame}

\begin{frame}{Эффект финансового рычага (ЭФР)}
Эффект финансового рычага показывает, на сколько процентов увеличивается рентабельность собственного капитала за счет привлечения заемных средств. 
\begin{align}
\label{degree_of_financial_leverage2}
\text{ЭФР}=(ROA-r_{\text{ПК}}) \times (1-T)\times \frac{\text{ЗС}}{\text{СС}}
\end{align}
Рекомендуемое значение ЭФР равняется 0.33 - 0.5.
\end{frame}

\begin{frame}{Изменение чистой прибыли на акцию}
\begin{align}
\Delta\text{ЧПа} &= \frac{\Delta\text{ВП} - \Delta\text{ПК}}{N_{\text{акций}}}\times (1 - T)\\
\Delta\text{ЧПа} &= \frac{\Delta\text{ВП}}{N_{\text{акций}}}\times (1 - T)
\end{align}
$\Delta\text{ПК}=0$, т.к. сумма процентов за кредит в рассматриваемый промежуток времени не меняется
\end{frame}

\begin{frame}[shrink=20]{Уровень финансового рычага}
\setbeamercovered{transparent}
\begin{align}
\text{DFL} & = \frac{\Delta\text{ЧПа}}{\text{ЧПа}}\div\frac{\Delta\text{ВП}}{\text{ВП}}
\end{align}
\begin{align}
\text{DFL} & \nonumber = \frac{\frac{\Delta\text{ВП}\times(1 - T)}{N_{\text{акций}}}}{\frac{(\text{ВП}-\text{ПК})\times(1 - T)}{N_{\text{акций}}}}\div\frac{\Delta\text{ВП}}{\text{ВП}}
\\ \nonumber &
\onslide<2->{
=\frac{\Delta\text{ВП}}{\text{ВП}-\text{ПК}}\times \frac{\text{ВП}}{\Delta\text{ВП}}\\ \nonumber &
}
\onslide<3->{
=\frac{\text{ВП}}{\text{ВП}-\text{ПК}}\\ \nonumber &
}
\onslide<4->{
=\frac{\text{ПДН}+\text{ПК}}{\text{ПДН}}\\ &
}
\onslide<5->{
=
1+\frac{\text{ПК}}{\text{ПДН}}
}
\end{align}
\setbeamercovered{invisible}
При DFL = 1 предприятие не имеет заемных средств.

\end{frame}
\begin{frame}[shrink=10]{Произведение финансового и операционного рычагов}

\begin{align}
\text{DOL} & = \frac{\text{ВП}+\text{Постоянные издержки}}{\text{ВП}}\\
\text{DFL} & = \frac{\text{ВП}}{\text{ВП}-\text{ПК}}\\
\text{DCL} & = \text{DOL}\times \text{DFL} = \frac{\text{ВП}+\text{Постоянные издержки}}{\text{ВП}-\text{ПК}}
\end{align}

    DOL - Уровень операционного рычага (Degree of Operating Leverage);\\
    DFL - Уровень финансового рычага (Degree of Financial Leverage)\\
    DCL - Уровень совокупного рычага (Degree of Combined Leverage).
%\end{align*}
\end{frame}
\begin{frame}{Если в бухгалтерской отчетности не разделены постоянные и переменные издержки}{}
Уровень совокупного рычага определяется по формуле:
\begin{align}
\text{DCL}=\frac{\Delta\text{ВП}}{\text{ВП}}\div \frac{\Delta\text{Выр}}{\text{Выр}}
\end{align}
Уровень операционного рычага определяется по формуле:
\begin{align}
\text{DOL}=\frac{DCL}{DFL}
\end{align}

\end{frame}
\begin{frame}[shrink=10]{Финансовый и операционный риски}{Совокупный риск}
\begin{itemize}
\item
\textbf{Финансовый рычаг} приводит к неустойчивости финансовых условий кредитования, неуверенность владельцев обыкновенных акций в их надежности, инфляционные процессы при кредитовании и т.п., что создает финансовый риск.
\item
\textbf{Операционный риск}, связанный с конкретным бизнесом в его рыночной нише, обусловлен неустойчивостью спроса и цен на готовую продукцию, а также цен сырья и энергии, недостаточной себестоимостью для получения приемлемой прибыли и т.п.
\end{itemize}
\textit{Совместное воздействие этих двух рисков дает совокупный риск, который может оказаться губительным для компании, так как предпринимательский и финансовый риски взаимно умножаются, усиливая неблагоприятные эффекты.
}
\end{frame}

\begin{frame}[shrink=10]{Расчет финансового риска}

\begin{exampleblock}{Компания А. Баланс}
% Table generated by Excel2LaTeX from sheet 'Лист1'
\begin{table}[htbp]
  \centering
%  \caption{Баланс}
    \begin{tabular}{lr}
    \toprule
    Статья & Сумма \\
    \midrule
    Активы, руб. &       1 000 000    \\
    Долг, руб. &          500 000    \\
    Собственные средства, руб. &          500 000    \\
    \textit{Справочно:} &  \\
    Количество выпущенных акций, руб. &       1 000 000    \\
    \bottomrule
    \end{tabular}%
  \label{tab:addlabel}%
\end{table}

\end{exampleblock}

\end{frame}

\setbeamercovered{invisible}
\begin{frame}[shrink=15]{Расчет финансового риска}

\begin{exampleblock}{Компания А. Отчет о прибылях и убытках}
% Table generated by Excel2LaTeX from sheet 'Лист1'
\begin{table}[htbp]
  \centering
%  \caption{Отчет о прибылях и убытках}
    \begin{tabular}{lr}
    \toprule
    Статья & Сумма \\
    \midrule
    Выручка, руб. &    10 000 000    \\
    Постоянные затраты, руб. &          500 000    \\
    Переменные затраты, руб. &       9 000 000    \\
    Валовая прибыль, руб. &          500 000    \\
    Проценты к уплате, руб. & \onslide<2->{75 000}    \\
    Прибыль до налогообложения, руб. & \onslide<3->{425 000}    \\
    Сумма налога на прибыль, руб. &\onslide<4->{76 500}    \\
    Чистая прибыль, руб. &\onslide<5->{348 500} \\
    \textit{Справочно:} &  \\
    Ставка процента по кредитам, \% & 15,00\% \\
    Ставка налога на прибыль, \% & 18\% \\
    Предельный доход, руб. &              1 000    \\
    \bottomrule
    \end{tabular}%
  \label{tab:addlabel}%
\end{table}

\end{exampleblock}

\end{frame}

\begin{frame}[shrink=10]{Расчет финансового риска}

\begin{exampleblock}{Компания А. Расчет коэффициентов рычага}
% Table generated by Excel2LaTeX from sheet 'Лист1'
\begin{table}[htbp]
  \centering
%  \caption{Расчет коэффициентов рычага}
    \begin{tabular}{lr}
    \toprule
    Коэффициент & Значение\\
    \midrule
    ROA, \% & \onslide<2->{34,85\%} \\
    ROE, \% & \onslide<3->{69,70\%} \\
    ЭОР, руб. & \onslide<4->{2 000}    \\
    ЭФР, \%пкт & \onslide<5-> {16,28\%} \\
    DOL, \% & \onslide<6->{200,00\%} \\
    DFL, \% & \onslide<7->{117,65\%} \\
    DCL, \% & \onslide<8->{235,29\%} \\
	ЧПа, \% & \onslide<9->{34,85\%} \\
    \bottomrule
    \end{tabular}%
  \label{tab:addlabel}%
\end{table}%
\end{exampleblock}
\end{frame}
\setbeamercovered{transparent}
\end{document}