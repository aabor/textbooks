\documentclass[12pt,a4paper]{article}

\usepackage[rus]{borochkin_article}

\begin{document}
\huge{Контрольная работа 1.}
\normalsize
\section{Вероятность убытков}
\begin{taskrus}
Средняя доходность фондового индекса равна 15\% годовых, стандартное отклонение доходности 24\%. Предполагается, что доходность имеет нормальное распределение. Инвестор формирует портфель, копирующий данный индекс. Определить вероятность того, что в следующем году портфель принесет ему убыток.

\textbf{Решение.}

Инвестор получит убыток, если доходность портфеля окажется меньше нуля. Согласно формуле 
\begin{align}
\label{prob_loss}
P(\alpha<r<\beta)=\Phi\left(\frac{\beta-\overline{x}}{\sigma}\right)-\Phi\left(\frac{\alpha-\overline{x}}{\sigma}\right),
\end{align}
вероятность того, что доходность актива окажется меньше нуля равна
\begin{align*}
P(-\infty<r<0)&=\Phi\left(\frac{0-15}{24}\right)-\Phi\left(\frac{-\infty-15}{24}\right)\\
&=\Phi(-0.625)-\Phi(-\infty)\\
&=0.266-0\\
&=26.6\%.
\end{align*}

\textbf{Варианты }
% Table generated by Excel2LaTeX from sheet 'Задача4'
\begin{table}[htbp]
  \centering
  \caption{Определение вероятности убытков}
    \begin{tabular}{lrrrr}
    \toprule
          & Вар1  & Вар2  & Вар3  & Вар4 \\
    \midrule
    Средняя доходность фондового индекса & 15\%  & 18\%  & 13\%  & 16\% \\
    Стандартное отклонение доходности & 24\%  & 15\%  & 30\%  & 35\% \\
    \midrule
    \textbf{Решение} &       &       &       &  \\
    \midrule
    Вероятность убытка в следующем году & 26,60\% & 11,51\% & 33,24\% & 32,38\% \\
    \bottomrule
    \end{tabular}%
  \label{tab:addlabel}%
\end{table}%

\end{taskrus}

\section{Доверительный интервал доходности, если известно истинное значение ст. отклонения доходности актива}
\begin{taskrus}
Доходность актива имеет нормальное распределение. На основе наблюдений за~252 дня была определена ожидаемая доходность в расчете на день. Она составила 0.87\%. Пусть известно, что истинное значение стандартного отклонения доходности актива в расчете на день равно 1,6\%. В каком интервале с надежностью 0,9 располагается истинное значение ожидаемой доходности актива?

\textbf{Решение.}

По данным статистики было определено не истинное значение ожидаемой доходности актива, а ее точечная оценка на основе выборочных данных. Для получения ответа о надежности такой оценки определяют доверительный интервал, который с заданным уровнем вероятности накрывает точечную оценку. В результате, с заданным уровнем надежности можно быть уверенным, что действительное значение ожидаемой доходности актива лежит в границах рассчитанного интервала.

В примере необходимо определить доверительный интервал для нормально распределенной случайной величины при известном значении ее истинного стандартного отклонения с коэффициентом доверия 0.9.

Верхнюю и нижнюю границы доверительного интервала для математического ожидания с известной дисперсией можно определить по следующим формулам:
\begin{align}
\label{eq:yield_low}
\overline{r}_L &=\overline{r}-\frac{\sigma}{\sqrt{n}}u_{1-\alpha/2},\\[8pt]
\label{eq:yield_upper}
\overline{r}_U &=\overline{r}+\frac{\sigma}{\sqrt{n}}u_{1-\alpha/2},
\end{align}
где

$\overline{r}_L, \overline{r}_U$ - нижняя и верхняя  границы доверительного интервала;

$\overline{r}$ - точечная оценка ожидаемой доходности на основе осуществленной выборки;

$\sigma$ - истинное значение стандартного отклонения доходности актива;

$n$ - объем выборки;

$u_{1-\alpha/2}$ - квантиль уровня $1-\alpha/2$ стандартного нормального распределения;

$\alpha$ - уровень значимости, соответствующий выбранной доверительной вероятности $\gamma$.

Из соотношения $\alpha=1-\gamma$ находим значение $\alpha$, соответствующее коэффициенту доверия 90\%.
$$\alpha=1-0.9=0.1$$
По таблице квантилей нормального распределения находим квантили $u_{1-\alpha/2}=u_{0.95}=1,65$.
По формулам \eqref{eq:yield_low} \eqref{eq:yield_upper} получаем:
\begin{align*}
\overline{r}_L&=0.704\%, \\
\overline{r}_U&=1.036\%
\end{align*}
\textbf{Варианты}
% Table generated by Excel2LaTeX from sheet 'З2_2'
\begin{table}[H]
  \centering
  \caption{Доверительный интервал доходности, если известно истинное значение ст. отклонения доходности актива}
    \begin{tabular}{lrrrr}
    \toprule
          & Вар1  & Вар2  & Вар3  & Вар4 \\
    \midrule
    \specialcell{Ожидаемая доходность\\в расчете на день } & 0,87\% & 1,60\% & 0,95\% & 2,10\% \\
    Количество наблюдений & 252   & 500   & 90    & 120 \\
    \specialcell{Истинное значение\\стандартного отклонения доходности\\в расчете на день} & 1,60\% & 3,10\% & 1,50\% & 5,50\% \\
    Уровень надежности гамма & 90\%  & 95\%  & 98\%  & 95\% \\
    \midrule
    Решение &       &       &       &  \\
    \midrule
    Доверительная вероятность альфа & 10\%  & 5\%   & 2\%   & 5\% \\
    Квантиль ст. норм. распред. & 1,644854 & 1,959964 & 2,326348 & 1,959964 \\
    Нижний доверительный интервал & 0,704\% & 1,328\% & 0,582\% & 1,116\% \\
    Верхний доверительный интервал & 1,036\% & 1,872\% & 1,318\% & 3,084\% \\
    \bottomrule
    \end{tabular}%
  \label{tab:addlabel}%
\end{table}%

\end{taskrus}

\section{Доверительный интервал доходности актива, если не известно значение стандартного отклонения актива}
\begin{taskrus}
Доходность актива имеет нормальное распределение. Данные о его доходности за прошедшие 10 месяцев представлены в таблице
% Table generated by Excel2LaTeX from sheet 'Лист2'
\begin{table}[H]
  \centering
  \caption{Доходность актива за 10 месяцев}
    \begin{tabular}{lrrrrrrrrrr}
    \toprule
    Месяцы & 1     & 2     & 3     & 4     & 5     & 6     & 7     & 8     & 9     & 10 \\
    \midrule
    Доходность & 4     & 1     & -1,5  & 0,5   & 3     & -2    & -1    & 2,8   & 1,2   & 1,5 \\
    \bottomrule
    \end{tabular}%
  \label{tab:addlabel}%
\end{table}%
Определите ожидаемую доходность актива. В каком интервале с надежностью 0,9 располагается истинное значение ожидаемой доходности актива?

\textbf{Решение.}

Ожидаемая доходность актива равна
$$\overline{r}=0.95\%,$$
В данном примере не известно истинное значение стандартного отклонения актива, поэтому при определении доверительного интервала используем правило математической статистики для математического ожидания нормально распределенной случайной 
величины при неизвестной дисперсии.

Верхнюю и нижнюю границы доверительного интервала для математического ожидания с неизвестной дисперсией можно определить по следующим формулам:
\begin{align}
\label{yield_low_corrected}
\overline{r}_L &=\overline{r}-\frac{s}{\sqrt{n}}t_{1-\alpha/2;n-1},\\[8pt]
\label{yield_upper_corrected}
\overline{r}_U &=\overline{r}+\frac{s}{\sqrt{n}}t_{1-\alpha/2;n-1},
\end{align}
где

$s$- исправленное стандартное отклонение;

$\alpha$ - уровень значимости, соответствующий выбранной доверительной вероятности $\gamma$, $\alpha=1-\gamma$.

$t_{1-\alpha/2;n-1}$- квантиль уровня $1-\alpha/2$ распределения Стьюдента с $n-1$ степенями свободы.

Исправленное стандартное отклонение равно:
\begin{align}
\label{st_dev_corrected}
s&=\sqrt{\frac{1}{n-1}\sum_{i=1}^n \left(r_i-\overline{r}\right)^2}\\
s&=2\% \nonumber
\end{align}
По  таблице квантилей распределения Стьюдента находим
значение статистики $t_{1-\alpha/2;n-1}=t_{0.95;9}=1.83$ и по формулам \eqref{yield_low_corrected} и \eqref{yield_upper_corrected} определяем границы доходности актива:
\begin{align*}
\overline{r}_L &=-0.209\%,\\
\overline{r}_U &=2,109\%.
\end{align*}
Доверителышй интервал равен: -0,209\%; 2.109\%.

\textbf{Варианты}
% Table generated by Excel2LaTeX from sheet 'З1_3'
\begin{table}[H]
  \centering
  \caption{Варианты}
    \begin{tabular}{lrrrrrrrrrr}
    \toprule
    \multicolumn{9}{c}{Вариант 3}                                             &       &  \\
\midrule
    \multicolumn{1}{c}{\multirow{2}[1]{*}{}} & \multicolumn{10}{c}{Период} \\ \cmidrule{2-11}
    \multicolumn{1}{c}{} & 1     & 2     & 3     & 4     & 5     & 6     & 7     & 8     & 9     & 10 \\
    \multicolumn{1}{l}{Доходность актива} & 18    & 11    & 8     & 13    & 18    & 25    & 16    & 7     & 10    & 8 \\
    \midrule
    \multicolumn{9}{c}{Вариант 4}                                             &       &  \\
\midrule
    \multicolumn{1}{c}{\multirow{2}[1]{*}{}} & \multicolumn{9}{c}{Период}                                            &  \\ \cmidrule{2-10}
    \multicolumn{1}{c}{} & 1     & 2     & 3     & 4     & 5     & 6     & 7     & 8     & 9     &  \\
    \multicolumn{1}{l}{Доходность актива} & 7     & 27    & 5     & -4    & -7    & 5     & -12   & -15   & -8    &  \\
    \bottomrule
    \end{tabular}%
  \label{tab:addlabel}%
\end{table}%

% Table generated by Excel2LaTeX from sheet 'З1_3'
\begin{table}[H]
  \centering
  \caption{Определение доверительного интервала доходности актива, если не известно значение стандартного отклонения актива
}
    \begin{tabular}{lrrrrr}
    \toprule
          &       & Вар1  & Вар2  & Вар3  & Вар4 \\
    \midrule
    Количество наблюдений &       & 10    & 8     & 10    & 9 \\
    Уровень надежности гамма &       & 90\%  & 95\%  & 98\%  & 95\% \\
    \midrule
    Решение &       &       &       &       &  \\
    \midrule
    Доверительная вероятность альфа &       & 10\%  & 5\%   & 2\%   & 5\% \\
    Ожидаемая доходность актива &       & 0,950\% & 4,500\% & 13,400\% & -0,222\% \\
    \specialcell{Исправленное ст. откл.\\дох-ти актива} &       & 2,000\% & 6,719\% & 5,777\% & 12,872\% \\
    Квантиль распред. Стьюдента &       & 1,833113 & 2,364624 & 2,821438 & 2,306004 \\
    Нижний доверительный интервал &       & -0,209\% & -1,117\% & 8,245\% & -10,117\% \\
    Верхний доверительный интервал &       & 2,109\% & 10,117\% & 18,555\% & 9,672\% \\
    \bottomrule
    \end{tabular}%
  \label{tab:addlabel}%
\end{table}%
\end{taskrus}


\section{Оценка границ дисперсии доходности актива при неизвестном математическом ожидании доходности}
\begin{taskrus}К/р. 
Доходность актива имеет нормальное распределение. На основе наблюдений за 31 день было рассчитано исправленное стандартное отклонение в расчете на день. Оно составило 1,5\%. В каком интервале с надежностью 0.95 располагается истинное значение дисперсии и стандартного отклонения доходности актива?

\textbf{Решение.}

Верхнюю и нижнюю границы доверительного интервала для дисперсии можно определить по следующим формулам:

\begin{align}
\sigma_L^2&=\frac{s^2(n-1)}{\chi_{1-\alpha/2;n-1}^2};\\[8pt]
\sigma_U^2&=\frac{s^2(n-1)}{\chi_{\alpha/2;n-1}^2};
\end{align}
где

$\sigma_L^2$, $\sigma_U^2$ - нижняя и верхняя границы доверительного интервала, соответственно;

$s^2$ - исправленная дисперсия доходности актива;

$n$ - объем выборки;

$\alpha$ - уровень значимости, соответствующий выбранной доверительной вероятности $\gamma$, $\alpha=1-\gamma$. 

$\chi_{\alpha/2;n-1}^2$ - $\alpha/2$-квантиль распределения хи-квадрат с $n-1$ степенями свободы.

Значение $\alpha$, соответствующее коэффициенту доверия 95\%.
$$\alpha=1-0.95=0.05.$$

Количество наблюдений случайной величины составило 31 день. Поэтому количество степеней свободы в примере равно 30. По таблице квантилей распределения $\chi^2$ находим квантили 46.98 и 16.79.

Границы доверительного интервала дисперсии доходности актива равны 1.44 и 4.02 и стандартного отклонения доходности актива равны 1.199\% и 2,005\%.


% Table generated by Excel2LaTeX from sheet 'З1_4'
\begin{table}[H]
  \centering
  \caption{Оценка границ дисперсии доходности актива при неизвестном математическом ожидании доходности}
    \begin{tabular}{lrrrrr}
    \toprule
          &       & Вар1  & Вар2  & Вар3  & Вар4 \\
    \midrule
    \specialcell{Исправленное ст. отклонение\\доходности актива} &       & 1,50\% & 2,50\% & 3,50\% & 1,50\% \\
    Объем выборки, дней &       & 31    & 41    & 51    & 61 \\
    Уровень надежности &       & 95\%  & 95\%  & 99\%  & 90\% \\
    Решение &       &       &       &       &  \\
    \multicolumn{1}{l}{\multirow{2}[0]{*}{\specialcell{Квантили распределения\\хи-квадрат}}} & Нижний & 46,979 & 59,342 & 79,490 & 79,082 \\
    \multicolumn{1}{l}{} & Верхний & 16,791 & 24,433 & 27,991 & 43,188 \\
    \multicolumn{1}{l}{\multirow{2}[0]{*}{\specialcell{Границы доверительного интервала\\дисперсии доходности}}} & Нижний & 1,437 & 4,213 & 7,705 & 1,707 \\
    \multicolumn{1}{l}{} & Верхний & 4,020 & 10,232 & 21,882 & 3,126 \\
    \multicolumn{1}{l}{\multirow{2}[0]{*}{\specialcell{Границы доверительного интервала ст.\\отклонения доходности, \%}}} & Нижний & 1,199 & 2,053 & 2,776 & 1,307 \\
    \multicolumn{1}{l}{} & Верхний & 2,005 & 3,199 & 4,678 & 1,768 \\
    \bottomrule
    \end{tabular}%
  \label{tab:addlabel}%
\end{table}%

\end{taskrus}

\section{Проверка гипотезы о равенстве дисперсий}
\begin{taskrus}
Доходности активов имеют нормальное распределение. На основе данных о доходности активов \textit{X} за 61 день и \textit{Y} за 51 день были рассчитаны исправленные стандартные отклонения доходности: $s_X=1,74\%$, $s_Y = 1,56\%$. Проверить гипотезу о равенстве дисперсий активов при уровне значимости 0,05.

\textbf{Решение.}

При проверке гипотезы о равенстве дисперсий рассматривают две гипотезы: $H_0: \sigma_X^2=\sigma_Y^2$, и $H_1: \sigma_X^2 > \sigma_Y^2$, $H_0$ - это основная (нулевая) гипотеза, $H_1$, - альтернативная гипотеза. 
Основная гипотеза говорит о том, что дисперсии доходности активов равны. Альтернативная гипотеза говорит о том, что они не равны. Если в результате проверки гипотеза $H_0$ отклоняется в пользу гипотезы $H_1$, то это означает, что дисперсии активов отличаются. Если гипотеза $H_0$ не отклоняется, то нет оснований отрицать равенство дисперсий.

В качестве критерия проверки гипотезы используют случайную величину, которая представляет собой отношение большей исправленной дисперсии к меньшей:
\begin{align}
\label{dev_equal_test}
F=\frac{s_X^2}{s_Y^2}
\end{align}
где
$F$ - случайная величина, имеющая распределение Фишера со степенями свободы $n-1$ и $k-1$;

$n$ - объем выборки доходности актива~$X$;

$k$ - объем выборки доходности актива~$Y$.

Если $F<f_{1-\alpha;n-1;k-1}$, где $f_{1-\alpha;n-1;k-1}$ - квантиль распределения Фишера, то параметр $F$ попадает в область принятия гипотезы, и нулевая гипотеза принимается на уровне значимости $\alpha$.

Если  $F>=f_{1-\alpha;n-1;k-1}$ параметр $F$ попадает в критическую область. Следовательно $H_0$, отклоняется в пользу гипотезы $H_1$.

Рассчитаем значение параметра $F$ согласно формуле \eqref{dev_equal_test}:
$$F=\frac{1.74^2}{1.56^2}=1.244$$

По таблице квантилей распределения Фишера находим $f_{1-0.05;61-1;61-1}=f_{0.95;60;50}=1.576$. 

Поскольку $1.244<1.576$, нулевая гипотеза принимается, дисперсии активов равны.

% Table generated by Excel2LaTeX from sheet 'З1_5'
\begin{table}[H]
  \centering
  \caption{Проверка гипотезы о равенстве дисперсий}
    \begin{tabular}{lcrrrr}
    \toprule
          &       & Вар1  & Вар2  & Вар3  & Вар4 \\
    \midrule
    \multicolumn{1}{c}{\multirow{2}[0]{*}{\specialcell{Стандартные отклонения\\доходности активов}}} & X     & 1,74\% & 1,50\% & 5,30\% & 0,69\% \\
    \multicolumn{1}{c}{} & Y     & 1,56\% & 1,25\% & 4,30\% & 0,56\% \\
    \multicolumn{1}{l}{\multirow{2}[0]{*}{Объем выборки}} & X     & 61    & 31    & 41    & 21 \\
    \multicolumn{1}{l}{} & Y     & 51    & 91    & 81    & 101 \\
    Уровень значимости &       & 5\%   & 10\%  & 5\%   & 10\% \\
    Решение &       &       &       &       &  \\
    Величина F &       &    1,244    &    1,440    &    1,519    &    1,518    \\
    Статистика Фишера &       & 1,576 & 1,432 & 1,545 & 1,494 \\
    Равны ли дисперсии? &       & Да    & Нет   & Да    & Нет \\
    \bottomrule
    \end{tabular}%
  \label{tab:addlabel}%
\end{table}%

\end{taskrus}

\section{Проверка гипотезы значимости коэфф. корреляции доходностей активов}
\begin{taskrus}
\label{task_sign_corrXY}
Доходность активов имеет нормальное распределение. Доходности активов \textit{X} и \textit{Y} за 8 периодов представлены в таблице.
 % Table generated by Excel2LaTeX from sheet 'Задача3'
 \begin{table}[H]
   \centering
   \caption{Доходность активов}
     \begin{tabular}{lcrrrrrrrr}
     \toprule
     \multicolumn{2}{c}{\multirow{2}[1]{*}{}} & \multicolumn{8}{c}{Период} \\ \cmidrule{3-10}
     \multicolumn{2}{c}{} & 1 & 2 & 3 & 4 & 5 & 6 & 7 & 8 \\
     \midrule
     \multicolumn{1}{l}{\multirow{2}[1]{*}{Доходность актива}} & X & 8 & 10 & 11 & 7 & -5 & -3 & 4 & 8 \\
     \multicolumn{1}{l}{} & Y & 12 & 15 & 10 & 11 & -2 & -4 & -3 & 5 \\
     \bottomrule
     \end{tabular}%
   \label{tab:addlabel}%
 \end{table}%


Определить коэффициент выборочной корреляции доходности активов. Проверить гипотезу о значимости коэффициента корреляции при уровне значимости 0.1.

\textbf{Решение.}

Коэффициент выборочной ковариации определяется по формуле:
\begin{align}
\label{covxy}
cov_{XY}=\frac{\sum_{i=1}^n \left(r_{X_i}-\overline{r} \right)\left(r_{Y_i}-\overline{r} \right)}{n},
\end{align}
где 

$r_{X_i}, r_{Y_i}$ - доходности активов \textit{X} и \textit{Y}, соответственно.

Коэффициент выборочной ковариации по формуле \eqref{covxy} равен 32,875.

Выборочные стандартные отклонения активов составляют 5.568, 7.089.

Коэффициент корреляции 0,833.

Основная гипотеза говорит о том, что коэффициент корреляции равен нулю. Альтернативная гипотеза говорит о том, что коэффициент корреляции не равен нулю.

Для проверки нулевой гипотезы на основе выборки строится статистика:
\begin{align}
\label{studentxy}
T=\frac{corr_{XY}\sqrt{n-2}}{\sqrt{1-corr_{XY}^2}}
\end{align}
Величина \textit{Т} имеет распределение Стьюдента с \textit{n-2} степенями свободы. Если рассчитанное на основе \eqref{studentxy} значение критерия \textit{Т }принадлежит критической области, то нулевую гипотезу отклоняют. 
При условии $\left|\frac{corr_{XY}\sqrt{n-2}}{\sqrt{1-corr_{XY}^2}}\right|<t_{1-\alpha/2;n-2}$ параметр \textit{T} попадает в область принятия нулевой гипотезы, следовательно, нулевая гипотеза принимается на уровне значимости $\alpha$.

При условии $\left|\frac{corr_{XY}\sqrt{n-2}}{\sqrt{1-corr_{XY}^2}}\right|\geq t_{1-\alpha/2;n-2}$ параметр \textit{T} попадает в критическую область, следовательно нулевая гипотеза отклоняется в пользу альтернативной гипотезы на уровне значимости $\alpha$.

Значение параметра \textit{T} равно 3.688, по таблице квантилей распределения Стьюдента находим $t_{0.95;6}=1.943$.

Значение критерия попадает в критическую область. Поэтому мы не принимаем нулевую гипотезу, т.е. коэффициент корреляции статистически значим.

\textbf{Варианты}

% Table generated by Excel2LaTeX from sheet 'Задача3'
\begin{table}[H]
  \centering
  \caption{Вариант 1}
    \begin{tabular}{rrrrrrrrrrrr}
    \toprule
    \multicolumn{2}{c}{\multirow{2}[1]{*}{}} & \multicolumn{10}{c}{Период} \\\cmidrule{3-12}
    \multicolumn{2}{c}{} & 1     & 2     & 3     & 4     & 5     & 6     & 7     & 8     & 9     & 10 \\
    \midrule
    \multicolumn{1}{l}{\multirow{2}[1]{*}{Доходность актива}} & X     & 18    & 11    & 8     & 13    & 18    & 25    & 16    & 7     & 10    & 8 \\
    \multicolumn{1}{l}{} & Y     & -4    & 15    & 18    & 1     & 17    & -2    & 10    & -3    & -5    & 3 \\
    \bottomrule
    \end{tabular}%
  \label{tab:addlabel}%
\end{table}%

% Table generated by Excel2LaTeX from sheet 'Задача3'
\begin{table}[H]
  \centering
  \caption{Вариант 2}
    \begin{tabular}{rrrrrrrrrrr}
    \toprule
    \multicolumn{2}{c}{\multirow{2}[1]{*}{}} & \multicolumn{9}{c}{Период} \\\cmidrule{3-11}
    \multicolumn{2}{c}{} & 1     & 2     & 3     & 4     & 5     & 6     & 7     & 8     & 9 \\
    \midrule
    \multicolumn{1}{l}{\multirow{2}[1]{*}{Доходность актива}} & X     & 7     & 27    & 5     & -4    & -7    & 5     & -12   & -15   & -8 \\
    \multicolumn{1}{l}{} & Y     & 14    & 0     & 32    & -11   & 5     & 8     & 4     & -3    & -2 \\
    \bottomrule
    \end{tabular}%
  \label{tab:addlabel}%
\end{table}%


\textbf{Решение}

% Table generated by Excel2LaTeX from sheet 'Задача3'
\begin{table}[H]
  \centering
  \caption{Определение значимости коэффициента корреляции}
    \begin{tabular}{lcrrrr}
    \toprule
          & Компания       & Вар1  & Вар2  & Вар3  & Вар4 \\
    \midrule
    Уровень значимости &       & 10\%  & 5\%   & 5\%   & 50\% \\
    Число степеней свободы &       & 10    & 8     & 10    & 9 \\
    \midrule
    \textbf{Решение }&       &       &       &       &  \\
    \midrule
	\specialcell{Коэффициент\\ выборочной ковариации}&       & 32,875 & -11,125 & -5,100 & 36,716 \\
\multicolumn{1}{l}{\multirow{2}[0]{*}{Стандартные отклонения}} & X     & 5,568 & 6,285 & 5,481 & 12,136 \\
    \multicolumn{1}{l}{} & Y     & 7,089 & 8,062 & 8,672 & 11,612 \\
    Коэффициент корреляции &       & 0,833 & -0,220 & -0,107 & 0,261 \\
    Параметр \textit{T }&       & 4,258 & -0,551 & -0,305 & 0,714 \\
    \specialcell{Статистика Стьюдента }&       & 1,860 & 2,447 & 2,306 & 0,711 \\
    \specialcell{Коэффициент корреляции\\ значим? }&       & Да    & Нет   & Нет   & Да \\
    \bottomrule
    \end{tabular}%
  \label{tab:addlabel}%
\end{table}%
\end{taskrus}

\end{document}