\documentclass[12pt,a4paper]{article}

\usepackage[rus]{borochkin_article}

\begin{document}
\section{Контрольная работа}
\normalsize
\begin{taskrus}
Средняя доходность фондового индекса равна 15\% годовых, стандартное отклонение доходности 24\%. Предполагается, что доходность имеет нормальное распределение. Инвестор формирует портфель, копирующий данный индекс. Определить вероятность того, что в следующем году портфель принесет ему убыток.
% Table generated by Excel2LaTeX from sheet 'З1_1'
\begin{table}[H]
  \centering
  \caption{Определение вероятности убытков при инвестициях в фондовый индекс}
    \begin{tabular}{lrrrr}
    \toprule
          & Вар1  & Вар2  & Вар3  & Вар4 \\
    \midrule
    Средняя доходность фондового индекса & 15\%  & 18\%  & 13\%  & 16\% \\
    Стандартное отклонение доходности & 24\%  & 15\%  & 30\%  & 35\% \\
    \midrule
    Вероятность убытка в следующем году & 26,60\% & 11,51\% & 33,24\% & 32,38\% \\
    \bottomrule
    \end{tabular}%
  \label{tab:addlabel}%
\end{table}%
\end{taskrus}

\begin{taskrus}
Доходность актива имеет нормальное распределение. На основе наблюдений за~252 дня была определена ожидаемая доходность в расчете на день. Она составила 0.87\%. Пусть известно, что истинное значение стандартного отклонения доходности актива в расчете на день равно 1,6\%. В каком интервале с надежностью 0,9 располагается истинное значение ожидаемой доходности актива?
% Table generated by Excel2LaTeX from sheet 'З2_2'
\begin{table}[H]
  \centering
  \caption{Доверительный интервал доходности, если известно истинное значение ст. отклонения доходности актива}
    \begin{tabular}{lrrrr}
    \toprule
          & Вар1  & Вар2  & Вар3  & Вар4 \\
    \midrule
    \specialcell{Ожидаемая доходность\\в расчете на день } & 0,87\% & 1,60\% & 0,95\% & 2,10\% \\
    Количество наблюдений & 252   & 500   & 90    & 120 \\
    \specialcell{Истинное значение\\стандартного отклонения доходности\\в расчете на день} & 1,60\% & 3,10\% & 1,50\% & 5,50\% \\
    Уровень надежности гамма & 90\%  & 95\%  & 98\%  & 95\% \\
    \midrule
    Доверительная вероятность альфа & 10\%  & 5\%   & 2\%   & 5\% \\
    Квантиль ст. норм. распред. & 1,644854 & 1,959964 & 2,326348 & 1,959964 \\
    Нижний доверительный интервал & 0,704\% & 1,328\% & 0,582\% & 1,116\% \\
    Верхний доверительный интервал & 1,036\% & 1,872\% & 1,318\% & 3,084\% \\
    \bottomrule
    \end{tabular}%
  \label{tab:addlabel}%
\end{table}%
\end{taskrus}

\pagebreak

\begin{taskrus}
Доходность актива имеет нормальное распределение. Данные о его доходности за прошедшие 10 месяцев представлены в таблице
Определите ожидаемую доходность актива. В каком интервале с надежностью 0,9 располагается истинное значение ожидаемой доходности актива?
% Table generated by Excel2LaTeX from sheet 'З1_3'
\begin{table}[H]
  \centering
  \caption{Определение доверительного интервала доходности актива, если не известно значение стандартного отклонения актива
}
    \begin{tabular}{lrrrrr}
    \toprule
          &       & Вар1  & Вар2  & Вар3  & Вар4 \\
    \midrule
    Количество наблюдений &       & 10    & 8     & 10    & 9 \\
    Уровень надежности гамма &       & 90\%  & 95\%  & 98\%  & 95\% \\
    \midrule
    Решение &       &       &       &       &  \\
    \midrule
    Доверительная вероятность альфа &       & 10\%  & 5\%   & 2\%   & 5\% \\
    Ожидаемая доходность актива &       & 0,950\% & 4,500\% & 13,400\% & -0,222\% \\
    \specialcell{Исправленное ст. откл.\\дох-ти актива} &       & 2,000\% & 6,719\% & 5,777\% & 12,872\% \\
    Квантиль распред. Стьюдента &       & 1,833113 & 2,364624 & 2,821438 & 2,306004 \\
    Нижний доверительный интервал &       & -0,209\% & -1,117\% & 8,245\% & -10,117\% \\
    Верхний доверительный интервал &       & 2,109\% & 10,117\% & 18,555\% & 9,672\% \\
    \bottomrule
    \end{tabular}%
  \label{tab:addlabel}%
\end{table}%
\end{taskrus}

\begin{taskrus}
Доходность актива имеет нормальное распределение. На основе наблюдений за 31 день было рассчитано исправленное стандартное отклонение в расчете на день. Оно составило 1,5\%. В каком интервале с надежностью 0.95 располагается истинное значение дисперсии и стандартного отклонения доходности актива?
% Table generated by Excel2LaTeX from sheet 'З1_4'
\begin{table}[H]
  \centering
  \caption{Оценка границ дисперсии доходности актива при неизвестном математическом ожидании доходности}
    \begin{tabular}{lrrrrr}
    \toprule
          &       & Вар1  & Вар2  & Вар3  & Вар4 \\
    \midrule
    \specialcell{Исправленное ст. отклонение\\доходности актива} &       & 1,50\% & 2,50\% & 3,50\% & 1,50\% \\
    Объем выборки, дней &       & 31    & 41    & 51    & 61 \\
    Уровень надежности &       & 95\%  & 95\%  & 99\%  & 90\% \\
    Решение &       &       &       &       &  \\
    \multicolumn{1}{l}{\multirow{2}[0]{*}{\specialcell{Квантили распределения\\хи-квадрат}}} & Нижний & 46,979 & 59,342 & 79,490 & 79,082 \\
    \multicolumn{1}{l}{} & Верхний & 16,791 & 24,433 & 27,991 & 43,188 \\
    \multicolumn{1}{l}{\multirow{2}[0]{*}{\specialcell{Границы доверительного интервала\\дисперсии доходности}}} & Нижний & 1,437 & 4,213 & 7,705 & 1,707 \\
    \multicolumn{1}{l}{} & Верхний & 4,020 & 10,232 & 21,882 & 3,126 \\
    \multicolumn{1}{l}{\multirow{2}[0]{*}{\specialcell{Границы доверительного интервала ст.\\отклонения доходности, \%}}} & Нижний & 1,199 & 2,053 & 2,776 & 1,307 \\
    \multicolumn{1}{l}{} & Верхний & 2,005 & 3,199 & 4,678 & 1,768 \\
    \bottomrule
    \end{tabular}%
  \label{tab:addlabel}%
\end{table}%
\end{taskrus}

\pagebreak

\begin{taskrus}
Доходности активов имеют нормальное распределение. На основе данных о доходности активов \textit{X} за 61 день и \textit{Y} за 51 день были рассчитаны исправленные стандартные отклонения доходности: $s_X=1,74\%$, $s_Y = 1,56\%$. Проверить гипотезу о равенстве дисперсий активов при уровне значимости 0,05.
% Table generated by Excel2LaTeX from sheet 'З1_5'
\begin{table}[H]
  \centering
  \caption{Проверка гипотезы о равенстве дисперсий}
    \begin{tabular}{lcrrrr}
    \toprule
          &       & Вар1  & Вар2  & Вар3  & Вар4 \\
    \midrule
    \multicolumn{1}{c}{\multirow{2}[0]{*}{\specialcell{Стандартные отклонения\\доходности активов}}} & X     & 1,74\% & 1,50\% & 5,30\% & 0,69\% \\
    \multicolumn{1}{c}{} & Y     & 1,56\% & 1,25\% & 4,30\% & 0,56\% \\
    \multicolumn{1}{l}{\multirow{2}[0]{*}{Объем выборки}} & X     & 61    & 31    & 41    & 21 \\
    \multicolumn{1}{l}{} & Y     & 51    & 91    & 81    & 101 \\
    Уровень значимости &       & 5\%   & 10\%  & 5\%   & 10\% \\
    Решение &       &       &       &       &  \\
    Величина F &       &    1,244    &    1,440    &    1,519    &    1,518    \\
    Статистика Фишера &       & 1,576 & 1,432 & 1,545 & 1,494 \\
    Равны ли дисперсии? &       & Да    & Нет   & Да    & Нет \\
    \bottomrule
    \end{tabular}%
  \label{tab:addlabel}%
\end{table}%
\end{taskrus}
\begin{taskrus}
\label{task_sign_corrXY}
Доходность активов имеет нормальное распределение. Доходности активов \textit{X} и \textit{Y} за 8 периодов представлены в таблице.
Определить коэффициент выборочной корреляции доходности активов. Проверить гипотезу о значимости коэффициента корреляции при уровне значимости 0.1.
% Table generated by Excel2LaTeX from sheet 'Задача3'
\begin{table}[H]
  \centering
  \caption{Определение значимости коэффициента корреляции}
    \begin{tabular}{lcrrrr}
    \toprule
          & Компания       & Вар1  & Вар2  & Вар3  & Вар4 \\
    \midrule
    Уровень значимости &       & 10\%  & 5\%   & 5\%   & 50\% \\
    Число степеней свободы &       & 10    & 8     & 10    & 9 \\
    \midrule
    \textbf{Решение }&       &       &       &       &  \\
    \midrule
	\specialcell{Коэффициент\\ выборочной ковариации}&       & 32,875 & -11,125 & -5,100 & 36,716 \\
\multicolumn{1}{l}{\multirow{2}[0]{*}{Стандартные отклонения}} & X     & 5,568 & 6,285 & 5,481 & 12,136 \\
    \multicolumn{1}{l}{} & Y     & 7,089 & 8,062 & 8,672 & 11,612 \\
    Коэффициент корреляции &       & 0,833 & -0,220 & -0,107 & 0,261 \\
    Параметр \textit{T }&       & 4,258 & -0,551 & -0,305 & 0,714 \\
    \specialcell{Статистика Стьюдента }&       & 1,860 & 2,447 & 2,306 & 0,711 \\
    \specialcell{Коэффициент корреляции\\ значим? }&       & Да    & Нет   & Нет   & Да \\
    \bottomrule
    \end{tabular}%
  \label{tab:addlabel}%
\end{table}%
\end{taskrus}
\end{document}