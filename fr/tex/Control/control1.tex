\documentclass[12pt,a4paper]{article}

\usepackage[rus]{borochkin_article}
  
\begin{document}
\section{Вариант}
\begin{taskrus}
Средняя доходность фондового индекса равна 13\% годовых, стандартное отклонение доходности 30\%. Предполагается, что доходность имеет нормальное распределение. Инвестор формирует портфель, копирующий данный индекс. Определить вероятность того, что в следующем году портфель принесет ему убыток.
\end{taskrus}

\begin{taskrus}
Доходность актива имеет нормальное распределение. На основе наблюдений за 90 дней была определена ожидаемая доходность в расчете на день. Она составила 0.95\%. Пусть известно, что истинное значение стандартного отклонения доходности актива в расчете на день равно 1,5\%. В каком интервале с надежностью 0,98 располагается истинное значение ожидаемой доходности актива?
\end{taskrus}

\begin{taskrus}
Доходность актива имеет нормальное распределение. Данные о его доходности за прошедшие 10 месяцев представлены в таблице
% Table generated by Excel2LaTeX from sheet 'З1_3'
\begin{table}[H]
  \centering
  \caption{Доходность актива за 10 месяцев}
    \begin{tabular}{lrrrrrrrrrr}
    \toprule
    \multicolumn{1}{c}{} & \multicolumn{10}{c}{Период} \\\cmidrule{2-11}
    \multicolumn{1}{c}{} & 1     & 2     & 3     & 4     & 5     & 6     & 7     & 8     & 9     & 10 \\
    \midrule
    \multicolumn{1}{l}{Доходность актива} & 18    & 11    & 8     & 13    & 18    & 25    & 16    & 7     & 10    & 8 \\
    \bottomrule
    \end{tabular}%
  \label{tab:addlabel}%
\end{table}%
Определите ожидаемую доходность актива. В каком интервале с надежностью 0,9 располагается истинное значение ожидаемой доходности актива?

\end{taskrus}

\begin{taskrus}
Доходность актива имеет нормальное распределение. На основе наблюдений за 51 день было рассчитано исправленное стандартное отклонение в расчете на день. Оно составило 3,5\%. В каком интервале с надежностью 0.99 располагается истинное значение дисперсии и стандартного отклонения доходности актива?
\end{taskrus}

\begin{taskrus}
Доходности активов имеют нормальное распределение. На основе данных о доходности активов \textit{X} за 41 день и \textit{Y} за 81 день были рассчитаны исправленные стандартные отклонения доходности: $s_X=5,3\%$, $s_Y = 4,3\%$. Проверить гипотезу о равенстве дисперсий активов при уровне значимости 0,05.
\end{taskrus}

\begin{taskrus} 
\label{task_sign_corrXY}
Доходность активов имеет нормальное распределение. Доходности активов \textit{X} и \textit{Y} за 10 периодов представлены в таблице.
% Table generated by Excel2LaTeX from sheet 'Задача3'
\begin{table}[H]
  \centering
  \caption{Доходность активов}
    \begin{tabular}{rrrrrrrrrrrr}
    \toprule
    \multicolumn{2}{c}{\multirow{2}[1]{*}{}} & \multicolumn{10}{c}{Период} \\\cmidrule{3-12}
    \multicolumn{2}{c}{} & 1     & 2     & 3     & 4     & 5     & 6     & 7     & 8     & 9     & 10 \\
    \midrule
    \multicolumn{1}{l}{\multirow{2}[1]{*}{Доходность актива}} & X     & 18    & 11    & 8     & 13    & 18    & 25    & 16    & 7     & 10    & 8 \\
    \multicolumn{1}{l}{} & Y     & -4    & 15    & 18    & 1     & 17    & -2    & 10    & -3    & -5    & 3 \\
    \bottomrule
    \end{tabular}%
  \label{tab:addlabel}%
\end{table}%
Определить коэффициент выборочной корреляции доходности активов. Проверить гипотезу о значимости коэффициента корреляции при уровне значимости 0.05.
\end{taskrus}


\pagebreak
\section{Вариант}
\begin{taskrus}
Средняя доходность фондового индекса равна 16\% годовых, стандартное отклонение доходности 35\%. Предполагается, что доходность имеет нормальное распределение. Инвестор формирует портфель, копирующий данный индекс. Определить вероятность того, что в следующем году портфель принесет ему убыток.

\end{taskrus}
\begin{taskrus}
Доходность актива имеет нормальное распределение. На основе наблюдений за 120 дней была определена ожидаемая доходность в расчете на день. Она составила 2.1\%. Пусть известно, что истинное значение стандартного отклонения доходности актива в расчете на день равно 5,5\%. В каком интервале с надежностью 0,95 располагается истинное значение ожидаемой доходности актива?
\end{taskrus}

\begin{taskrus}
Доходность актива имеет нормальное распределение. Данные о его доходности за прошедшие 9 месяцев представлены в таблице
% Table generated by Excel2LaTeX from sheet 'З1_3'
\begin{table}[H]
  \centering
  \caption{Доходность актива за 9 месяцев}
    \begin{tabular}{rrrrrrrrrr}
    \toprule
    \multicolumn{1}{c}{} & \multicolumn{9}{c}{Период} \\\cmidrule{2-10}
    \multicolumn{1}{c}{} & 1     & 2     & 3     & 4     & 5     & 6     & 7     & 8     & 9 \\
    \midrule
    \multicolumn{1}{l}{Доходность актива} & 7     & 27    & 5     & -4    & -7    & 5     & -12   & -15   & -8 \\
    \bottomrule
    \end{tabular}%
  \label{tab:addlabel}%
\end{table}%

Определите ожидаемую доходность актива. В каком интервале с надежностью 0,95 располагается истинное значение ожидаемой доходности актива?

\end{taskrus}

\begin{taskrus}
Доходность актива имеет нормальное распределение. На основе наблюдений за 61 день было рассчитано исправленное стандартное отклонение в расчете на день. Оно составило 1,5\%. В каком интервале с надежностью 0.90 располагается истинное значение дисперсии и стандартного отклонения доходности актива?
\end{taskrus}

\begin{taskrus}
Доходности активов имеют нормальное распределение. На основе данныч о доходности активов \textit{X} и \textit{Y} за 21 день были рассчитаны исправленные стандартные отклонения доходности: $X=0,72\%$, $Y = 0,56\%$. Проверить гипотезу о равенстве дисперсий активов при уровне значимости 0,10.
\end{taskrus}

\begin{taskrus} 
\label{task_sign_corrXY}
Доходность активов имеет нормальное распределение. Доходности активов \textit{X} и \textit{Y} за 9 периодов представлены в таблице.
% Table generated by Excel2LaTeX from sheet 'Задача3'
\begin{table}[H]
  \centering
  \caption{Доходность активов}
    \begin{tabular}{rrrrrrrrrrr}
    \toprule
    \multicolumn{2}{c}{\multirow{2}[1]{*}{}} & \multicolumn{9}{c}{Период} \\\cmidrule{3-11}
    \multicolumn{2}{c}{} & 1     & 2     & 3     & 4     & 5     & 6     & 7     & 8     & 9 \\
    \midrule
    \multicolumn{1}{l}{\multirow{2}[1]{*}{Доходность актива}} & X     & 7     & 27    & 5     & -4    & -7    & 5     & -12   & -15   & -8 \\
    \multicolumn{1}{l}{} & Y     & 14    & 0     & 32    & -11   & 5     & 8     & 4     & -3    & -2 \\
    \bottomrule
    \end{tabular}%
  \label{tab:addlabel}%
\end{table}%
Определить коэффициент выборочной корреляции доходности активов. Проверить гипотезу о значимости коэффициента корреляции при уровне значимости 0.5.
\end{taskrus}

\end{document}