\documentclass[12pt,a4paper]{article}

\usepackage[rus]{borochkin_article}

\begin{document}
\huge{Контрольная работа 2.}
\normalsize
\section{Определение доверительного интервала для коэффициента корреляции при малых объемах выборки}
\begin{taskrus}
Коэффициент выборочной корреляции доходности активов равен 0,832946. Объем выборки составляет 8 наблюдений. Найти доверительные интервалы для коэффициента корреляции с коэффициентом доверия $\gamma=0,9$. 

% Table generated by Excel2LaTeX from sheet 'З2_1'
\begin{table}[H]
  \centering
    \begin{tabular}{lrrrr}
    \toprule
          & Вар1  & Вар2  & Вар3  & Вар4 \\
    \midrule
    Коэффициент выборочной корреляции & 0,832946 & -0,21956 & -0,1073 & 0,260537 \\
    Объем выборки & 8     & 8     & 10    & 9 \\
    Коэффициент доверия & 90\%  & 95\%  & 99\%  & 98\% \\
    Решение &       &       &       &  \\
    Уровень значимости & 10\%  & 5\%   & 1\%   & 2\% \\
    Квантиль ст. норм. Распр. & 1,644854 & 1,959964 & 2,575829 & 2,326348 \\
    z\_L  & 0,402585 & -1,08403 & -1,07533 & -0,69933 \\
    z\_U  & 1,873786 & 0,669016 & 0,871816 & 1,200129 \\
    corr\_L & 0,382158 & -0,79469 & -0,79146 & -0,60394 \\
    corr\_U & 0,953936 & 0,584332 & 0,702295 & 0,833694 \\
    \bottomrule
    \end{tabular}%
  \label{tab:addlabel}%
\end{table}%


\end{taskrus}

\section{Определение доверительного интервала для коэффициента корреляции при малых объемах выборки}
\begin{taskrus}
Выборочный коэффициент корреляции равен 0,833. Он был определен на основе 600 наблюдений. Найти доверительный интервал для коэффициента корреляции с коэффициентом доверия $\gamma=0.9$.

\begin{table}[H]
  \centering
    \begin{tabular}{lrrrr}
    \toprule
          & Вар1  & Вар2  & Вар3  & Вар4 \\
    \midrule
    Выборочный коэффициент корреляции & 0,833 & 0,7   & 0,65  & 0,9 \\
    Объем выборки & 600   & 1000  & 5000  & 550 \\
    Коэффициент доверия & 90\%  & 95\%  & 99\%  & 98\% \\
    Решение &       &       &       &  \\
    Уровень значимости & 10\%  & 5\%   & 1\%   & 2\% \\
    Квантиль ст. норм. Распр. & 1,6449 & 1,9600 & 2,5758 & 2,3263 \\
    corr\_L & 0,8127 & 0,6686 & 0,6290 & 0,8813 \\
    corr\_U & 0,8538 & 0,7318 & 0,6711 & 0,9190 \\
    \bottomrule
    \end{tabular}%
  \label{tab:addlabel}%
\end{table}%

\end{taskrus}
\pagebreak
\section{Определение VaR портфеля с учетом валютного курса}
\begin{taskrus}
Российский инвестор купил акции компании \textit{А} на 400 тыс. долл. Стандартное отклонение доходности акции в расчете на день составляет 1.26\%. Курс доллара 1 долл.=29 руб., стандартное отклонение валютою курса в расчете на один день 0.35\%. Коэффициент корреляции между курсом доллара и доходностью акции компании \textit{А} равен 0.25. Определить \textit{VaR }портфеля инвестора в рублях с доверительной вероятностью 95\%.

% Table generated by Excel2LaTeX from sheet 'З2_3'
\begin{table}[H]
  \centering
    \begin{tabular}{lrrrr}
    \toprule
          & Вар1  & Вар2  & Вар3  & Вар4 \\
    \midrule
    Стоимость акций, долл. & 400000 & 25000 & 500000 & 125000 \\
    Курс доллара к рублю & 29    & 56    & 83    & 75 \\
    \specialcell{Ст. откл.\\
    доходности акции за день }& 1,26\% & 1,50\% & 2\%   & 1,50\% \\
    Ст. откл. валютного курса & 0,35\% & 2\%   & 3,50\% & 0,70\% \\
    \specialcell{Коэфф. корр. между\\
    доходностью акции\\и курсом доллара} & 0,25  & 0,7   & 0,8   & 0,4 \\
    Доверительная вероятность & 95\%  & 90\%  & 95\%  & 90\% \\
    Решение &       &       &       &  \\
    \specialcell{Рублевая стоимость портфеля,\\тыс. руб.} & 11600 & 1400  & 41500 & 9375 \\
    Дисперсия доходности портфеля & 0,0193\% & 0,1045\% & 0,2745\% & 0,0358\% \\
    Ст. откл. дох-ти портфеля & 1,3895\% & 3,2326\% & 5,2393\% & 1,8921\% \\
    \specialcell{Однодневный \textit{VaR }портфеля,\\тыс. руб.} & 265,113 & 57,999 & 3576,404 & 227,326 \\
    \bottomrule
    \end{tabular}%
  \label{tab:addlabel}%
\end{table}%
\end{taskrus}
\pagebreak
\section{EaR}
\begin{taskrus} 
Определить величину средних ожидаемых потерь для одного дня для портфеля стоимостью 10 млн. руб., в который входят акции двух компаний. Уд. вес первой акции в стоимости портфеля составляет 60\%, второй - 40\%. Стандартное отклонение доходности первой акции в расчете на один день равно 1.58\%, второй 1.9\%, коэффициент корреляции доходностей акций равен 0.8. Предполагается, что доходность акций имеет нормальное распределение. Доверительная вероятность равна 95\%.

% Table generated by Excel2LaTeX from sheet 'Задача5'
\begin{table}[H]
  \centering
  \caption{Определение величины средних ожидаемых потерь\textit{EaR}}
    \begin{tabular}{lrrrrr}
    \toprule
          & Компания & Вар1  & Вар2  & Вар3  & Вар4 \\
    \midrule
    \multicolumn{2}{l}{\specialcell{Стоимость портфеля,\\ млн. руб.}} & 10    & 2     & 5     & 15 \\
    \multirow{2}[0]{*}{\specialcell{Удельные веса\\ акций в портфеле}} & \multicolumn{1}{c}{A} & 60\%  & 50\%  & 20\%  & 35\% \\
          & \multicolumn{1}{c}{B} & 40\%  & 50\%  & 80\%  & 65\% \\
    \multirow{2}[0]{*}{\specialcell{Стандартное отклонение\\ доходности акций}} & \multicolumn{1}{c}{A} & 1,58\% & 2,50\% & 2,70\% & 3,80\% \\
          & \multicolumn{1}{c}{B} & 1,90\% & 1,00\% & 0,60\% & 1,60\% \\
    \multicolumn{2}{l}{\specialcell{Коэффициент корреляции\\ доходностей акций}} & 0,8   & 0,2   & -0,5  & 0,9 \\
    \multicolumn{2}{l}{\specialcell{Доверительная \\вероятность $\gamma$, \%}} & 95\%  & 90\%  & 99\%  & 90\% \\
    \midrule
    \multicolumn{6}{l}{\textbf{Решение}} \\
    \midrule
    \multicolumn{2}{l}{\specialcell{Стандартное отклонение\\ доходности портфеля, \%}} & 1,621\% & 1,436\% & 0,513\% & 2,311\% \\
    \multicolumn{2}{l}{\specialcell{Стандартное отклонение дохода\\ портфеля, млн. руб.}} & 0,162144 & 0,028723 & 0,025632 & 0,346635 \\
    \multicolumn{2}{l}{$z_{\gamma}$} & 1,644854 & 1,281552 & 2,326348 & 1,281552 \\
    \multicolumn{2}{l}{\specialcell{Однодневный VaR\\ портфеля, млн. руб.}} & 0,266703 & 0,03681 & 0,059629 & 0,444231 \\
    \multicolumn{2}{l}{\specialcell{Величина средних ожидаемых\\ потерь, млн. руб.}} & -0,33446 & -0,05041 & -0,06831 & -0,60834 \\
    \bottomrule
    \end{tabular}%
  \label{tab:addlabel}%
\end{table}%

\end{taskrus}
\pagebreak
\section{Определение VaR на основе стандартных факторов риска банка J.P. Morgan}

\begin{taskrus}
Портфель состоит из акций пяти компаний, купленных на суммы   $V=[1\;3\;5\;6\;8]$ млн. рублей. Бета акций относительно фондового индекса равны \\$\beta=[0.6\;0.75\;0.9\;1.2\;1.3]$. Стандартное отклонение рыночного портфеля для одного дня составляет 1.8\%. Определить однодневный  \textit{VaR }портфеля с доверительной вероятностью 90\% согласно методике, принятой в Рискметриках банка J.P. Morgan, на основе стандартных факторов риска. 

% Table generated by Excel2LaTeX from sheet 'Задача6'
\begin{table}[htbp]
  \centering
    \begin{tabular}{lcrrrr}
    \toprule
          &Компания    & Вар1  & Вар2  & Вар3  & Вар4 \\
    \midrule
    \multicolumn{1}{l}{\multirow{5}[0]{*}{\specialcell{ Стоимость пакета \\акций, млн. руб.}}} & 1     & 1     & 5     & 2     & 20 \\
    \multicolumn{1}{c}{} & 2     & 3     & 10    & 3     & 40 \\
    \multicolumn{1}{c}{} & 3     & 5     & 12    & 7     & 80 \\
    \multicolumn{1}{c}{} & 4     & 6     & 15    & 8     & 90 \\
    \multicolumn{1}{c}{} & 5     & 8     & 30    & 10    & 120 \\
    \multicolumn{1}{c}{\multirow{5}[0]{*}{\specialcell{ Бета относительно\\фондового индекса}}} & 1     & 0,6   & 0,1   & 0,2   & 0,8 \\
    \multicolumn{1}{c}{} & 2     & 0,75  & 0,8   & 0,5   & 0,85 \\
    \multicolumn{1}{c}{} & 3     & 0,9   & 0,7   & 0,9   & 0,95 \\
    \multicolumn{1}{c}{} & 4     & 1,2   & 1,1   & 1,5   & 1,1 \\
    \multicolumn{1}{c}{} & 5     & 1,3   & 2     & 1,7   & 1,4 \\
    \multicolumn{2}{l}{\specialcell{Стандартное отклонение\\рыночного портфеля\\ для одного дня}} & 1,80\% & 2,20\% & 0,55\% & 1,20\% \\
    \multicolumn{2}{l}{\specialcell{Доверительная\\вероятность \textit{VaR}}} & 90\%  & 95\%  & 99\%  & 95\% \\
    \midrule
    \multicolumn{6}{l}{\textbf{Решение}} \\
    \midrule
    z статистика &       & 1,281552 & 1,644854 & 2,326348 & 1,644854 \\
    VaR портфеля &       & 0,575545 & 3,379845 & 0,475971 & 7,75713 \\
    \bottomrule
    \end{tabular}%
  \label{tab:addlabel}%
\end{table}%

\end{taskrus}

\pagebreak


\section{Ответы к тестам}

% Table generated by Excel2LaTeX from sheet 'Лист1'
\begin{table}[H]
  \centering
  \caption{Вариант 1}
    \begin{tabular}{ccccc}
    \toprule
    \textbf{1} & \textbf{2} & \textbf{3} & \textbf{4} & \textbf{5} \\
    А     & В     & Г     & Б     & А \\
    \midrule
    \textbf{6} & \textbf{7} & \textbf{8} & \textbf{9} & \textbf{10} \\
    Б     & Б     & А     & Г     & Б \\
    \midrule
    \textbf{11} & \textbf{12} & \textbf{13} & \textbf{14} & \textbf{15} \\
    В     & В     & А     & А     & Б \\
    \bottomrule
    \end{tabular}%
  \label{tab:addlabel}%
\end{table}%

% Table generated by Excel2LaTeX from sheet 'Лист1'
\begin{table}[H]
  \centering
  \caption{Вариант 2}
    \begin{tabular}{ccccc}
    \toprule
    \textbf{1} & \textbf{2} & \textbf{3} & \textbf{4} & \textbf{5} \\
    Г     & Б     & Б     & Г     & Г \\
    \midrule
    \textbf{6} & \textbf{7} & \textbf{8} & \textbf{9} & \textbf{10} \\
    Г     & В     & Г     & А     & Г \\
    \midrule
    \textbf{11} & \textbf{12} & \textbf{13} & \textbf{14} & \textbf{15} \\
    Г     & Г     & В     & Г     & Г \\
    \bottomrule
    \end{tabular}%
  \label{tab:addlabel}%
\end{table}%

\end{document}