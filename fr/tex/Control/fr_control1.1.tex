% !TeX program = lualatex -synctex=1 -interaction=nonstopmode --shell-escape %.tex

\documentclass[12pt, table]{exam}
\usepackage[rus]{borochkin}

\usepackage{borochkin_exam}

%%%%%%%%%%%%%%%%%%%%%%%%%%%%%%%%%%%%%%%%%
\professor
\iftagged{professor}{ \printanswers }

%%%%%%%%%%%%%%%%%%%%%%%%%%%%%%%%%%%%%%%%%

\examrunningheader{Финансовые риски в кредитной сфере, к/р №1}
\examrunningfooter{А. А. Борочкин}



\begin{document}
\setcounter{section}{0\relax}%
\section{Вариант}

\noindent
\studentpersonalinfo{ФРвКС}


\normalsize
\begin{questions}
\question[40] Тест
\answerstotest

	
\question[20] 
\noaddpoints
\begin{subparts}
\subpart[10] 

\begin{solution}[21em]
\end{solution}

\subpart[10] 

\begin{solution}[12em]
\end{solution}

\end{subparts}
\addpoints

\question[10] 
\noaddpoints
\begin{subparts}
	\subpart[8] 
	\begin{solution}[8em]

	\end{solution}

	\subpart[2] 
	\begin{solution}[4em]
	\end{solution}
\end{subparts}
\addpoints

\question[15] 
\begin{subparts}
	\subpart[5] 
	\begin{solution}[8em]
		
	\end{solution}
	
	\subpart[5] 
	\begin{solution}[18em]
	\end{solution}
	\subpart[5] 
	\begin{solution}[18em]
	\end{solution}
	
\end{subparts}
\addpoints

\question[10] 
\begin{solution}[10em]
\end{solution}


\end{questions}

\pagebreak
\noindent\textbf{Тестовые вопросы (выберите один правильный ответ)}

\begin{questions}
\begin{multicols}{2}
\setlength{\columnsep}{1cm}

\question Укажите наиболее полное определение риска:
	 \begin{choices}
	 \CC Риск – это деятельность, связанная с преодолением неопределенности в ситуации неизбежного выбора, в процессе которой имеется возможность количественно и качественно оценить вероятность достижения предполагаемого результата, неудачи и отклонения от цели;
	 \choice Сочетание вероятности и последствий наступления неблагоприятных событий;
	 \choice Риск — характеристика ситуации, имеющей неопределённость исхода, при обязательном наличии неблагоприятных последствий;
	 \choice Риск — это вероятность возможной нежелательной потери чеголибо при плохом стечении обстоятельств.
	 \end{choices}
\question Особенностью чистых рисков является то, что
	 \begin{choices}
	 \choice они возникают только в сфере налогообложения;
	 \choice они несут в себе как потери, так и прибыли для предпринимательской деятельности;
	 \CC их причинами могут быть стихийные бедствия, несчастные случаи, недееспособность руководителей фирм и тп;
	 \choice их причинами могут быть изменение конъюнктуры рынка, курсов валют и тп.
	 \end{choices}
\question В зависимости от сферы деятельности выделяют следующие виды рисков:
	 \begin{choices}
	 \choice производственный;
	 \choice коммерческий;
	 \choice финансовый;
	 \CC всё вышеперечисленное.
	 \end{choices}
\question Под критическим риском понимают:
	 \begin{choices}
	 \choice уровень риска в пределах его среднего уровня;
	 \CC риск, уровень которого выше среднего, но в пределах максимально допустимых значений в данной экономической системе;
	 \choice риск, который превышает максимально допустимую границу;
	 \choice верны варианты Б и В.
	 \end{choices}
\question Неопределенность – это:
	 \begin{choices}
	 \CC неполное или неточное представление о значениях различных параметров в будущем, порождаемых различными причинами и, прежде всего, неполнотой или неточностью информации об условиях реализации решения, в том числе о затратах и результатах;
	 \choice возможность возникновения в ходе реализации решения неблагоприятных ситуаций и последствий;
	 \choice ситуация, когда последствия принимаемых решений являются неконтролируемыми;
	 \choice ситуация, когда последствия принимаемых решений являются неизвестными.
	 \end{choices}
\question Под инвестиционным портфелем понимается
	 \begin{choices}
	 \choice совокупность рискованных активов могущих приносить как доход так и убыток;
	 \CC целенаправленно сформированная в соответствии с определенной инвестиционной стратегией совокупность вложений в инвестиционные объекты;
	 \choice вложения в финансовые инструменты, торгуемые на высоко ликвидных рынках;
	 \choice любые вложения, связанные с риском, имеющие диверсификацию и потенциал увеличения первоначального капитала.
	 \end{choices}
\question Портфель роста формируется из
	 \begin{choices}
	 \choice высокорисковых финансовых инструментов;
	 \CC акций быстрорастущих компаний;
	 \choice краткосрочных спекулятивных позиций на фондовом рынке;
	 \choice производных финансовых инструментов.
	 \end{choices}
\question Оценить риск инвестиционного портфеля при вводе в его состав актива без риска можно по формуле ($D_p$ – доля прежнего портфеля в формируемом, $\sigma$ – стандартное отклонение доходности):
	 \begin{choices}
	 \CC $\sigma_f=D_p \cdot \sigma_p$;
	 \choice $\sigma_f=\frac{D_p}{\sigma_p}$;
	 \choice риск инвестиционного портфеля не меняется при вводе в его состав актива без риска;
	 \choice для решения указанной задачи требуется решение матричных уравнений.
	 \end{choices}
\question Коэффициент Шарпа …
	 \begin{choices}
	 \choice учитывает доходность портфеля, полученную сверх ставки без риска, и весь риск:;
	 \choice $\frac{r_p - r_f}{\sigma_p}$ , где $r_p$ – средняя доходность портфеля ценных бумаг за рассматриваемый период $r_f$ – средняя ставка без риска за данный период $\sigma_p$ – стандартное отклонение доходности портфеля;
	 \choice целесообразно использовать для оценки эффективности управления менее диверсифицированных портфелей;
	 \CC всё вышеперечисленное верно.
	 \end{choices}
\question Какие модели и способы не используются для оценки кредитных деривативов:
	 \begin{choices}
	 \choice на основе стоимости хеджирования;
	 \CC на основе модели оценки стоимости капитальных активов, CAPM;
	 \choice на основе кредитного рейтинга фирмы;
	 \choice на основе интенсивности дефолтов.
	 \end{choices}
\question Рыночный риск, который несет банк, НЕ включает в себя:
	 \begin{choices}
	 \choice фондовый риск;
	 \choice валютный риск;
	 \CC риск ликвидности;
	 \choice процентный риск.
	 \end{choices}
\question Расчет процентного риска НЕ осуществляется банком в отношении:
	 \begin{choices}
	 \choice долговых ценных бумаг;
	 \choice долевых ценных бумаг с правом конверсии в долговые ценные бумаги;
	 \CC «золотых» акций;
	 \choice неконвертируемых привилегированных акций, размер дивиденда по которым определен.
	 \end{choices}
\question Факторы, НЕ влияющие на возникновение риска потери деловой репутации кредитной организацией:
	 \begin{choices}
	 \CC неучастие кредитной организации в благотворительной и общественной деятельности;
	 \choice несоблюдением кредитной организацией законодательства Российской Федерации, учредительных и внутренних документов кредитной организации;
	 \choice несоблюдением кредитной организацией обычаев делового оборота, принципов профессиональной этики;
	 \choice несоблюдением кредитной организацией договорных обязательств перед кредиторами, вкладчиками и иными клиентами и контрагентами.
	 \end{choices}
\question Управление операционным риском в банке НЕ включает в себя:
	 \begin{choices}
	 \CC прогнозирование риска;
	 \choice выявления риска;
	 \choice оценки риска;
	 \choice мониторинга риска.
	 \end{choices}
\question Признаками ухудшения общего качества работы кредитной организации по взысканию проблемных и безнадежных ссуд обычно НЕ являются:
	 \begin{choices}
	 \choice увеличение объема ссуд, предоставленных с нарушениями кредитной политики (в том числе политики в отношении обеспечения по ссудам);
	 \CC сотрудничество банка с коллекторскими агентствами в части принудительного взыскания проблемной задолженности;
	 \choice наличие в ссудном портфеле проблемных и безнадежных ссуд, не имеющих видимых признаков улучшения;
	 \choice высокая концентрация кредитного риска (значительное увеличение объема ссуд, предоставленных отдельному заемщику, группе связанных заемщиков, заемщикам, принадлежащим к одним и тем же отраслям либо географическим регионам, или заемщикам, уязвимым к одним и тем же экономическим факторам).
	 \end{choices}



\end{multicols}
\end{questions}

\end{document}
