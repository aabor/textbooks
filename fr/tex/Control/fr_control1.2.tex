% !TeX program = lualatex -synctex=1 -interaction=nonstopmode --shell-escape %.tex

\documentclass[12pt, table]{exam}
\usepackage[rus]{borochkin}

\usepackage{borochkin_exam}

%%%%%%%%%%%%%%%%%%%%%%%%%%%%%%%%%%%%%%%%%
\professor
\iftagged{professor}{ \printanswers }

%%%%%%%%%%%%%%%%%%%%%%%%%%%%%%%%%%%%%%%%%

\examrunningheader{Финансовые риски в кредитной сфере, к/р №1}
\examrunningfooter{А. А. Борочкин}

\begin{document}
\setcounter{section}{1\relax}%
\section{Вариант}

\noindent
\studentpersonalinfo{ФРвКС}

\normalsize
\begin{questions}
\question[40] Тест
\answerstotest

	
\question[20] 
\noaddpoints
\begin{subparts}
\subpart[10] 

\begin{solution}[21em]
\end{solution}

\subpart[10] 

\begin{solution}[12em]
\end{solution}

\end{subparts}
\addpoints

\question[10] 
\noaddpoints
\begin{subparts}
	\subpart[8] 
	\begin{solution}[8em]

	\end{solution}

	\subpart[2] 
	\begin{solution}[4em]
	\end{solution}
\end{subparts}
\addpoints

\question[15] 
\begin{subparts}
	\subpart[5] 
	\begin{solution}[8em]
		
	\end{solution}
	
	\subpart[5] 
	\begin{solution}[18em]
	\end{solution}
	\subpart[5] 
	\begin{solution}[18em]
	\end{solution}
	
\end{subparts}
\addpoints

\question[10] 
\begin{solution}[10em]
\end{solution}


\end{questions}

\pagebreak
\noindent\textbf{Тестовые вопросы (выберите один правильный ответ)}

\begin{questions}
\begin{multicols}{2}
\setlength{\columnsep}{1cm}

\question Чем отличается риск от неопределенности:
	 \begin{choices}
	 \CC В ситуации неопределенности отсутствует какаялибо информация о последствиях возможного события;
	 \choice В случае риска имеются вероятностные характеристики неконтролируемых переменных, при неопределенности такие характеристики отсутствуют;
	 \choice В случае рискованной ситуации существует возможность выделить возможные варианты событий и количественно оценить их вероятность;
	 \choice Верны ответы Б и В.
	 \end{choices}
\question Какие риски НЕ связаны с покупательной способностью денег:
	 \begin{choices}
	 \choice инфляционные;
	 \choice инвестиционные;
	 \CC дефляционные;
	 \choice валютные.
	 \end{choices}
\question К каким видам риска относится риск упущенной выгоды:
	 \begin{choices}
	 \choice финансовый;
	 \choice инвестиционный;
	 \choice валютный;
	 \CC транспортный.
	 \end{choices}
\question Выделяют следующие виды неопределенностей в зависимости от времени возникновения:
	 \begin{choices}
	 \choice ретроспективные;
	 \CC текущие;
	 \choice перспективные;
	 \choice всё вышеперечисленное.
	 \end{choices}
\question Что НЕ используется в качестве методов оценки экономических рисков:
	 \begin{choices}
	 \CC теория вероятностей;
	 \choice теория игр;
	 \choice экспертные подходы;
	 \choice суждение на основе личного опыта.
	 \end{choices}
\question Целями инвестиционного портфеля являются:
	 \begin{choices}
	 \choice максимизация роста капитала;
	 \CC максимизация роста дохода;
	 \choice минимизация инвестиционных рисков;
	 \choice любая цель из перечисленных, при условии, что она обозначена в инвестиционной стратегии.
	 \end{choices}
\question Как связаны между собой ковариация и коэффициент корреляции доходностей активов:
	 \begin{choices}
	 \choice эти понятия эквивалентны;
	 \CC коэффициент корреляции равен корню квадратному из ковариации;
	 \choice ковариация доходностей двух активов равна коэффициенту корреляции умноженному на стандартные отклонения доходностей каждого из двух активов;
	 \choice коэффициент корреляции доходностей двух активов равен ковариации умноженной на стандартные отклонения доходностей каждого из двух активов.
	 \end{choices}
\question Согласно подходу Capital Asset Pricing Model, CAPM, риск инвестиций измеряется величиной $\beta$ по формуле $cov_{iM}$ – ковариация доходности акции с доходностью рыночного индекса $corr_{iM}$  коэффициент корреляции доходности акции с доходностью рыночного индекса $\sigma_i$ – стандартное отклонение доходности акции $\sigma_M$ – стандартное отклонение доходности рыночного индекса):
	 \begin{choices}
	 \CC $\beta_i=\frac{cov_{iM}}{\sigma_M^2}$;
	 \choice $\beta_i=\frac{\sigma_i}{\sigma_M} \cdot corr_{iM}$;
	 \choice $\beta$ – коэффициент публикуется поставщиком биржевых котировок;
	 \choice верны два первых ответа.
	 \end{choices}
\question Коэффициент Трейнора …
	 \begin{choices}
	 \choice это отношение средней доходности, превышающей безрисковую процентную ставку, к систематическому риску $\beta$;
	 \choice применяется для оценки эффективности управления мало диверсифицированного портфеля;
	 \choice в отличие от коэффициента Шарпа, коэффициент Трейнора соотносит доходность не с общим риском, а только с риском отдельного набора финансовых инструментов;
	 \CC всё вышеперечисленное верно.
	 \end{choices}
\question К типичным банковским рискам относятся:
	 \begin{choices}
	 \choice кредитный риск;
	 \CC страновой риск;
	 \choice риск потери репутации;
	 \choice всё вышеперечисленное.
	 \end{choices}
\question Правовой риск может возникнуть у банка вследствие:
	 \begin{choices}
	 \choice несоблюдения требований нормативных правовых актов и заключенных договоров;
	 \choice допускаемых правовых ошибок при осуществлении деятельности (неправильные юридические консультации или неверное составление документов, в том числе при рассмотрении спорных вопросов в судебных органах);
	 \CC несовершенства правовой системы (противоречивость законодательства, отсутствие правовых норм по регулированию отдельных вопросов);
	 \choice всё вышепречисленное.
	 \end{choices}
\question Оценка фондового риска осуществляется кредитной организацией в отношении:
	 \begin{choices}
	 \choice обыкновенных акций;
	 \choice депозитарных расписок;
	 \CC производных финансовых инструментов, базисным активом которых являются ценные бумаги, указанные выше, а также фондовые индексы;
	 \choice всё вышеперечисленное.
	 \end{choices}
\question При оценке уровня риска потери деловой репутации кредитными организациями могут использоваться следующие показатели:
	 \begin{choices}
	 \CC изменение финансового состояния кредитной организации (например, изменение структуры активов кредитной организации, их обесценение в целом или в части отдельных групп, изменение структуры собственных средств кредитной организации);
	 \choice возрастание или сокращение количества жалоб и претензий к кредитной организации, в том числе относительно качества обслуживания клиентов и контрагентов, соблюдения обычаев делового оборота;
	 \choice верны два первых ответа;
	 \choice на уровень риска потери деловой репутации не влияет финансовое состояние кредитной организации, более важно количество жалоб и претензий.
	 \end{choices}
\question Признаками ухудшения качества ссуды могут служить:
	 \begin{choices}
	 \CC ухудшение ликвидности заемщика;
	 \choice высокая текучесть кадров;
	 \choice вовлеченность заемщика в рискованные проекты;
	 \choice всё вышеперечисленное.
	 \end{choices}
\question Ликвидность залога зависит от следующих факторов:
	 \begin{choices}
	 \choice количество потенциальных покупателей объекта залога в регионе;
	 \CC территориальное месторасположение объекта залога, обеспеченность социально-бытовой инфраструктурой, окружение объекта;
	 \choice степень удаленности и удобство подъезда к объекту;
	 \choice всё вышеперечисленное.
	 \end{choices}


\end{multicols}
\end{questions}

\end{document}
