\documentclass[12pt,a4paper]{article}

\usepackage[rus]{borochkin_article}
  
\begin{document}
\section{Вариант}
\begin{taskrus}
Коэффициент выборочной корреляции доходности активов равен -0,1073. Объем выборки составляет 10 наблюдений. Найти доверительные интервалы для коэффициента корреляции с коэффициентом доверия $\gamma=0,99$. 
\end{taskrus}

\begin{taskrus}
Выборочный коэффициент корреляции равен 0,65. Он был определен на основе 5000 наблюдений. Найти доверительный интервал для коэффициента корреляции с коэффициентом доверия $\gamma=0.99$.
\end{taskrus}

\begin{taskrus}
Российский инвестор купил акции компании \textit{А} на 500 тыс. долл. Стандартное отклонение доходности акции в расчете на день составляет 2\%. Курс доллара 1 долл.=83 руб., стандартное отклонение валютою курса в расчете на один день 3.5\%. Коэффициент корреляции между курсом доллара и доходностью акции компании \textit{А} равен 0.8. Определить \textit{VaR }портфеля инвестора в рублях с доверительной вероятностью 95\%.
\end{taskrus}

\begin{taskrus} 
Определить величину средних ожидаемых потерь для одного дня для портфеля стоимостью 5 млн. руб., в который входят акции двух компаний. Уд. вес первой акции в стоимости портфеля составляет 20\%, второй - 80\%. Стандартное отклонение доходности первой акции в расчете на один день равно 2.7\%, второй 0.6\%, коэффициент корреляции доходностей акций равен -0.5. Предполагается, что доходность акций имеет нормальное распределение. Доверительная вероятность равна 99\%.
\end{taskrus}


\begin{taskrus}
Портфель состоит из акций пяти компаний, купленных на суммы\newline
{$V=[2\;3\;7\;8\;10]$} млн. рублей. Бета акций относительно фондового индекса равны \newline
{$\beta=[0.2\;0.5\;0.9\;1.5\;1.7]$}. Стандартное отклонение рыночного портфеля для одного дня составляет 0.55\%. Определить однодневный  \textit{VaR }портфеля с доверительной вероятностью 99\% согласно методике, принятой в Рискметриках банка J.P. Morgan, на основе стандартных факторов риска. 
\end{taskrus}


\pagebreak
\section{Вариант}
\begin{taskrus}
Коэффициент выборочной корреляции доходности активов равен 0,260537. Объем выборки составляет 9 наблюдений. Найти доверительные интервалы для коэффициента корреляции с коэффициентом доверия $\gamma=0,98$. 
\end{taskrus}

\begin{taskrus}
Выборочный коэффициент корреляции равен 0,9. Он был определен на основе 550 наблюдений. Найти доверительный интервал для коэффициента корреляции с коэффициентом доверия $\gamma=0.98$.
\end{taskrus}

\begin{taskrus}
Российский инвестор купил акции компании \textit{А} на 125 тыс. долл. Стандартное отклонение доходности акции в расчете на день составляет 1.5\%. Курс доллара 1 долл.=75 руб., стандартное отклонение валютою курса в расчете на один день 0.70\%. Коэффициент корреляции между курсом доллара и доходностью акции компании \textit{А} равен 0.4. Определить \textit{VaR }портфеля инвестора в рублях с доверительной вероятностью 90\%.
\end{taskrus}

\begin{taskrus} 
Определить величину средних ожидаемых потерь для одного дня для портфеля стоимостью 15 млн. руб., в который входят акции двух компаний. Уд. вес первой акции в стоимости портфеля составляет 35\%, второй - 65\%. Стандартное отклонение доходности первой акции в расчете на один день равно 3.8\%, второй 1.6\%, коэффициент корреляции доходностей акций равен 0.9. Предполагается, что доходность акций имеет нормальное распределение. Доверительная вероятность равна 90\%.

\end{taskrus}


\begin{taskrus}
Портфель состоит из акций пяти компаний, купленных на суммы   \newline $V=[20\;40\;80\;90\;120]$ млн. рублей. Бета акций относительно фондового индекса равны \newline $\beta=[0.8\;0.85\;0.95\;1.1\;1.4]$. Стандартное отклонение рыночного портфеля для одного дня составляет 1.2\%. Определить однодневный  \textit{VaR }портфеля с доверительной вероятностью 95\% согласно методике, принятой в Рискметриках банка J.P. Morgan, на основе стандартных факторов риска. 
\end{taskrus}
\end{document}