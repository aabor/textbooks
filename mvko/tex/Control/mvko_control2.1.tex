% !TeX program = lualatex -synctex=1 -interaction=nonstopmode --shell-escape %.tex

\documentclass[12pt, table]{exam}
\usepackage[rus]{borochkin}

\usepackage{borochkin_exam}

%%%%%%%%%%%%%%%%%%%%%%%%%%%%%%%%%%%%%%%%%
\professor
\iftagged{professor}{ \printanswers }
%%%%%%%%%%%%%%%%%%%%%%%%%%%%%%%%%%%%%%%%%

\examrunningheader{Международные валютно-кредитные отношения, к/р №2}
\examrunningfooter{А. А. Борочкин}

\begin{document}
\setcounter{section}{0\relax}%
\section{Вариант}

\noindent
\studentpersonalinfo{МВКО}

\normalsize
\begin{questions}
\question[40] Тест
\answerstotest

	
\question[20] 
\noaddpoints
\begin{subparts}
\subpart[10] 

\begin{solution}[21em]
\end{solution}

\subpart[10] 

\begin{solution}[12em]
\end{solution}

\end{subparts}
\addpoints

\question[10] 
\noaddpoints
\begin{subparts}
	\subpart[8] 
	\begin{solution}[8em]

	\end{solution}

	\subpart[2] 
	\begin{solution}[4em]
	\end{solution}
\end{subparts}
\addpoints

\question[15] 
\begin{subparts}
	\subpart[5] 
	\begin{solution}[8em]
		
	\end{solution}
	
	\subpart[5] 
	\begin{solution}[18em]
	\end{solution}
	\subpart[5] 
	\begin{solution}[18em]
	\end{solution}
	
\end{subparts}
\addpoints

\question[10] 
\begin{solution}[10em]
\end{solution}


\end{questions}

\pagebreak
\noindent\textbf{Тестовые вопросы (выберите один правильный ответ)}

\begin{questions}
\begin{multicols}{2}
\setlength{\columnsep}{1cm}

\question Основным источником погашения международного кредита является:
	 \begin{choices}
	 \CC рефинансирование заимствований;
	 \choice амортизация;
	 \choice производственная прибыль;
	 \choice спекулятивный доход.
	 \end{choices}
\question Рост объемов международного кредитования ведет:
	 \begin{choices}
	 \CC к углублению отраслевых диспропорций в национальных экономиках;
	 \choice ослаблению отраслевых диспропорций в национальных экономиках;
	 \choice сохранению отраслевых диспропорций в национальных экономиках на неизменном уровне;
	 \choice непредсказуемым изменениям отраслевых диспропорций в национальных экономиках.
	 \end{choices}
\question В качестве обеспечения кредита не могут выступать:
	 \begin{choices}
	 \choice ценные бумаги;
	 \choice незавершенное производство;
	 \choice товарно-материальные ценности;
	 \CC кредиторская задолженность.
	 \end{choices}
\question Основными заемщиками на международном рынке являются:
	 \begin{choices}
	 \choice национальные правительства;
	 \CC компании и корпорации;
	 \choice банки;
	 \choice частные лица.
	 \end{choices}
\question Наиболее простая финансовая схема характеризует:
	 \begin{choices}
	 \CC акцептно-рамбурсный кредит;
	 \choice овердрафт;
	 \choice ипотечный кредит;
	 \choice факторинг.
	 \end{choices}
\question Привлечение средств для краткосрочных финансовых операций целесообразнее всего осуществлять посредством:
	 \begin{choices}
	 \choice коммерческого кредита;
	 \choice форфейтинга;
	 \choice ипотечного кредита;
	 \CC бланкового финансового кредита.
	 \end{choices}
\question Процентная ставка по бланковым кредитам:
	 \begin{choices}
	 \CC выше, чем по обеспеченным;
	 \choice ниже, чем по обеспеченным;
	 \choice такая же, как и по обеспеченным;
	 \choice отличается от ставки по обеспеченным кредитам непредсказуемым образом.
	 \end{choices}
\question Кредит по открытому счету используется:
	 \begin{choices}
	 \CC при устойчивых торговых связях экспортера и импортера;
	 \choice для обеспечения расчетов по единичной сделке между экспортером и импортером;
	 \choice при поставках инвестиционных товаров длительного пользования;
	 \choice при обеспечении обязательств заемщика банковской гарантией.
	 \end{choices}
\question Для современной мировой практики характерна тенденция:
	 \begin{choices}
	 \CC к удлинению сроков международных кредитов;
	 \choice сокращению сроков международных кредитов;
	 \choice сохранению сроков международных кредитов на сложившемся уровне;
	 \choice непредсказуемых колебаний сроков международных кредитов.
	 \end{choices}
\question Обострение политической напряженности в стране ведет:
	 \begin{choices}
	 \choice к росту ее кредитного рейтинга;
	 \CC снижению ее кредитного рейтинга;
	 \choice неизменному значению ее кредитного рейтинга;
	 \choice непредсказуемым колебаниям ее кредитного рейтинга.
	 \end{choices}
\question При увеличении темпов инфляции в стране и сохранении номинальных ставок неизменными, реальная ставка по кредитам в национальной валюте:
	 \begin{choices}
	 \choice вырастет;
	 \CC снизится;
	 \choice останется неизменной;
	 \choice изменится непредсказуемым образом.
	 \end{choices}
\question Увеличение национальным правительством официальной учетной ставки способствует:
	 \begin{choices}
	 \CC повышению номинальных кредитных ставок в национальной валюте;
	 \choice снижению номинальных кредитных ставок в национальной валюте;
	 \choice сохранению номинальных кредитных ставок в национальной валюте;
	 \choice непредсказуемым колебаниям номинальных кредитных ставок в национальной валюте.
	 \end{choices}
\question Основными заемщиками МВФ по состоянию на начало XXI в являются:
	 \begin{choices}
	 \choice промышленно развитые страны;
	 \choice развивающиеся страны и страны с переходной экономикой — экспортеры нефти;
	 \CC развивающиеся страны и страны с переходной экономикой — импортеры нефти;
	 \choice страны с плановой экономикой.
	 \end{choices}
\question Обязательным условием кредитования НЕ является:
	 \begin{choices}
	 \choice возврат привлеченных средств;
	 \choice обеспечение обязательств заемщика залогом или гарантией;
	 \CC наличие кредитной истории заемщика;
	 \choice целевой характер заимствований.
	 \end{choices}
\question К методам кредитной дискриминации НЕ относится:
	 \begin{choices}
	 \choice ограничение объемов предоставления кредитов;
	 \choice повышение кредитных ставок;
	 \CC предоставление кредита только в свободно конвертируемых валютах;
	 \choice ужесточение требований к обеспечению кредита.
	 \end{choices}
\question Банковские кредиты выдаются в:
	 \begin{choices}
	 \CC денежной форме;
	 \choice товарной форме;
	 \choice форме отсрочки платежа;
	 \choice форме патентов и лицензий.
	 \end{choices}



\end{multicols}
\end{questions}

\end{document}
