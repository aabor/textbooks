% !TeX program = lualatex -synctex=1 -interaction=nonstopmode --shell-escape %.tex

\documentclass[12pt, table]{exam}
\usepackage[rus]{borochkin}

\usepackage{borochkin_exam}

%%%%%%%%%%%%%%%%%%%%%%%%%%%%%%%%%%%%%%%%%
\professor
\iftagged{professor}{ \printanswers }
%%%%%%%%%%%%%%%%%%%%%%%%%%%%%%%%%%%%%%%%%

\examrunningheader{Международные валютно-кредитные отношения, к/р №2}
\examrunningfooter{А. А. Борочкин}

\begin{document}
\setcounter{section}{1\relax}%
\section{Вариант}

\noindent
\studentpersonalinfo{МВКО}

\normalsize
\begin{questions}
\question[40] Тест
\answerstotest

	
\question[20] 
\noaddpoints
\begin{subparts}
\subpart[10] 

\begin{solution}[21em]
\end{solution}

\subpart[10] 

\begin{solution}[12em]
\end{solution}

\end{subparts}
\addpoints

\question[10] 
\noaddpoints
\begin{subparts}
	\subpart[8] 
	\begin{solution}[8em]

	\end{solution}

	\subpart[2] 
	\begin{solution}[4em]
	\end{solution}
\end{subparts}
\addpoints

\question[15] 
\begin{subparts}
	\subpart[5] 
	\begin{solution}[8em]
		
	\end{solution}
	
	\subpart[5] 
	\begin{solution}[18em]
	\end{solution}
	\subpart[5] 
	\begin{solution}[18em]
	\end{solution}
	
\end{subparts}
\addpoints

\question[10] 
\begin{solution}[10em]
\end{solution}


\end{questions}

\pagebreak
\noindent\textbf{Тестовые вопросы (выберите один правильный ответ)}

\begin{questions}
\begin{multicols}{2}
\setlength{\columnsep}{1cm}

\question Наиболее значимой функцией кредита является:
	 \begin{choices}
	 \CC рыночное регулирование финансовых рынков;
	 \choice перераспределение финансовых ресурсов;
	 \choice финансирование государственного долга;
	 \choice минимизация трансакционных издержек.
	 \end{choices}
\question Причиной для применения кредитной дискриминации или кредитной блокады может служить:
	 \begin{choices}
	 \CC недостаточная надежность заемщика;
	 \choice объявление дефолта заемщиком;
	 \choice политические расхождения кредитора с заемщиком;
	 \choice все перечисленное.
	 \end{choices}
\question Рынки еврокредитов характеризует:
	 \begin{choices}
	 \choice стабильность процентных ставок;
	 \choice активное использование валют развивающихся стран;
	 \choice значительный объем обращающихся средств;
	 \CC государственное регулирование процентных ставок.
	 \end{choices}
\question Основными кредиторами на мировом кредитном рынке являются:
	 \begin{choices}
	 \choice национальные правительства;
	 \CC компании и корпорации;
	 \choice банки;
	 \choice частные лица.
	 \end{choices}
\question Формой коммерческого кредита не является:
	 \begin{choices}
	 \CC кредит на цели проведения торговых операций;
	 \choice отсрочка платежей, предоставляемая покупателю;
	 \choice аванс покупателя;
	 \choice предварительная оплата.
	 \end{choices}
\question Процентная ставка по краткосрочным кредитам, как правило:
	 \begin{choices}
	 \choice выше, чем по долгосрочным;
	 \choice ниже, чем по долгосрочным;
	 \choice такая же, как и по долгосрочным;
	 \CC отличается от ставки по долгосрочным кредитам непредсказуемым образом.
	 \end{choices}
\question В международной практике к краткосрочным кредитам относятся кредиты сроком:
	 \begin{choices}
	 \CC до 1 недели;
	 \choice до 1 месяца;
	 \choice до 1 квартала;
	 \choice до 1 года.
	 \end{choices}
\question Наиболее гибко управлять привлеченными средствами, использовать их по любому назначению позволяет:
	 \begin{choices}
	 \CC коммерческий кредит;
	 \choice банковский экспортный кредит;
	 \choice банковский финансовый кредит;
	 \choice двусторонний государственный кредит.
	 \end{choices}
\question Ставка по государственным международным кредитам, как правило:
	 \begin{choices}
	 \CC больше, чем по частным кредитам;
	 \choice меньше, чем по частным кредитам;
	 \choice такая же, как по частным кредитам;
	 \choice непредсказуемым образом отличается от ставки по частным кредитам.
	 \end{choices}
\question При увеличении рентабельности проекта срок его окупаемости:
	 \begin{choices}
	 \choice растет;
	 \CC снижается;
	 \choice остается неизменным;
	 \choice изменяется непредсказуемым образом.
	 \end{choices}
\question Увеличение кредитного рейтинга страны ведет:
	 \begin{choices}
	 \choice к росту процентных ставок по кредитам, предоставляемым резидентам данной страны;
	 \CC снижению процентных ставок по кредитам, предоставляемым резидентам данной страны;
	 \choice сохранению процентных ставок по кредитам, предоставляемым резидентам данной страны;
	 \choice непредсказуемым изменениям процентных ставок по кредитам, предоставляемым резидентам данной страны.
	 \end{choices}
\question Увеличение темпов инфляции в стране способствует:
	 \begin{choices}
	 \CC росту номинальных ставок по кредитам в национальной валюте;
	 \choice снижению номинальных ставок по кредитам в национальной валюте;
	 \choice сохранению номинальных ставок по кредитам в национальной валюте;
	 \choice непредсказуемым колебаниям номинальных ставок по кредитам в национальной валюте.
	 \end{choices}
\question Если курс валюты, в которой выдан кредит, растет, стоимость обслуживания кредита:
	 \begin{choices}
	 \choice растет;
	 \choice снижается;
	 \CC не изменяется;
	 \choice изменяется непредсказуемым образом.
	 \end{choices}
\question Кредитные риски по двусторонним кредитам:
	 \begin{choices}
	 \choice больше, чем по многосторонним;
	 \choice меньше, чем по многосторонним;
	 \CC такие же, как по многосторонним;
	 \choice непредсказуемым образом отличаются от рисков по много сторонним кредитам.
	 \end{choices}
\question При прочих равных обстоятельствах глобализация мирового кредитного рынка способствует:
	 \begin{choices}
	 \choice повышению стоимости обслуживания кредитов;
	 \choice снижению стоимости обслуживания кредитов;
	 \CC сохранению неизменной стоимости обслуживания кредитов;
	 \choice резким колебаниям стоимости обслуживания кредитов.
	 \end{choices}
\question Активными участниками мирового кредитного рынка, выступающими преимущественно в качестве заемщиков, а не кредиторов, являются:
	 \begin{choices}
	 \CC промышленно развитые страны;
	 \choice развивающиеся страны — экспортеры нефти;
	 \choice страны с переходной экономикой;
	 \choice страны с плановой экономикой.
	 \end{choices}



\end{multicols}
\end{questions}

\end{document}
