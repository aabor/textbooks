% !TeX program = lualatex -synctex=1 -interaction=nonstopmode --shell-escape %.tex

\documentclass[12pt, table]{exam}
\usepackage[rus]{borochkin}

\usepackage{borochkin_exam}

%%%%%%%%%%%%%%%%%%%%%%%%%%%%%%%%%%%%%%%%%
\professor
\iftagged{professor}{ \printanswers }
%%%%%%%%%%%%%%%%%%%%%%%%%%%%%%%%%%%%%%%%%

\examrunningheader{Международные валютно-кредитные отношения, к/р №3}
\examrunningfooter{А. А. Борочкин}

\begin{document}
\setcounter{section}{0\relax}%
\section{Вариант}

\noindent
\studentpersonalinfo{МВКО}

\normalsize
\begin{questions}
\question[40] Тест
\answerstotest

	
\question[20] 
\noaddpoints
\begin{subparts}
\subpart[10] 

\begin{solution}[21em]
\end{solution}

\subpart[10] 

\begin{solution}[12em]
\end{solution}

\end{subparts}
\addpoints

\question[10] 
\noaddpoints
\begin{subparts}
	\subpart[8] 
	\begin{solution}[8em]

	\end{solution}

	\subpart[2] 
	\begin{solution}[4em]
	\end{solution}
\end{subparts}
\addpoints

\question[15] 
\begin{subparts}
	\subpart[5] 
	\begin{solution}[8em]
		
	\end{solution}
	
	\subpart[5] 
	\begin{solution}[18em]
	\end{solution}
	\subpart[5] 
	\begin{solution}[18em]
	\end{solution}
	
\end{subparts}
\addpoints

\question[10] 
\begin{solution}[10em]
\end{solution}


\end{questions}

\pagebreak
\noindent\textbf{Тестовые вопросы (выберите один правильный ответ)}

\begin{questions}
\begin{multicols}{2}
\setlength{\columnsep}{1cm}

\question  Доля ценных бумаг развитых стран в оборотах мирового фондового рынка в последние десятилетия:
	 \begin{choices}
	 \choice растет;
	 \CC снижается;
	 \choice остается неизменной;
	 \choice случайным образом колеблется.
	 \end{choices}
\question  Наибольшее число мелких частных акционеров приходится:
	 \begin{choices}
	 \CC на развитые страны;
	 \choice развивающиеся страны;
	 \choice страны с переходной экономикой;
	 \choice одинаково для всех трех групп.
	 \end{choices}
\question  Зарубежные облигации в мировой практике применяются:
	 \begin{choices}
	 \CC дольше, чем еврооблигации;
	 \choice столько же, сколько еврооблигации;
	 \choice меньше, чем еврооблигации;
	 \choice не применяются.
	 \end{choices}
\question  Повышение дивидендов по акции способствует:
	 \begin{choices}
	 \CC росту курса акции;
	 \choice снижению курса акции;
	 \choice сохранению курса акции на неизменном уровне;
	 \choice непредсказуемым колебаниям курса акции.
	 \end{choices}
\question  А — котировка купонной облигации с фиксированной ставкой, Б — котировка бескупонной облигации Если ставки международного финансового рынка возрастут, то:
	 \begin{choices}
	 \choice А и Б увеличатся;
	 \CC А и Б уменьшатся;
	 \choice А увеличится, Б уменьшится;
	 \choice А уменьшится, Б увеличится.
	 \end{choices}
\question  Доходность вложений в бескупонную облигацию составляет 25\% годовых За год до погашения эта облигация будет продаваться по курсу:
	 \begin{choices}
	 \choice 50\% номинала;
	 \CC 75\% номинала;
	 \choice 80\% номинала;
	 \choice 90\% номинала.
	 \end{choices}
\question  Политические риски по сравнению с валютными рисками характеризуются:
	 \begin{choices}
	 \choice большими потенциальными потерями и вероятностью их понести;
	 \choice меньшими потенциальными потерями и вероятностью их понести;
	 \CC большими потенциальными потерями и меньшей вероятностью их понести;
	 \choice меньшими потенциальными потерями и большей вероятностью их понести.
	 \end{choices}
\question  При росте политических рисков в стране котировки ее международных ценных бумаг:
	 \begin{choices}
	 \choice увеличиваются;
	 \CC снижаются;
	 \choice остаются неизменными;
	 \choice изменяются непредсказуемым образом.
	 \end{choices}
\question  Основное направление деятельности хеджевых фондов на международном фондовом рынке это:
	 \begin{choices}
	 \choice долгосрочные вложения в акции;
	 \choice долгосрочные вложения в облигации;
	 \CC спекуляции с ценными бумагами;
	 \choice посреднические операции.
	 \end{choices}
\question  К международному фондовому рынку НЕ относятся:
	 \begin{choices}
	 \choice покупка российским резидентом казначейских обязательств США у резидента США;
	 \CC покупка российским резидентом казначейских обязательств США у российского резидента;
	 \choice покупка резидентом США российских государственных ценных бумаг у резидента России;
	 \choice покупка резидентом США российских государственных ценных бумаг у резидента Германии.
	 \end{choices}
\question  Резидент Германии в соответствии с законодательством Германии приобретает у другого резидента Германии российскую облигацию федерального займа, ранее приобретенную им на российском рынке В этом случае ОФЗ выступает как:
	 \begin{choices}
	 \choice национальная облигация;
	 \CC иностранная облигация;
	 \choice международная облигация;
	 \choice еврооблигация.
	 \end{choices}
\question  Для еврооблигаций нехарактерно:
	 \begin{choices}
	 \choice размещение облигаций через международные финансовые синдикаты;
	 \choice свободно конвертируемая валюта номинала;
	 \CC государственная регистрация выпуска;
	 \choice одновременное размещение облигаций на фондовых рынках нескольких стран.
	 \end{choices}
\question  Характерной особенностью современного международного фондового рынка является:
	 \begin{choices}
	 \choice сокращение доли срочных инструментов в оборотах рынка;
	 \choice сокращение транснациональных филиальных сетей профессиональных фондовых посредников;
	 \choice снижение налоговых ставок на операции с ценными бумагами;
	 \CC рост диверсификации инвестиционных портфелей отдельных инвесторов по странам и регионам.
	 \end{choices}
\question  К росту политических рисков по ценным бумагам страны НЕ ведут:
	 \begin{choices}
	 \choice осуществление экономических реформ;
	 \choice введение визовых ограничений;
	 \choice смена правящей партии;
	 \CC девальвация национальной валюты.
	 \end{choices}
\question  Ликвидность финансовых инвестиций, как правило:
	 \begin{choices}
	 \CC больше, чем реальных;
	 \choice меньше, чем реальных;
	 \choice такая же, как у реальных;
	 \choice непредсказуемым образом отличается от реальных.
	 \end{choices}
\question  Товарные кредиты относятся:
	 \begin{choices}
	 \CC к реальным инвестициям;
	 \choice прямым инвестициям;
	 \choice портфельным инвестициям;
	 \choice прочим инвестициям.
	 \end{choices}
\question  Соглашение о разделе продукции относится:
	 \begin{choices}
	 \choice к реальным инвестициям;
	 \CC прямым инвестициям;
	 \choice портфельным инвестициям;
	 \choice прочим инвестициям.
	 \end{choices}
\question  Для современного международного рынка инвестиций характерна следующая структура участников:
	 \begin{choices}
	 \choice развитые и развивающиеся страны являются нетто экспортерами капитала;
	 \choice развитые и развивающиеся страны являются нетто импортерами капитала;
	 \CC развитые страны являются нетто экспортерами капитала, развивающиеся страны являются нетто импортерами капитала;
	 \choice развитые страны являются нетто импортерами капитала, развивающиеся страны являются нетто экспортерами капитала.
	 \end{choices}
\question  Притоку современных технологий способствует расширение объемов:
	 \begin{choices}
	 \CC прямых иностранных инвестиций в национальную экономику;
	 \choice портфельных иностранных инвестиций в национальную экономику;
	 \choice прочих иностранных инвестиций в национальную экономику;
	 \choice ничто из перечисленного.
	 \end{choices}
\question  Рост уровня заработной платы в стране способствует:
	 \begin{choices}
	 \CC улучшению инвестиционного климата;
	 \choice ухудшению инвестиционного климата;
	 \choice сохранению инвестиционного климата неизменным;
	 \choice непредсказуемым изменениям  инвестиционного климата.
	 \end{choices}
\question  В последние десятилетия объем международных слияний и поглощений:
	 \begin{choices}
	 \CC растет;
	 \choice снижается;
	 \choice остается неизменным;
	 \choice стохастически колеблется.
	 \end{choices}
\question  Наибольший объем прямых иностранных инвестиций приходится на вложения в экономику:
	 \begin{choices}
	 \CC промышленно развитых стран;
	 \choice развивающихся стран;
	 \choice стран с переходной экономикой;
	 \choice стран с плановой экономикой.
	 \end{choices}
\question  После заключения межгосударственного договора об устранении двойного налогообложения объем экспорта и импорта инвестиций между этими странами:
	 \begin{choices}
	 \CC возрастет;
	 \choice снизится;
	 \choice останется неизменным;
	 \choice будет стохастически изменяться.
	 \end{choices}
\question  Для национальной экономики:
	 \begin{choices}
	 \CC прямые иностранные инвестиции предпочтительней портфельных;
	 \choice портфельные иностранные инвестиции предпочтительней прямых;
	 \choice и те и другие инвестиции равно полезны;
	 \choice и те и другие инвестиции равно вредны.
	 \end{choices}
\question  Приобретение контрольного пакета акций компании относится:
	 \begin{choices}
	 \choice к реальным инвестициям;
	 \choice финансовым инвестициям;
	 \CC прямым инвестициям;
	 \choice портфельным инвестициям.
	 \end{choices}
\question  Характерными особенностями прямых инвестиций являются:
	 \begin{choices}
	 \CC долгосрочный характер;
	 \choice краткосрочный характер;
	 \choice спекулятивный характер;
	 \choice низкая доходность вложений.
	 \end{choices}
\question  Инвестиции в драгоценные металлы, камни и предметы коллекционирования, как правило, относятся:
	 \begin{choices}
	 \choice к ссудным инвестициям;
	 \choice реальным инвестициям;
	 \choice финансовым инвестициям;
	 \CC долгосрочным инвестициям.
	 \end{choices}
\question  Инвестиции в развивающиеся страны отличаются от инвестиций в развитые страны:
	 \begin{choices}
	 \choice более низкой доходностью;
	 \choice большей долей портфельных инвестиций;
	 \CC большей долей прямых инвестиций;
	 \choice большей долей инвестиций в нематериальные активы.
	 \end{choices}



\end{multicols}
\end{questions}

\end{document}
