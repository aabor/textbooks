% !TeX program = lualatex -synctex=1 -interaction=nonstopmode --shell-escape %.tex

\documentclass[12pt, table]{exam}
\usepackage[rus]{borochkin}

\usepackage{borochkin_exam}

%%%%%%%%%%%%%%%%%%%%%%%%%%%%%%%%%%%%%%%%%
\professor
\iftagged{professor}{ \printanswers }
%%%%%%%%%%%%%%%%%%%%%%%%%%%%%%%%%%%%%%%%%

\examrunningheader{Международные валютно-кредитные отношения, к/р №1}
\examrunningfooter{А. А. Борочкин}


\begin{document}
\setcounter{section}{0\relax}%
\section{Вариант}

\noindent
\studentpersonalinfo{МВКО}

\normalsize
\begin{questions}
\question[40] Тест
\answerstotest

	
\question[20] 
\noaddpoints
\begin{subparts}
\subpart[10] 

\begin{solution}[21em]
\end{solution}

\subpart[10] 

\begin{solution}[12em]
\end{solution}

\end{subparts}
\addpoints

\question[10] 
\noaddpoints
\begin{subparts}
	\subpart[8] 
	\begin{solution}[8em]

	\end{solution}

	\subpart[2] 
	\begin{solution}[4em]
	\end{solution}
\end{subparts}
\addpoints

\question[15] 
\begin{subparts}
	\subpart[5] 
	\begin{solution}[8em]
		
	\end{solution}
	
	\subpart[5] 
	\begin{solution}[18em]
	\end{solution}
	\subpart[5] 
	\begin{solution}[18em]
	\end{solution}
	
\end{subparts}
\addpoints

\question[10] 
\begin{solution}[10em]
\end{solution}


\end{questions}

\pagebreak
\noindent\textbf{Тестовые вопросы (выберите один правильный ответ)}

\begin{questions}
\begin{multicols}{2}
\setlength{\columnsep}{1cm}

\question Какой аспект научно-технического прогресса в наибольшей степени способствовал мировой экономической глобализации?
	 \begin{choices}
	 \CC рост вычислительной мощности компьютеров и увеличение их доступности;
	 \choice  внедрение энергосберегающих технологий в производственных отраслях;
	 \choice освоение космического пространства;
	 \choice механизация сельского хозяйства и рыболовства.
	 \end{choices}
\question Наибольшая степень интернационализации характерна для:
	 \begin{choices}
	 \choice товарного производства;
	 \choice сферы услуг;
	 \choice торговли;
	 \CC финансовой сферы.
	 \end{choices}
\question В условиях экономической глобализации доля прямых иностранных инвестиций в общем объеме мировых инвестиций:
	 \begin{choices}
	 \CC растет;
	 \choice снижается;
	 \choice остается неизменной;
	 \choice изменяется непредсказуемым образом.
	 \end{choices}
\question Показателем степени экономической глобализации является:
	 \begin{choices}
	 \choice доля англоязычных материалов в общем объеме печатной продукции;
	 \CC доля международной торговли в мировом ВВП;
	 \choice доля глобальных телекоммуникаций в общем объеме мировых телекоммуникаций;
	 \choice все перечисленное.
	 \end{choices}
\question В условиях финансовой глобализации устойчивость финансовых рынков развивающихся стран:
	 \begin{choices}
	 \choice увеличивается;
	 \CC уменьшается;
	 \choice остается неизменной;
	 \choice изменяется непредсказуемым образом.
	 \end{choices}
\question Какие из перечисленных мер способствуют ограничению международного движения спекулятивного капитала:
	 \begin{choices}
	 \choice расширение состава обязательной отчетности банков в части международных финансовых операций;
	 \choice введение налога на валюто обменные операции;
	 \choice завышение национальных процентных ставок;
	 \CC все вышеперечисленное.
	 \end{choices}
\question Глобальный финансовый рынок включает в себя:
	 \begin{choices}
	 \choice национальные финансовые рынки развитых стран;
	 \choice все национальные финансовые рынки;
	 \CC все национальные финансовые рынки и международный финансовый рынок;
	 \choice международный финансовый рынок, финансовые структуры ООН и МВФ.
	 \end{choices}
\question Международный финансовый рынок функционирует:
	 \begin{choices}
	 \choice с 8 до 18 часов по времени Лондона;
	 \choice с 6 до 20 часов по времени Лондона;
	 \choice с 4 до 22 часов по времени Лондона;
	 \CC круглосуточно.
	 \end{choices}
\question Страна А интенсивно вовлечена в международное разделение труда Страна Б проводит политику самообеспечения и минимизации объемов внешнеэкономических связей Национальный финансовый рынок какой из двух стран при прочих равных обстоятельствах в большей степени будет интегрирован в мировой?
	 \begin{choices}
	 \CC страны А;
	 \choice страны Б;
	 \choice обеих стран;
	 \choice непредсказуемо.
	 \end{choices}
\question Наибольшим уровнем риска характеризуются операции следующей категории участников международного финансового рынка:
	 \begin{choices}
	 \CC спекулянтытрейдеры;
	 \choice арбитражеры;
	 \choice посредники;
	 \choice хеджеры.
	 \end{choices}
\question В условиях глобального финансового рынка локальные финансовые кризисы:
	 \begin{choices}
	 \choice резко ослабляются;
	 \choice не претерпевают изменений;
	 \CC перерастают в региональные и мировые кризисы;
	 \choice ничто из перечисленного.
	 \end{choices}
\question Демократизация страны способствует:
	 \begin{choices}
	 \choice усилению интеграции национального финансового рынка в мировой;
	 \choice ослаблению интеграции национального финансового рынка в мировой;
	 \choice сохранению степени интеграции национального финансового рынка в мировой на неизменном уровне;
	 \CC непредсказуемым изменениям степени интеграции национального финансового рынка в мировой.
	 \end{choices}
\question На современном международном финансовом рынке конкуренция между участниками:
	 \begin{choices}
	 \CC растет;
	 \choice снижается;
	 \choice остается на неизменном уровне;
	 \choice изменяется без определенной тенденции.
	 \end{choices}
\question Наименьшими возможностями по привлечению средств на международном финансовом рынке обладают:
	 \begin{choices}
	 \choice национальные правительства развитых государств;
	 \CC национальные правительства развивающихся государств;
	 \choice транснациональные банки;
	 \choice международные финансовые организации.
	 \end{choices}
\question Интеграции национального финансового рынка в мировой способствует:
	 \begin{choices}
	 \choice ужесточение законодательства о движении капитала;
	 \choice увеличение объемов международной торговли;
	 \CC переход к экспансионистской денежно-кредитной политике;
	 \choice увеличение объемов эмиссии государственных ценных бумаг.
	 \end{choices}
\question Основными факторами развития мирового финансового рынка на современном этапе являются:
	 \begin{choices}
	 \CC демонетизация золота;
	 \choice интеграция национальных финансовых рынков развивающихся стран в международный;
	 \choice интенсивный рост наукоемких отраслей мировой экономики;
	 \choice ослабление государственного контроля за международным движением капитала.
	 \end{choices}
\question Последствиями финансовой глобализации для мирового финансового рынка НЕ являются:
	 \begin{choices}
	 \choice повышение роли биржевого рынка;
	 \choice увеличение числа участников;
	 \choice диверсификация обращающихся на рынке инструментов;
	 \CC снижение значимости ценных бумаг для рынка.
	 \end{choices}
\question Валютный рынок наиболее тесно связан:
	 \begin{choices}
	 \choice с производством;
	 \CC внешней торговлей;
	 \choice внутренней торговлей;
	 \choice потреблением.
	 \end{choices}
\question Крупнейшие участники валютного рынка имеют больше преимуществ по сравнению с мелкими и средними участниками на:
	 \begin{choices}
	 \choice биржевом рынке;
	 \CC внебиржевом рынке;
	 \choice срочном рынке;
	 \choice текущем рынке.
	 \end{choices}
\question Аббревиатурой ОТС обозначается:
	 \begin{choices}
	 \choice биржевой валютный рынок;
	 \CC внебиржевой валютный рынок;
	 \choice срочный валютный рынок;
	 \choice текущий валютный рынок.
	 \end{choices}
\question Валютный курс выражает:
	 \begin{choices}
	 \choice покупательную способность валюты;
	 \choice полезность валюты;
	 \CC пропорцию обмена данной валюты на другие;
	 \choice некоторую социальную условность.
	 \end{choices}
\question Курс покупки ниже курса продажи:
	 \begin{choices}
	 \choice всегда;
	 \CC в прямой котировке;
	 \choice в косвенной котировке;
	 \choice никогда.
	 \end{choices}
\question Существенные расхождения официального и рыночного курсов характерны:
	 \begin{choices}
	 \choice для стран с экспортно ориентированной экономикой;
	 \choice стран с экономикой, ориентированной на самообеспечение;
	 \CC стран с государственным планированием экономики;
	 \choice стран со свободной рыночной экономикой.
	 \end{choices}
\question Для прогнозирования валютных курсов в долгосрочной перспективе эффективнее:
	 \begin{choices}
	 \choice анализ трендов;
	 \CC фундаментальный анализ;
	 \choice графический анализ;
	 \choice все три метода равно эффективны.
	 \end{choices}
\question При усилении инфляционных ожиданий в России курс рубля к ведущим мировым валютам:
	 \begin{choices}
	 \choice возрастет;
	 \CC уменьшится;
	 \choice не изменится;
	 \choice изменится непредсказуемым образом.
	 \end{choices}
\question Долгосрочное снижение курса национальной валюты окажет следующее влияние на структуру национальной экономики:
	 \begin{choices}
	 \CC рост доли экспортных отраслей, снижение доли отраслей, ориентированных на импорт;
	 \choice снижение доли экспортных отраслей, рост доли отраслей, ориентированных на импорт;
	 \choice отсутствие изменений;
	 \choice непредсказуемые изменения.
	 \end{choices}
\question Средством повышения конвертируемости национальной валюты является:
	 \begin{choices}
	 \choice повышение курса национальной валюты;
	 \CC отмена валютных ограничений;
	 \choice отказ от регулирования курса валюты;
	 \choice эмиссия национальной валюты.
	 \end{choices}



\end{multicols}
\end{questions}

\end{document}
