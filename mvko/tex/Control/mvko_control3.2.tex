% !TeX program = lualatex -synctex=1 -interaction=nonstopmode --shell-escape %.tex

\documentclass[12pt, table]{exam}
\usepackage[rus]{borochkin}

\usepackage{borochkin_exam}

%%%%%%%%%%%%%%%%%%%%%%%%%%%%%%%%%%%%%%%%%
\professor
\iftagged{professor}{ \printanswers }
%%%%%%%%%%%%%%%%%%%%%%%%%%%%%%%%%%%%%%%%%

\examrunningheader{Международные валютно-кредитные отношения, к/р №3}
\examrunningfooter{А. А. Борочкин}

\begin{document}
\setcounter{section}{1\relax}%
\section{Вариант}

\noindent
\studentpersonalinfo{МВКО}

\normalsize
\begin{questions}
\question[40] Тест
\answerstotest

	
\question[20] 
\noaddpoints
\begin{subparts}
\subpart[10] 

\begin{solution}[21em]
\end{solution}

\subpart[10] 

\begin{solution}[12em]
\end{solution}

\end{subparts}
\addpoints

\question[10] 
\noaddpoints
\begin{subparts}
	\subpart[8] 
	\begin{solution}[8em]

	\end{solution}

	\subpart[2] 
	\begin{solution}[4em]
	\end{solution}
\end{subparts}
\addpoints

\question[15] 
\begin{subparts}
	\subpart[5] 
	\begin{solution}[8em]
		
	\end{solution}
	
	\subpart[5] 
	\begin{solution}[18em]
	\end{solution}
	\subpart[5] 
	\begin{solution}[18em]
	\end{solution}
	
\end{subparts}
\addpoints

\question[10] 
\begin{solution}[10em]
\end{solution}


\end{questions}

\pagebreak
\noindent\textbf{Тестовые вопросы (выберите один правильный ответ)}

\begin{questions}
\begin{multicols}{2}
\setlength{\columnsep}{1cm}

\question  Государственное регулирование:
	 \begin{choices}
	 \choice оказывает более сильное влияние на национальный фондовый рынок;
	 \CC оказывает более сильное влияние на международный фондовый рынок;
	 \choice оказывает одинаковое влияние на оба сегмента мирового фондового рынка;
	 \choice не оказывает никакого влияния на оба сегмента мирового фондового рынка.
	 \end{choices}
\question  В течение последних десятилетий темпы роста оборотов рынка производных финансовых инструментов на ценные бумаги:
	 \begin{choices}
	 \CC отстают от темпов роста оборотов мирового фондового рынка в целом;
	 \choice близки к темпам роста оборотов мирового фондового рынка в целом;
	 \choice опережают темпы роста оборотов мирового фондового рынка в целом;
	 \choice отрицательны.
	 \end{choices}
\question  Международные портфельные инвестиции отличаются от прямых тем, что:
	 \begin{choices}
	 \CC к портфельным инвестициям относится покупка акций, а к прямым — облигаций;
	 \choice к портфельным инвестициям относится покупка облигаций, а к прямым — акций;
	 \choice к портфельным инвестициям относится покупка крупного пакета акций, а к прямым — покупка меньшего объема акций;
	 \choice к прямым инвестициям относится покупка крупного пакета акций, а к портфельным — покупка меньшего объема акций.
	 \end{choices}
\question  Российское предприятие эмитирует облигации с номиналом в фунтах стерлингов Великобритании и размещает их на фондовом рынке Великобритании Эти облигации являются:
	 \begin{choices}
	 \CC еврооблигациями;
	 \choice национальными облигациями;
	 \choice зарубежными облигациями;
	 \choice ничем из перечисленного.
	 \end{choices}
\question  Если рыночная стоимость акции и дивиденд по ней выросли, как изменится доходность вложений в акцию?
	 \begin{choices}
	 \choice увеличится;
	 \CC уменьшится;
	 \choice останется неизменным;
	 \choice изменится непредсказуемым образом.
	 \end{choices}
\question  На рынке обращаются две облигации одного эмитента: купонная со ставкой 10\% и бескупонная Валюта номинала и срок обращения по облигациям равны Если доходность вложений в обе облигации равны, то:
	 \begin{choices}
	 \choice котировка купонной облигации больше котировки бескупонной облигации;
	 \CC котировка бескупонной облигации больше котировки купонной облигации;
	 \choice котировка купонной облигации равна котировке купонной облигации;
	 \choice котировка купонной облигации непредсказуемым образом отличается от котировки купонной облигации.
	 \end{choices}
\question  Если суверенный кредитный рейтинг страны, резидентом которой является эмитент еврооблигации растет, то котировка еврооблигации:
	 \begin{choices}
	 \choice увеличивается;
	 \choice снижается;
	 \CC остается неизменным;
	 \choice изменяется непредсказуемым образом.
	 \end{choices}
\question  При росте курса валюты, в которой номинированы акции, стоимость АДР на эти акции:
	 \begin{choices}
	 \choice увеличивается;
	 \CC снижается;
	 \choice остается неизменным;
	 \choice изменяется непредсказуемым образом.
	 \end{choices}
\question  Замещение валют европейских стран единой европейской валютой евро способствует:
	 \begin{choices}
	 \choice росту валютных рисков на мировом фондовом рынке;
	 \choice снижению валютных рисков на мировом фондовом рынке;
	 \CC сохранению валютных рисков на мировом фондовом рынке неизменными;
	 \choice непредсказуемым колебаниям валютных рисков на мировом фондовом рынке.
	 \end{choices}
\question  Основным источником средств хеджевых фондов является:
	 \begin{choices}
	 \choice уставный капитал;
	 \CC доход от продажи собственных облигаций;
	 \choice доход от привлечения депозитов;
	 \choice заемные средства.
	 \end{choices}
\question  Мировой рынок акций характеризуется:
	 \begin{choices}
	 \choice большей степенью глобализации, чем мировой рынок облигаций;
	 \CC меньшей степенью глобализации, чем мировой рынок облигаций;
	 \choice такой же степенью глобализации, как и мировой рынок облигаций;
	 \choice практически полным отсутствием глобализации.
	 \end{choices}
\question  Характерной особенностью современного мирового фондового рынка являются:
	 \begin{choices}
	 \choice усиление государственного регулирования рынка;
	 \choice рост значимости неорганизованного сегмента рынка;
	 \CC уменьшение объемов долга развивающихся стран в форме ценных бумаг;
	 \choice упрощение финансовых схем.
	 \end{choices}
\question  Финансовый вексель российского банка с номиналом в евро, реализованный российскому предприятию, является:
	 \begin{choices}
	 \choice обычной ценной бумагой;
	 \choice евробумагой;
	 \choice зарубежной ценной бумагой;
	 \CC международной ценной бумагой.
	 \end{choices}
\question  Основными факторами возникновения кризисов мирового фондового рынка являются:
	 \begin{choices}
	 \choice налогообложение фондовых операций;
	 \choice неэффективность технической инфраструктуры рынка;
	 \choice занижение нормы прибыли в финансовом секторе;
	 \CC экономически необоснованные операции с фиктивным капиталом на фондовом рынке.
	 \end{choices}
\question  Деятельность институциональных инвесторов основывается на следующих принципах:
	 \begin{choices}
	 \CC минимизация рисков портфеля активов;
	 \choice широкое использование валютных спекуляций;
	 \choice диверсификация инвестиционного портфеля;
	 \choice ограниченное использование заемных средств.
	 \end{choices}
\question  Доходность прямых инвестиций для инвестора:
	 \begin{choices}
	 \CC важна больше, чем доходность портфельных инвестиций;
	 \choice важна меньше, чем доходность портфельных инвестиций;
	 \choice важна так же, как доходность портфельных инвестиций;
	 \choice вообще не важна.
	 \end{choices}
\question  При прочих равных обстоятельствах диверсификация рисков по прямым инвестициям:
	 \begin{choices}
	 \choice больше, чем по портфельным инвестициям;
	 \CC меньше, чем по портфельным инвестициям;
	 \choice такая же, как по портфельным инвестициям;
	 \choice непредсказуемо отличается от диверсификации рисков по портфельным инвестициям.
	 \end{choices}
\question  Спекулятивный характер, как правило, носят:
	 \begin{choices}
	 \choice реальные инвестиции;
	 \choice прямые инвестиции;
	 \CC краткосрочные инвестиции;
	 \choice смешанные инвестиции.
	 \end{choices}
\question  Мобильность прямых инвестиций:
	 \begin{choices}
	 \CC больше, чем портфельных инвестиций;
	 \choice меньше, чем портфельных инвестиций;
	 \choice такая же, как у портфельных инвестиций;
	 \choice непредсказуемо отличается от мобильности портфельных инвестиций.
	 \end{choices}
\question  Ведущими участниками рынка прямых иностранных инвестиций являются:
	 \begin{choices}
	 \CC транснациональные компании;
	 \choice международные банки;
	 \choice инвестиционные фонды;
	 \choice национальные правительства.
	 \end{choices}
\question  В последние десятилетия доля портфельных инвестиций в общем объеме мирового инвестиционного рынка:
	 \begin{choices}
	 \CC растет;
	 \choice снижается;
	 \choice остается неизменной;
	 \choice стохастически колеблется.
	 \end{choices}
\question  Снижение суверенного кредитного рейтинга страны, как правило, сопровождается:
	 \begin{choices}
	 \CC улучшением инвестиционного климата;
	 \choice ухудшением инвестиционного климата;
	 \choice сохранением инвестиционного климата неизменным;
	 \choice непредсказуемыми изменениями инвестиционного климата.
	 \end{choices}
\question  Характерной особенностью развития мирового инвестиционного рынка в последние десятилетия является:
	 \begin{choices}
	 \CC ужесточение инвестиционного законодательства большинства стран;
	 \choice либерализация  инвестиционного законодательства большинства стран;
	 \choice ужесточение инвестиционного законодательства в странах с либеральным законодательством и либерализация инвестиционного законодательства в странах с жестким законодательством;
	 \choice ужесточение инвестиционного законодательства в странах с жестким законодательством и либерализация инвестиционного законодательства в странах с либеральным законодательством.
	 \end{choices}
\question  Международные инвестиции влияют на конкуренцию на внутреннем товарном рынке следующим образом:
	 \begin{choices}
	 \CC любые инвестиции увеличивают конкуренцию;
	 \choice прямые инвестиции уменьшают конкуренцию, а портфельные — увеличивают;
	 \choice портфельные инвестиции увеличивают конкуренцию, а прямые не влияют на нее;
	 \choice прямые инвестиции увеличивают конкуренцию, а портфельные не влияют на нее.
	 \end{choices}
\question  Доходность государственных инвестиций, как правило:
	 \begin{choices}
	 \choice выше доходности частных;
	 \choice ниже доходности частных;
	 \CC равны доходности частных;
	 \choice непредсказуемо отличаются от доходности частных.
	 \end{choices}
\question  К прямым инвестициям относятся:
	 \begin{choices}
	 \CC приобретение контрольного, более 51\%, пакета акций компании;
	 \choice приобретение малого, до 5\%, пакета акций компании;
	 \choice приобретение облигаций компании;
	 \choice предоставление кредита компании.
	 \end{choices}
\question  Невозможны такие формы международных инвестиций, как:
	 \begin{choices}
	 \choice прямые реальные инвестиции;
	 \choice портфельные реальные инвестиции;
	 \choice прямые финансовые инвестиции;
	 \CC портфельные финансовые инвестиции.
	 \end{choices}
\question  Улучшению инвестиционного климата в стране способствует:
	 \begin{choices}
	 \choice расширение государственного внутреннего долга;
	 \choice либерализация валютного законодательства;
	 \CC принятие государственного бюджета;
	 \choice снижение налоговых ставок.
	 \end{choices}



\end{multicols}
\end{questions}

\end{document}
