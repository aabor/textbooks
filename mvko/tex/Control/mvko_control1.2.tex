% !TeX program = lualatex -synctex=1 -interaction=nonstopmode --shell-escape %.tex

\documentclass[12pt, table]{exam}
\usepackage[rus]{borochkin}

\usepackage{borochkin_exam}

%%%%%%%%%%%%%%%%%%%%%%%%%%%%%%%%%%%%%%%%%
\professor
\iftagged{professor}{ \printanswers }
%%%%%%%%%%%%%%%%%%%%%%%%%%%%%%%%%%%%%%%%%

\examrunningheader{Международные валютно-кредитные отношения, к/р №1}
\examrunningfooter{А. А. Борочкин}

\begin{document}
\setcounter{section}{1\relax}%
\section{Вариант}

\noindent
\studentpersonalinfo{МВКО}

\normalsize
\begin{questions}
\question[40] Тест
\answerstotest

	
\question[20] 
\noaddpoints
\begin{subparts}
\subpart[10] 

\begin{solution}[21em]
\end{solution}

\subpart[10] 

\begin{solution}[12em]
\end{solution}

\end{subparts}
\addpoints

\question[10] 
\noaddpoints
\begin{subparts}
	\subpart[8] 
	\begin{solution}[8em]

	\end{solution}

	\subpart[2] 
	\begin{solution}[4em]
	\end{solution}
\end{subparts}
\addpoints

\question[15] 
\begin{subparts}
	\subpart[5] 
	\begin{solution}[8em]
		
	\end{solution}
	
	\subpart[5] 
	\begin{solution}[18em]
	\end{solution}
	\subpart[5] 
	\begin{solution}[18em]
	\end{solution}
	
\end{subparts}
\addpoints

\question[10] 
\begin{solution}[10em]
\end{solution}


\end{questions}

\pagebreak
\noindent\textbf{Тестовые вопросы (выберите один правильный ответ)}

\begin{questions}
\begin{multicols}{2}
\setlength{\columnsep}{1cm}

\question Рост международного финансового рынка преимущественно за счет спекулятивных операций — характерная черта:
	 \begin{choices}
	 \CC XIX в;
	 \choice нач XX в;
	 \choice середины XX в;
	 \choice к XX  нач XXI в.
	 \end{choices}
\question Доля торговли технологиями, патентами и лицензиями в общем объеме мировой торговли в конце XX — начале XXI в:
	 \begin{choices}
	 \choice растет;
	 \choice снижается;
	 \choice остается неизменной;
	 \CC изменяется непредсказуемым образом.
	 \end{choices}
\question Доля внутри отраслевой торговли в общем объеме мировой торговли в нач XXI в:
	 \begin{choices}
	 \CC растет;
	 \choice снижается;
	 \choice остается неизменной;
	 \choice изменяется непредсказуемым образом.
	 \end{choices}
\question В условиях финансовой глобализации конкуренция на внутренних рынках финансовых услуг:
	 \begin{choices}
	 \choice увеличивается;
	 \CC уменьшается;
	 \choice остается неизменной;
	 \choice изменяется непредсказуемым образом.
	 \end{choices}
\question В условиях современной финансовой глобализации наибольшими темпами растут обороты:
	 \begin{choices}
	 \choice рынка акций;
	 \CC рынка облигаций;
	 \choice кредитного рынка;
	 \choice рынка производных финансовых инструментов.
	 \end{choices}
\question Характерной чертой финансовой глобализации является:
	 \begin{choices}
	 \choice повышение устойчивости процентных ставок;
	 \choice рост концентрации банковского капитала;
	 \choice ужесточение государственного контроля за международными финансовыми операциями;
	 \CC увеличение срочности международных финансовых операций.
	 \end{choices}
\question Обороты современного мирового финансового рынка:
	 \begin{choices}
	 \choice больше, чем обороты мировой товарной торговли;
	 \choice равны оборотам мировой товарной торговли;
	 \CC меньше, чем обороты мировой товарной торговли;
	 \choice непредсказуемо отличаются от оборотов мировой товарной торговли.
	 \end{choices}
\question Повышение суверенного кредитного рейтинга страны способствует:
	 \begin{choices}
	 \choice усилению интеграции национального финансового рынка в мировой;
	 \choice ослаблению интеграции национального финансового рынка в мировой;
	 \choice сохранению степени интеграции национального финансового рынка в мировой на неизменном уровне;
	 \CC непредсказуемым изменениям степени интеграции национального финансового рынка в мировой.
	 \end{choices}
\question Затраты прямых участников международного финансового рынка:
	 \begin{choices}
	 \CC больше, чем у опосредованных;
	 \choice меньше, чем у опосредованных;
	 \choice такие же, как у опосредованных;
	 \choice непредсказуемым образом отличаются от затрат опосредованных.
	 \end{choices}
\question В наибольшей степени ориентированы на осуществление операций с производными финансовыми инструментами:
	 \begin{choices}
	 \CC трейдеры;
	 \choice арбитражеры;
	 \choice  посредники;
	 \choice хеджеры.
	 \end{choices}
\question Спекулятивные операции на мировом финансовом рынке:
	 \begin{choices}
	 \choice увеличивают его ликвидность и ценовую стабильность;
	 \choice уменьшают его ликвидность и ценовую стабильность;
	 \CC увеличивают его ликвидность и уменьшают ценовую стабильность;
	 \choice уменьшают его ликвидность и увеличивают ценовую стабильность.
	 \end{choices}
\question Роль институциональных инвесторов в функционировании международного финансового рынка в последние десятилетия:
	 \begin{choices}
	 \choice росла;
	 \choice снижалась;
	 \choice оставалась неизменной;
	 \CC изменялась без определенной тенденции.
	 \end{choices}
\question В наибольшей степени формирование международного финансового рынка выгодно для:
	 \begin{choices}
	 \CC промышленно развитых стран;
	 \choice развивающихся стран — экспортеров нефти;
	 \choice развивающихся стран — импортеров нефти;
	 \choice стран с переходной экономикой.
	 \end{choices}
\question Характерными чертами современного международного финансового рынка НЕ являются:
	 \begin{choices}
	 \choice широкое использование компьютерных технологий связи;
	 \CC высокая степень диверсификации обращающихся на рынке инструментов;
	 \choice стабильность котировок и ставок по обращающимся на рынке инструментам;
	 \choice высокий уровень налоговых ставок.
	 \end{choices}
\question Характерными чертами современного мирового финансового рынка являются:
	 \begin{choices}
	 \choice масштабные спекулятивные операции;
	 \choice свободное перемещение капиталов между сегментами рынка;
	 \CC формирование финансовых пузырей;
	 \choice все вышеперечисленное.
	 \end{choices}
\question Функциями современного мирового финансового рынка является:
	 \begin{choices}
	 \CC страхование рисков участников рынка;
	 \choice перераспределение финансовых ресурсов мировой экономики;
	 \choice формирование равновесного уровня процентных ставок, валютных курсов и котировок ценных бумаг;
	 \choice все вышеперечисленное.
	 \end{choices}
\question На степень интеграции страны в мировую экономику влияют:
	 \begin{choices}
	 \choice географическое положение страны;
	 \choice численность населения страны;
	 \choice этническая структура населения страны;
	 \CC все вышеперечисленное.
	 \end{choices}
\question Доля форвардного рынка в структуре международного валютного рынка на протяжении последних десятилетий:
	 \begin{choices}
	 \choice росла;
	 \CC снижалась;
	 \choice оставалась неизменной;
	 \choice стохастически колебалась.
	 \end{choices}
\question Стандартизация условий контракта характерна:
	 \begin{choices}
	 \choice для биржевого валютного рынка;
	 \CC внебиржевого валютного рынка;
	 \choice срочного валютного рынка;
	 \choice текущего валютного рынка.
	 \end{choices}
\question Рынок наличной иностранной валюты обслуживает преимущественно:
	 \begin{choices}
	 \choice производственные компании;
	 \CC финансовые учреждения;
	 \choice государственные органы;
	 \choice население.
	 \end{choices}
\question Произведение прямой котировки национальной валюты на косвенную в кризисной экономике:
	 \begin{choices}
	 \choice растет;
	 \choice падает;
	 \CC неизменно;
	 \choice меняется непредсказуемым образом.
	 \end{choices}
\question Крос-скурс выражается:
	 \begin{choices}
	 \choice в прямой котировке;
	 \CC косвенной котировке;
	 \choice как в прямой, так и в косвенной котировке;
	 \choice в специфической котировке кросс-курса.
	 \end{choices}
\question Курсовая маржа на менее стабильную валюту равна А, маржа на более стабильную равна Б В этом случае:
	 \begin{choices}
	 \choice А больше Б;
	 \choice А меньше Б;
	 \CC А незначительно отличается от Б;
	 \choice А отличается непредсказуемым образом от Б.
	 \end{choices}
\question 1 долл = 30 руб Это:
	 \begin{choices}
	 \choice прямая котировка рубля и доллара;
	 \CC косвенная котировка рубля и доллара;
	 \choice прямая котировка рубля и косвенная доллара;
	 \choice прямая котировка доллара и косвенная рубля.
	 \end{choices}
\question При снижении внешней задолженности России курс рубля к ведущим мировым валютам:
	 \begin{choices}
	 \choice возрастет;
	 \CC уменьшится;
	 \choice не изменится;
	 \choice изменится непредсказуемым образом.
	 \end{choices}
\question Частые, трудно предсказуемые колебания курса национальной валюты окажут следующее воздействие на уровень иностранных инвестиций в национальную экономику:
	 \begin{choices}
	 \CC он возрастет;
	 \choice он снизится;
	 \choice он не изменится;
	 \choice он изменится непредсказуемым образом.
	 \end{choices}
\question Курс доллара США к рублю составляет 22,00 руб за 1 долл Курс евро к рублю составляет 27,50 руб за 1 евро Курс доллара к евро составляет:
	 \begin{choices}
	 \choice 1 долл = 0,75 евро;
	 \CC 1 долл = 0,80 евро;
	 \choice 1 долл = 1,20 евро;
	 \choice 1 долл = 1,25 евро.
	 \end{choices}



\end{multicols}
\end{questions}

\end{document}
