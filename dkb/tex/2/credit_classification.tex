\documentclass[_DKB_p2_Credit.tex]{subfiles}

\begin{document}

\setbeamercovered{transparent}
\subsection{Формы кредита}
\begin{frame}{Формы кредита}
\begin{itemize}[<+->]
\item
банковский;
\item
коммерческий;
\item
потребительский;
\item
государственный;
\item
международный кредит.
\end{itemize}
\end{frame}

\begin{frame}{Формы кредита}{Банковский кредит}
\begin{block}{Банковский кредит}
\quad
— это кредит, предоставляемый юридическим лицам исключительно в денежной форме специализированными кредитно-финансовыми организациями (имеющими лицензию ЦБ на осуществление кредитных операций). 

\end{block}
\end{frame}

\begin{frame}
Банковский кредит оформляют кредитным договором. Плату за его предоставление взимают в виде ссудного процента, ставку которого определяют по соглашению сторон (с учетом ее средней нормы на данный период и конкретных условий кредитования).
\end{frame}

\begin{frame}{Формы кредита}{Коммерческий кредит}
\begin{block}{Коммерческий кредит}
\quad 
— это кредит, предоставленный одним юридическим лицом другому в виде товаров (работ, услуг) с отсрочкой платежа. Юридически оформляется долговым соглашением или векселем. 

Дешевле банковского кредита, но имеет ряд недостатков.
\end{block}
\end{frame}

\begin{frame}{Формы кредита}{Потребительский кредит}
\begin{block}{Потребительский кредит}
\quad
— это кредит, предоставляемый физическим лицам для удовлетворения их потребительских нужд. Он может быть предоставлен как в денежной форме, так и в форме продажи товаров (выполнения работ, оказания услуг) на условиях отсрочки или рассрочки платежа.


Кредит индивидуальным предпринимателям к потребительскому не относится.
\end{block}
\end{frame}

\begin{frame}{Формы кредита}{Государственный кредит}
\begin{block}{Государственный кредит}
\quad
— это кредит, который предоставляется государством. Он является особой формой государственного финансирования, при которой средства выделяются на возвратной основе и только в том случае, если возможности безвозмездного бюджетного финансирования уже исчерпаны или если в таковом нет крайней необходимости. 
\end{block}
\end{frame}

\begin{frame}
Государственный кредит предоставляется на льготных условиях, поэтому он служит также инструментом регулирования экономики. Предоставляется либо в качестве поощрения либо поддержки.
\end{frame}

\begin{frame}{Государственный кредит}{Предоставляется в качестве поощрения и/или поддержки.}
\textbf{Для поощрения }может быть предоставлен по конкурсу отдельным отраслям и предприятиям, играющим значительную роль в развитии экономики страны.

Для поддержки направляют:
\begin{itemize}
\item
в отдельные регионы страны, нуждающиеся в дотациях;
\item
в низкорентабельные, но необходимые отрасли народного хозяйства;
\item
на развитие экономической и социальной инфраструктуры;
\item
в отдельные стратегически важные объекты.
\end{itemize}

\end{frame}

\begin{frame}{Формы государственного кредита}
\begin{itemize}
\item
денежная (прямое выделение бюджетных средств)
\item
налоговый
\item
инвестиционный налоговый кредит. 
\end{itemize}
\end{frame}
\begin{frame}
\begin{block}{Налоговый кредит }
\quad 
- отсрочка по уплате налога.
\end{block}
\begin{block}{Инвестиционный налоговый кредит}
\quad
— возможность в течение определенного срока и в определенных пределах уменьшить свои налоговые платежи с последующей поэтапной уплатой суммы кредита и начисленных процентов.
\end{block}
Порядок и условия предоставления таких кредитов определяются Налоговым кодексом РФ.
\end{frame}

\begin{frame}{Формы кредита}{Международный кредит}
\begin{block}{Международный кредит}
\quad
— это кредит, представленный заемщику кредитором из другой страны.
\end{block}
\begin{itemize}
\item
Международные кредиты государственного уровня.
\item
Межправительственные кредиты.
\item
Международные кредиты частного уровня.
\end{itemize}
\end{frame}
\begin{frame}[allowframebreaks]{Экономическая роль международного кредита }
\begin{itemize}
\item
способствует развитию экспортно-импортных связей и образованию мировых рынков и тем самым стимулирует повышение конкурентоспособности товаров;
\item
обеспечивает бесперебойность международных расчетов и таким образом укрепляет мировые хозяйственные связи;

\pagebreak
\item
увеличивает возможность концентрации и централизации капиталов и соответственно повышает экономические возможности владельцев этих капиталов;
\item
осуществляет перераспределение капиталов между странами;

\pagebreak
\item
выравнивает норму ссудного процента, что приводит к оптимальному размещению ресурсов с точки зрения единого мирового экономического пространства.
\end{itemize}
\end{frame}

\subsection{Классификация кредитов}
\begin{frame}[allowframebreaks]{{Классификация кредитов}}{По экономическому назначению: целевые и нецелевые}
Целевой (связанный):
\begin{itemize}
\item 
платежные (на проведение конкретной коммерческой сделки или удовлетворение временной нужды):
\begin{enumerate}
\item[-] 
на оплату расчетных (платежных) документов контрагентов клиента;
\item[-] 
на приобретение ценных бумаг;
\item[-] 
на авансовые платежи;
\item[-] 
на платежи в бюджет;
\item[-] 
на заработную плату;
\item[-] 
другие;
\end{enumerate}
\pagebreak
\item
на финансирование производственных затрат:
\begin{enumerate}
\item[-] 
формирование запасов;
\item[-] 
финансирование текущих производственных затрат;
\item[-] 
финансирование инвестиционных затрат, включая кредиты на \item[-] 
лизинговые и т.п. операции;
\end{enumerate}
\item
учет (покупка) векселей, включая операции репо (покупка с обязательством обратной продажи);
\item
потребительские кредиты (физическим лицам);
\end{itemize}
\end{frame}

\begin{frame}{Классификация кредитов}{По форме предоставления}
\begin{itemize}
\item
в безналичной форме:
\begin{enumerate}
\item[-]
зачисление безналичных денег на соответствующий счет заемщика, в том числе реструктуризация ранее выданного кредита и предоставление нового;
\item[-]
кредитование с использованием векселей банка;
\item[-]
в смешанной форме.
\end{enumerate}
\item
в налично-денежной форме (как правило, физическим лицам).
\end{itemize}
\end{frame}

\begin{frame}{Классификация кредитов}{По технике предоставления}
\begin{itemize}
\item
Одной суммой.
\item
С овердрафтом. 
\item
В виде кредитной линии:
\begin{enumerate}
\item[-]
простая (невозобновляемая) кредитная линия;
\item[-]
возобновляемая (револьверная) кредитная линия (онкольная и контокоррентная).
\end{enumerate}
\item
Комбинированные варианты.
\end{itemize}
\end{frame}

\begin{frame}{Классификация кредитов}{По способу предоставления}
\begin{itemize}
\item
Индивидуальный (предоставляется одним банком).
\item
Синдицированный.
\end{itemize}
\end{frame}

\begin{frame}[allowframebreaks]{Классификация кредитов}{По времени и технике погашения кредита}
\begin{itemize}
\item
Погашаемый одной суммой в конце срока.
\item
Погашаемые равными долями через равные промежутки времени.

\pagebreak
\item
Погашаемые неравными доля через различные промежутки времени:
\begin{enumerate}
\item[-]
сложный кредит ( с выплатой от 20 до 50 процентов суммы кредита в конце срока);
\item[-]
прогрессивный кредит ( с прогрессивно нарастающими к концу срока действия кредитного договора выплатами);
\item[-]
сезонный кредит (кредит для сезонных производств с выплатами только в те месяцы, на которые приходятся максимальные суммы выручки)
\end{enumerate}
\end{itemize}
\end{frame}
\begin{frame}{Онкольный кредит}
\begin{block}{Онкольная кредитная линия} 
\quad
предполагает установление лимита суммы (обусловленного, к примеру, величиной оценки векселей заемщика, заложенных им в банке) и в рамках согласованного периода времени, причем таким образом, что по мере погашения взятых ранее кредитов лимит может непрерывно и автоматически (без заключения дополнительного договора/соглашения) восстанавливаться.
\end{block}
\end{frame}
\begin{frame}{Контокоррентный кредит}
\begin{block}{Контокоррентная кредитная линия }
\quad
кредитование производственных нужд заемщика в пределах оговоренного лимита сумм и установленного срока действия соглашения. При этом кредиты непрерывно и автоматически выдаются и погашаются, отражаясь на едином контокоррентном счете, сочетающем в себе свойства ссудного и расчетного счетов; лимит при этом каждый раз восстанавливается.
\end{block}
\end{frame}
\begin{frame}{Другие виды кредитных услуг}
\begin{itemize}
\item
гарантийные операции банков в части кредитования своих клиентов;
\item
консультационные услуги по вопросам кредитования.
\end{itemize}
\end{frame}
\setbeamercovered{transparent}
\end{document}
