\documentclass[_DKB_p2_Credit.tex]{subfiles}
\begin{document}

\setbeamercovered{transparent}

\subsection{Необходимость кредита}
\begin{frame}{Необходимость кредита}
На базе неравномерности кругооборота и оборота капиталов естественным становится появление отношений, которые устраняют несоответствие между временем производства и временем обращения средств. Таким отношением является кредит
\end{frame}

\begin{frame}{}
\begin{itemize}[<+->]
\item
кредит становится необходимым в том случае, если происходит совпадение интересов кредитора и заемщика, т.е. оба субъекта должны быть взаимно заинтересованы в кредитной сделке;
\item
участники кредитной сделки - кредитор и заемщик - должны выступать как юридически самостоятельные субъекты, материально гарантирующие выполнение обязательств, вытекающих из экономических связей.

\end{itemize}
\end{frame}

\subsection{Отличие кредита от денег и других экономических категорий}
\begin{frame}{\setfontsize{12pt}Отличие кредита от денег и других экономических категорий}{Отличие кредита от денег}
\begin{itemize}
\item
Различный состав участников. 
\item
Наблюдается при отсрочке платежа за тот или иной товар. 
\item
Различием в удовлетворяемых потребностях.
\item
Прослеживается в их движении: в кредит можно предоставить не только деньги, но и любой товар.
\end{itemize}
\end{frame}

\begin{frame}{Кредит и аванс}
Неотъемлемым свойством кредита выступает доверие. Однако, доверие присуще не только кредиту, но и другим отношениям между людьми.

Для кредитной сделки возвратность является непременным условием.
\end{frame}

\begin{frame}{Различия между кредитом и наймом}
\begin{itemize}
\item
Оплата труда и выполнение работы не связаны напрямую во времени.
\item
Работодатель, нанимая работника, принимает на себя ряд социальных обязательств. Расторжение и изменение трудового договора сложнее, чем кредитного.
\end{itemize}
\end{frame}

\begin{frame}{Различия между кредитом и страхованием}
При наступлении страхового случая страховая компания выплачивает страхователю определенную сумму средств, причем зачастую большую, чем уплаченная ранее сумма.
\begin{itemize}
\item
возвратность здесь необязательный атрибут, так как страхователю средства выплачиваются только при возникновении страхового случая. 
\item
При уплате страхового взноса собственность на него переходит к страховщику, в то время как при кредите собственность на ссужаемую стоимость лишь временно уступается, ее собственником всегда является одно и то же юридическое лицо - кредитор.
\end{itemize}
\end{frame}

\begin{frame}{Различие кредита и финансов}
Финансы в отличие от кредита являются порождением распределительных, а не перераспределительных процессов, приводят к смене собственника передаваемой стоимости, директивны, обусловливают отношения между субъектами.
\end{frame}

\begin{frame}[allowframebreaks]{Различие кредита, займа и ссуды}
\begin{block}{Кредит }
\quad
– это предоставление денежных средств или товаров (работ, услуг) на условиях последующего возврата этих денежных средств или оплаты предоставленных товаров (работ, услуг) в установленный срок, включая оплату процентов за их использование.
\end{block}
Независимо от того, что предоставляется в кредит – товар или деньги, возврат предполагается только в денежной форме.
\pagebreak

\begin{block}{Заем }
\quad 
– это предоставление денег или каких-либо вещей на условиях возврата в установленный срок той же суммы денег (суммы займа) или равного количества полученных вещей того же рода и качества (возможно, с уплатой процента).
\end{block}
Если в займы или в кредит предоставляются деньги, то понятие кредит применимо в тех случаях, когда денежные средства получены от организаций, которые занимаются их предоставлением на возвратной основе регулярно и профессионально.
\pagebreak
\begin{block}{Ссуда}
\quad
- предмет кредитных отношений (или займа), т.е. совокупность денежных или товарно-материальных средств, передаваемых на определенных условиях другому лицу. Ссуда таким образом, не условия сделки, а те средства которые предаются взаймы или в кредит.
\end{block}
\end{frame}

\subsection{Структура кредита}
\begin{frame}{Структура кредита}
\begin{itemize}
\item
Кредитор - сторона кредитных отношений, предоставляющая ссуду. 
\item
Заемщик - сторона кредитных отношений, получающая кредит и обязанная возвратить полученную ссуду.
\item
Объектом передачи выступает ссуженная стоимость, как особая часть стоимости. 
\end{itemize}
\end{frame}
\subsection{Стадии движения кредита}
\begin{frame}{Стадии движения кредита}
1) размещение кредита;

2) получение кредита заемщиками;

3) его использование;

4) высвобождение ресурсов;

5) возврат временно позаимствованной стоимости;

6) получение кредитором средств, размещенных в форме кредита.

\end{frame}
\subsection{Характеристика сущности кредита}
\begin{frame}[allowframebreaks]{Характеристика сущности кредита}
\begin{itemize}
\item
\textbf{Возвратность }– необходимость погашения кредита. 
\begin{enumerate}
\item[-]
Практически выражается в перечислении соответствующей суммы денежных средств на счет кредитной организации, выдавшей кредит.
\item[-]
При нарушении этого условия кредитор применяет экономические санкции (увеличение процента), а при дальнейшей отсрочке погашения кредита (в нашей стране свыше трех месяцев) предъявлять требования в судебном порядке.
\end{enumerate}
\pagebreak
\item
\textbf{Платность }– за предоставленный кредит заемщик уплачивает кредитору вознаграждение в форме процента.
\item
\textbf{Обеспеченность }– необходимость обеспечения защиты имущественных интересов кредитора при возможном нарушении заемщиком принятых на себя обязательств. (залог имущества, гарантии и пр.)
\pagebreak
\item
\textbf{Целевое использование }– для оценки риска невозврата кредита кредитор должен быть ознакомлен с целью его получения заемщиком. Использование целевого кредита не по назначению может стать основанием для досрочного отзыва кредита или введения штрафного (повышенного) процента.
\item
\textbf{Дифференцированность }– различный подход к разным категориям заемщиков. Это может быть вызвано интересами самих кредиторов или проводимой государством политики поддержки отдельных отраслей и сфер деятельности.
\item
социально-экономическая сторона кредита.
\end{itemize}
\end{frame}

\setbeamercovered{transparent}

\end{document}