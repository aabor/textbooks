% !TeX program = lualatex -synctex=1 -interaction=nonstopmode --shell-escape %.tex

\documentclass[_DKB_p2_Credit.tex]{subfiles}
\begin{document}

\setbeamercovered{transparent}

\subsection{Функции кредита}
\begin{frame}{Функции кредита}{Методологические принципы анализа функций кредита}
\begin{itemize}
\item
функция носит объективный характер.
\item
в каждый момент кредит проявляет сущность не всеми своими функциями.
\item
Функция - это изменяющаяся категория.
\item
\end{itemize}
\end{frame}

\begin{frame}
Функция выражает специфическое взаимодействие кредита как целостного процесса:
	- относится к кредитному отношению в целом, а не отдельно к кредитору или заемщику;
	
	- должна характеризовать специфическое взаимодействие всех форм кредита;
	
	-функция направлена на сохранение кредита как целостного образования.
\end{frame}

\begin{frame}{Функции кредита}
\begin{itemize}[<+->]
\item
Перераспределительная:
\begin{enumerate}
\item
- деньги "работают";
\item
- бесперебойность процесса производства;
\item
- ускоряется научно-техническое развитие экономики.
\end{enumerate}
\item
Контрольная (оценка кредитоспособности заемщиков).
\item
Эмиссионная (центральный банк, банки, коммерческие предприятия).
\end{itemize}
\end{frame}

\subsection{Законы кредита}
\begin{frame}{Законы кредита}
\begin{block}{Экономические законы }
предполагают обнаружение устойчивой взаимосвязи между экономическими явлениями, в том числе между кредитом и другими экономическими категориями.
\end{block}
\end{frame}
\begin{frame}{Признаки закона, в т.ч. экономического}
\begin{itemize}[<+->]
\item
Необходимость.
\item
Существенность. 
\item
Объективность.
\item
Всеобщность.
\item
Конкретность.
\end{itemize}
\end{frame}
\begin{frame}{Законы кредита}
\begin{itemize}[<+->]
\item
Закон возвратности кредита.
\item
Закон равновесия между высвобождаемыми и перераспределяемыми на началах возвратности ресурсами.
\item
Закон сохранения ссуженной стоимости. 
\item
Временный характер функционирования кредита. 
\end{itemize}
\end{frame}

\subsection{Теории кредита}
\begin{frame}[allowframebreaks]{Теории кредита}
\begin{itemize}
\item
Натуралистическая (А. Смит (1723—1790) и Д. Рикардо (1772—1823) (анг.), Ж. Сэй, Ф. Бастия (фр.), Д. Мак-Куллох (амер.)):
\begin{enumerate}
\item
- объектом кредита являются натуральные, т.е. неденежные вещественные блага;
\item
- кредит представляет собой движение натуральных благ, и поэтому он есть лишь способ перераспределения существующих в данном обществе материальных ценностей;
\pagebreak
\item
- ссудный капитал тождествен действительному, следовательно, накопление ссудного капитала есть проявление накопления действительного капитала, а движение первого полностью совпадает с движением производительного капитала;
\item
- поскольку кредит выполняет пассивную роль, то коммерческие банки являются всего лишь скромными посредниками.
\end{enumerate}
\pagebreak
\item
Капиталотворческая (Дж. Ло (1671 — 1729) (фр.), Г. Маклеод (1821—1902) (анг.), И. Шумпетер (Австрия), А. Ган (герм.), Дж. Кейнс и Р. Хоутри (анг.)):
\begin{enumerate}
\item
- Кредит отождествлялся с деньгами и богатством
\item
- Кредит приносит прибыль и поэтому является «производительным капиталом», а банки — это «фабрики кредита»
\item
- Инфляционный кредит (т. е. кредит, способный к безграничному расширению) содействует постоянному экономическому росту
\end{enumerate}

\pagebreak
\item
Кейнсианская теория кредитного регулирования экономики (После кризиса в 1929—1933 гг.)
\item
Неокейнсианская школа денежно-кредитного регулирования (П. Самуэльсон, Л. Лернер, С. Харрис, Э. Хансен, Дж. Гэлбрейт.) (после 2-ой Мировой войны);
\pagebreak
\item
школа рынка ссудных капиталов (Р. Голдсмит, С. Кузнец, X. Дугел, Д. Кример): корпорации, государство, население не могут развиваться на базе собственных финансовых ресурсов и требуют постоянных вливаний денежных средств рынка капитала;
\item
монетаристская школа М. Фридмана: основными инструментами регулирования экономики являются изменения денежной массы и процентных ставок, что дает возможность чередовать кредитную экспансию и рестрикцию. Установление среднегодовых темпов роста денежной массы в сочетании с определенным уровнем процентных ставок позволяет влиять на динамику производства и цен.
\end{itemize}
\end{frame}

\setbeamercovered{transparent}
\end{document}