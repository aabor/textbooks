% !TeX program = lualatex -synctex=1 -interaction=nonstopmode --shell-escape %.tex

\documentclass[12pt, table]{exam}
\usepackage[rus]{borochkin}

\usepackage{borochkin_exam}

%%%%%%%%%%%%%%%%%%%%%%%%%%%%%%%%%%%%%%%%%

\professor
\iftagged{professor}{ \printanswers }
%%%%%%%%%%%%%%%%%%%%%%%%%%%%%%%%%%%%%%%%%

\begin{document}
\noindent
% Контр/р № #1, Вариант #2, Предмет #3
\studentpersonalinfo{1}{1}{ДКБ}
\normalsize

\begin{questions}
\question[40] Тест
\answerstotest

\pagebreak
\question[15] Рассчитайте размер денежной базы, основные агрегаты денежной массы (M0, M1, M2) и величину денежного мультипликатора, исходя из приведенных в таблице показателей.
	
\small
\begin{tabularx}{\linewidth}[b]{@{}>{\raggedright\arraybackslash}Xr@{}}		& млрд. руб.\\
	\toprule
	Обязательные резервы & 509,7 \\
	Срочные депозиты нефинансовых и финансовых (кроме кредитных) организаций & 6295,3 \\
	Депозиты населения & 3901,9 \\
	Наличные деньги в обращении & 7946,8 \\
	Остатки средств в кассах кредитных организаций & 805,9 \\
	Корреспондентские счета кредитных организаций в Банке России & 1675,3 \\
	Депозиты нефинансовых и финансовых (кроме кредитных) организаций & 6191,6 \\
	Переводные депозиты & 10093,6 \\
	Срочные депозиты & 21585,1 \\
	Срочные депозиты населения & 15289,8 \\
	Депозиты банков в Банке России & 658,6 \\
	\bottomrule
\end{tabularx}%
\normalsize
	
\begin{solution}[12em] В результате перегруппировки балансовых статей получаем:
	
\small
\begin{tabularx}{\linewidth}[b]{@{}>{\raggedright\arraybackslash}Xr@{}}				& млрд. руб.\\
	\toprule
	Наличные деньги в обращении & 7946,8 \\
	Остатки средств в кассах кредитных организаций & 805,9 \\
	Корреспондентские счета кредитных организаций в Банке России & 1675,3 \\
	Обязательные резервы & 509,7 \\
	Депозиты банков в Банке России & 658,6 \\
	Переводные депозиты & 10093,6 \\
	Депозиты населения & 3901,9 \\
	Депозиты нефинансовых и финансовых (кроме кредитных) организаций & 6191,6 \\
	Срочные депозиты & 21585,1 \\
	Срочные депозиты населения & 15289,8 \\
	Срочные депозиты нефинансовых и финансовых (кроме кредитных) организаций & 6295,3 \\
	\midrule
	Денежная база & 11596,3 \\
	М0    & 7946,8 \\
	М1    & 18040,4 \\
	М2    & 39625,5 \\
	Денежный мультипликатор & 3,42 \\
	\bottomrule
\end{tabularx}%
\normalsize
\end{solution}

\pagebreak	
\question[10] Предположим, что спрос и предложение капитала (без учета инфляционных ожиданий) заданы функциями 	
$r^d=14-0.5 \cdot q^d$, $r^s=7 + 2 \cdot q^s$, где 	
$r$ – процентная ставка, 	
$q$ – объем капитала.	
\noaddpoints
\begin{subparts}
	\subpart[2] Определите равновесные значения $q^*$ и $r^*$.

	\begin{solution}[12em]
		
		\raggedright		
		Пусть уравнения спроса и предложения на капитал задаются, соответственно, функциями: $r^d=k^d\cdot q^d +c^d$ и $r^s=k^s\cdot q^s +c^s$, тогда точка пересечения прямых определяется по формулам:
		\begin{align*}
		q^*&=\frac{c^d-c^s}{k^s-k^d}=2.8\\
		r^*&=c^d + k^d \cdot q^* =12.6.
		\end{align*}
	\end{solution}
	
	\subpart[4] Что произойдет, если доходность государственных краткосрочных облигаций на финансовом рынке внезапно повысится до 13\%? Нарисуйте график.
	\begin{solution}[12em]
		\begin{multicols}{2}
		\setlength{\columnsep}{1cm}
		Спрос на капитал, $q^d$,  и предложение капитала, $q^s$, при искусственном завышении процентных ставок:
		\begin{align*}
		q^d&=\frac{r^{max}-c^d}{k^d}=2\\
		q^s&=\frac{r^{max}-c^s}{k^s}=3.
		\end{align*}
		\centering
		\includegraphics[scale=.7
		% trim={<left> <lower> <right> <upper>}				
		,trim={0cm 3cm 4cm 0cm},clip]
		{../../../texExercises/tikz/credit_excessive_supply}
		\end{multicols}
	\end{solution}
	
	\subpart[4] Какова будет величина избыточного предложения капитала?
	\begin{solution}[12em]
		Избыточное предложение капитала:
		\begin{align*}
		q^s-q^d=1.
		\end{align*}
	\end{solution}
	
\end{subparts}
\addpoints

\pagebreak
\question[10] Предположим, что бессрочная облигация дает ежегодный доход 2500~₽.
\noaddpoints
\begin{subparts}
	\subpart[5] Определите ожидаемую рыночную цену такой облигации, если доходность других активов с тем же уровнем риска составляет 5\%.
	
	\begin{solution}[12em]
		\begin{align*}
		P_1=\frac{C}{r_1}=500~000~₽,
		\end{align*}	
		где
		
		$P_1$ - рыночная цена облигации;
		
		$C$ - купонный доход, ₽;
		
		$r_1$ - процентная ставка, годовых.
	\end{solution}
	
	\subpart[5] Если доходность других активов вырастет до 7.5\%, то повысится или снизится рыночная цена этой облигации и до какого уровня.
	
	\begin{solution}[12em]
		\begin{align*}
		P_2=\frac{C}{r_2}=333~000~₽.
		\end{align*}	
		
	\end{solution}
\end{subparts}
\addpoints

\pagebreak
\question[20] Номинал облигации 5000~₽, купон 3\%, выплачивается один раз в год. До погашения облигации 10 лет. Облигация стоит 4800~₽. 
\noaddpoints
\begin{subparts}
	\subpart[10] Определите ориентировочно доходность до погашения облигации.
	\begin{solution}[12em]
		\begin{align*}
		r^*&=\frac{(N-P)/m \cdot n + C/m }{(N+P)/2}\\
		r^*&=3,47\%
		\end{align*}
		где
		
		$r^*$ - ориентировочная доходность облигации;
		
		$N$ - номинал облигации;
		
		$P$ - рыночная цена облигации;
		
		$C$ - сумма купона;
		
		$n, m$ - срок облигации и количество выплат процентов в году, соответственно.
	\end{solution}
	
	\subpart[10] Определите доходность до погашения облигации методом линейной интерполяции.
	
	\begin{solution}[12em]
		\begin{align*}
		r_1&=4\%\\
		r_2&=3\%\\
		P_1&=\frac{C}{r_1} \cdot \left(1 - \frac{1}{(1+r_1)^n} \right) + \frac{N}{(1+r_1)^n}=4594,5~₽\\
		P_2&=\frac{C}{r_2} \cdot \left(1 - \frac{1}{(1+r_2)^n} \right) + \frac{N}{(1+r_1)^n}=5000~₽\\
		r^{**}&=r_1+(r_2-r_1) \cdot \frac{P_1-P}{P_1-P_2}=3.49\%
		\end{align*}	
		где
		
		$r_1, r_2$ - соответственно, округленная вверх и вниз ориентировочная доходность облигации;
		
		$P_1, P_2$ - соответственно, нижняя и верхняя границы рыночной цены облигации;
		
		$r^{**}$ - доходность до погашения облигации, определенная методом линейной интерполяции.
	\end{solution}
	
\end{subparts}
\addpoints

\pagebreak
\question[10] Предположим, что и реальная и номинальная процентные ставки по однолетним контрактам равны 0.5\% (без учета инфляционных ожиданий).
\noaddpoints

\begin{subparts}
	\subpart[5] Определите новую ожидаемую номинальную процентную ставку, если предполагается, что в течение следующего года цены вырастут на 0.75\%.
	
	\begin{solution}[12em]
		\begin{align*}
			r_n&=r_r+\tau + r_r \cdot \tau\\
			&=1.25\%.\nonumber
		\end{align*}
		где

		$r_n$ - номинальная ставка процента;
		
		$r_r$ - реальная ставка процента;
		
		$\tau$ - ожидаемые темпы инфляции.
	\end{solution}
	
	\subpart[5] Определите новую ожидаемую номинальную процентную ставку, если предполагается, что цены понизятся на 1.25\%.
	
	\begin{solution}[12em]
		
		\raggedright
		Ответ: 0,05\%.
	\end{solution}
	
\end{subparts}
\addpoints

\end{questions}

\pagebreak
\noindent\textbf{Тестовые вопросы (выберите один правильный ответ)}

\begin{questions}
\begin{multicols}{2}
\setlength{\columnsep}{1cm}

\question Необходимой основой, базой существования денег являются:
	 \begin{choices}
	 \choice Финансовые нужды государства;
	 \choice Внешнеэкономические связи;
	 \CC Товарное производство и обращение товаров;
	 \choice Потребности центрального и коммерческих банков.
	 \end{choices}
\question Полноценные деньги  это деньги, у которых номинальная стоимость:
	 \begin{choices}
	 \choice Устанавливается стихийно на рынке;
	 \choice Ниже реальной стоимости;
	 \choice Превышает реальную стоимость;
	 \CC Соответствует реальной стоимости.
	 \end{choices}
\question Знаки стоимости  это деньги, у которых номинальная стоимость:
	 \begin{choices}
	 \choice Не устанавливается;
	 \choice Соответствует реальной стоимости;
	 \choice Ниже реальной стоимости;
	 \CC Превышает реальную стоимость.
	 \end{choices}
\question Деньги служат средством для удовлетворения  потребностей
	 \begin{choices}
	 \choice Только духовных;
	 \choice Ограниченного числа;
	 \CC Всех;
	 \choice Только материальных.
	 \end{choices}
\question Функцию меры стоимости выполняют  деньги
	 \begin{choices}
	 \CC Только полноценные;
	 \choice Только неполноценные;
	 \choice Полноценные и неполноценные;
	 \choice Бумажные и кредитные.
	 \end{choices}
\question Функция денег как  гарантирует реализацию всех других денежных функций
	 \begin{choices}
	 \choice Мировых денег;
	 \choice Средства платежа;
	 \CC Меры стоимости;
	 \choice Средства обращения.
	 \end{choices}
\question При обращении полноценных денег масштаб цен:
	 \begin{choices}
	 \choice Не устанавливался;
	 \choice Представлял собой потребительную стоимость денежной единицы;
	 \CC Совпадал с весовым количеством металла, закрепленным за денежной единицей;
	 \choice Представлял собой величину денежной единицы, стихийно складывающуюся в результате формирования в стране определенного уровня цен.
	 \end{choices}
\question Деньги правильно выполняют функцию средства платежа, если:
	 \begin{choices}
	 \choice Фактический платеж по долговым обязательствам меньше договорного;
	 \choice Беспрепятственно совершается процесс обращения товаров в деньги и наоборот;
	 \CC Фактический платеж по долговым обязательствам соответствует договорному;
	 \choice Наблюдается соответствие между денежными накоплениями и образованием реальных материальных запасов.
	 \end{choices}
\question В плановой экономике деньги рассматривались главным образом как:
	 \begin{choices}
	 \choice Активный инструмент воздействия на все сферы экономических процессов, реальное средство стимулирования эффективности производства;
	 \CC Инструмент учета и контроля со стороны центральных и других органов управления экономикой;
	 \choice Капитал или самовозрастающая стоимость;
	 \choice Средство накопления и приумножения богатства.
	 \end{choices}
\question Развитие теории денег было вызвано:
	 \begin{choices}
	 \choice Широким развитием внешнеэкономических связей и необходимостью исследования роли денег в обеспечении их функционирования;
	 \CC Появлением неполноценных денежных знаков и необходимостью анализа влияния массы денег на уровень цен;
	 \choice Необходимостью исследования причин перехода от полноценных денег к знакам стоимости;
	 \choice Широким развитием воспроизводственного процесса и необходимостью исследования роли денег в обеспечении его функционирования.
	 \end{choices}
\question Теория А Филлипса предполагает, что уровень цен изменяется:
	 \begin{choices}
	 \CC В зависимости от уровней занятости населения и заработной платы;
	 \choice Пропорционально изменению массы денег в обращении;
	 \choice Стихийно под свободным воздействием спроса и предложения;
	 \choice В зависимости от изменения золотовалютных запасов страны.
	 \end{choices}
\question Сторонники количественной теории денег считают, что:
	 \begin{choices}
	 \choice Размер заработной платы оказывает решающее воздействие на уровень занятости;
	 \choice Рост занятости не сопровождается увеличением заработной платы и ростом цен;
	 \choice Рост занятости сказывается на увеличении заработной платы, но не сопровождается ростом цен;
	 \CC Рост занятости и увеличение заработной платы сопровождаются ростом цен, а при уменьшении оплаты труда цены снижаются.
	 \end{choices}
\question В условиях современной рыночной экономики первичной является эмиссия денег
	 \begin{choices}
	 \choice Наличных;
	 \choice Бумажных;
	 \choice Металлических;
	 \CC Безналичных.
	 \end{choices}
\question Наличные деньги поступают в оборот путем:
	 \begin{choices}
	 \choice Выплаты предприятиями заработной платы рабочим;
	 \choice Перевода расчетно-кассовыми центрами оборотной кассы денежных средств в резервные фонды;
	 \CC Осуществления кассовых операций коммерческими банками;
	 \choice Передачи центральным банком резервных денежных фондов расчетно-кассовым центрам.
	 \end{choices}
\question  мультипликатор представляет собой процесс увеличения денег на депозитных счетах коммерческих банков в период их движения от одного коммерческого банка к другому
	 \begin{choices}
	 \CC Банковский;
	 \choice Депозитный;
	 \choice Кредитный;
	 \choice Ссудный.
	 \end{choices}
\question  мультипликатор отражает объект мультипликации  деньги на депозитных счетах коммерческих банков
	 \begin{choices}
	 \choice Кредитный;
	 \CC Депозитный;
	 \choice Банковский;
	 \choice Финансовый.
	 \end{choices}
\question Безналичные расчеты производятся юридическими и физическими лицами через:
	 \begin{choices}
	 \CC Коммерческие банки;
	 \choice Расчетно-кассовые центры;
	 \choice Региональные депозитарии;
	 \choice Уличные банкоматы.
	 \end{choices}
\question Весь безналичный оборот является:
	 \begin{choices}
	 \choice Неплатежным;
	 \choice Наличным;
	 \choice Сезонным;
	 \CC Платежным.
	 \end{choices}
\question Необходимой предпосылкой осуществления безналичных расчетов служит наличие у плательщика и получателя:
	 \begin{choices}
	 \choice Лимита оборотной кассы;
	 \CC Банковских счетов;
	 \choice Лицензии на право совершения безналичных расчетов;
	 \choice Генеральной лицензии Центрального банка РФ.
	 \end{choices}
\question Экономические процессы в народном хозяйстве опосредуются преимущественно  оборотом
	 \begin{choices}
	 \CC Безналичным;
	 \choice Наличным;
	 \choice Сезонным;
	 \choice Валютным.
	 \end{choices}
\question Корреспондентские счета банков открываются:
	 \begin{choices}
	 \choice По указанию Центрального банка РФ;
	 \choice По указанию муниципалитетов;
	 \CC На основе межбанковских соглашений;
	 \choice По указанию Министерства финансов РФ.
	 \end{choices}
\question Для расчетного обслуживания между банком и клиентом заключается:
	 \begin{choices}
	 \choice Кредитный договор;
	 \choice Договор приема денежных средств;
	 \choice Трастовый договор;
	 \CC Договор банковского счета.
	 \end{choices}
\question Безналичный оборот охватывает  платежи
	 \begin{choices}
	 \choice Только товарные;
	 \CC Товарные и нетоварные;
	 \choice Только нетоварные;
	 \choice Только финансовые.
	 \end{choices}
\question В настоящее время наиболее распространенной формой безналичных расчетов в России являются:
	 \begin{choices}
	 \choice Аккредитивы;
	 \choice Платежные требования;
	 \CC Платежные поручения;
	 \choice Чеки.
	 \end{choices}
\question Недостатком аккредитивной формы расчетов является:
	 \begin{choices}
	 \choice Быстрота и простота проведения расчетной операции;
	 \CC Замедление товарооборота, отвлечение средств покупателя из хозяйственного оборота на срок действия аккредитива;
	 \choice Отсутствие для поставщика гарантии оплаты покупателем поставленной ему продукции;
	 \choice Необходимость получения специального разрешения Банка России на право проведения расчетов аккредитивом.
	 \end{choices}
\question Особенностью обращения аккредитивов в России является то, что они:
	 \begin{choices}
	 \choice Могут использоваться для расчетов с несколькими поставщиками и могут быть переадресованы;
	 \CC Могут использоваться для расчетов только с одним поставщиком и не могут быть переадресованы;
	 \choice Оплачиваются только наличными деньгами;
	 \choice Используются только в сделках между физическими лицами.
	 \end{choices}
\question Чек, платеж по которому совершается как в пользу лица, указанного в чеке, так и путем передаточной надписи другому лицу, именуется:
	 \begin{choices}
	 \choice Предъявительским;
	 \CC Ордерным;
	 \choice Именным;
	 \choice Ассигнационным.
	 \end{choices}
\question Чеки, передаваемые путем оформления передаточной надписи (индоссамента), именуются:
	 \begin{choices}
	 \CC Ордерными;
	 \choice Именными;
	 \choice Предъявительскими;
	 \choice Ассигнационными.
	 \end{choices}
\question Наличный денежный оборот — это процесс:
	 \begin{choices}
	 \choice Эмиссии и изъятия наличных денег из обращения;
	 \choice Подготовки, эмиссии и выпуска наличных денег в обращение;
	 \choice Перехода наличных денег в безналичные и наоборот;
	 \CC Непрерывного движения наличных денежных знаков.
	 \end{choices}
\question В случае превышения лимита оборотной кассы расчетно-кассового центра деньги в сумме, превышающей лимит:
	 \begin{choices}
	 \CC Переводятся в резервный фонд;
	 \choice Уничтожаются;
	 \choice Обмениваются на иностранную валюту;
	 \choice Обмениваются на государственные ценные бумаги.
	 \end{choices}
\question Цикл наличного денежного оборота начинается при:
	 \begin{choices}
	 \choice Выдаче коммерческими банками денег предприятиям;
	 \CC Выдаче расчетно-кассовыми центрами денег коммерческим банкам;
	 \choice Выплате предприятиями заработной платы работникам;
	 \choice Оплате населением товаров в торговой сети.
	 \end{choices}
\question При превышении лимита оборотной кассы предприятия должны сдать излишек наличных денег в:
	 \begin{choices}
	 \choice Расчетно-кассовый центр;
	 \choice Региональный депозитарий;
	 \CC Обслуживающий их коммерческий банк;
	 \choice Министерство финансов РФ.
	 \end{choices}
\question  векселя взаимно выписываются предпринимателями друг на друга без товарных поставок с целью получения наличных денег путем учета векселей в банке
	 \begin{choices}
	 \choice Дисконтные;
	 \choice Банковские;
	 \CC Дружеские;
	 \choice Бронзовые.
	 \end{choices}
\question Номинальная стоимость  денег соответствует их реальной стоимости
	 \begin{choices}
	 \CC Полноценных;
	 \choice Бумажных;
	 \choice Кредитных;
	 \choice Идеальных.
	 \end{choices}
\question Бумажные деньги выполняют функции:
	 \begin{choices}
	 \CC Средства обращения и средства платежа;
	 \choice Меры стоимости;
	 \choice Меры стоимости и средства накопления;
	 \choice Средства накопления и средства платежа.
	 \end{choices}
\question При избытке денежной массы излишняя часть  денег уходила в сокровища
	 \begin{choices}
	 \CC Металлических;
	 \choice Бумажных;
	 \choice Кредитных;
	 \choice Идеальных.
	 \end{choices}
\question Кредитные деньги выполняют функции:
	 \begin{choices}
	 \choice Меры стоимости и средства накопления;
	 \choice Меры стоимости и средства платежа;
	 \CC Средства обращения и средства платежа;
	 \choice Средства накопления и средства платежа.
	 \end{choices}
\question Золотодевизный стандарт предусматривал:
	 \begin{choices}
	 \choice Размен банкнот на золотые слитки большого веса;
	 \choice Обращение золотых монет, выполняющих все денежные функции;
	 \CC Обмен банкнот на иностранные валюты, беспрепятственно разменные на золото;
	 \choice Одновременное обращение золотых и серебряных монет.
	 \end{choices}
\question При функционировании биметаллизма система двойной валюты предусматривала, что соотношение между золотыми и серебряными монетами устанавливается:
	 \begin{choices}
	 \choice Стихийно;
	 \choice По согласованию хозяйствующих субъектов;
	 \CC Государством;
	 \choice Коммерческими банками.
	 \end{choices}
\question Добавочный лист к векселю называется:
	 \begin{choices}
	 \choice Цессией;
	 \choice Домициляцией;
	 \CC Аллонжем;
	 \choice Индоссаментом.
	 \end{choices}



\end{multicols}
\end{questions}

\end{document}
