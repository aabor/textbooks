% !TeX program = lualatex -synctex=1 -interaction=nonstopmode --shell-escape %.tex

\documentclass[12pt, table]{exam}
\usepackage[rus]{borochkin}

\usepackage{borochkin_exam}

%%%%%%%%%%%%%%%%%%%%%%%%%%%%%%%%%%%%%%%%%

\professor
\iftagged{professor}{ \printanswers }
%%%%%%%%%%%%%%%%%%%%%%%%%%%%%%%%%%%%%%%%%


\begin{document}
\setcounter{section}{0\relax}%
\noindent
% Контр/р № #1, Вариант #2, Предмет #3
\studentpersonalinfo{2}{1}{ДКБ}
\normalsize

\begin{questions}
\question[40] Тест
\answerstotest

\pagebreak
\question[10] Составьте баланс Центрального Банка РФ, используя статьи из финансовой отчетности Банка России, приведенные в таблице. 

\small
\begin{tabularx}{\linewidth}[b]{@{}>{\raggedright\arraybackslash}Xr@{}}
	млрд. руб.\\
	\toprule
	Драгоценные металлы &               3 889    \\
	Прочие пассивы &                  216    \\
	Средства кредитных организаций - резидентов на счетах ЦБ РФ &               3 047    \\
	Требования к МВФ &               1 509    \\
	Прочие активы &               4 830    \\
	Наличные деньги в обращении &               8 339    \\
	Долговые обязательства Правительства РФ &                  302    \\
	Кредиты и депозиты &               3 467    \\
	Обязательства перед МВФ &               1 396    \\
	Средства Правительства РФ на счетах ЦБ РФ &               6 077    \\
	Средства на счетах в Банке России &             10 214    \\
	Ценные бумаги &                  517    \\
	Капитал Банка России &             12 512    \\
	Средства, размещенные у нерезидентов, и ценные бумаги иностранных эмитентов &             18 472    \\
	Средства в расчетах &                      6    \\
	\bottomrule
\end{tabularx}%
\normalsize
\begin{solution}[12em] В результате перегруппировки балансовых статей получаем:
	
	\small
	\begin{tabularx}{\linewidth}[b]{@{}>{\raggedright\arraybackslash}Xr@{}}		& млрд. руб.\\
		\toprule
	    Драгоценные металлы &               3 889    \\
		Средства, размещенные у нерезидентов, и ценные бумаги иностранных эмитентов &             18 472    \\
		Кредиты и депозиты &               3 467    \\
		Ценные бумаги &                  517    \\
		в т.ч. долговые обязательства Правительства РФ &                  302    \\
		Требования к МВФ &               1 509    \\
		Прочие активы &               4 830    \\
		Итого по активу &             32 683    \\
		\midrule
		Наличные деньги в обращении &               8 339    \\
		Средства на счетах в Банке России &             10 214    \\
		в т.ч. средства Правительства РФ на счетах ЦБ РФ &               6 077    \\
		Средства кредитных организаций - резидентов на счетах ЦБ РФ &               3 047    \\
		Средства в расчетах &                      6    \\
		Обязательства перед МВФ &               1 396    \\
		Прочие пассивы &                  216    \\
		Капитал Банка России &             12 512    \\
		Итого по пассиву &             32 683    \\
		\bottomrule
	\end{tabularx}%
	\normalsize
\end{solution}

\pagebreak
\question[15] Предположим, что кривая спроса на депозиты задается как 
$D^d = 57 - 300 \cdot r_D$, кривая предложения депозитов задается как 
$D^s = 4.5 + 120 \cdot r_D$, кривая спроса на кредит задается как 
$L^d = 31 - 100 \cdot r_L $, а кривая предложения кредита задается как 
$L^s= 13 + 70 \cdot r_L$.
\noaddpoints

\begin{subparts}
	\subpart[5] Постройте примерный график рыночного спроса и предложения депозитов и определите равновесный объем депозитов и процентную ставку по депозитам на данном рынке.
	
	\begin{solution}[12em]
		\begin{multicols}{2}
		\setlength{\columnsep}{1cm}
		Пусть уравнение кривой задается формулой $D=c+k \cdot r_D$, где $D$ это объем депозитов, $r_D$ процентная ставка по депозитам. Индексы $d$ и $s$ - спрос и предложение.
		\begin{align*}
		r_D^*&=\frac{c_D^d-c_D^s}{k_D^s-k_D^d}=12.5\%\\
		D^*&= c_D^d + r_D^* \cdot k_D^d = 19.5.
		\end{align*}
		\end{multicols}
	\end{solution}
	
	\subpart[5] Постройте примерный график рыночного спроса и предложения кредита и найдите равновесный объем депозитов и процентную ставку по ссудам на данном рынке.
	
	\begin{solution}[12em]
		\begin{multicols}{2}
		\setlength{\columnsep}{1cm}
		Пусть уравнение кривой задается формулой $L=c+k \cdot r_L$, где $L$ это объем депозитов, $r_L$ процентная ставка по кредитам. 		
		\begin{align*}
		r_L^*&=\frac{c_L^d-c_L^s}{k_L^s-k_L^d}=11.5\%\\
		L^*&= c_L^d + r_L^* \cdot k_L^d = 19.5.
		\end{align*}
		
		\centering
		\includegraphics[scale=.7
		% trim={<left> <lower> <right> <upper>}				
		,trim={0cm .5cm 2cm 0cm},clip]
		{../../../texExercises/tikz/banking_system_marge}
		\end{multicols}
	\end{solution}
	
	\subpart[5] Каков объем маржи в процентах и в абсолютном выражении получает банковская система в целом на разнице доходов от кредитования и расходов по привлечению средств.
	
	\begin{solution}[12em] Процентная маржа определяется как разница процентных ставок по кредитам и депозитам: $marge=r_L^*-r_D^*=-1\%$. Прибыль банковской системы равна: $L^* \cdot marge = -0.195.$
	\end{solution}
	
\end{subparts}
\addpoints

\pagebreak
\question[15] Допустим, что требуемая норма резервного покрытия равна 5\% по депозитам до вос­требования и 7\% по срочным депозитам. 
\noaddpoints
\begin{subparts}
	\subpart[2] Если клиент депонирует 300~000~₽ на вклад до востребования, как будет реагировать на это отдельный банк?
	
	\begin{solution}[12em]
		Банк сохранит резервы в сумме:
		
		$\Delta D \cdot d_1 = 15~000~₽$
		
		где 
		
		$\Delta D$ - изменение банковских депозитов; $d_1$ - норма резервного покрытия по депозитам до востребования.
		
		Остальные деньги будут выданы клиентам банка в ввиде ссуд.
		
	\end{solution}
	
	\subpart[10]  Если текущие поступления от кредитования распределяются в пропорции – 75\% на депозиты до востребования и 25\% на срочные депозиты, определите возможные изменения консолидированного баланса.
	
	\begin{solution}[12em]
		Определяем значение депозитного мультипликатора:
		\begin{align*}
		\Delta TR&= d_1 \cdot w_1 \cdot \Delta D + d_2 \cdot w_2 \cdot \Delta D\\
		\Delta D&=\frac{1}{d_1 \cdot w_1 +d_2 \cdot w_2}\cdot \Delta TR\\
		m&= \frac{1}{d_1 \cdot w_1 +d_2 \cdot w_2} = 18.2
		\end{align*}
		где
		
		$\Delta TR$ - изменение резервов в банковской системе;	$\Delta D$ - изменение объема депозитов; $m$ - депозитный мультипликатор.
		
		Всего будет создано 5~454~545~₽ депозитов;
		
		в т.ч.
		
		$\Delta D \cdot w_1= 4~090~909~₽$ - депозитов до востребования;
		
		$\Delta D \cdot w_2= 1~363~636~₽$ - срочных депозитов.
		
		Резервы и ссуды составят, соответственно, 300~000~₽ и 5~154~545~₽.
		
		
	\end{solution}
	
	\subpart[3] Как изменится консолидированный баланс банков, если центральный банк введет единую норму резервирования по всем депозитам в размере 2.5\%?
	\begin{solution}[12em]
		\begin{align*}
		\frac{\Delta D }{d}=12~000~000~₽
		\end{align*}
		где $d$ - единая норма резервов.
		
	\end{solution}
	
\end{subparts}
\addpoints

\pagebreak
\question[15] Заемщик обращается в банк за получением ломбардного\linebreak кредита 12.01.20XX (год невисокосный). В залог он предоставил 25 шт. ценных бумаг, текущий курс которых составляет 500~₽/шт. Максимальная сумма кредита, которую может выдать банк не должна превышать 80\% стоимости залога. Банк берет единовременную комиссию в момент выдачи кредита в размере 200~₽. Кредит предоставляется до 13.04.20XX по ставке 8.5\% годовых. При расчете процентов принимается точное число дней пользования кредитом и точное число дней в году. Проценты за кредит начисляются, начиная со дня выдачи кредита, проценты в день погашения кредита не начисляются. 
\noaddpoints
\begin{subparts}
	\subpart[5]Какую сумму денег заемщик получит на руки?
	\begin{solution}[12em]
		
		Год невисокосный, база равна 365 дн. 
		
		Срок кредита: 91 дн.
		
		Максимальная сумма кредита: 10~000.00~₽
		
		Заемщик получит на руки: 9~588,08~₽
		
	\end{solution}
	
	\subpart[5] В день погашения кредита заемщик внес только 3000~₽ в банк, продлив оставшуюся сумму долга до 14.07.20XX.  Определите остаток долга.
	\begin{solution}[12em]
		\begin{multicols}{2}
		\setlength{\columnsep}{1cm}
		Срок кредита: 92 дн.
		\begin{align}
		N_1&=\frac{N_0-R}{1-r \cdot \frac{n}{base}}\\
		&=7153.26~₽\nonumber
		\end{align}
		где $N_0$ - первоначальный остаток долга; $N_1$ - остаток долга после погашения части суммы; $R$ - погашенная сумма долга; $r$ - ставка процента банка; $n$ - срок кредита в днях; $base$ - количество дней в году.
		\end{multicols}
	\end{solution}
	
	\subpart[5] В день погашения кредита заемщик в банк не явился. В день 21.07.20XX заемщик внес в банк только 2000~₽, продлив оставшуюся сумму долга до 15.10.20XX.  Определите сумму процентов за просрочку и остаток долга.
	\begin{solution}[6em]
		Срок просрочки: 7 дн.
		
		Проценты за просрочку 12.35~₽.
		
		Сумма, уплаченная за продление долга 1~987.65~₽. 
		
		Остаток долга 5~271.17~₽.
	\end{solution}
	
\end{subparts}
\addpoints

\pagebreak
\question[10] Ставка привлечения кредитов в рублях на межбанковском рынке 13\% годовых. Определите, какая ставка может быть предложена по депозитному вкладу клиенту при нормативе обязательных резервов 3\%. Средства, полученные банком на межбанковском рынке, не подлежат обязательному резервированию в ЦБ.
\begin{solution}[10em]
	\begin{align}
	r_D&=r_{LB} \cdot (1-d)\\
	&=12.61\%\nonumber
	\end{align}
	где
	
	$r_{LB}$ - ставка по кредитам на межбанковском рынке;
	
	$d$ - норма обязательных резервов;
	
	$r_D$ - ставка по депозитам физических лиц.
	
\end{solution}
	
\end{questions}

\pagebreak
\noindent\textbf{Тестовые вопросы (выберите один правильный ответ)}

\begin{questions}
\begin{multicols}{2}
\setlength{\columnsep}{1cm}

\question Сущность банков определяется тем, что они:
	 \begin{choices}
	 \CC изменяют денежную массу в обращении путем организации безналичного обращения;
	 \choice занимаются приемом вкладов;
	 \choice уменьшают денежную массу в обращении путем организации безналичного обращения;
	 \choice осуществляют эмиссию денежных знаков.
	 \end{choices}
\question Коммерческие банки, созданные с участием государственного капитала:
	 \begin{choices}
	 \choice должны предоставлять льготные кредиты государственным предприятиям;
	 \choice являются проводниками государственной социальной политики;
	 \choice не могут стремиться к получению прибыли;
	 \CC все сказанное неверно.
	 \end{choices}
\question Членство коммерческих банков в саморегулируемых организациях:
	 \begin{choices}
	 \choice обязательно в силу закона;
	 \choice почетно;
	 \choice желательно;
	 \CC коммерческий банк может быть членом нескольких саморегулируемых организаций.
	 \end{choices}
\question Обязательные резервы в Российской Федерации:
	 \begin{choices}
	 \choice установлены для всех кредитных организаций;
	 \CC для всех банков;
	 \choice для всех банков, кроме Сбербанка России;
	 \choice не используются.
	 \end{choices}
\question Учредители кредитной организации не имеют право выходить из состава участников:
	 \begin{choices}
	 \choice два года;
	 \choice один год;
	 \CC три года;
	 \choice полгода.
	 \end{choices}
\question Учредителем кредитной организации может быть:
	 \begin{choices}
	 \choice любое юридическое лицо;
	 \choice любое физическое лицо;
	 \CC любое юридическое или физическое лицо;
	 \choice только юридические и физические лица, удовлетворяющие определенным требованиям.
	 \end{choices}
\question Документы, представляемые учредителями в Территориальное управление Банка России, подписывает:
	 \begin{choices}
	 \choice самый крупный учредитель;
	 \CC лицо, уполномоченное собранием учредителей;
	 \choice председатель совета директоров (наблюдательного совета);
	 \choice главный бухгалтер.
	 \end{choices}
\question Укажите, какую функцию НЕ выполняет собственный капитал коммерческого банка:
	 \begin{choices}
	 \choice страховую;
	 \choice оперативную;
	 \choice регулирующую;
	 \CC развития.
	 \end{choices}
\question К депозитным источникам привлеченных средств НЕ относятся:
	 \begin{choices}
	 \CC кредиты центрального банка;
	 \choice продажа депозитных сертификатов;
	 \choice текущие счета предприятий;
	 \choice вклады населения.
	 \end{choices}
\question Какие виды вкладов в банках, вошедших в систему страхования вкладов, НЕ являются застрахованными:
	 \begin{choices}
	 \choice срочные вклады и вклады до востребования, включая валютные вклады;
	 \CC банковский вклад, удостоверенный сберегательным сертификатом на предъявителя;
	 \choice текущие счета, в том числе используемые для расчетов по банковским картам, для получения зарплаты, пенсии или стипендии;
	 \choice банковский вклад, удостоверенный именным сберегательным сертификатом.
	 \end{choices}
\question Размер возмещения по вкладам составляет:
	 \begin{choices}
	 \choice 100\% суммы вкладов, но не более 400 тыс руб;
	 \choice 100\% суммы вкладов, но не более 700 тыс руб;
	 \CC 100\% суммы вкладов, но не более 1400 тыс руб;
	 \choice 100\% суммы вкладов.
	 \end{choices}
\question Активы банка — это:
	 \begin{choices}
	 \choice вклады до востребования, акции и резервы;
	 \choice наличные деньги, собственность и резервы;
	 \choice наличные деньги, собственность и акции;
	 \CC выданные кредиты, размещенные в других банках депозиты.
	 \end{choices}
\question Доходы от финансовой деятельности включают
	 \begin{choices}
	 \choice штрафы и неустойки, взысканные банком;
	 \CC доходы от продажи производных инструментов;
	 \choice доход от реализации залогов;
	 \choice доходы по операциям прошлых лет, поступившие в текущем финансовом году.
	 \end{choices}
\question Основным инструментом безналичных платежей в России является:
	 \begin{choices}
	 \CC платежное поручение;
	 \choice платежное требование;
	 \choice аккредитив;
	 \choice инкассовое поручение.
	 \end{choices}
\question В основе банковского кредитования НЕ лежит принцип
	 \begin{choices}
	 \choice срочности;
	 \choice платности;
	 \CC безопасности;
	 \choice обеспеченности.
	 \end{choices}
\question К способам обеспечения возвратности кредитов НЕ относится:
	 \begin{choices}
	 \choice залог;
	 \choice банковская гарантия;
	 \choice поручительство;
	 \CC нет верного ответа.
	 \end{choices}
\question Мониторинг выданных кредитов осуществляет:
	 \begin{choices}
	 \choice специальное подразделение банка;
	 \CC сотрудник банка, выдававший кредит;
	 \choice сотрудник банка, проводивший оценку кредитоспособности;
	 \choice служба безопасности банка.
	 \end{choices}
\question Секьюритизация — это:
	 \begin{choices}
	 \choice рефинансирование задолженности;
	 \choice продажа неработающих активов;
	 \choice выпуск долговых обязательств;
	 \CC выпуск долговых обязательств, обеспеченных активами.
	 \end{choices}
\question Карта, которая позволяет ее держателю только распоряжаться средствами, находящимися на счете, называется:
	 \begin{choices}
	 \choice предоплаченной;
	 \choice кредитной;
	 \CC дебетовой;
	 \choice овердрафтной.
	 \end{choices}
\question К признакам, характеризующим возникновение проблем в банке, относятся:
	 \begin{choices}
	 \choice продажа основных фондов банка;
	 \CC отток вкладов из банка;
	 \choice неожиданное изменение организационной структуры банка;
	 \choice получение долгосрочного кредита в центральном банке.
	 \end{choices}
\question Российским банкам запрещается заниматься:
	 \begin{choices}
	 \choice страхованием и торговлей;
	 \CC страховой, торговой и производственной деятельностью;
	 \choice торговой и производственной деятельностью;
	 \choice профессиональной деятельностью на рынке ценных бумаг.
	 \end{choices}
\question Каков статус банковской ассоциации:
	 \begin{choices}
	 \choice коммерческая организация;
	 \CC некоммерческая организация;
	 \choice государственная корпорация;
	 \choice не является юридическим лицом.
	 \end{choices}
\question Под прямыми количественными ограничениями понимается:
	 \begin{choices}
	 \choice установление для банков ограничений проведения отдельных банковских операций;
	 \choice установление лимитов на участие иностранного капитала в банковской системе;
	 \choice ограничение в кредитовании;
	 \CC все перечисленное выше.
	 \end{choices}
\question Наиболее гибким инструментом денежно-кредитной политики, которым располагает Банк России, являются:
	 \begin{choices}
	 \CC процентные ставки по кредитам;
	 \choice валютные интервенции;
	 \choice норма обязательного резервирования;
	 \choice операции на открытом рынке.
	 \end{choices}
\question Банковская лицензия выдается:
	 \begin{choices}
	 \CC бессрочно;
	 \choice на три года;
	 \choice на пять лет;
	 \choice на срок, указанный в лицензии.
	 \end{choices}
\question Банк России предъявляет следующие требования к источникам средств, вносимым учредителями в уставный капитал банка:
	 \begin{choices}
	 \choice они могут быть привлечены учредителем на рынке капиталов;
	 \choice они могут быть привлечены посредством банковского кредита;
	 \choice они могут быть привлечены от партнеров учредителей;
	 \CC они должны быть исключительно собственными средствами учредителей.
	 \end{choices}
\question Кредитное управление банка ответственно:
	 \begin{choices}
	 \choice лишь за оформление кредитных договоров;
	 \CC за все операции, связанные с кредитованием и погашением ссуд;
	 \choice размещение имеющихся избыточных резервов;
	 \choice поиск и привлечение дополнительных ресурсов.
	 \end{choices}
\question Значение норматива достаточности собственного капитала банка не может быть ниже:
	 \begin{choices}
	 \choice 8\%;
	 \choice 2\%;
	 \choice 4\%;
	 \CC 10\%.
	 \end{choices}
\question К пассивам банка относятся:
	 \begin{choices}
	 \choice резервы, размещенные в центральном банке;
	 \choice наличность;
	 \CC выпущенные депозитные сертификаты;
	 \choice ссуды другим банкам.
	 \end{choices}
\question Какие виды вкладов в банках, вошедших в систему страхования вкладов, являются застрахованными:
	 \begin{choices}
	 \choice средства на счетах индивидуальных предпринимателей;
	 \choice средства на номинальных счетах опекунов/попечителей, бенефициарами по которым являются подопечные;
	 \CC средства на обезличенных металлических счетах;
	 \choice средства на счетах эскроу для расчетов по сделкам куплипродажи недвижимости на период их государственной регистрации.
	 \end{choices}
\question Качество активов определяется в зависимости:
	 \begin{choices}
	 \choice от степени риска;
	 \choice степени ликвидности;
	 \choice степени доходности;
	 \CC всего вышесказанного.
	 \end{choices}
\question К первоклассным ликвидным средствам не относятся:
	 \begin{choices}
	 \choice средства в кассе;
	 \choice средства на корреспондентских счетах в ЦБ РФ;
	 \CC векселя первоклассных эмитентов;
	 \choice средства на корреспондентских счетах в других банках.
	 \end{choices}
\question Капитальные расходы банка включают расходы:
	 \begin{choices}
	 \choice на покупку акций торгового портфеля;
	 \choice вложения в инвестиционный портфель;
	 \choice инвестиции в недвижимость;
	 \CC вступление в систему SWIFT.
	 \end{choices}
\question Банковская тайна — это тайна:
	 \begin{choices}
	 \CC о видах операций банка;
	 \choice счетах и вкладах его клиентов;
	 \choice счетах его корреспондентов;
	 \choice обо всем вышеперечисленном.
	 \end{choices}
\question Страхование кредитного риска банка — это страхование:
	 \begin{choices}
	 \choice наличных денег в кассе;
	 \choice ценностей в хранилище банка;
	 \CC имущества, принимаемого в залог;
	 \choice выданной ссуды.
	 \end{choices}
\question На кредитный комитет выносятся:
	 \begin{choices}
	 \CC все согласованные кредитные договоры;
	 \choice только крупные кредитные сделки;
	 \choice инвестиционные кредиты;
	 \choice соглашения о долевом финансировании.
	 \end{choices}
\question В качестве управляющей компании банк создает:
	 \begin{choices}
	 \choice фонды рынка денег;
	 \choice паевые фонды;
	 \CC общие фонды банковского управления;
	 \choice фонды фондов.
	 \end{choices}
\question Официальная цена на драгоценные металлы устанавливается:
	 \begin{choices}
	 \choice исходя из результатов торгов на межбанковской золотой бирже;
	 \choice итогов торгов на утренней сессии ММВБ;
	 \CC фиксинга Лондонской биржи драгоценных металлов;
	 \choice официального обменного курса рубля.
	 \end{choices}
\question POSтерминал предназначен:
	 \begin{choices}
	 \choice для проведения авторизации;
	 \choice получения наличных денег;
	 \choice получения информации о счете;
	 \CC оплаты покупки.
	 \end{choices}
\question Если планом финансового оздоровления предусмотрено изменение организационной структуры банка, то это предполагает следующие меры:
	 \begin{choices}
	 \choice закрытие филиалов банка;
	 \choice сокращение сотрудников банка;
	 \choice присоединение банка к другому банку;
	 \CC все вышеуказанное.
	 \end{choices}



\end{multicols}
\end{questions}

\end{document}
