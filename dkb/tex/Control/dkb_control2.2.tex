% !TeX program = lualatex -synctex=1 -interaction=nonstopmode --shell-escape %.tex

\documentclass[12pt, table]{exam}
\usepackage[rus]{borochkin}

\usepackage{borochkin_exam}

%%%%%%%%%%%%%%%%%%%%%%%%%%%%%%%%%%%%%%%%%

\professor
\iftagged{professor}{ \printanswers }
%%%%%%%%%%%%%%%%%%%%%%%%%%%%%%%%%%%%%%%%%


\begin{document}
\setcounter{section}{0\relax}%
\noindent
% Контр/р № #1, Вариант #2, Предмет #3
\studentpersonalinfo{2}{2}{ДКБ}
\normalsize

\begin{questions}
\question[40] Тест
\answerstotest

\pagebreak
\question[10] Составьте баланс Центрального Банка РФ, используя статьи из финансовой отчетности Банка России, приведенные в таблице. 

\small
\begin{tabularx}{\linewidth}[b]{@{}>{\raggedright\arraybackslash}Xr@{}}
	млрд. руб.\\
	\toprule
    Кредиты и депозиты &               2 971    \\
	Драгоценные металлы &               4 271    \\
	Средства на счетах в Банке России &             10 708    \\
	Капитал Банка России &               8 652    \\
	Средства, размещенные у нерезидентов, и ценные бумаги иностранных эмитентов &             19 755    \\
	Прочие активы &               1 427    \\
	Требования к МВФ &               1 542    \\
	Обязательства перед МВФ &               1 443    \\
	Долговые обязательства Правительства РФ &                  288    \\
	Средства Правительства РФ на счетах ЦБ РФ &               5 919    \\
	Прочие пассивы &                  789    \\
	Средства кредитных организаций - резидентов на счетах ЦБ РФ &               3 043    \\
	Средства в расчетах &                      7    \\
	Наличные деньги в обращении &               8 866    \\
	Ценные бумаги &                  499    \\
	\bottomrule
\end{tabularx}%
\normalsize
\begin{solution}[12em] В результате перегруппировки балансовых статей получаем:
	
	\small
	\begin{tabularx}{\linewidth}[b]{@{}>{\raggedright\arraybackslash}Xr@{}}		& млрд. руб.\\
		\toprule
	    Драгоценные металлы &               4 271    \\
		Средства, размещенные у нерезидентов, и ценные бумаги иностранных эмитентов &             19 755    \\
		Кредиты и депозиты &               2 971    \\
		Ценные бумаги &                  499    \\
		в т.ч. долговые обязательства Правительства РФ &                  288    \\
		Требования к МВФ &               1 542    \\
		Прочие активы &               1 427    \\
		Итого по активу &             30 465    \\
		\midrule
		Наличные деньги в обращении &               8 866    \\
		Средства на счетах в Банке России &             10 708    \\
		в т.ч. средства Правительства РФ на счетах ЦБ РФ &               5 919    \\
		Средства кредитных организаций - резидентов на счетах ЦБ РФ &               3 043    \\
		Средства в расчетах &                      7    \\
		Обязательства перед МВФ &               1 443    \\
		Прочие пассивы &                  789    \\
		Капитал Банка России &               8 652    \\
		Итого по пассиву &             30 465    \\
		\bottomrule
	\end{tabularx}%
	\normalsize
\end{solution}

\pagebreak
\question[15] Предположим, что кривая спроса на депозиты задается как 
$D^d = 15 - 30 \cdot r_D$, кривая предложения депозитов задается как 
$D^s = 3 + 30 \cdot r_D$, кривая спроса на кредит задается как 
$L^d = 19.75 - 50 \cdot r_L $, а кривая предложения кредита задается как 
$L^s= 3.5 + 35 \cdot r_L$.
\noaddpoints

\begin{subparts}
	\subpart[5] Постройте примерный график рыночного спроса и предложения депозитов и определите равновесный объем депозитов и процентную ставку по депозитам на данном рынке.
	
	\begin{solution}[12em]
		\begin{multicols}{2}
			\setlength{\columnsep}{1cm}
			Пусть уравнение кривой задается формулой $D=c+k \cdot r_D$, где $D$ это объем депозитов, $r_D$ процентная ставка по депозитам. Индексы $d$ и $s$ - спрос и предложение.
			\begin{align*}
			r_D^*&=\frac{c_D^d-c_D^s}{k_D^s-k_D^d}=20\%\\
			D^*&= c_D^d + r_D^* \cdot k_D^d = 9.
			\end{align*}
		\end{multicols}
	\end{solution}
	
	\subpart[5] Постройте примерный график рыночного спроса и предложения кредита и найдите равновесный объем депозитов и процентную ставку по ссудам на данном рынке.
	
	\begin{solution}[12em]
		\begin{multicols}{2}
			\setlength{\columnsep}{1cm}
			Пусть уравнение кривой задается формулой $L=c+k \cdot r_L$, где $L$ это объем депозитов, $r_L$ процентная ставка по кредитам. 		
			\begin{align*}
			r_L^*&=\frac{c_L^d-c_L^s}{k_L^s-k_L^d}=21.5\%\\
			L^*&= c_L^d + r_L^* \cdot k_L^d = 9.
			\end{align*}
			
			\centering
			\includegraphics[scale=.7
			% trim={<left> <lower> <right> <upper>}				
			,trim={0cm .5cm 2cm 0cm},clip]
			{../../../texExercises/tikz/banking_system_marge}
		\end{multicols}
	\end{solution}
	
	\subpart[5] Каков объем маржи в процентах и в абсолютном выражении получает банковская система в целом на разнице доходов от кредитования и расходов по привлечению средств.
	
	\begin{solution}[12em] Процентная маржа определяется как разница процентных ставок по кредитам и депозитам: $marge=r_L^*-r_D^*=1.5\%$. Прибыль банковской системы равна: $L^* \cdot marge = 0.135.$
	\end{solution}
	
\end{subparts}
\addpoints

\pagebreak
\question[15] Допустим, что требуемая норма резервного покрытия равна 1.5\% по депозитам до вос­требования и 2\% по срочным депозитам. 
\noaddpoints
\begin{subparts}
	\subpart[2] Если клиент депонирует 500~000~₽ на вклад до востребования, как будет реагировать на это отдельный банк?
	
	\begin{solution}[12em]
		Банк сохранит резервы в сумме:
		
		$\Delta D \cdot d_1 = 7500~₽$
		
		где 
		
		$\Delta D$ - изменение банковских депозитов; $d_1$ - норма резервного покрытия по депозитам до востребования.
		
		Остальные деньги будут выданы клиентам банка в ввиде ссуд.
		
	\end{solution}
	
	\subpart[10]  Если текущие поступления от кредитования распределяются в пропорции – 55\% на депозиты до востребования и 45\% на срочные депозиты, определите возможные изменения консолидированного баланса.
	
	\begin{solution}[12em]
		Определяем значение депозитного мультипликатора:
		\begin{align*}
		\Delta TR&= d_1 \cdot w_1 \cdot \Delta D + d_2 \cdot w_2 \cdot \Delta D\\
		\Delta D&=\frac{1}{d_1 \cdot w_1 +d_2 \cdot w_2}\cdot \Delta TR\\
		m&= \frac{1}{d_1 \cdot w_1 +d_2 \cdot w_2} = 58
		\end{align*}
		где
		
		$\Delta TR$ - изменение резервов в банковской системе; $\Delta D$ - изменение объема депозитов; $m$ - депозитный мультипликатор.
		
		Всего будет создано 28~985~507~₽ депозитов;
		
		в т.ч.
		
		$\Delta D \cdot w_1= 15~942~029~₽$ - депозитов до востребования;
		
		$\Delta D \cdot w_2= 13~043~478~₽$ - срочных депозитов.
		
		Резервы и ссуды составят, соответственно, 500~000~₽ и 28~485~507~₽.
		
		
	\end{solution}
	
	\subpart[3] Как изменится консолидированный баланс банков, если центральный банк введет единую норму резервирования по всем депозитам в размере 1.25\%?
	\begin{solution}[12em]
		\begin{align*}
		\frac{\Delta D }{d}=40~000~000~₽
		\end{align*}
		где $d$ - единая норма резервов.
		
	\end{solution}
	
\end{subparts}
\addpoints

\pagebreak
\question[15] Заемщик обращается в банк за получением ломбардного\linebreak кредита 23.02.20XX (год високосный). В залог он предоставил 600 шт. ценных бумаг, текущий курс которых составляет 300~₽/шт. Максимальная сумма кредита, которую может выдать банк не должна превышать 80\% стоимости залога. Банк берет единовременную комиссию в момент выдачи кредита в размере 200~₽. Кредит предоставляется до 24.05.20XX по ставке 7.5\% годовых. При расчете процентов принимается точное число дней пользования кредитом и точное число дней в году. Проценты за кредит начисляются, начиная со дня выдачи кредита, проценты в день погашения кредита не начисляются. 
\noaddpoints
\begin{subparts}
	\subpart[5]Какую сумму денег заемщик получит на руки?
	\begin{solution}[12em]
		
		Год високосный, база равна 366 дн. 
		
		Срок кредита: 91 дн.
		
		Максимальная сумма кредита: 144 000.00~₽
		
		Заемщик получит на руки: 141 114,75~₽
		
	\end{solution}
	
	\subpart[5] В день погашения кредита заемщик внес только 50~000~₽ в банк, продлив оставшуюся сумму долга до 25.08.20XX.  Определите остаток долга.
	\begin{solution}[12em]
		\begin{multicols}{2}
			\setlength{\columnsep}{1cm}
			Срок кредита: 93 дн.
			\begin{align}
			N_1&=\frac{N_0-R}{1-r \cdot \frac{n}{base}}\\
			&=95~826,20~₽\nonumber
			\end{align}
			где $N_0$ - первоначальный остаток долга; $N_1$ - остаток долга после погашения части суммы; $R$ - погашенная сумма долга; $r$ - ставка процента банка; $n$ - срок кредита в днях; $base$ - количество дней в году.
		\end{multicols}
	\end{solution}
	
	\subpart[5] В день погашения кредита заемщик в банк не явился. В день 09.09.20XX заемщик внес в банк только 25~000~₽, продлив оставшуюся сумму долга до 26.11.20XX.  Определите сумму процентов за просрочку и остаток долга.
	\begin{solution}[6em]
		Срок просрочки: 15 дн.
		
		Проценты за просрочку 333,82~₽.
		
		Сумма, уплаченная за продление долга 24 666,18~₽. 
		
		Остаток долга 72~315,89~₽.
	\end{solution}
	
\end{subparts}
\addpoints

\pagebreak
\question[10] Ставка привлечения кредитов в рублях на межбанковском рынке 10\% годовых. Определите, какая ставка может быть предложена по депозитному вкладу клиенту при нормативе обязательных резервов 5\%. Средства, полученные банком на межбанковском рынке, не подлежат обязательному резервированию в ЦБ.
\begin{solution}[10em]
	\begin{align}
	r_D&=r_{LB} \cdot (1-d)\\
	&=9.5\%\nonumber
	\end{align}
	где
	
	$r_{LB}$ - ставка по кредитам на межбанковском рынке;
	
	$d$ - норма обязательных резервов;
	
	$r_D$ - ставка по депозитам физических лиц.
	
\end{solution}


\end{questions}

\pagebreak
\noindent\textbf{Тестовые вопросы (выберите один правильный ответ)}

\begin{questions}
\begin{multicols}{2}
\setlength{\columnsep}{1cm}

\question Универсальный банк:
	 \begin{choices}
	 \CC выполняет весь перечень банковских операций;
	 \choice обслуживает и физических, и юридических лиц;
	 \choice имеет рублевую и валютную лицензию;
	 \choice имеет генеральную лицензию.
	 \end{choices}
\question Какой деятельностью не занимаются банковские ассоциации:
	 \begin{choices}
	 \choice оказанием членам и участникам правовой, консультационной помощи;
	 \choice анализом состояния и тенденций развития экономики, банковского дела и рынка финансовых услуг;
	 \choice предоставлением крупных инвестиционных кредитов;
	 \CC обеспечением функционирования негосударственной системы разрешения споров между членами.
	 \end{choices}
\question Банк России может издавать следующие нормативные акты:
	 \begin{choices}
	 \choice законы, положения, инструкции, указания;
	 \choice письма, положения, инструкции, указания;
	 \choice положения, инструкции, указания;
	 \CC телеграммы, письма.
	 \end{choices}
\question Банк России обязан отозвать лицензию на осуществление банковской деятельности, если достаточность капитала становится ниже:
	 \begin{choices}
	 \choice 8\%;
	 \CC 10\%;
	 \choice 6\%;
	 \choice 2\%.
	 \end{choices}
\question Генеральная банковская лицензия может быть получена кредитной организацией:
	 \begin{choices}
	 \choice сразу при оплате необходимой для этого величины уставного капитала;
	 \choice только через два года успешной деятельности при соблюдении необходимых для этого требований;
	 \CC только через пять лет успешной деятельности при соблюдении необходимых для этого требований;
	 \choice сразу после оплаты необходимой для этого величины уставного капитала и при наличии гарантии, выданной зарубежным банком.
	 \end{choices}
\question При создании кредитной организации учредители могут вносить в уставный капитал:
	 \begin{choices}
	 \choice только денежные средства;
	 \choice только денежные средства и здание (помещение);
	 \CC только денежные средства и имущество в неденежной форме согласно перечню, утвержденному Банком России;
	 \choice денежные средства, здание (помещение), имущество в неденежной форме.
	 \end{choices}
\question Капитал банка считается достаточным в зависимости:
	 \begin{choices}
	 \choice от структуры его пассивов;
	 \CC качества его активов;
	 \choice резервных требований;
	 \choice его абсолютной величины.
	 \end{choices}
\question Средства резервного фонда должны использоваться:
	 \begin{choices}
	 \choice на покрытие убытков отчетного года;
	 \choice выплату дивидендов по акциям банка;
	 \choice выплату процентов, начисленных по вкладам;
	 \CC материальное поощрение сотрудников.
	 \end{choices}
\question Что такое банковский вклад:
	 \begin{choices}
	 \CC денежные средства, размещаемые физическими и юридическими лицами в банках;
	 \choice денежные средства, размещаемые физическими лицами  гражданами России в банках;
	 \choice денежные средства, размещаемые физическими лицами в банках, включая капитализированные проценты;
	 \choice любые денежные средства в банке.
	 \end{choices}
\question Источниками формирования фонда обязательного страхования вкладов являются:
	 \begin{choices}
	 \choice страховые взносы банков, доходы от размещения фонда и иные доходы;
	 \CC средства федерального бюджета;
	 \choice средства Банка России;
	 \choice средства обязательных резервов банков.
	 \end{choices}
\question Активные банковские операции — это:
	 \begin{choices}
	 \choice выдача ссуд;
	 \choice формирование капитала;
	 \CC прием депозитов;
	 \choice формирование резервного фонда.
	 \end{choices}
\question Под долгосрочными активами понимаются активы со сроком погашения:
	 \begin{choices}
	 \choice свыше 180 дней, но не более 360 дней;
	 \choice свыше одного года;
	 \choice от двух до пяти лет;
	 \CC от пяти до 30 лет.
	 \end{choices}
\question Нераспределенную прибыль банк:
	 \begin{choices}
	 \choice может использовать по собственному усмотрению;
	 \CC обязан направлять на формирование резервного фонда;
	 \choice должен выплатить учредителям (собственникам);
	 \choice должен использовать на расширение кредитования.
	 \end{choices}
\question Ломбардным называется кредит, выдаваемый:
	 \begin{choices}
	 \CC под залог справки из ломбарда;
	 \choice специальным ломбардным коммерческим банком;
	 \choice Банком России под залог ценных бумаг, перечень которых он же и устанавливает;
	 \choice ломбардом.
	 \end{choices}
\question Банковские услуги кредитования, если их классифицировать по технике предоставления кредита, бывают:
	 \begin{choices}
	 \choice индивидуальными;
	 \choice синдицированными;
	 \CC двусторонними;
	 \choice нет верного ответа.
	 \end{choices}
\question Способы возврата кредита согласуются на стадии:
	 \begin{choices}
	 \choice обсуждения условий кредита;
	 \choice оценки кредитоспособности заемщика;
	 \choice заседания кредитного комитета;
	 \CC подписания кредитного договора.
	 \end{choices}
\question Переводной вексель отличается от простого тем, что:
	 \begin{choices}
	 \choice на нем проставлен аваль;
	 \CC он допускает возможность цессии;
	 \choice он не будет оплачен без акцепта;
	 \choice верны пункты «Б» и «В».
	 \end{choices}
\question Организация, осуществляющая расчетную деятельность с предприятиями торговли и услуг по операциям, совершаемым с использованием карт, — это:
	 \begin{choices}
	 \choice мерчантдайзер;
	 \choice эквайер;
	 \choice расчетный банк;
	 \CC эмитент.
	 \end{choices}
\question Карта, на которой присутствуют данные в виде рельефных знаков, — это:
	 \begin{choices}
	 \choice эмбосированная карта;
	 \choice интеллектуальная карта;
	 \CC магнитная карта;
	 \choice персонифицированная карта.
	 \end{choices}
\question Временная администрация вводится в банк в соответствии с решением, принятым:
	 \begin{choices}
	 \choice Банком России;
	 \CC Правительством РФ;
	 \choice арбитражным судом;
	 \choice Агентством страхования вкладов.
	 \end{choices}
\question Банковское законодательство включает:
	 \begin{choices}
	 \choice только специальные банковские законы;
	 \CC банковские законы и законы общего действия;
	 \choice банковские законы, законы общего действия и нормативные документы Банка России;
	 \choice все законы, затрагивающие какиелибо аспекты деятельности банков.
	 \end{choices}
\question Банковская система России в соответствии с Законом о банках и банковской деятельности включает в себя:
	 \begin{choices}
	 \choice Банк России;
	 \CC коммерческие банки;
	 \choice филиалы и представительства иностранных банков;
	 \choice все вышеназванное верно.
	 \end{choices}
\question Обязательные резервы как инструмент денежно-кредитной политики Банк России использует:
	 \begin{choices}
	 \choice для регулирования денежной массы в обращении;
	 \choice страхования вкладчиков от потерь;
	 \choice возмещения собственных потерь;
	 \CC регулирования ликвидности банка.
	 \end{choices}
\question Генеральная лицензия дает банку право:
	 \begin{choices}
	 \CC вести все банковские операции;
	 \choice осуществлять все банковские операции в любой валюте;
	 \choice открывать свои подразделения за границей РФ;
	 \choice привлекать средства частных лиц.
	 \end{choices}
\question Банк России принимает решение о лицензировании кредитной организации:
	 \begin{choices}
	 \CC самостоятельно;
	 \choice с учетом рекомендаций Министерства экономического развития и торговли Российской Федерации (Минэкономразвития России);
	 \choice совместно с Федеральной налоговой службой Российской Федерации (ФНС России);
	 \choice совместно с Минфином России.
	 \end{choices}
\question Банк России предъявляет определенные требования к будущим руководителям кредитной организации, в том числе:
	 \begin{choices}
	 \choice о наличии у кандидатов высшего образования;
	 \choice наличии у кандидатов высшего экономического образования;
	 \choice наличии у кандидатов высшего юридического образования;
	 \CC наличии у кандидатов высшего экономического или юридического образования и опыта руководящей работы в банке.
	 \end{choices}
\question В состав дополнительного капитала банка входят:
	 \begin{choices}
	 \choice стоимость обыкновенных акций;
	 \CC стоимость привилегированных акций;
	 \choice резервный фонд;
	 \choice нераспределенная прибыль.
	 \end{choices}
\question Фонды банка формируются за счет:
	 \begin{choices}
	 \choice привлеченных средств;
	 \choice средств Банка России;
	 \choice прибыли;
	 \CC всего вышеперечисленного.
	 \end{choices}
\question Привлекать деньги от физических лиц во вклады имеет право:
	 \begin{choices}
	 \choice любой банк с момента создания и получения лицензии на осуществление банковской деятельности;
	 \choice банк, работающий не менее одного года;
	 \CC банк, работающий не менее двух лет и имеющий соответствующую лицензию;
	 \choice банк, вошедший в систему страхования вкладов.
	 \end{choices}
\question Страховым случаем является:
	 \begin{choices}
	 \choice отзыв лицензии у банка;
	 \choice введение Банком России временной администрации в банке;
	 \CC введение Банком России моратория на удовлетворение требований кредиторов;
	 \choice одновременно все вышеуказанное.
	 \end{choices}
\question Банк проводит активные операции:
	 \begin{choices}
	 \choice для привлечения новых вкладчиков;
	 \choice кредитования нуждающихся в финансировании предприятий;
	 \choice извлечения прибыли;
	 \CC выполнения указаний регулирующих органов.
	 \end{choices}
\question Комиссионные доходы банка включают:
	 \begin{choices}
	 \choice доходы от переоценки инвалютных ресурсов;
	 \choice тарифы за расчетно-кассовое обслуживание;
	 \CC стоимость обязательства банка предоставить кредит;
	 \choice все вышеперечисленное.
	 \end{choices}
\question В целях осуществления межбанковских расчетных операций банки устанавливают
	 \begin{choices}
	 \choice дружеские отношения;
	 \choice корреспондентские отношения;
	 \choice кредитные отношения;
	 \CC комиссионные отношения.
	 \end{choices}
\question Орудием краткосрочного коммерческого кредита является:
	 \begin{choices}
	 \CC кредитный договор;
	 \choice вексель;
	 \choice долговая расписка;
	 \choice аккредитив.
	 \end{choices}
\question Страховая компания обязана выплатить банку страховое возмещение при страховании ответственности заемщика за невозврат кредита в следующих случаях:
	 \begin{choices}
	 \choice заемные средства не возвращены вследствие совершения работниками банка преступления, находящегося в прямой причинной связи со страховым случаем;
	 \choice заемные средства не возвращены, так как использовались не по целевому назначению;
	 \CC заемщик не получил прибыль и обанкротился;
	 \choice верны пункты «Б» и «В».
	 \end{choices}
\question Уровень кредитного риска банка:
	 \begin{choices}
	 \CC остается неизменным после оценки кредитоспособности заемщика;
	 \choice меняется вслед за изменением финансового положения заемщика;
	 \choice меняется в зависимости от точности исполнения обязательств по кредитному договору заемщиком;
	 \choice меняется вслед за изменением стоимости залога.
	 \end{choices}
\question Ипотечные облигации — это:
	 \begin{choices}
	 \choice старшие облигации, поскольку обеспечены залогом недвижимости;
	 \choice облигации пониженного статуса, поскольку стоимость обеспечения может меняться;
	 \CC необеспеченные облигации, поскольку у банка нет имущественных прав на предмет залога;
	 \choice облигации, обеспеченные закладной на объект недвижимости.
	 \end{choices}
\question Карта, которая позволяет ее держателю оплачивать покупки при наличии нулевого остатка на счете держателя, является:
	 \begin{choices}
	 \choice предоплаченной;
	 \choice кредитной;
	 \CC дебетовой;
	 \choice международной.
	 \end{choices}
\question Импринтер используется:
	 \begin{choices}
	 \choice для введения ПИН-кода;
	 \choice переноса оттиска рельефных знаков на слип;
	 \choice получения информации о состоянии счета;
	 \CC проведения авторизации карты.
	 \end{choices}
\question Ликвидация банка-банкрота происходит по решению:
	 \begin{choices}
	 \choice Банка России;
	 \choice Агентства страхования вкладов;
	 \choice арбитражного суда;
	 \CC суда общей инстанции.
	 \end{choices}



\end{multicols}
\end{questions}

\end{document}
