% !TeX program = lualatex -synctex=1 -interaction=nonstopmode --shell-escape %.tex

\documentclass[12pt, table]{exam}
\usepackage[rus]{borochkin}

\usepackage{borochkin_exam}

%%%%%%%%%%%%%%%%%%%%%%%%%%%%%%%%%%%%%%%%%

\professor
\iftagged{professor}{ \printanswers }
%%%%%%%%%%%%%%%%%%%%%%%%%%%%%%%%%%%%%%%%%


\begin{document}
\setcounter{section}{0\relax}%
\noindent
% Контр/р № #1, Вариант #2, Предмет #3
\studentpersonalinfo{1}{2}{ДКБ}
\normalsize

\begin{questions}
\question[40] Тест
\answerstotest

\pagebreak
\question[15] Рассчитайте размер денежной базы, основные агрегаты денежной массы (M0, M1, M2) и величину денежного мультипликатора, исходя из приведенных в таблице показателей.

\small
\begin{tabularx}{\linewidth}[b]{@{}>{\raggedright\arraybackslash}Xr@{}}		& млрд. руб.\\
	\toprule
	Срочные депозиты нефинансовых и финансовых (кроме кредитных) организаций & 5948,1 \\
	Депозиты банков в Банке России & 702,3 \\
	Наличные деньги в обращении & 8033,3 \\
	Корреспондентские счета кредитных организаций в Банке России & 1834,8 \\
	Обязательные резервы & 505,1 \\
	Срочные депозиты & 21217,8 \\
	Депозиты населения & 3923,6 \\
	Депозиты нефинансовых и финансовых (кроме кредитных) организаций & 6100,3 \\
	Переводные депозиты & 10023,9 \\
	Срочные депозиты населения & 15269,7 \\
	Остатки средств в кассах кредитных организаций & 832,6 \\
	\bottomrule
\end{tabularx}%
\normalsize
\begin{solution}[23em] В результате перегруппировки балансовых статей получаем:

\small	
\begin{tabularx}{\linewidth}[b]{@{}>{\raggedright\arraybackslash}Xr@{}}				& млрд. руб.\\
	\toprule
	Наличные деньги в обращении & 8033,3 \\
	Остатки средств в кассах кредитных организаций & 832,6 \\
	Корреспондентские счета кредитных организаций в Банке России & 1834,8 \\
	Обязательные резервы & 505,1 \\
	Депозиты банков в Банке России & 702,3 \\
	Переводные депозиты & 10023,9 \\
	Депозиты населения & 3923,6 \\
	Депозиты нефинансовых и финансовых (кроме кредитных) организаций & 6100,3 \\
	Срочные депозиты & 21217,8 \\
	Срочные депозиты населения & 15269,7 \\
	Срочные депозиты нефинансовых и финансовых (кроме кредитных) организаций & 5948,1 \\
	\midrule
	Денежная база & 11908,1 \\
	М0    & 8033,3 \\
	М1    & 18057,2 \\
	М2    & 39275 \\
	Денежный мультипликатор & 3,30 \\
	\bottomrule
\end{tabularx}%
\normalsize
	
\end{solution}

\pagebreak	
\question[10] Предположим, что спрос и предложение капитала (без учета инфляционных ожиданий) заданы функциями 	
$r^d=13-2 \cdot q^d$, $r^s=5 + 2 \cdot q^s$, где 	
$r$ – процентная ставка, 	
$q$ – объем капитала.	
\noaddpoints
\begin{subparts}
	\subpart[2] Определите равновесные значения $q^*$ и $r^*$.

	\begin{solution}[12em]

		\raggedright
		Пусть уравнения спроса и предложения на капитал задаются, соответственно, функциями: $r^d=k^d\cdot q^d +c^d$ и $r^s=k^s\cdot q^s +c^s$, тогда точка пересечения прямых определяется по формулам:
		\begin{align*}
		q^*&=\frac{c^d-c^s}{k^s-k^d}=2\\
		r^*&=c^d + k^d \cdot q^* =9.
		\end{align*}
	\end{solution}
	
	\subpart[4] Что произойдет, если доходность государственных краткосрочных облигаций на финансовом рынке внезапно повысится до 10\%? Нарисуйте график.
	\begin{solution}[12em]
		\begin{multicols}{2}
			\setlength{\columnsep}{1cm}
		Спрос на капитал, $q^d$,  и предложение капитала, $q^s$, при искусственном завышении процентных ставок:
		\begin{align*}
		q^d&=\frac{r^{max}-c^d}{k^d}=1.5\\
		q^s&=\frac{r^{max}-c^s}{k^s}=2.5.
		\end{align*}
		\centering
		\includegraphics[scale=.7
		% trim={<left> <lower> <right> <upper>}				
		,trim={0cm 3cm 4cm 0cm},clip]
		{../../../texExercises/tikz/credit_excessive_demand}
		\end{multicols}
	\end{solution}
	
	\subpart[4] Какова будет величина избыточного предложения капитала?
	\begin{solution}[12em]
		Избыточное предложение капитала:
		\begin{align*}
		q^s-q^d=1.
		\end{align*}
	\end{solution}
	
\end{subparts}
\addpoints

\pagebreak
\question[10] Предположим, что бессрочная облигация дает ежегодный доход 50~000~₽.
\noaddpoints
\begin{subparts}
	\subpart[5] Определите ожидаемую рыночную цену такой облигации, если доходность других активов с тем же уровнем риска составляет 2.5\%.
	
	\begin{solution}[12em]
		\begin{align*}
		P_1=\frac{C}{r_1}=2~000~000~₽,
		\end{align*}	
		где
		
		$P_1$ - рыночная цена облигации;
		
		$C$ - купонный доход, ₽;
		
		$r_1$ - процентная ставка, годовых.
	\end{solution}
	
	\subpart[5] Если доходность других активов вырастет до 2\%, то повысится или снизится рыночная цена этой облигации и до какого уровня.
	
	\begin{solution}[12em]
		\begin{align*}
		P_2=\frac{C}{r_2}=2~500~000~₽.
		\end{align*}	
		
	\end{solution}
\end{subparts}
\addpoints

\pagebreak
\question[20] Номинал облигации 25~000~₽, купон 2.5\%, выплачивается один раз в год. До погашения облигации 20 лет. Облигация стоит 20~000~₽. 
\noaddpoints
\begin{subparts}
	\subpart[10] Определите ориентировочно доходность до погашения облигации.
	\begin{solution}[12em]
		\begin{align*}
		r^*&=\frac{(N-P)/m \cdot n + C/m }{(N+P)/2}\\
		r^*&=3,89\%
		\end{align*}
		где
		
		$r^*$ - ориентировочная доходность облигации;
		
		$N$ - номинал облигации;
		
		$P$ - рыночная цена облигации;
		
		$C$ - сумма купона;
		
		$n, m$ - срок облигации и количество выплат процентов в году, соответственно.
	\end{solution}
	
	\subpart[10] Определите доходность до погашения облигации методом линейной интерполяции.
	
	\begin{solution}[12em]
		\begin{align*}
		r_1&=9\%\\
		r_2&=10\%\\
		P_1&=\frac{C}{r_1} \cdot \left(1 - \frac{1}{(1+r_1)^n} \right) + \frac{N}{(1+r_1)^n}=19903,6~₽\\
		P_2&=\frac{C}{r_2} \cdot \left(1 - \frac{1}{(1+r_2)^n} \right) + \frac{N}{(1+r_1)^n}=23140,3~₽\\
		r^{**}&=r_1+(r_2-r_1) \cdot \frac{P_1-P}{P_1-P_2}=3,97\%
		\end{align*}	
		где
		
		$r_1, r_2$ - соответственно, округленная вверх и вниз ориентировочная доходность облигации;
		
		$P_1, P_2$ - соответственно, нижняя и верхняя границы рыночной цены облигации;
		
		$r^{**}$ - доходность до погашения облигации, определенная методом линейной интерполяции.
	\end{solution}
	
\end{subparts}
\addpoints

\pagebreak
\question[10] Предположим, что и реальная и номинальная процентные ставки по однолетним контрактам равны 8\% (без учета инфляционных ожиданий).
\noaddpoints

\begin{subparts}
	\subpart[5] Определите новую ожидаемую номинальную процентную ставку, если предполагается, что в течение следующего года цены вырастут на 12\%.
	
	\begin{solution}[12em]
		\begin{align*}
		r_n&=r_r+\tau + r_r \cdot \tau\\
		&=20.96\%.\nonumber
		\end{align*}
		где
	
		$r_n$ - номинальная ставка процента;
		
		$r_r$ - реальная ставка процента;
		
		$\tau$ - ожидаемые темпы инфляции.
	\end{solution}
	
	\subpart[5] Определите новую ожидаемую номинальную процентную ставку, если предполагается, что цены понизятся на 4\%.
	
	\begin{solution}[12em]
		
		\raggedright
		Ответ: 3.68\%.
	\end{solution}
	
\end{subparts}
\addpoints

\end{questions}

\pagebreak
\noindent\textbf{Тестовые вопросы (выберите один правильный ответ)}

\begin{questions}
\begin{multicols}{2}
\setlength{\columnsep}{1cm}

\question К непосредственным предпосылкам появления денег относятся:
	 \begin{choices}
	 \choice Открытия золотых месторождений и появление рынков продовольственных товаров;
	 \choice Переход от натурального хозяйства к производству и обмену товарами и имущественное обособление производителей товаров;
	 \CC Формирование централизованных государств и открытие золотых месторождений;
	 \choice Наличие частной собственности на средства производства и появление крупных оптовых рынков.
	 \end{choices}
\question   это товарообменная сделка с передачей права собственности на товар без оплаты деньгами
	 \begin{choices}
	 \choice Форфейтинг;
	 \choice Демпинг;
	 \choice Бартер;
	 \CC Факторинг.
	 \end{choices}
\question  концепция происхождения денег считает, что деньги возникли в результате специального соглашения между людьми
	 \begin{choices}
	 \choice Монетаристская;
	 \choice Эволюционная;
	 \choice Рационалистическая;
	 \CC Психологическая.
	 \end{choices}
\question К знакам стоимости относятся:
	 \begin{choices}
	 \choice Металлические деньги, у которых номинальная стоимость соответствует реальной стоимости;
	 \choice Бумажные и кредитные деньги, стершаяся металлическая монета;
	 \CC Золотые деньги;
	 \choice Только кредитные деньги.
	 \end{choices}
\question Сущность функции меры стоимости проявляется в том, что деньги выступают:
	 \begin{choices}
	 \CC Посредником при обмене товаров;
	 \choice Средством оплаты долговых обязательств;
	 \choice Всеобщим стоимостным эталоном;
	 \choice Средством накопления и сбережения.
	 \end{choices}
\question Сущность функции средства обращения проявляется в том, что деньги выступают:
	 \begin{choices}
	 \choice Посредником при обмене товаров;
	 \choice Средством оплаты долговых обязательств;
	 \CC Средством накопления и сбережения;
	 \choice Всеобщим эквивалентом, мерой стоимости всех остальных товаров.
	 \end{choices}
\question Международными счетными денежными единицами являются:
	 \begin{choices}
	 \choice СДР, ЭКЮ и евро;
	 \choice ЭКЮ и доллар;
	 \CC СДР и евро;
	 \choice СДР.
	 \end{choices}
\question В функции  используются только полноценные деньги
	 \begin{choices}
	 \choice Средства накопления;
	 \choice Средства платежа,;
	 \CC Меры стоимости;
	 \choice Средства обращения.
	 \end{choices}
\question Общим для современных теорий денег является признание:
	 \begin{choices}
	 \choice Возможности саморегулирования количества денег в обращении;
	 \CC Нетоварного происхождения денег;
	 \choice Роли денег в развитии экономики и необходимости регулирования массы денег в обращении;
	 \choice Возможности использования денег для оценки товаров и обмена на них.
	 \end{choices}
\question Количественная теория денег предполагает, что уровень цен определяется:
	 \begin{choices}
	 \choice Зависимостью от изменения золотовалютных запасов страны;
	 \CC Уровнем занятости населения и соответствующим ему совокупным фондом заработной платы;
	 \choice Стихийно, под воздействием спроса и предложения;
	 \choice Массой денег в обращении.
	 \end{choices}
\question Кейнсианская теория денег предполагает:
	 \begin{choices}
	 \CC Стихийность процессов ценообразования и полное невмешательство государства в процесс установления цен;
	 \choice Активное участие государства в регулировании денежной массы и возможность ее увеличения для стимулирования занятости и деловой активности;
	 \choice Установление зависимости количества денег от золотовалютных запасов страны;
	 \choice Недопущение роста денежной массы за счет проведения государством жесткой денежнокредитной политики.
	 \end{choices}
\question Монетаристская теория денег предполагает, что количество денег в обращении:
	 \begin{choices}
	 \choice Подвержено саморегулированию, а государство должно лишь сдерживать рост денежной массы;
	 \choice Зависит от размера золотовалютных запасов страны;
	 \choice Не устанавливается стихийно, а формируется в зависимости от экономических потребностей государства;
	 \CC Зависит от степени интернационализации национальной экономики.
	 \end{choices}
\question Безналичные деньги выпускаются в оборот:
	 \begin{choices}
	 \choice Центральным банком путем предоставления ссуд расчетно-кассовым центрам;
	 \choice Предприятиями, имеющими счета в коммерческих банках;
	 \choice Расчетно-кассовыми центрами путем предоставления ссуд предприятиям;
	 \CC Коммерческими банками путем предоставления ссуд их клиентам.
	 \end{choices}
\question В основе денежной эмиссии лежат операции
	 \begin{choices}
	 \choice Финансовые;
	 \choice Кредитные;
	 \CC Валютные;
	 \choice Фондовые.
	 \end{choices}
\question  мультипликатор предполагает, что мультипликация может осуществляться только в результате кредитования народного хозяйства
	 \begin{choices}
	 \CC Кредитный;
	 \choice Депозитный;
	 \choice Банковский;
	 \choice Финансовый.
	 \end{choices}
\question В условиях рыночной экономики размер эмиссии наличных денег определяется:
	 \begin{choices}
	 \choice Коммерческими банками на основе прогнозов денежных доходов и расходов населения;
	 \CC Центральным банком на основе прогнозов кассовых оборотов коммерческих банков;
	 \choice Местными органами власти на основе исследования социально-экономической ситуации в регионе;
	 \choice Коммерческими банками на основе прогнозов денежных доходов и расходов предприятий.
	 \end{choices}
\question Платежный оборот осуществляется:
	 \begin{choices}
	 \CC В наличной и безналичной формах;
	 \choice Только в наличной форме;
	 \choice Только в безналичной форме;
	 \choice В наличной форме в порядке, установленном Центральным банком РФ.
	 \end{choices}
\question В безналичном обороте функционируют деньги в качестве:
	 \begin{choices}
	 \choice Средства обращения;
	 \choice Средства накопления;
	 \choice Средства платежа;
	 \CC Меры стоимости.
	 \end{choices}
\question Банки и другие кредитные организации для проведения расчетов внутри страны открывают друг у друга  счета
	 \begin{choices}
	 \choice Корреспондентские;
	 \CC ЛОРО;
	 \choice НОСТРО;
	 \choice Бюджетные.
	 \end{choices}
\question Установление правил, сроков и стандартов осуществления безналичных расчетов, координация, регулирование и лицензирование организации расчетных систем возлагаются на:
	 \begin{choices}
	 \CC Регистрационную палату;
	 \choice Коммерческие банки;
	 \choice Клиринговые центры;
	 \choice Центральный банк РФ.
	 \end{choices}
\question Основную часть денежного оборота составляет  оборот
	 \begin{choices}
	 \choice Наличный;
	 \choice Платежный;
	 \CC Неплатежный;
	 \choice Сезонный.
	 \end{choices}
\question Безналичные расчеты проводятся:
	 \begin{choices}
	 \choice На основании расчетных документов установленной формы и с соблюдением соответствующего документооборота;
	 \choice На основании расписок плательщика и получателя средств;
	 \choice В порядке, оговоренном плательщиком и получателем де нежных средств;
	 \CC В порядке, который самостоятельно устанавливают коммерческие банки, плательщики и получатели средств.
	 \end{choices}
\question В безналичном денежном обороте, в сравнении с наличным оборотом, издержки обращения:
	 \begin{choices}
	 \choice Чрезвычайно велики;
	 \CC Отсутствуют совсем;
	 \choice Гораздо меньше;
	 \choice Гораздо больше.
	 \end{choices}
\question  форма расчетов представляет собой банковскую операцию, посредством которой банк эмитент по поручению и за счет клиента на основании расчетных документов осуществляет действия по получению от плательщика платежа
	 \begin{choices}
	 \choice Инкассовая;
	 \choice Аккредитивная;
	 \CC Чековая;
	 \choice Вексельная.
	 \end{choices}
\question Выплата с аккредитива наличными деньгами:
	 \begin{choices}
	 \choice Не допускается;
	 \CC Допускается;
	 \choice Допускается при разрешении банка эмитента;
	 \choice Допускается при разрешении территориального управления Банка России.
	 \end{choices}
\question  пластиковая карточка позволяет ее владельцу осуществлять расчеты только в пределах той суммы, которая находится на его отдельном карточном счете в банке
	 \begin{choices}
	 \choice Срочная;
	 \CC Дебетная;
	 \choice Кредитно-дебетная;
	 \choice Кредитная.
	 \end{choices}
\question Разрешение банка осуществить операцию с применением банковской пластиковой карточки, порождающее обязательство банка перечислить деньги по расчетному документу, составленному с ее помощью, называется:
	 \begin{choices}
	 \choice Эквайрингом;
	 \CC Инкассированием;
	 \choice Эмбоссированием;
	 \choice Авторизацией.
	 \end{choices}
\question Деятельность коммерческого банка по обслуживанию пластиковых карточек называется:
	 \begin{choices}
	 \CC Эмбоссированием;
	 \choice Авторизацией;
	 \choice Эквайрингом;
	 \choice Домициляцией.
	 \end{choices}
\question Предприятия могут получить наличные деньги:
	 \begin{choices}
	 \choice Только в Центральном банке РФ;
	 \choice В любом коммерческом банке;
	 \choice Только в обслуживающем их коммерческом банке;
	 \CC В любом расчетно-кассовом центре.
	 \end{choices}
\question Все предприятия и организации должны хранить наличные деньги, за исключением установленного лимита в:
	 \begin{choices}
	 \CC Казначействе;
	 \choice Центральном банке РФ;
	 \choice Коммерческих банках;
	 \choice Министерстве финансов РФ.
	 \end{choices}
\question В случае превышения лимита оборотной кассы коммерческих банков, деньги в сумме, превышающей лимит:
	 \begin{choices}
	 \choice Уничтожаются;
	 \CC Обмениваются на иностранную валюту;
	 \choice Обмениваются на государственные ценные бумаги;
	 \choice Сдаются в расчетно-кассовый центр.
	 \end{choices}
\question Деньги из оборотных касс расчетно-кассового центра направляются в:
	 \begin{choices}
	 \choice Кассы предприятий и организаций;
	 \choice Операционные кассы коммерческих банков;
	 \CC Региональный депозитарий;
	 \choice Региональное казначейство.
	 \end{choices}
\question чеки могут передаваться с помощью индоссамента
	 \begin{choices}
	 \choice Именные;
	 \choice Необращающиеся;
	 \CC Ордерные;
	 \choice Предъявительские.
	 \end{choices}
\question Эмитентами бумажных денег выступают:
	 \begin{choices}
	 \CC Коммерческие банки;
	 \choice Казначейство и эмиссионный банк;
	 \choice Казначейство и предприятия;
	 \choice Предприятия.
	 \end{choices}
\question деньги  это знаки стоимости, обычно не разменные на металл, имеющие принудительный курс и выпускаемые государством для покрытия своих расходов
	 \begin{choices}
	 \CC Бумажные;
	 \choice Кредитные;
	 \choice Металлические;
	 \choice Товарные.
	 \end{choices}
\question Вексельное обращение ограничено:
	 \begin{choices}
	 \CC Только суммами сделок;
	 \choice Только сроками сделок;
	 \choice Сроками и суммами сделок;
	 \choice Невозможностью передачи векселей.
	 \end{choices}
\question Классические банкноты:
	 \begin{choices}
	 \choice Не имеют никаких гарантий;
	 \choice Не размениваются на золото и имеют только коммерческую гарантию;
	 \CC Имеют двойную гарантию: золотую и коммерческую;
	 \choice Имеют только золотую гарантию.
	 \end{choices}
\question Золотослитковый стандарт предусматривал:
	 \begin{choices}
	 \choice Одновременное обращение золотых и серебряных монет;
	 \choice Размен банкнот на золото большого веса;
	 \CC Размен банкнот на бумажные и кредитные деньги;
	 \choice Размен банкнот на девизы  иностранную валюту, беспрепятственно разменную на любое количество золота.
	 \end{choices}
\question В рыночной модели экономики центральный банк обычно предоставляет денежные средства правительству:
	 \begin{choices}
	 \choice В порядке кредитования под определенное обеспечение;
	 \choice На безвозмездной основе;
	 \CC Только при наличии гарантий международных валютнофинансовых организаций;
	 \choice На кредитной основе без всякого обеспечения.
	 \end{choices}
\question Фидуциарная банкнотная эмиссия  это эмиссия банкнот:
	 \begin{choices}
	 \choice Необеспеченная металлическим запасом эмиссионного банка;
	 \choice Обеспеченная металлическим запасом эмиссионного банка;
	 \CC Обеспеченная всеми активами коммерческих банков;
	 \choice Независящая от размера золотого запаса страны.
	 \end{choices}



\end{multicols}
\end{questions}

\end{document}
