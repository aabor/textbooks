% !TeX program = lualatex -synctex=1 -interaction=nonstopmode %.tex

\documentclass[_DKB_p1_Money.tex]{subfiles}
\begin{document}

\setbeamercovered{invisible}
\subsection{Правовые основы}
\begin{frame}{Правовые основы деятельности Центрального Банка}
Статус, цели деятельности, функции и полномочия Центрального банка Российской Федерации (Банка России) определяются:
\begin{itemize}
\item
Конституцией Российской Федерации;

\item
Федеральным законом "О Центральном банке Российской Федерации (Банке России)"; 

\item
другими федеральными законами.
\end{itemize}
\end{frame}

\subsection{Цели деятельности }
\begin{frame}{Цели деятельности Банка России }
\begin{itemize}[<+->]
\item
защита и обеспечение устойчивости рубля;
\item
развитие и укрепление банковской системы Российской Федерации;
\item
обеспечение стабильности и развитие национальной платежной системы;
\item
развитие финансового рынка Российской Федерации;
\item
обеспечение стабильности финансового рынка Российской Федерации.
\item
Получение прибыли не является целью деятельности Банка России.
\end{itemize}
\end{frame}

\subsection{Функции }
\begin{frame}[ allowframebreaks]{Функции Банка России}
\begin{itemize}
\item
во взаимодействии с Правительством Российской Федерации разрабатывает и проводит единую государственную денежно-кредитную политику; 
\item
во взаимодействии с Правительством Российской Федерации разрабатывает и проводит политику развития и обеспечения стабильности функционирования финансового рынка Российской Федерации;
\item
монопольно осуществляет эмиссию наличных денег и организует наличное денежное обращение; утверждает графическое обозначение рубля в виде знака;
\item
является кредитором последней инстанции для кредитных организаций, организует систему их рефинансирования;

\pagebreak
\item
устанавливает правила осуществления расчетов в Российской Федерации;
\item
осуществляет надзор и наблюдение в национальной платежной системе;
\item
устанавливает правила проведения банковских операций;
\item
осуществляет обслуживание счетов бюджетов всех уровней бюджетной системы Российской Федерации;

\pagebreak
\item
осуществляет эффективное управление золотовалютными резервами Банка России;

\item
принимает решение о государственной регистрации кредитных организаций, негосударственных пенсионных фондов (выдача и отзыв лицензий);
\item
осуществляет надзор за деятельностью кредитных организаций и банковских групп,  некредитных финансовых организаций в соответствии с федеральными законами;

\pagebreak
\item
осуществляет регистрацию выпусков эмиссионных ценных бумаг и проспектов ценных бумаг, регистрацию отчетов об итогах выпусков эмиссионных ценных бумаг;
\item
осуществляет контроль и надзор за соблюдением эмитентами требований законодательства Российской Федерации об акционерных обществах и ценных бумагах и в сфере корпоративных отношений в акционерных обществах;

\pagebreak
\item
осуществляет самостоятельно или по поручению Правительства Российской Федерации все виды банковских операций и иных сделок, необходимых для выполнения функций Банка России;
\item
организует и осуществляет валютное регулирование и валютный контроль в соответствии с законодательством Российской Федерации;

\pagebreak
\item
определяет порядок осуществления расчетов с международными организациями, иностранными государствами, а также с юридическими и физическими лицами;

\pagebreak
\item
утверждает отраслевые стандарты и планы счетов бухгалтерского учета для кредитных организаций, Банка России и некредитных финансовых организаций;
\item
устанавливает и публикует официальные курсы иностранных валют по отношению к рублю;

\pagebreak
\item
принимает участие в разработке прогноза платежного баланса Российской Федерации и организует составление платежного баланса Российской Федерации;

\pagebreak
\item
проводит анализ и прогнозирование состояния экономики Российской Федерации, публикует соответствующие материалы и статистические данные;
\item
осуществляет выплаты Банка России по вкладам физических лиц в признанных банкротами банках, не участвующих в системе обязательного страхования вкладов физических лиц в банках Российской Федерации, в случаях и порядке, которые предусмотрены федеральным законом;

\pagebreak
\item
является депозитарием средств Международного валютного фонда в валюте Российской Федерации, осуществляет операции и сделки, предусмотренные статьями Соглашения Международного валютного фонда и договорами с Международным валютным фондом;

\pagebreak
\item
осуществляет контроль за соблюдением требований законодательства Российской Федерации о противодействии неправомерному использованию инсайдерской информации и манипулированию рынком;

\pagebreak
\item
осуществляет защиту прав и законных интересов акционеров и инвесторов на финансовых рынках, страхователей, застрахованных лиц и выгодоприобретателей;

\pagebreak
\item
организовывает оказание услуг по передаче электронных сообщений по финансовым операциям (далее - финансовые сообщения);
\item
осуществляет иные функции в соответствии с федеральными законами.
\end{itemize}
\end{frame}

\subsection{Денежно-кредитная политика}
\begin{frame}[ allowframebreaks]{Инструменты денежно-кредитной политики}
\begin{itemize}
\item
процентные ставки по операциям Банка России;
\item
обязательные резервные требования;
\item
операции на открытом рынке;
\item
рефинансирование кредитных организаций;

\pagebreak
\item
валютные интервенции;
\item
установление ориентиров роста денежной массы;
\item
прямые количественные ограничения;
\item
эмиссия облигаций от своего имени;
\item
другие инструменты, определенные Банком России.

\end{itemize}
\end{frame}

\begin{frame}[shrink=25]
% Table generated by Excel2LaTeX from sheet 'Лист1'
\begin{table}[htbp]
\caption{Текущие параметры денежно-кредитной политики ЦБ РФ}
  \centering
\begin{tabularx}{\linewidth}[b]{@{}>{\raggedright\arraybackslash}Xr@{}}    \toprule
    Параметр & Значение \\
    \midrule
    Ключевая ставка & 10,50\% \\
    Инфляция, июль 2016 года, \%  & 7,20\% \\
    Цель по инфляции, 2017 год, \%  & 4\% \\
    MIACR на 25.08.2016  & 10,25\% \\
    RUONIA на 25.08.2016 & 10,28\% \\
    MosPrime  Rate на 26.08.2016 & 10,59\% \\
    ROISfix  на 26.08.2016 & 10,26\% \\
    Нормативы обязательных резервов & 5-7\% \\
    Международные резервы РФ, млрд. & \$398,2\\
    Доходы по вкладам свыше года & \\
    в рубл.  & 10,476\%\\
    в долл.  & 2,853\%\\
    в Евро  & 1,650\%\\
    \bottomrule
    \end{tabularx}%
  \label{tab:addlabel}%

\raggedright
Источник: Центральный банк РФ, \url{http://www.cbr.ru/}
\end{table}%
\end{frame}

\begin{frame}{RUONIA (Ruble OverNight Index Average) }
Индикативная взвешенная рублевая депозитная ставка «овернайт» российского межбанковского рынка RUONIA (Ruble OverNight Index Average) отражает оценку стоимости необеспеченного заимствования банков с минимальным кредитным риском.
\end{frame}

\begin{frame}{MosPrime Rate (Moscow Prime Offered Rate)}
 — независимая индикативная ставка предоставления рублёвых кредитов (депозитов) на московском денежном рынке. 
 
Формируется Национальной финансовой ассоциацией на основе ставок предоставления рублёвых кредитов (депозитов), объявляемых банками-ведущими участниками российского денежного рынка первоклассным финансовым организациям со сроками «overnight», 1 неделя, 2 недели, 1, 2, 3 и 6 месяцев. 
\end{frame}

\begin{frame}[shrink=15]
\begin{figure}
\center
\includegraphics[scale=0.5]{img/interest_rates_Russia}
\caption{Процентные ставки на российском рынке}
\end{figure}
\end{frame}

\begin{frame}[shrink=15]
\begin{figure}
\center
\includegraphics[scale=0.5]{img/price_indexes_Russia}
\caption{Индексы цен на российском рынке}
\end{figure}
\end{frame}


\begin{frame}[shrink=15]
\begin{figure}
\center
\includegraphics[scale=0.5]{img/total_reserves_Russia}
\caption{Международные резервы России,\\ за исключением золота с 1993 по 2016 гг.}
\end{figure}
\end{frame}

\begin{frame}[shrink=15]
\begin{figure}
\center
\includegraphics[scale=0.5]{img/unemployment_Russia}
\caption{Зарегистрированный уровень безработицы в России\\ с 1991 по 2016 гг.}
\end{figure}
\end{frame}

\begin{frame}[shrink=15]
\begin{figure}
\center
\includegraphics[scale=0.5]{img/salaries_Russia}
\caption{Заработная плата в России с 1996 по 2016 гг.\\ (данные без снятой сезонности)}
\end{figure}
\end{frame}

\begin{frame}[shrink=15]
\begin{figure}
\center
\includegraphics[scale=0.5]{img/production_Russia}
\caption{Индекс промышленного производства в России\\ с 1993 по 2016 гг.}
\end{figure}
\end{frame}

\begin{frame}[shrink=15]
\begin{figure}
\center
\includegraphics[scale=0.5]{img/exp_imp_ratio_Russia}
\caption{Отношение экспорта к импорту в России с 1991 по 2016 гг.}
\end{figure}
\end{frame}

\begin{frame}[shrink=15]
\begin{figure}
\center
\includegraphics[scale=0.5]{img/gdp_Russia}
\caption{Валовой внутрненний продукт России с 2003 по 2016 гг.}
\end{figure}
\end{frame}

\subsection{Операции}
\begin{frame}[ allowframebreaks]{Операции Банка России}
\begin{itemize}
\item
предоставлять кредиты под обеспечение ценными бумагами и другими активами; предоставлять кредиты без обеспечения на срок не более одного года российским кредитным организациям, имеющим рейтинг не ниже установленного уровня;

\pagebreak
\item
покупать и продавать ценные бумаги на открытом рынке, а также продавать ценные бумаги, выступающие обеспечением кредитов Банка России;
\item
покупать и продавать облигации, эмитированные Банком России, и депозитные сертификаты;

\pagebreak
\item
покупать и продавать иностранную валюту, а также платежные документы и обязательства, номинированные в иностранной валюте, выставленные российскими и иностранными кредитными организациями;

\pagebreak
\item
покупать, хранить, продавать драгоценные металлы и иные виды валютных ценностей;
\item
проводить расчетные, кассовые и депозитные операции, принимать на хранение и в управление ценные бумаги и другие активы;

\pagebreak
\item
выдавать поручительства и банковские гарантии;
\item
осуществлять операции с финансовыми инструментами, используемыми для управления финансовыми рисками;

\pagebreak
\item
открывать счета в российских и иностранных кредитных организациях на территории Российской Федерации и территориях иностранных государств;
\item
выставлять чеки и векселя в любой валюте;

\pagebreak
\item
осуществлять другие банковские операции и сделки от своего имени в соответствии с обычаями делового оборота, принятыми в международной банковской практике.
\end{itemize}
\end{frame}

\begin{frame}{Обеспечением кредитов Банка России могут выступать}
\begin{itemize}
\item
золото и другие драгоценные металлы в стандартных и мерных слитках;
\item
иностранная валюта;
\item
векселя, номинированные в российской или иностранной валюте;
\item
государственные ценные бумаги.
\end{itemize}
\end{frame}
\begin{frame}{Финансовая стабильность}{Банк России вправе...}
1) компенсировать кредитным организациям часть убытков (расходов), возникших у них по сделкам с кредитными организациями, у которых была отозвана лицензия;

2) компенсировать часть убытков (расходов) центральных контрагентов (кредитными организациями, некредитными финансовыми организациями), возникших у них по сделкам с участниками клиринга, у которых была отозвана (аннулирована) лицензия.
\end{frame}

\subsection{Организация ЦБ РФ}
\begin{frame}{Организация Банка России}
\begin{itemize}
\item
Центральный аппарат, 
\item
территориальные учреждения, 
\item
расчетно-кассовые центры, 
\item
вычислительные центры, 
\item
полевые учреждения, 
\item
образовательные и другие организации, в том числе подразделения безопасности,
\item
Российское объединение инкассации.
\end{itemize}
\end{frame}

\setbeamercovered{transparent}
\end{document}