% !TeX program = lualatex -synctex=1 -interaction=nonstopmode --shell-escape %.tex

\documentclass[_DKB_p1_Money.tex]{subfiles}
\begin{document}

\setbeamercovered{invisible}

\subsection{Издержки обращения}
\begin{frame}{Издержки обращения}
\begin{block}{Издержки обращения}
\quad
- это расходы, связанные с приобретением товаров и услуг, подразделяются на издержки ожидания и трансакционные издержки
\end{block}
\begin{itemize}[<+->]
\item
Издержки ожидания — это расходы, которые несет индивид по мере увеличения времени до получения необходимого товара
\item
Трансакционпые издержки — это расходы, которые индивид несет при совершении торговой сделки
\end{itemize}
\end{frame}

\begin{frame}{}
\begin{figure}
	\centering
	\begin{overprint}
		\forloop{slideno}{1}{\value{slideno} < 4}{%
			\only<\value{slideno}>{
				\includegraphics[page=\value{slideno},
				scale=.7
				% trim={<left> <lower> <right> <upper>}				
				,trim={1cm 4cm 4cm 0cm},clip]
				{tikz/distribution_cost}}}
	\end{overprint}
	\vspace*{-1.5em}
	\caption{\captionf{Издержки обращения}}
\end{figure}
\vspace*{-1.5em}
\begin{itemize}[<+->]
	\item
	$C_T$ - издержки обращения;
	\item
	$C_W$ - издержки ожидания;
	\item
	$C_E$ - общие издержки.
\end{itemize}
\end{frame}

\begin{frame}{}
\begin{figure}
	\centering
	\begin{overprint}
		\forloop{slideno}{1}{\value{slideno} < 2}{%
			\only<\value{slideno}>{
				\includegraphics[page=\value{slideno},
				scale=.8
				% trim={<left> <lower> <right> <upper>}				
				,trim={1cm 4cm 4cm 0cm},clip]
				{tikz/minimum_cost_exchange}}}
	\end{overprint}
	\vspace*{-1.5em}
	\caption{\captionf{Обмен с минимальными издержками}}
\end{figure}
\vspace*{-1.5em}
Минимум издержек ($C^*_E$) достигается, если сделка была совершена за время $T^*$. 
\end{frame}


\begin{frame}{}
\begin{figure}
	\centering
	\begin{overprint}
		\forloop{slideno}{1}{\value{slideno} < 6}{%
			\only<\value{slideno}>{
				\includegraphics[page=\value{slideno},
				scale=.6
				% trim={<left> <lower> <right> <upper>}				
				,trim={0cm 4cm 0cm 0cm},clip]
				{tikz/barter_to_money_evolution}}}
	\end{overprint}
	\vspace*{-2em}
	\caption{\captionf{Переход от чистого бартера к неразменным деньгам}}
\end{figure}
\vspace*{-2em}
\begin{itemize}
	\small
	\item Издержки ожидания не связаны с типом системы торговли, используемой в данной экономике
	\item Трансакционные издержки существенно меняются в зависимости от используемой системы торговли.
\end{itemize}
\end{frame}


\subsection{Товарные деньги}
\begin{frame}{Экономика с системой товарных денег}
\begin{itemize}
\item
Цена золота ($P_G$) определяется количеством единиц товаров и услуг, которые необходимо отдать за одну весовую единицу золота $G_N$.

\item
Золото имеет два различных применения: оно может использоваться как деньги или в других целях. 
\end{itemize}
\end{frame}

\begin{frame}
\begin{figure}
\center
	\begin{subfigure}[t]{4cm}
		\centering
		%trim={<left> <lower> <right> <upper>}				
		\includegraphics[trim={0.5cm 4.5cm 6cm 0cm},clip, scale=0.8]{tikz/gold_for_money}
		\caption{\captionf{для денежных\\ целей $G_M^d$.}}\label{fig:GD_money}	
	\end{subfigure}
	\quad
	\begin{subfigure}[t]{4cm}
		\centering
		%trim={<left> <lower> <right> <upper>}				
		\includegraphics[trim={0.5cm 4.5cm 6cm 0cm},clip, scale=0.8]{tikz/gold_for_industry}
		\caption{\captionf{для других\\ целей $G_N^d$.}}\label{fig:GD_other}
	\end{subfigure}
\caption{\captionf{Кривые спроса на золото.}}\label{fig:gold_demand}
\end{figure}
\end{frame}

\begin{frame}{}
\begin{figure}
	\centering
	\begin{overprint}
		\forloop{slideno}{1}{\value{slideno} < 2}{%
			\only<\value{slideno}>{
				\includegraphics[page=\value{slideno},
				scale=.85
				% trim={<left> <lower> <right> <upper>}				
				,trim={0cm 4cm 3cm 0cm},clip]
				{tikz/gold_total_demand}}}
	\end{overprint}
	\vspace*{-2.5em}
	\caption{\captionf{Кривая совокупного спроса на золото}}
\end{figure}
	\vspace*{-1.5em}
\footnotesize{Кривая совокупного спроса на золото $G^d$ строится суммированием по горизонтали кривых спроса на золото для денежных $G_M^d$ и других целей $G_N^d$.}
\end{frame}


\begin{frame}{}
\begin{figure}
	\centering
	\begin{overprint}
		\forloop{slideno}{1}{\value{slideno} < 2}{%
			\only<\value{slideno}>{
				\includegraphics[page=\value{slideno},
				scale=.75
				% trim={<left> <lower> <right> <upper>}				
				,trim={0cm 4cm 3cm 0cm},clip]
				{tikz/gold_equilibrium_price}}}
	\end{overprint}
	\vspace*{-2.5em}
	\caption{\captionf{Равновесная цена золота}}
\end{figure}
\vspace*{-1.5em}
\footnotesize{
	Равновесие на рынке золота возникает, когда совокупный спрос на золото равен совокупному предложению золота. 
	
	Это происходит в точке ($G_0,P_G^0$) пересечения графика спроса на золото $G^d$ с  графиком предложения золота $G^s$.}
\end{frame}


\begin{frame}{}
\begin{figure}
	\centering
	\begin{overprint}
		\forloop{slideno}{1}{\value{slideno} < 3}{%
			\only<\value{slideno}>{
				\includegraphics[page=\value{slideno},
				scale=1
				% trim={<left> <lower> <right> <upper>}				
				,trim={0cm 6cm 4cm 0cm},clip]
				{tikz/new_gold_supply}}}
	\end{overprint}
	\vspace*{-2.5em}
	\caption{\captionf{Открытие нового золотого месторождения}}
\end{figure}
\vspace*{-1.5em}
\footnotesize{
	Равновесная цена золота уменьшается $P_G^1$, а уровень цен на товары и услуги (обратная величина) – увеличивается, следовательно, возникает инфляция.}
\end{frame}


\begin{frame}{}
\begin{figure}
	\centering
	\begin{overprint}
		\forloop{slideno}{1}{\value{slideno} < 3}{%
			\only<\value{slideno}>{
				\includegraphics[page=\value{slideno},
				scale=.8
				% trim={<left> <lower> <right> <upper>}				
				,trim={0cm 5cm 4cm 0cm},clip]
				{tikz/new_gold_demand}}}
	\end{overprint}
	\vspace*{-2.5em}
	\caption{\captionf{Влияние роста спроса на золото на цену золота и общий уровень цен товаров и услуг}}
\end{figure}
\vspace*{-1.5em}
\footnotesize{
	Равновесная цена золота вырастет до величины $P_G^1$, уровень цен на товары и услуги снизится, т. е. произойдет дефляция или рост покупательной способности золотых денег.}
\end{frame}


\subsection{Золотой стандарт}
\begin{frame}{Система товарного (золотого) стандарта}
\begin{itemize}
\item
Международный золотой стандарт эффективно функционировал с начала XIX в. до первой мировой войны, а в США использовался до 1933 г., всего на два года дольше, чем Великобритания. 

\item
Введение золотого стандарта означает, что в основу национальной денежной системы положено определенное количество золота, называемого золотым резервом.

\item
При полном золотом обеспечении обращающихся денег величина золотого резерва образует национальную денежную базу.
\end{itemize}
\end{frame}

\begin{frame}{Денежная масса в обращении в условиях золотого стандарта}

1) реальное золото в золотом резерве, обычно в форме золотых монет или тщательно взвешенных и промаркированных слитков золота, которые могут служить средством обращения и, следовательно, являются деньгами. 

2) обеспеченные золотом денежные знаки, используемые в качестве средства обращения (обеспеченные золотом монеты, денежные знаки, выпущенные частными банками, или денежные знаки, выпущенные правительством). 
\end{frame}

\begin{frame}{Товарный стандарт}{Снижение трансакционных издержек}
1. Денежные знаки, выпущенные частными банками или правительством, гораздо легче использовать как средство платежа, чем тяжелый металл вроде золота.

2. Правительство (или центральный банк) может устанавливать соотношение между количеством банкнот, которое выпускает банк, и золотым резервом, т. е. золотое покрытие. Это означает, что в условиях золотого стандарта правительство может вести определенную политику регулирования количества денег в обращении. 
\end{frame}


\begin{frame}{}
\begin{figure}
	\centering
	\begin{overprint}
		\forloop{slideno}{1}{\value{slideno} < 2}{%
			\only<\value{slideno}>{
				\includegraphics[page=\value{slideno},
				scale=.6
				% trim={<left> <lower> <right> <upper>}				
				,trim={0cm 4cm 3cm 0cm},clip]
				{tikz/money_monopoly}}}
	\end{overprint}
	\vspace*{-2.5em}
	\caption{\captionf{Монопольная эмиссия денег}}
\end{figure}
\vspace*{-1.5em}
\footnotesize{
	$GC^d$ — кривая спроса на золотые монеты.
	
	Будучи единственным эмитентом золотых монет, правительство может устанавливать любую цену $P_{GC}^*$ на них, определяемую как число товарных единиц в расчете на одну золотую монету.
	
	Средние общие издержки \textit{АТС} выпуска золотых монет, постоянны и равны предельным издержкам \textit{МС}.}
\end{frame}


\begin{frame}{}
\begin{figure}
	\centering
	\begin{overprint}
		\forloop{slideno}{1}{\value{slideno} < 3}{%
			\only<\value{slideno}>{
				\includegraphics[page=\value{slideno},
				scale=.8
				% trim={<left> <lower> <right> <upper>}				
				,trim={0cm 4cm 2.5cm 0cm},clip]
				{tikz/money_falsification}}}
	\end{overprint}
	\vspace*{-2.5em}
	\caption{\captionf{Фальсификация монет и инфляция}}
\end{figure}
\vspace*{-1.5em}
\footnotesize{
	\onslide<2->{
		Снижая предельные и средние издержки выпуска денег, фальсификация монет делает для правительства выгодным выпуск их большого количества.
}}
\end{frame}


\begin{frame}{}
\begin{figure}
	\centering
	\begin{overprint}
		\forloop{slideno}{1}{\value{slideno} < 2}{%
			\only<\value{slideno}>{
				\includegraphics[page=\value{slideno},
				scale=.7
				% trim={<left> <lower> <right> <upper>}				
				,trim={0cm 4cm 2.5cm 0cm},clip]
				{tikz/money_optimum_supply}}}
	\end{overprint}
	\vspace*{-2.5em}
	\caption{\captionf{Общественно оптимальный выпуск денег}}
\end{figure}
\vspace*{-1.5em}
\footnotesize{
	Экономика достигает эффективности, если цена золотых монет равна предельным издержкам их выпуска.
	
	1. Предоставить частным фирмам право выпуска золотых монет в условиях свободной конкуренции.
	
	2. Правительство откажется от монопольной прибыли от эмиссии денег.
}
\end{frame}

\setbeamercovered{transparent}
\end{document}
