% !TeX program = lualatex -synctex=1 -interaction=nonstopmode --shell-escape %.tex

\documentclass[_DKB_p1_Money.tex]{subfiles}
\begin{document}

\setbeamercovered{transparent}

\subsection{Для чего нужно изучать проблемы денег?}
\begin{frame}{Для чего нужно изучать проблемы денег?}
\begin{itemize}[<+->]
\item
Изменился способ, которым люди совершают передачу денег;
\item
Важность вопросов, связанных с денежным обращением и банковской системой;
\item
Изучение проблем денег - это увлекательное занятие;
\item
Личные мотивы.
\end{itemize}
\textbf{Деньги могут влиять на состояние экономики, воздействуя на уровень экономической активности: ВВП и уровень цен.}
\end{frame}

\subsection{Сущность денег}
\begin{frame}{Сущность денег}
\begin{block}{Деньги}
\quad
- это любой предмет или поддающаяся контролю запись, которые обычно принимаются для оплаты товаров и услуг, а также возмещения долгов в отдельно взятой стране или социально-экономических обстоятельствах, либо нечто иное, что можно легко конвертировать в такую форму.
\end{block}
\end{frame}
\begin{frame}{Использование слова денег, как синонима слов богатство и доход}
\begin{block}{Богатство}
\quad
— совокупный набор элементов собственности, представляющих собой накопления. Богатство включает в себя не только деньги, но и другие активы, такие, как, например, облигации, акции, предметы искусства, земля, мебель, машины и дома.
\end{block}
\begin{block}{Доход }
\quad
— это поток поступлений за определенный промежуток времени. Деньги же, напротив, — это запас, т.е. определенное количество в определенный момент времени.
\end{block}
\end{frame}

\subsection{Роль денег}
\begin{frame}{Роль денег}
\begin{block}
\quad
Аристотель говорил: ``Все... должно быть оценено в деньгах, потому что это позволяет людям всегда обмениваться услугами и таким образом делает возможным существование общества''.
\end{block}
\begin{block}{Актив называется ликвидным}
\quad
если его можно обменять на товар или услугу при низких трансакционных издержках и относительной определенности номинальной стоимости этого актива.
\end{block}
\end{frame}

\subsection{Функции денег}

\begin{frame}{}
\begin{figure}
	\centering
	\begin{overprint}
		\forloop{slideno}{1}{\value{slideno} < 6}{%
			\only<\value{slideno}>{
				\includegraphics[page=\value{slideno},
				scale=.8
				% trim={<left> <lower> <right> <upper>}				
				,trim={0.5cm 2.8cm 4cm 0cm},clip]
				{tikz/money_functions}}}
	\end{overprint}
	\vspace*{-1.5em}
	\caption{\captionf{Функции денег}}
\end{figure}
\end{frame}


\begin{frame}{Средство обращения }
\begin{block}{Средство обращения }
\quad
- это посредник, используемый в торговле, во избежание недостатков системы чистого бартера. Бумажные деньги (fiat currencies, анг.), обычно принимаются в качестве средства обращения. 
\end{block}
\end{frame}

\begin{frame}[shrink=10]{Единица счета (мера стоимости) }
Экономисты используют деньги как единицу счета для измерения стоимости данных товаров и услуг относительно других товаров и услуг.

На рынке нет других способов установления, определения или измерения стоимости различных товаров и услуг. Определение цены - это существенное условие справедливой торговли, эффективного размещения ресурсов, экономического роста, богатства и правосудия.

Использование денег как относительно стабильной единицы измерения стоимости улучшает эффективность рынка.

Понятия единицы счета и меры стоимости часто используются как синонимы.

Как мера стоимости деньги используются в бухгалтерском учете. 
\end{frame}

\begin{frame}{Средство сохранения стоимости   }
\begin{block}{Средство сохранения стоимости}
\quad
- это функция актива, который может быть сохранен, возвращен и обменен в дальнейшем и может быть предсказуемо полезным после возврата.
\end{block}
Альтернативные формы сохранения стоимости:

недвижимость, драгоценные металлы, драгоценные камни, антикварные коллекции, кокаин, живой скот, акции биржевых компаний, предоплаченные карты, экономика дара (стоимость сохраняется в виде общественной репутации)
\end{frame}

\begin{frame}{Средство платежа }
Эта функция предполагает одновременное использование денег как средства обращения и как единицы счета. Мы обычно устанавливаем размер задолженности в единицах счета, а выплачиваем эту задолженность с помощью денег как средства обращения. 
\begin{block}{Деньги как средство платежа}
\quad
- это законное платежное средство (legal tender, анг.), признаваемое законодательной системой действительным для погашения финансовых обязательств.
\end{block}
\end{frame}

\subsection{Желательные свойства денег}
\begin{frame}{}
\begin{figure}
	\centering
	\begin{overprint}
		\forloop{slideno}{1}{\value{slideno} < 11}{%
			\only<\value{slideno}>{
				\includegraphics[page=\value{slideno},
				scale=.65
				% trim={<left> <lower> <right> <upper>}				
				,trim={0.5cm 0cm 2cm 0cm},clip]
				{tikz/money_desirable_characteristics}}}
	\end{overprint}
	\vspace*{-1.5em}
	\caption{\captionf{Желательные свойства денег}}
\end{figure}
\end{frame}

\setbeamercovered{transparent}
\end{document}