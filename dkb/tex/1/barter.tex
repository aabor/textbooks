% !TeX program = lualatex -synctex=1 -interaction=nonstopmode --shell-escape %.tex

\documentclass[_DKB_p1_Money.tex]{subfiles}
\begin{document}

\setbeamercovered{invisible}

\subsection{Чистый бартер}
\begin{frame}{Система чистого бартера}
\begin{block}{Бартер}
\quad
 – это прямой обмен товаров и услуг на другие товары и услуги. При системе чистого бартера индивид, которому нужен определенный товар (услуга), должен найти другого индивида, который готов предоставить такой товар в обмен на товар, предлагаемый первым индивидом Таким образом, должно существовать двойное совпадение потребностей, т.е. два индивида должны одновременно иметь соответствующие товары (услуги) и хотеть их продать.
\end{block}
\end{frame}

\begin{frame}[shrink=15]{Недостатки системы чистого бартера}
\begin{itemize}
\item Отсутствие способа сохранения общей покупательной способности.

\item Отсутствие единого масштаба измерения стоимости

Определим число цен, которые будут существовать в экономике, в которой производится и продается 1000 товаров, но отсутствуют деньги и денежная единица счета.

Решение.
\begin{align*}
\onslide<2->{\text{Число курсов (цен)}&=}\onslide<3->{\frac{N\cdot(N-1)}{2}}\\
\onslide<4->{&=\frac{1000\cdot 999}{2} =}\onslide<5->{ 499 500}
\end{align*}

\item Отсутствие определенной единицы платежа для использования в контрактах, предусматривающих платежи в будущем.
\end{itemize}

\end{frame}

\setbeamercovered{transparent}

\subsection{Организованный бартер}
\begin{frame}[shrink=10]{Система организованного бартера}
\begin{itemize}[<+->]
\item
Введены специальные места торговли — обменные пункты, на которых представлены конкретные товары и услуги, предлагаемые к обмену. 

\item
Учреждение обменных пунктов позволяет потенциальным покупателям заранее знать, где можно найти продавцов конкретных товаров. 

\item
По прежнему остается проблема двойного совпадения потребностей: индивид знает, что именно он найдет на определенном обменном пункте, но он не всегда знает, какой товар (услугу) продавец захочет получить в обмен.

\item
Установление одного товара, который бы широко принимался на всех обменных пунктах $\Rightarrow$ переход от бартера к денежному обращению.
\end{itemize}
\end{frame}
\subsection{Товарные деньги}
\begin{frame}[shrink=20]{Товарные деньги}
% Table generated by Excel2LaTeX from sheet 'Лист1'
\begin{table}[htbp]
  \centering
  \caption{\captionf{Виды товарных денег}}
\begin{tabularx}{\linewidth}[b]{@{}>{\raggedright\arraybackslash}XXX@{}}
	\toprule
	Железо              & Хохолок красного дятла         & Золото        \\
	Медь                & Птичьи перья                   & Серебро       \\
	Латунь              & Стекло                         & Ножи          \\
	Вино                & Полированные шарики (ожерелья) & Горшки        \\
	Зерно               & Ром                            & Лодки         \\
	Соль                & Меласса (черная патока)        & Смола         \\
	Лошади              & Табак                          & Рис           \\
	Овцы                & С/х орудия                     & Коровы        \\
	Козы                & Круглые камни с отверстиями    & Рабы          \\
	Черепаховые панцири & Крупные кристаллы соли         & Бумага        \\
	Зубы морской свиньи & Раковины улиток                & Сигареты      \\
	Китовый ус          & Игральные карты                & Кабаньи клыки \\
	Кожа                &                                &               \\ \bottomrule
\end{tabularx}%
  \label{tab:addlabel}%
\end{table}%
\end{frame}

\subsection{Товарный стандарт}
\begin{frame}[allowframebreaks]{Товарный стандарт}
\begin{itemize}
\item
Первые товарные деньги (например, шерсть, лодки, овцы, зерно) имели одинаковую меновую и потребительную стоимость.

\item
С расширением употребления денег были придуманы монеты, содержание металла в которых (используемого, например, в ювелирном деле) равнялось их меновой стоимости как денег. 

\item
Номинал монеты, указанный на лицевой стороне, равнялся рыночной стоимости металла, содержащегося в монете. 

\item
Граждане могли на законном основании переплавлять монеты для использования их в качестве денег.

\pagebreak
\item
В условиях товарного стандарта используются денежные знаки, стоимость которых частично или полностью основана на стоимости какого-то реального товара, например золота или серебра. 

\item
Полноценные деньги, (золотые монеты; банкноты, разменные на фиксированное количество золота), являются формой товарного стандарта, в которую трансформируется реальный товар (золото). 

\end{itemize}
\end{frame}
\subsection{Бумажные деньги}
\begin{frame}{Бумажные деньги (fiat money, анг.)}
Бумажные деньги подразделяются на два типа: 

1) выпускаемые правительством и центральным банком;

2) выпускаемые депозитными учреждениями.
\begin{block}{Казначейские билеты}
\quad
- это бумажные деньги, выпускаемые правительством (казначейством, ведающим кассовым исполнением бюджета).
\end{block}
\end{frame}

\begin{frame}
\begin{block}{Бумажно-денежный (фидуциарный) стандарт}
\quad
стоимость платежа покоится на вере людей в то, что они могут обменять бумажные деньги на товары и услуги.
\end{block}
\begin{itemize}
\item
Для того чтобы стоимость денег была предсказуемой, соотношение между спросом и предложением денег не должно меняться часто, резко или значительно.
\item
Ожидания в отношении уменьшения стоимости денег изменяют виды и объемы финансовых активов, которые люди склонны сберегать, в том числе денежные суммы, которые они предпочитают хранить.

\end{itemize}
\end{frame}

\setbeamercovered{transparent}
\end{document}