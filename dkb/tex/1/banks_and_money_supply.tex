% !TeX program = lualatex -synctex=1 -interaction=nonstopmode --shell-escape %.tex

\documentclass[_DKB_p1_Money.tex]{subfiles}

\begin{document}

\setbeamercovered{invisible}

\subsection{Резервы банка}
\begin{frame}[shrink=5]{Банковские резервы и депозиты}
\begin{itemize}[<+->]
\item
В соответствии с законодательством банки должны хранить обязательные резервы, т.е. резервы на специальном счете в Банке России, которые составляют определенный процент от общего объема депозитов. 

\item
Все резервы сверх того экономисты называют избыточными резервами.

\item
Отдельно взятый банк может кредитовать своих клиентов только исходя из избыточных резервов.
\end{itemize}
\begin{block}{Ссуды }
\quad
— это обещания клиентов выплатить определенную сумму в будущем, т. е. это их долговые обязательства.
\end{block}
\end{frame}

\begin{frame}[allowframebreaks]{\setfontsize{12pt}Как отдельно взятый банк реагирует на увеличение резервов?}
{Допущения}
1. Требуемая норма обязательных резервов по депозитам до востребования равна 10\%.

2. Банки не склонны иметь избыточные резервы, поскольку последние не приносят процентов, поэтому они превращают их в приносящие проценты активы, такие, как ссуды. 

\pagebreak
3. Депозиты до востребования являются единственными банковскими пассивами; резервы в Банке России и ссуды — единственные активы банка. 

4. Спрос на кредит таков, что у банка нет проблем с выдачей дополнительных средств в виде займов. 

\pagebreak
5. Каждый раз, когда ссуду получает частное или юридическое лицо, вся сумма поступает на текущие счета, клиенты банка не снимают со счетов наличных денег.
\end{frame}

\begin{frame}[shrink=25]{Первоначальный баланс банка}
% Table generated by Excel2LaTeX from sheet 'Лист1'
\begin{table}[htbp]
  \centering
  \caption{Первоначальный баланс Банка 1}
\begin{tabularx}{\linewidth}[b]{@{}>{\raggedright\arraybackslash}XrXr@{}}
	\toprule
	        \multicolumn{2}{c}{Активы}         &           \multicolumn{2}{c}{Пассивы}           \\ \midrule
	Суммарные резервы    & 100000~₽            & Депозиты до востребования & 1000000~₽           \\
	Обязательные резервы & 100000~₽            &                           &                     \\
	Избыточные резервы   & 0~₽                 &                           &                     \\
	Ссуды                & 900000~₽            &                           &                     \\ \midrule
	\bfseries Итого      & \bfseries 1000000~₽ &                           & \bfseries 1000000~₽ \\ \bottomrule
\end{tabularx}%
  \label{tab:addlabel}%
\end{table}%

\end{frame}

\begin{frame}[shrink=25]{Новый вклад в Банк 1}
% Table generated by Excel2LaTeX from sheet 'Лист1'
\begin{table}[htbp]
  \centering
  \caption{Баланс Банка 1}
\begin{tabularx}{\linewidth}[b]{@{}>{\raggedright\arraybackslash}XrXr@{}}
	\toprule
	        \multicolumn{2}{c}{Активы}         &           \multicolumn{2}{c}{Пассивы}           \\ \midrule
	Суммарные резервы    & 200000~₽            & Депозиты до востребования & 1100000~₽           \\
	Обязательные резервы & 110000~₽            &                           &                     \\
	Избыточные резервы   & 90000~₽             &                           &                     \\
	Ссуды                & 900000~₽            &                           &                     \\ \midrule
	\bfseries Итого      & \bfseries 1100000~₽ &                           & \bfseries 1100000~₽ \\ \bottomrule
\end{tabularx}%
  \label{tab:addlabel}%
\end{table}%
\end{frame}

\begin{frame}[shrink=25]{Ссудная операция Банка 1}
% Table generated by Excel2LaTeX from sheet 'Лист1'
\begin{table}[htbp]
  \centering
  \caption{Баланс Банка 1}
\begin{tabularx}{\linewidth}[b]{@{}>{\raggedright\arraybackslash}XrXr@{}}    \toprule
    \multicolumn{2}{c}{Активы} & \multicolumn{2}{c}{Пассивы} \\
    \midrule
    Суммарные резервы & 200000~₽ & Депозиты до востребования & 1100000~₽\\
    Обязательные резервы & 110000~₽ &       &  \\
    Избыточные резервы & 0~₽     &       &  \\
    Ссуды & 990000~₽ &       &  \\
    \midrule
    \bfseries Итого &  \bfseries 1100000~₽ & & \bfseries 1100000~₽\\
    \bottomrule
    \end{tabularx}%
  \label{tab:addlabel}%
\end{table}%

\textbf{Перемещение средств из одного банка в другой не влияет на денежную массу и общий объем кредитования в экономике}
\end{frame}

\begin{frame}[shrink=25]{Операции центрального банка}{Покупка Банком России государственных облигаций}
% Table generated by Excel2LaTeX from sheet 'Лист2'
\begin{table}[htbp]
  \centering
  \caption{Банк России}
\begin{tabularx}{\linewidth}[b]{@{}>{\raggedright\arraybackslash}XrXr@{}}
	\toprule
	       \multicolumn{2}{c}{Активы}         &            \multicolumn{2}{c}{Пассивы}            \\ \midrule
	Государственные ценные бумаги & +100000~₽ & Резервы коммерческого банка (корсчет) & +100000~₽ \\ \bottomrule
\end{tabularx}%
  \label{tab:addlabel}%
\end{table}%

% Table generated by Excel2LaTeX from sheet 'Лист2'
\begin{table}[htbp]
  \centering
  \caption{Коммерческий банк}
\begin{tabularx}{\linewidth}[b]{@{}>{\raggedright\arraybackslash}XrXr@{}}
	\toprule
	    \multicolumn{2}{c}{Активы}     &           \multicolumn{2}{c}{Пассивы}           \\ \midrule
	Корсчет в Банке России & +100000~₽ & Депозиты до востребования  продавца & +100000~₽ \\ \bottomrule
\end{tabularx}%
  \label{tab:addlabel}%
\end{table}%
\end{frame}

\begin{frame}[shrink=25]{Операции центрального банка}{Продажа Банком России государственных облигаций}
% Table generated by Excel2LaTeX from sheet 'Лист2'
\begin{table}[htbp]
  \centering
  \caption{Банк России}
\begin{tabularx}{\linewidth}[b]{@{}>{\raggedright\arraybackslash}XrXr@{}}
	\toprule
	       \multicolumn{2}{c}{Активы}         &            \multicolumn{2}{c}{Пассивы}            \\ \midrule
	Государственные ценные бумаги & -100000~₽ & Резервы коммерческого банка (корсчет) & -100000~₽ \\ \bottomrule
\end{tabularx}%
  \label{tab:addlabel}%
\end{table}%


% Table generated by Excel2LaTeX from sheet 'Лист2'
\begin{table}[htbp]
  \centering
  \caption{Коммерческий банк}
\begin{tabularx}{\linewidth}[b]{@{}>{\raggedright\arraybackslash}XrXr@{}}
	\toprule
	    \multicolumn{2}{c}{Активы}     &           \multicolumn{2}{c}{Пассивы}           \\ \midrule
	Корсчет в Банке России & -100000~₽ & Депозиты до востребования  продавца & -100000~₽ \\ \bottomrule
\end{tabularx}%
  \label{tab:addlabel}%
\end{table}%
\end{frame}

\begin{frame}
\begin{itemize}
\item
Никто не обязан заключать сделки с центральным банком.

\item
Центральный банк может выкупить столько государственных ценных бумаг, сколько захочет, поскольку он (в отличие от нас) может платить за них, увеличивая резервы других банков. 

\item
Центральный банк не должен беспокоиться о минимизации капитального убытка или максимизации дохода от капитала. Его обязанность — вести операции для общественной выгоды.
\end{itemize}

\end{frame}


\begin{frame}[shrink=25]{Расширение депозитов}
{Покупка облигаций со стороны Банка России }
% Table generated by Excel2LaTeX from sheet 'Лист1'
\begin{table}[htbp]
  \centering
  \caption{Баланс Банка 1}
\begin{tabularx}{\linewidth}[b]{@{}>{\raggedright\arraybackslash}XrXr@{}}
	\toprule
	        \multicolumn{2}{c}{Активы}         &           \multicolumn{2}{c}{Пассивы}           \\ \midrule
	Суммарные резервы    & 200000~₽            & Депозиты до востребования & 1100000~₽           \\
	Обязательные резервы & 110000~₽            &                           &                     \\
	Избыточные резервы   & 90000~₽             &                           &                     \\
	Ссуды                & 900000~₽            &                           &                     \\ \midrule
	\bfseries Итого      & \bfseries 1100000~₽ &                           & \bfseries 1100000~₽ \\ \bottomrule
\end{tabularx}%
  \label{tab:addlabel}%
\end{table}%

\end{frame}

\begin{frame}[shrink=25]{Расширение депозитов}
{Покупка облигаций со стороны Банка России }
% Table generated by Excel2LaTeX from sheet 'Лист1'
\begin{table}[htbp]
  \centering
  \caption{Баланс Банка 1}
\begin{tabularx}{\linewidth}[b]{@{}>{\raggedright\arraybackslash}XrXr@{}}
	\toprule
	        \multicolumn{2}{c}{Активы}         &           \multicolumn{2}{c}{Пассивы}           \\ \midrule
	Суммарные резервы    & 200000~₽            & Депозиты до востребования & 1100000~₽           \\
	Обязательные резервы & 110000~₽            &                           &                     \\
	Избыточные резервы   & 0~₽                 &                           &                     \\
	Ссуды                & 990000~₽            &                           &                     \\ \midrule
	\bfseries Итого      & \bfseries 1100000~₽ &                           & \bfseries 1100000~₽ \\ \bottomrule
\end{tabularx}%
  \label{tab:addlabel}%
\end{table}%

\end{frame}


\begin{frame}[shrink=25]{Расширение депозитов}
{Кредитная операция Банка 1}
% Table generated by Excel2LaTeX from sheet 'Лист1'
\begin{table}[htbp]
  \centering
  \caption{Изменения в балансе Банка 2}
\begin{tabularx}{\linewidth}[b]{@{}>{\raggedright\arraybackslash}XrXr@{}}
	\toprule
	       \multicolumn{2}{c}{Активы}         &          \multicolumn{2}{c}{Пассивы}           \\ \midrule
	Суммарные резервы    & +90000~₽           & Депозиты до востребования & +90000~₽           \\
	Обязательные резервы & +9000~₽            &                           &                    \\
	Избыточные~резервы   & +81000~₽           &                           &                    \\
	Ссуды                & 0~₽                &                           &                    \\ \midrule
	\bfseries Итого      & \bfseries +90000~₽ &                           & \bfseries +90000~₽ \\ \bottomrule
\end{tabularx}%
  \label{tab:addlabel}%
\end{table}%
\end{frame}

\begin{frame}[shrink=25]{Расширение депозитов}
{Кредитная операция Банка 2}
% Table generated by Excel2LaTeX from sheet 'Лист1'
\begin{table}[htbp]
  \centering
  \caption{Изменения в балансе Банка 2}
\begin{tabularx}{\linewidth}[b]{@{}>{\raggedright\arraybackslash}XrXr@{}}
	\toprule
	       \multicolumn{2}{c}{Активы}         &          \multicolumn{2}{c}{Пассивы}           \\ \midrule
	Суммарные резервы    & +90000~₽           & Депозиты до востребования & +90000~₽           \\
	Обязательные резервы & +9000~₽            &                           &                    \\
	Избыточные резервы   & +0~₽               &  \\
	Ссуды                & +81000~₽           &                           &                    \\ \midrule
	\bfseries Итого      & \bfseries +90000~₽ &                           & \bfseries +90000~₽ \\ \bottomrule
\end{tabularx}%
  \label{tab:addlabel}%
\end{table}%

\end{frame}

\begin{frame}
\begin{itemize}
\item
Процесс продолжается с банками 3, 4, 5 и т. д. 

\item
Каждый банк будет получать все меньший и меньший прирост депозитов, поскольку 10\% всегда должно находиться в резервах.

\item
Выводы будут те же, как и в том случае, если бы в нашем примере банки использовали свои избыточные резервы вместо выдачи ссуд для покупки ценных бумаг, приносящих проценты.
\end{itemize}
\end{frame}

\subsection{Депозитный мультипликатор}
\begin{frame}{\setfontsize{12pt}Простейший депозитный (денежный) мультипликатор}
Мы можем выявить математическую зависимость между максимальным увеличением депозитов до востребования и изменением резервов.
\begin{align}
\Delta \text{TR}=d \cdot \Delta D
\end{align}
где $\Delta \text{TR}$ — изменение суммарных резервов;

$d$ — требуемая норма резервного покрытия по депозитам до востребования; 

$\Delta D$ — изменение объема депозитов до востребования.
\end{frame}

\begin{frame}{Простейший депозитный (денежный) мультипликатор}
Разделим обе части уравнения на требуемую норму резервного покрытия \textit{d}:
\begin{align}
\frac{1}{d}\cdot \Delta \text{TR}=\Delta D
\end{align}

\begin{itemize}
\item
Изменение суммарных резервов увеличит депозиты до востребования в $\frac{1}{d}\cdot \Delta \text{TR}$ раз. 

\item
Выражение $\frac{1}{d}$ — это простейший депозитный (денежный) мультипликатор. 
\end{itemize}
\end{frame}

\begin{frame}{\setfontsize{12pt}Недостатки простейшего депозитного мультипликатора}
Формула простейшего депозитного мультипликатора описывает весьма упрощенную ситуацию, в которой существуют только депозиты до востребования с определенной требуемой нормой резервного покрытия, клиенты банка не снимают наличность со счетов, а банковские избыточные резервы всегда равны нулю. 
\end{frame}

\begin{frame}{Депозитный мультипликатор}{Новые допущения}
\begin{itemize}
\item
На практике денежная масса, состоит из предложения денег со стороны банков — депозитов до востребования и из предложения денег со стороны государства — наличных денег. 

\item
Совокупное предложение денег со стороны государства состоит из банковских резервов и наличных денег в обращении. 

\item
Банки склонны иметь положительные избыточные резервы (из соображений собственной финансовой безопасности).
\end{itemize}
\end{frame}

\begin{frame}[shrink=25]{Расширение депозитов}
{Покупка облигаций со стороны Банка России }
% Table generated by Excel2LaTeX from sheet 'Лист1'
\begin{table}[htbp]
  \centering
  \caption{Баланс Банка 1}
\begin{tabularx}{\linewidth}[b]{@{}>{\raggedright\arraybackslash}XrXr@{}}
	\toprule
	        \multicolumn{2}{c}{Активы}         &           \multicolumn{2}{c}{Пассивы}           \\ \midrule
	Суммарные резервы    & 200000~₽            & Депозиты до востребования & 1100000~₽           \\
	Обязательные резервы & 110000~₽            &                           &                     \\
	Избыточные резервы   & 90000~₽             &                           &                     \\
	Ссуды                & 900000~₽            &                           &                     \\ \midrule
	\bfseries Итого      & \bfseries 1100000~₽ &                           & \bfseries 1100000~₽ \\ \bottomrule
\end{tabularx}%
  \label{tab:addlabel}%
\end{table}%

\end{frame}

\setbeamercovered{invisible}
\begin{frame}{Утечка наличности}
Предположим, что все представители небанковского сектора, включая продавца облигаций, захотят располагать суммой наличности, которая будет постоянной частью (25\%) их депозитов до востребования. 

Следовательно, сразу после внесения первоначального депозита в 100000~₽, продавец облигации снимает со счета
\onslide<2->{
\begin{align*}
80000~₽\cdot 0.25=20000~₽
\end{align*}}
\end{frame}

\setbeamercovered{transparent}

\begin{frame}[shrink=25]{Утечка наличности}
% Table generated by Excel2LaTeX from sheet 'Лист1'
\begin{table}[htbp]
  \centering
  \caption{Баланс Банка 1}
\begin{tabularx}{\linewidth}[b]{@{}>{\raggedright\arraybackslash}XrXr@{}}
	\toprule
	          \multicolumn{2}{c}{Активы}           &              \multicolumn{2}{c}{Пассивы}               \\ \midrule
	Суммарные резервы        & 200000~₽            & Депозиты до востребования        & 1100000~₽           \\
	Снятие наличности        & -20000~₽            & Снятие наличности                & -20000~₽            \\ \midrule
	Чистые суммарные резервы & 180000~₽            & Чистые депозиты до востребования & 1080000~₽           \\
	Обязательные резервы     & 108000~₽            &                                  &                     \\
	Избыточные резервы       & 72000~₽             &                                  &                     \\
	Ссуды                    & 900000~₽            &                                  &                     \\ \midrule
	\bfseries Итого          & \bfseries 1080000~₽ &                                  & \bfseries 1080000~₽ \\ \bottomrule
\end{tabularx}%
  \label{tab:addlabel}%
\end{table}%

\end{frame}

\setbeamercovered{invisible}
\begin{frame}
Предположим, что банки хранят избыточные резервы в постоянной пропорции 5\% от всех дополнительных депозитов до востребования. 

Если текущие избыточные резервы равны 72000~₽, то Банк 1 будет держать только 
\onslide<2->{
\begin{align*}
80000~₽ \cdot 0.05=4000~₽
\end{align*}}

\onslide<3->{
Таким образом, остается 68000~₽ нежелательных избыточных резервов, которые Банк 1 хотел бы выдать в виде ссуд.}

\end{frame}
\setbeamercovered{transparent}

\begin{frame}[shrink=25]{Утечка наличности}
% Table generated by Excel2LaTeX from sheet 'Лист1'
\begin{table}[htbp]
  \centering
  \caption{Баланс Банка 1}
\begin{tabularx}{\linewidth}[b]{@{}>{\raggedright\arraybackslash}XrXr@{}}
	\toprule
	          \multicolumn{2}{c}{Активы}           &              \multicolumn{2}{c}{Пассивы}               \\ \midrule
	Суммарные резервы        & 200000~₽            & Депозиты до востребования        & 1100000~₽           \\
	Снятие наличности        & -20000~₽            & Снятие наличности                & -20000~₽            \\ \midrule
	Чистые суммарные резервы & 180000~₽            & Чистые депозиты до востребования & 1080000~₽           \\
	Обязательные резервы     & 108000~₽            &                                  &                     \\
	Избыточные резервы       & 4000~₽              &                                  &                     \\
	Ссуды                    & 968000~₽            &                                  &                     \\ \midrule
	\bfseries Итого          & \bfseries 1080000~₽ &                                  & \bfseries 1080000~₽ \\ \bottomrule
\end{tabularx}%
  \label{tab:addlabel}%
\end{table}%

\end{frame}

\begin{frame}[allowframebreaks]{Формула депозитного мультипликатора}
Суммарные резервы банковской системы изменяются в размере:
\begin{align}
\Delta TR&=d\cdot \Delta D + e \cdot \Delta D \nonumber\\
&= (d+e)\cdot \Delta D
\end{align}
где \textit{e} — отношение избыточных резервов к депозитам до востребования.

\pagebreak
Утечка наличности равна:
\begin{align}
\Delta C = c \cdot \Delta D,
\end{align}
где 

\textit{C} — банковские резервы;

\textit{c} — величина отношения наличных денег к объему депозитов до востребования, ожидаемая небанковским сектором.

\pagebreak
Общий прирост предложения денег со стороны государства будет равен сумме уравнений, т.е.
\begin{align}
\Delta TR + \Delta C &= [(d+e) \cdot \Delta D] + c \cdot \Delta D \nonumber \\
&= (d+e+c) \cdot \Delta D
\end{align}

\pagebreak
\begin{itemize}
\item
Левая часть уравнения — общее изменение предложения денег со стороны государства. 

\item
Правая часть уравнения показывает, что такое изменение зависит от изменения объема депозитов в банковской системе, которое в свою очередь зависит от требуемой нормы резервного покрытия, уровня избыточных резервов, ожидаемого банками, и отношения наличных денег к объему депозитов до востребования, ожидаемого небанковским сектором.
\end{itemize}

\pagebreak
Предложение денег со стороны государства называется денежной базой, которая равна:
\begin{align}
MB=TR+C
\end{align}
ее изменение составляет:
\begin{align}
\Delta MB=\Delta TR+\Delta C
\end{align}

\pagebreak
Таким образом, денежная база изменяется в результате изменений суммарных резервов и количества наличных денег у небанковского сектора:
\begin{align}
\Delta MB = (d+e+c) \cdot \Delta D
\end{align}
Если мы разделим обе части уравнения на величину $(d+e+c)$, то получим:
\begin{align}
\Delta MB \cdot \frac{1}{d+e+c} = \Delta D
\end{align}

Величина $\frac{1}{d+e+c}$ называется депозитным мультипликатором
\end{frame}

\subsection{Денежный мультипликатор}
\begin{frame}[allowframebreaks]{Денежный мультипликатор}
Денежная масса в соответствии с определением Банка России денежного агрегата M1 равна:
\begin{align}
M=C+D
\end{align}
где \textit{М} — денежная масса. 

Это означает, что изменение денежной массы составит 
\begin{align}
\Delta M=\Delta C + \Delta D
\end{align}
т. е. денежная масса меняется в ответ на изменения объема наличности у небанковского сектора и депозитов до востребования.

\pagebreak
Общий денежный мультипликатор \textit{m} показывает, насколько изменилась денежная масса в ответ на изменение денежной базы (предложения денег со стороны государства).
\begin{align}
\Delta M &= m \cdot \Delta MB \nonumber\\
\Delta M &= \Delta C + \Delta D \nonumber\\
\Delta MB &= \Delta TR + \Delta C \nonumber\\
\Delta D + c\cdot \Delta D &= m \cdot (d+e+c)\cdot \Delta D \nonumber\\
1+c &= m \cdot (d+e+c)\nonumber \nonumber \\
m &=\frac{1+c}{d+e+c}
\end{align}

\pagebreak
На практике изменение денежной массы будет равно:
\begin{align}
M=\frac{1+c}{d+e+c}\cdot MB
\end{align}
Ограничения модели:
\begin{itemize}
\item
Только требуемая норма резервного покрытия \textit{d} приблизительно постоянна.
\item
Значения других переменных (\textit{e} и \textit{c}) полностью зависят от поведения банковской системы и небанковского сектора соответственно.
\end{itemize}

\end{frame}


\subsection{Кредитный мультипликатор}
\begin{frame}[allowframebreaks]{Общий кредитный\\ мультипликатор}
Если у банков в качестве активов только ссуды и резервы, а в качестве пассивов только депозиты до востребования, тогда справедливо, что:
\begin{align}
L+TR=D
\end{align}
где
\textit{L} - объем ссуд.

Прирост общего объема банковского кредитования должен быть равен:
\begin{align}
\Delta L = \Delta D - \Delta TR
\end{align}

\pagebreak
Предположим, что прирост объема банковского кредитования связан с приростом денежной базы с помощью мультипликатора $m_L$:
\begin{align}
\Delta L = m_L \cdot \Delta MB
\end{align}
Используем подстановки:
\begin{align}
\Delta L &= \Delta D - \Delta TR \nonumber\\
&= \Delta D - [(d+e) \cdot \Delta D] \nonumber\\
\Delta MB &= \Delta TR + \Delta C \nonumber\\
&= (d+e+c) \cdot \Delta D
\end{align}

\pagebreak
Тогда:
\begin{align}
\Delta D - [(d+e)\cdot \Delta D]=m_L \cdot (d+e+c) \cdot \Delta D
\end{align}

Если разделить обе части уравнения на $\Delta D$ то имеем:
\begin{align}
1-(d+e)=m_L \cdot (d+e+c)
\end{align}

\pagebreak
Наконец, разделим обе части последнего уравнения на величину $(d+e+c)$:
\begin{align}
m_L=\frac{1-(d+e)}{d+e+c}
\end{align}
и получим выражение общего кредитного мультипликатора. 

\pagebreak
Модель общего кредитного мультипликатора:
\begin{align}
\Delta L=\Delta MB \cdot \frac{1-(d+e)}{d+e+c}
\end{align}

\end{frame}
\end{document}