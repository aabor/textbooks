\documentclass[_DKB_p1_Money.tex]{subfiles}
\begin{document}

\subsection{Важность измерения денежной массы}
\setbeamercovered{transparent}

\begin{frame}{Важность определения денег и измерения денежной массы}
\begin{itemize}[<+->]
\item
Изменения объема денежной массы и темпов ее прироста влияют на важные экономические переменные: высокий уровень занятости населения, стабильность цен, экономический рост, сбалансированность платежного баланса. 

\item
Оптимальный объем и оптимальные темпы прироста денежной массы определяются как показатели, которые позволяют достичь целей развития национальной экономики.
\end{itemize}
\end{frame}

\begin{frame}{Денежно-кредитная политика }
\begin{block}{Денежно-кредитная политика }
\quad
определяется как система мер, которые изменяют объем денег, выпускаемых в обращение Центральным Банком, для достижения целей развития национальной экономики.
\end{block}
\begin{itemize}[<+->]
\item
Соответствие между теоретическим и эмпирическим определением денег;

\item
Центральный Банк может лишь изменять некоторые «денежные» переменные;

\item
в результате изменения денежной массы экономические переменные должны изменяться в желательном направлении.
\end{itemize}
\end{frame}

\subsection{Подходы к измерению денежной массы}
\begin{frame}{Подходы к определению денег и измерению денежной массы}
\begin{itemize}
\item
Трансакционный подход

\item
Ликвидный подход

\end{itemize}
\end{frame}

\begin{frame}
\begin{block}{Трансакционный подход }
к измерению денежной массы базируется на функции денег как средства обращения. 

Все активы служат средством сохранения стоимости, но только некоторые из них используются как средство обращения.

В состав денежной массы входят наличные и депозиты до востребования.

Люди стараются минимизировать сумму денег, которую они хранят в качестве средства обращения.

В результате увеличения совокупного предложения денег расходы общества в целом увеличиваются.
\end{block}
\end{frame}

\begin{frame}
\begin{block}{Ликвидный подход}
Ликвидный подход к измерению денежной массы исходит из того, что сущностью денег является их свойство быть наиболее ликвидным активом.

Подчеркивает функцию денег как средства сохранения стоимости.

Деньги считаются наиболее ликвидным из всех активов.

Включает в состав денег любые высоколиквидные активы, т. е. такие, которые могут быть быстро обращены в деньги без потери номинальной стоимости и без существенных затрат.
\end{block}
\end{frame}

\subsection{Современное измерение денежной массы}
\begin{frame}[shrink=20]{Современное измерение денежной массы}
% Table generated by Excel2LaTeX from sheet 'Лист1'
\begin{table}[htbp]
	\caption{Денежная база}
	\centering
  	\begin{tabularx}{\linewidth}[b]{@{}>{\raggedright\arraybackslash}Xrr@{}}
    \toprule
    \multicolumn{1}{c}{\textbf{Показатель, млрд. руб.}} & \multicolumn{1}{c}{\textbf{01.07.2016}} & \multicolumn{1}{c}{\textbf{01.08.2016}} \\
    \midrule
    \textbf{Денежная база (в широком определении)} & \textbf{10 785,60} & \textbf{10 600,60} \\
    \textit{—  наличные деньги в обращении с учетом остатков средств в кассах кредитных организаций} & \textit{8 241,90} & \textit{8 322,40} \\
    \textit{—  корреспондентские счета кредитных организаций в Банке России} & \textit{1 712,40} & \textit{1 491,30} \\
    \textit{—  обязательные резервы} & \textit{394,3} & \textit{394} \\
    \textit{—  депозиты кредитных организаций в Банке России} & \textit{436,9} & \textit{392,9} \\
    \textit{—  облигации Банка России у кредитных организаций} & \textit{0} & \textit{0} \\
    \bottomrule
    \end{tabularx}%
  \label{tab:addlabel}%
\end{table}%
\end{frame}

\begin{frame}{Денежная база, пояснения}
Наличные в кассах кредитных организаций - без учета наличных денег в кассах учреждений Банка России.

Облигации - по рыночной стоимости. 
\end{frame}

\begin{frame}{Денежная база (в узком определении)}
\begin{itemize}
\item
наличные деньги кредитных организаций в обращении вне Банка России; 	

\item
обязательные резервы кредитных организаций в Банке России.
\end{itemize}
\end{frame}

\begin{frame}[shrink=15]
% Table generated by Excel2LaTeX from sheet 'Лист3'
\begin{table}[htbp]
	\caption{Денежная масса (национальное определение)}
  \centering
  	\begin{tabularx}{\linewidth}[b]{@{}>{\raggedright\arraybackslash}Xrr@{}}
    \toprule
    \textbf{Показатель, млрд. руб.} & \textbf{01.06.2016} & \textbf{01.07.2016} \\
    \midrule
    \textbf{Наличные деньги в обращении вне банковской системы (денежный агрегат M0)} & \textbf{7 295,90} & \textbf{7 372,00} \\
    \textit{Переводные депозиты} & \textit{9 582,60} & \textit{9 678,30} \\
    \textbf{Денежный агрегат М1} & \textbf{16 878,50} & \textbf{17 050,30} \\
    \textit{Другие депозиты, входящие в состав денежного агрегата М2} & \textit{19 373,10} & \textit{19 436,90} \\
    \textbf{Денежная масса в национальном определении (денежный агрегат М2)} & \textbf{36 251,60} & \textbf{36 487,20} \\
    \bottomrule
    \end{tabularx}%
  \label{tab:addlabel}%
\end{table}%

\end{frame}

\begin{frame}{Денежная масса в России}
Денежный агрегат М1 включает наличные деньги в обращении вне банковской системы (денежный агрегат М0) и остатки средств в национальной валюте на расчетных, текущих и иных счетах до востребования населения, нефинансовых и финансовых (кроме кредитных) организаций, являющихся резидентами Российской Федерации.

Денежный агрегат М2 включает денежный агрегат М1 и остатки средств в национальной валюте на счетах срочных депозитов и иных привлеченных на срок средств населения, нефинансовых и финансовых (кроме кредитных) организаций, являющихся резидентами Российской Федерации.
\end{frame}

\begin{frame}[shrink=25]
% Table generated by Excel2LaTeX from sheet 'Лист4'
\begin{table}[htbp]
\caption{Широкая денежная масса (M2x)}
  \centering
  	\begin{tabularx}{\linewidth}[b]{@{}>{\raggedright\arraybackslash}Xrr@{}}
    \toprule
    Показатель, млрд. руб. & 01.06.2016 & 01.07.2016 \\
    \midrule
    \textbf{Депозиты, включаемые в широкую денежную массу} & 89,89 & 49,82 \\
    \textit{\textbf{Переводные депозиты}} & 79,98 & 45,32 \\
    \textit{ другие финансовые организации} & 66,21 & 27,17 \\
    \textit{ нефинансовые государственные  организации} & 13,53 & 17,84 \\
    \textit{ другие нефинансовые организации} & 0,24  & 0,31 \\
    \textit{\textbf{Другие депозиты}} & 9,90  & 4,50 \\
    \textit{ другие финансовые организации} & 9,90  & 4,50 \\
    \textit{ нефинансовые государственные  организации} & 0,00  & 0,00 \\
    \textit{ другие нефинансовые организации} & 0,00  & 0,00 \\
    \bottomrule
    \end{tabularx}%
  \label{tab:addlabel}%
\end{table}%

\end{frame}

\begin{frame}{Денежный агрегат М3}
М3 включает в себя агрегат M2x, а также:
\begin{itemize}
\item
- депозиты с уведомлением о возврате до 3 месяцев;

\item
- операции РЕПО;

\item
- счета денежного рынка;

\item
- долговые ценные бумаги сроком до 2-ух лет.
\end{itemize}
Денежный агрегат М3:
\begin{itemize}
\item
не рассчитывается Банком России. 

\item
на международном уровне определяется Organisation for Economic Co-operation and Development (OECD) для всех стран членов  по единой методике.
\end{itemize}
\end{frame}

\begin{frame}
\begin{figure}
\center
\includegraphics[scale=0.5]{img/money}
\caption{Денежные агрегаты в России}
\end{figure}
\end{frame}

\begin{frame}
\begin{figure}
\center
\includegraphics[scale=0.5]{img/m2_gdp_Russia}
\caption{Уровень монетизации экономики (М2/ВВП) России с 2003 по 2016 гг.}
\end{figure}
\end{frame}

\begin{frame}
\begin{block}{Теневая экономика}
\quad
предполагает проведение кассовых сделок, о доходах по которым участники сделки не сообщают налоговым органам.

По различным оценкам может составлять от 2\% до 50\% ВВП.
\end{block}
\end{frame}


\setbeamercovered{transparent}
\end{document}