% !TeX program = lualatex -synctex=1 -interaction=nonstopmode --shell-escape %.tex

\documentclass[_fin_decisions_lectures.tex]{subfiles}

\begin{document}

\section*{Литература}
\subsection*{Учебники}

\begin{frame}[allowframebreaks]
  \frametitle<presentation>{Учебники}
    
  \begin{thebibliography}{10}
    
  \beamertemplatebookbibitems
  % Start with overview books.
  \bibitem{nikonov2017}
	О.И.~Никонов, С.В.~Кругликов, М.А.~Медведева
    \newblock {\em Математическое моделирование и методы принятия решений: Учебное пособие}.
    \newblock М.:Флинта, Изд-во Урал. ун-та, 2017. - 100 с. ISBN 978-5-9765-3142-0.
    
   \bibitem{Slepuhina2017}
    Ю.Э.~Слепухина. 
    \newblock {\em Риск-менеджмент на финансовых рынках: Учебное пособие}.
    \newblock М.:Флинта, 2017. - 215 с.: ISBN 978-5-9765-3240-3.
    
    \pagebreak    
   \bibitem{Knyazeva2017}
	Е.Г.~Князева, Л.И.~Юзвович, Р.Ю.~Луговцов. 
	\newblock {\em Финансово-экономические риски: Учебное пособие }.
	\newblock М.:Флинта, Изд-во Урал. ун-та, 2017. - 112 с. ISBN 978-5-9765-3130-7.

   \bibitem{Kazakova2017}
	Н.А.~Казакова. 
	\newblock {\em Финансовая среда предпринимательства и предпринимательские риски}.
	\newblock М. : ИНФРА-М, 2017. — 208 с. — (Высшее образование: Бакалавриат).

\pagebreak
  \bibitem{Chetirkin2015}
	Е.М.~Четыркин. 
    \newblock {\em Финансовые риски: науч.-практич. пособие}.
    \newblock М.: Изд. дом «Дело» РАНХиГС, 2015.
  
  \end{thebibliography}
\end{frame}

\subsection*{Научные статьи}

\begin{frame}[shrink=15]
  \frametitle<presentation>{Научные статьи}
  \begin{thebibliography}{10}
  \beamertemplatearticlebibitems
  % Followed by interesting articles. Keep the list short. 
  \bibitem{Aven2010}
    \newblock T.~Aven 
    \newblock On how to define, understand and describe risk. {\em Reliability Engineering \& System Safety}. 2010. T.~95, №~6. C.~623-631.
  \bibitem{Aven2012}
    \newblock T.~Aven 
    \newblock The risk concept—historical and recent development trends.{\em Reliability Engineering \& System Safety}. 2012. T.~99. C.~33-44.
   \end{thebibliography}
\end{frame}

\subsection*{Интернет ресурсы}
\begin{frame}[allowframebreaks]
  \frametitle<presentation>{Интернет ресурсы}
    
  \begin{thebibliography}{10}
  
  \setbeamertemplate{bibliography item}[online]

  \bibitem{micex}
    Московская биржа.
    \newblock http://rts.micex.ru/.
  \bibitem{cbr}
    Центральный банк Российской федерации.
    \newblock http://www.cbr.ru/ .
  \bibitem{gks}
    Росстат.
    \newblock http://www.gks.ru.
  \end{thebibliography}
\end{frame}

\subsection*{Научные журналы}

\begin{frame}
  \frametitle<presentation>{Журналы}
    
  \begin{thebibliography}{10}
  
  \beamertemplatearticlebibitems
  \bibitem{}
  {\em Деньги и кредит}
  \bibitem{}
  	{\em Финансы и кредит}
  \bibitem{}
  	{\em Экономический анализ: теория и практика}
  \bibitem{}
  	{\em Эксперт}
  
  \end{thebibliography}
\end{frame}

\subsection*{Газеты}

\begin{frame}
  \frametitle<presentation>{Газеты}
    
  \begin{thebibliography}{10}
  
  \beamertemplatearticlebibitems
  \bibitem{}
  	\newblock{\em Коммерсантъ}
  \bibitem{}
  	\newblock{\em Ведомости}
 
  \end{thebibliography}
\end{frame}

\subsection*{Материалы лекций}
\begin{frame}
Электронные презентации лекций размещены в личном профиле на сайте www.academia.edu:
\begin{thebibliography}{10}
	
	\setbeamertemplate{bibliography item}[online]
	
	\bibitem{acad}
	Academia
	
	\footnotesize{\url{https://unn-ru.academia.edu/AlexanderBorochkin}}
\end{thebibliography}
\end{frame}

\end{document}