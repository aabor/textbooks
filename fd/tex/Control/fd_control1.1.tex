% !TeX program = lualatex -synctex=1 -interaction=nonstopmode --shell-escape %.tex

\documentclass[12pt, table, twoside, a4paper]{exam}
\usepackage[rus]{borochkin}

\usepackage{borochkin_exam}

%%%%%%%%%%%%%%%%%%%%%%%%%%%%%%%%%%%%%%%%%

\professor
\iftagged{professor}{ \printanswers }

%%%%%%%%%%%%%%%%%%%%%%%%%%%%%%%%%%%%%%%%%



\begin{document}
\setcounter{section}{0\relax}%

\noindent
% Контр/р № #1, Вариант #2, Предмет #3
\studentpersonalinfo{1}{1}{МОФР}

\normalsize
\begin{questions}
\question[9] Тест
\answerstotestShort

\question[20] Номинал облигации 5000~₽, купон 3\%, выплачивается один раз в год. До погашения облигации 10 лет. Облигация стоит 4800~₽. 
\noaddpoints
\begin{subparts}
	\subpart[10] Определите ориентировочно доходность до погашения облигации.
	\begin{solution}[10em]
		\begin{align*}
		r^*&=\frac{(N-P)/m \cdot n + C/m }{(N+P)/2}\\
		r^*&=3,47\%
		\end{align*}
		где
		
		$r^*$ - ориентировочная доходность облигации;
		
		$N$ - номинал облигации;
		
		$P$ - рыночная цена облигации;
		
		$C$ - сумма купона;
		
		$n, m$ - срок облигации и количество выплат процентов в году, соответственно.
	\end{solution}
	
	\subpart[10] Определите доходность до погашения облигации методом линейной интерполяции.
	
	\begin{solution}[10em]
		\begin{align*}
		r_1&=4\%\\
		r_2&=3\%\\
		P_1&=\frac{C}{r_1} \cdot \left(1 - \frac{1}{(1+r_1)^n} \right) + \frac{N}{(1+r_1)^n}=4594,5~₽\\
		P_2&=\frac{C}{r_2} \cdot \left(1 - \frac{1}{(1+r_2)^n} \right) + \frac{N}{(1+r_1)^n}=5000~₽\\
		r^{**}&=r_1+(r_2-r_1) \cdot \frac{P_1-P}{P_1-P_2}=3.49\%
		\end{align*}	
		где
		
		$r_1, r_2$ - соответственно, округленная вверх и вниз ориентировочная доходность облигации;
		
		$P_1, P_2$ - соответственно, нижняя и верхняя границы рыночной цены облигации;
		
		$r^{**}$ - доходность до погашения облигации, определенная методом линейной интерполяции.
	\end{solution}
	
\end{subparts}
\addpoints

\question[5] Средняя доходность фондового индекса равна 13\% годовых, стандартное отклонение доходности 30\%. Предполагается, что доходность имеет нормальное распределение. Инвестор формирует портфель, копирующий данный индекс. Определить вероятность того, что в следующем году портфель принесет ему убыток.

\begin{solution}[6em]
	
	\raggedright
	Вероятность   попадания   доходности   актива   в   заданный интервал определяется по формуле 
	\begin{align}
	\label{prob_loss}
	P(\alpha<r<\beta)=\Phi\left(\frac{\beta-\overline{x}}{\sigma}\right)-\Phi\left(\frac{\alpha-\overline{x}}{\sigma}\right),
	\end{align}
	где 
	
	$\Phi(.) $ - кумулятивная функция нормального стандартного распределения случайной величины (доходности актива);
	
	$\alpha $ - нижняя граница рассматриваемого интервала;
	
	$\beta $ - верхняя граница рассматриваемого интервала;
	
	$\sigma $ - стандартное отклонение доходности актива.
	
	Инвестор получит убыток, если доходность портфеля окажется меньше нуля. Согласно формуле \eqref{prob_loss} вероятность того, что доходность актива окажется меньше нуля равна
	\begin{align*}
	P(-\infty<r<0)&=\Phi\left(\frac{0-13}{30}\right)-\Phi\left(\frac{-\infty-13}{30}\right)\\
	&=\Phi(-0.433)-\Phi(-\infty)\\
	&=0.3324-0\\
	&=33.24\%.
	\end{align*}
\end{solution}

\addpoints

\question[10] Доходность актива имеет нормальное распределение. На основе наблюдений за 90 дней была определена ожидаемая доходность в расчете на день. Она составила 0.95\%. Пусть известно, что истинное значение стандартного отклонения доходности актива в расчете на день равно 1,5\%. В каком интервале с надежностью 0,98 располагается истинное значение ожидаемой доходности актива?

\begin{solution}[8em]
	
	\raggedright
	По данным статистики было определено не истинное значение ожидаемой доходности актива, а ее точечная оценка на основе выборочных данных. Для получения ответа о надежности такой оценки определяют доверительный интервал, который с заданным уровнем вероятности накрывает точечную оценку. В результате, с заданным уровнем надежности можно быть уверенным, что действительное значение ожидаемой доходности актива лежит в границах рассчитанного интервала.
	
	В примере необходимо определить доверительный интервал для нормально распределенной случайной величины при известном значении ее истинного стандартного отклонения с коэффициентом доверия 0.98.
	
	Верхнюю и нижнюю границы доверительного интервала для математического ожидания с известной дисперсией можно определить по следующим формулам:
	\begin{align}
	\label{eq:yield_low}
	\overline{r}_L &=\overline{r}-\frac{\sigma}{\sqrt{n}}u_{1-\alpha/2},\\[8pt]
	\label{eq:yield_upper}
	\overline{r}_U &=\overline{r}+\frac{\sigma}{\sqrt{n}}u_{1-\alpha/2},
	\end{align}
	где
	
	$\overline{r}_L, \overline{r}_U$ - нижняя и верхняя  границы доверительного интервала;
	
	$\overline{r}$ - точечная оценка ожидаемой доходности на основе осуществленной выборки;
	
	$\sigma$ - истинное значение стандартного отклонения доходности актива;
	
	$n$ - объем выборки;
	
	$u_{1-\alpha/2}$ - квантиль уровня $1-\alpha/2$ стандартного нормального распределения;
	
	$\alpha$ - уровень значимости, соответствующий выбранной доверительной вероятности $\gamma$.
	
	Из соотношения $\alpha=1-\gamma$ находим значение $\alpha$, соответствующее коэффициенту доверия 98\%.
	$$\alpha=1-0.98=0.02$$.
	
	По таблице квантилей нормального распределения находим квантили $u_{1-\alpha/2}=u_{0.99}=2,326$.
	
	По формулам \eqref{eq:yield_low} \eqref{eq:yield_upper} получаем:
	\begin{align*}
	\overline{r}_L&=0.95\% - \frac{1.5\%}{\sqrt{90}} \cdot 2.326 
	= 0.582\%, \\
	\overline{r}_U&=0.95\% + \frac{1.5\%}{\sqrt{90}} \cdot 2.326 =1.318\%
	\end{align*}
\end{solution}


\question[10] Доходность актива имеет нормальное распределение. Данные о его доходности за прошедшие 10 месяцев представлены в таблице

\begin{tabularx}{\linewidth}[b]{@{}>{\raggedright\arraybackslash}Xrrrrrrrrrr@{}}
	\toprule
	Месяцы            & 1  & 2  & 3 & 4  & 5  & 6  & 7  & 8 & 9  & 10 \\ \midrule
	Доходность актива & 18 & 11 & 8 & 13 & 18 & 25 & 16 & 7 & 10 & 8  \\ \bottomrule
\end{tabularx}%

Определите ожидаемую доходность актива. В каком интервале с надежностью 0,98 располагается истинное значение ожидаемой доходности актива?

\begin{solution}[8em]
	
	\raggedright
	Ожидаемая доходность актива равна
	$$\overline{r}=13.4\%,$$
	В данном примере не известно истинное значение стандартного отклонения актива, поэтому при определении доверительного интервала используем правило математической статистики для математического ожидания нормально распределенной случайной 
	величины при неизвестной дисперсии.
	
	Верхнюю и нижнюю границы доверительного интервала для математического ожидания с неизвестной дисперсией можно определить по следующим формулам:
	\begin{align}
	\label{yield_low_corrected}
	\overline{r}_L &=\overline{r}-\frac{s}{\sqrt{n}}t_{1-\alpha/2;n-1},\\[8pt]
	\label{yield_upper_corrected}
	\overline{r}_U &=\overline{r}+\frac{s}{\sqrt{n}}t_{1-\alpha/2;n-1},
	\end{align}
	где
	
	$s$- исправленное стандартное отклонение;
	
	$\alpha$ - уровень значимости, соответствующий выбранной доверительной вероятности $\gamma$, $\alpha=1-\gamma$.
	
	$t_{1-\alpha/2;n-1}$- квантиль уровня $1-\alpha/2$ распределения Стьюдента с $n-1$ степенями свободы.
	
	Исправленное стандартное отклонение равно:
	\begin{align}
	\label{st_dev_corrected}
	s&=\sqrt{\frac{1}{n-1}\sum_{i=1}^n \left(r_i-\overline{r}\right)^2}\\
	s&=5.777\% \nonumber
	\end{align}
	По  таблице квантилей распределения Стьюдента находим
	значение статистики $t_{1-\alpha/2;n-1}=t_{0.95;9}=2.821$ и по формулам \eqref{yield_low_corrected} и \eqref{yield_upper_corrected} определяем границы доходности актива:
	\begin{align*}
	\overline{r}_L &= 13.4\% - \frac{5.777\%}{\sqrt{10}} \cdot 2.821 =8.245\%,\\
	\overline{r}_U &=13.4\% + \frac{5.777\%}{\sqrt{10}} \cdot 2.821 = 18.555\%.
	\end{align*}
	Доверительный интервал равен: 8.245\%; 18.555\%.
\end{solution}

\pagebreak
\question[10] Доходность актива имеет нормальное распределение. На основе наблюдений за 51 день было рассчитано исправленное стандартное отклонение в расчете на день. Оно составило 3,5\%. В каком интервале с надежностью 0.99 располагается истинное значение дисперсии и стандартного отклонения доходности актива?

\begin{solution}[8em]
	\raggedright
	Верхнюю и нижнюю границы доверительного интервала для дисперсии можно определить по следующим формулам:
	
	\begin{align}
	\sigma_L^2&=\frac{s^2(n-1)}{\chi_{1-\alpha/2;n-1}^2};\\[8pt]
	\sigma_U^2&=\frac{s^2(n-1)}{\chi_{\alpha/2;n-1}^2};
	\end{align}
	где
	
	$\sigma_L^2$, $\sigma_U^2$ - нижняя и верхняя границы доверительного интервала, соответственно;
	
	$s^2$ - исправленная дисперсия доходности актива;
	
	$n$ - объем выборки;
	
	$\alpha$ - уровень значимости, соответствующий выбранной доверительной вероятности $\gamma$, $\alpha=1-\gamma$. 
	
	$\chi_{\alpha/2;n-1}^2$ - $\alpha/2$-квантиль распределения хи-квадрат с $n-1$ степенями свободы.
	
	Значение $\alpha$, соответствующее коэффициенту доверия 99\%.
	$$\alpha=1-0.99=0.01.$$
	
	Количество наблюдений случайной величины составило 51 день. Поэтому количество степеней свободы в примере равно 50. По таблице квантилей распределения $\chi^2$ находим квантили 79.49 и 27.991.
	
	\begin{align*}
		\sigma^2_L&=\frac{3.5\%^2 \cdot 50}{79.49}=7.705\\
		\sigma^2_U&=\frac{3.5\%^2 \cdot 50}{27.991}=21.882
	\end{align*}
	
	Границы доверительного интервала дисперсии доходности актива равны 7.705 и 21.882 и стандартного отклонения доходности актива равны 2.776\% и 4.678\%.
\end{solution}

\question[10] Доходности активов имеют нормальное распределение. На основе данных о доходности активов \textit{X} за 41 день и \textit{Y} за 81 день были рассчитаны исправленные стандартные отклонения доходности: $s_X=5.3\%$, $s_Y = 4.3\%$. Проверить гипотезу о равенстве дисперсий активов при уровне значимости 0,05.

\begin{solution}[6em]
	
	\raggedright
	При проверке гипотезы о равенстве дисперсий рассматривают две гипотезы: $H_0: \sigma_X^2=\sigma_Y^2$, и $H_1: \sigma_X^2 > \sigma_Y^2$, $H_0$ - это основная (нулевая) гипотеза, $H_1$, - альтернативная гипотеза. 
	Основная гипотеза говорит о том, что дисперсии доходности активов равны. Альтернативная гипотеза говорит о том, что они не равны. Если в результате проверки гипотеза $H_0$ отклоняется в пользу гипотезы $H_1$, то это означает, что дисперсии активов отличаются. Если гипотеза $H_0$ не отклоняется, то нет оснований отрицать равенство дисперсий.
	
	В качестве критерия проверки гипотезы используют случайную величину, которая представляет собой отношение большей исправленной дисперсии к меньшей:
	\begin{align}
	\label{dev_equal_test}
	F=\frac{s_X^2}{s_Y^2}
	\end{align}
	где
	$F$ - случайная величина, имеющая распределение Фишера со степенями свободы $n-1$ и $k-1$;
	
	$n$ - объем выборки доходности актива~$X$;
	
	$k$ - объем выборки доходности актива~$Y$.
	
	Если $F<f_{1-\alpha;n-1;k-1}$, где $f_{1-\alpha;n-1;k-1}$ - квантиль распределения Фишера, то параметр $F$ попадает в область принятия гипотезы, и нулевая гипотеза принимается на уровне значимости $\alpha$.
	
	Если  $F>=f_{1-\alpha;n-1;k-1}$ параметр $F$ попадает в критическую область. Следовательно $H_0$, отклоняется в пользу гипотезы $H_1$.
	
	Рассчитаем значение параметра $F$ согласно формуле \eqref{dev_equal_test}:
	$$F=\frac{5.3^2}{4.3^2}=1.519$$
	
	По таблице квантилей распределения Фишера находим $f_{1-0.05;41-1;81-1}=f_{0.95;40;80}=1.545$. 
	
	Поскольку $1.519<1.545$, нулевая гипотеза принимается, дисперсии активов равны.
\end{solution}

\question[20] Доходность активов имеет нормальное распределение. Доходности активов \textit{X} и \textit{Y} за 10 периодов представлены в таблице.
\begin{table}[htbp]
	\centering
	\begin{tabular}{llrrrrrrrrrr}
		\multicolumn{2}{c}{\multirow{2}[1]{*}{}} & \multicolumn{10}{c}{Период} \\\cmidrule{3-12}
		\multicolumn{2}{c}{} & 1     & 2     & 3     & 4     & 5     & 6     & 7     & 8     & 9     & 10 \\
		\midrule
		\multicolumn{1}{l}{\multirow{2}[1]{*}{Доходность актива}} & X     & 18    & 11    & 8     & 13    & 18    & 25    & 16    & 7     & 10    & 8 \\
		& Y     & -4    & 15    & 18    & 1     & 17    & -2    & 10    & -3    & -5    & 3 \\
	\end{tabular}%
	\label{tab:addlabel}%
\end{table}%
\noaddpoints
\begin{subparts}
	\subpart[10] Определить коэффициент выборочной корреляции доходности активов.
	
	\begin{solution}[6em]
		Коэффициент выборочной ковариации  определяется по формуле:
		\begin{align*}
		cov_{XY}=\frac{\sum_{i=1}^n \left(r_{X_i}-\overline{r} \right)\left(r_{Y_i}-\overline{r} \right)}{n},
		\end{align*}
		где 
		
		$r_{X_i}, r_{Y_i}$ - соответственно, доходности активов \textit{X} и \textit{Y}.
		
		Коэффициент выборочной ковариации равен -5.1.
		
		Выборочные стандартные отклонения активов составляют 5.481, 8.672.
		
		Коэффициент корреляции -0,107.
	\end{solution}
	
	\subpart[10] Проверить гипотезу о значимости коэффициента корреляции при уровне значимости 0.05.
	
	\begin{solution}[6em]
		
		\raggedright
		Основная гипотеза говорит о том, что коэффициент корреляции равен нулю. Альтернативная гипотеза говорит о том, что коэффициент корреляции не равен нулю.
		
		Для проверки нулевой гипотезы на основе выборки строится статистика:
		\begin{align}
		\label{studentxy}
		T=\frac{corr_{XY}\sqrt{n-2}}{\sqrt{1-corr_{XY}^2}}
		\end{align}
		Величина \textit{Т} имеет распределение Стьюдента с \textit{n-2} степенями свободы. Если рассчитанное на основе \eqref{studentxy} значение критерия \textit{Т }принадлежит критической области, то нулевую гипотезу отклоняют. 
		При условии $\left|\frac{corr_{XY}\sqrt{n-2}}{\sqrt{1-corr_{XY}^2}}\right|<t_{1-\alpha/2;n-2}$ параметр \textit{T} попадает в область принятия нулевой гипотезы, следовательно, нулевая гипотеза принимается на уровне значимости $\alpha$.
		
		При условии $\left|\frac{corr_{XY}\sqrt{n-2}}{\sqrt{1-corr_{XY}^2}}\right|\geq t_{1-\alpha/2;n-2}$ параметр \textit{T} попадает в критическую область, следовательно нулевая гипотеза отклоняется в пользу альтернативной гипотезы на уровне значимости $\alpha$.
		
		\begin{align*}
			T=-\frac{0.107 \cdot \sqrt{8}}{\sqrt{1-\left(-0.107^2\right)}}=-0.305
		\end{align*}
		Значение параметра \textit{T} равно -0.305, по таблице квантилей распределения Стьюдента находим $t_{0.95;8}=2.306$.
		
		Значение критерия не попадает в критическую область. Поэтому мы принимаем нулевую гипотезу, т.е. коэффициент корреляции статистически не значим.
		
	\end{solution}
	
\end{subparts}
\addpoints

\end{questions}

\pagebreak
\noindent\textbf{(выберите один правильный ответ)}

\begin{questions}
\begin{multicols}{2}
\setlength{\columnsep}{1cm}

\question Укажите наиболее полное определение риска:
	 \begin{choices}
	 \CC Риск – это деятельность, связанная с преодолением неопределенности в ситуации неизбежного выбора, в процессе которой имеется возможность количественно и качественно оценить вероятность достижения предполагаемого результата, неудачи и отклонения от цели;
	 \choice Сочетание вероятности и последствий наступления неблагоприятных событий;
	 \choice Риск — характеристика ситуации, имеющей неопределённость исхода, при обязательном наличии неблагоприятных последствий;
	 \choice Риск — это вероятность возможной нежелательной потери чеголибо при плохом стечении обстоятельств.
	 \end{choices}
\question Особенностью чистых рисков является то, что
	 \begin{choices}
	 \choice они возникают только в сфере налогообложения;
	 \choice они несут в себе как потери, так и прибыли для предпринимательской деятельности;
	 \CC их причинами могут быть стихийные бедствия, несчастные случаи, недееспособность руководителей фирм и тп;
	 \choice их причинами могут быть изменение конъюнктуры рынка, курсов валют и тп.
	 \end{choices}
\question В зависимости от сферы деятельности выделяют следующие виды рисков:
	 \begin{choices}
	 \choice производственный;
	 \choice коммерческий;
	 \choice финансовый;
	 \CC всё вышеперечисленное.
	 \end{choices}
\question Под критическим риском понимают:
	 \begin{choices}
	 \choice уровень риска в пределах его среднего уровня;
	 \CC риск, уровень которого выше среднего, но в пределах максимально допустимых значений в данной экономической системе;
	 \choice риск, который превышает максимально допустимую границу;
	 \choice верны варианты Б и В.
	 \end{choices}
\question Неопределенность – это:
	 \begin{choices}
	 \CC неполное или неточное представление о значениях различных параметров в будущем, порождаемых различными причинами и, прежде всего, неполнотой или неточностью информации об условиях реализации решения, в том числе о затратах и результатах;
	 \choice возможность возникновения в ходе реализации решения неблагоприятных ситуаций и последствий;
	 \choice ситуация, когда последствия принимаемых решений являются неконтролируемыми;
	 \choice ситуация, когда последствия принимаемых решений являются неизвестными.
	 \end{choices}
\question Под инвестиционным портфелем понимается
	 \begin{choices}
	 \choice совокупность рискованных активов могущих приносить как доход так и убыток;
	 \CC целенаправленно сформированная в соответствии с определенной инвестиционной стратегией совокупность вложений в инвестиционные объекты;
	 \choice вложения в финансовые инструменты, торгуемые на высоко ликвидных рынках;
	 \choice любые вложения, связанные с риском, имеющие диверсификацию и потенциал увеличения первоначального капитала.
	 \end{choices}
\question Портфель роста формируется из
	 \begin{choices}
	 \choice высокорисковых финансовых инструментов;
	 \CC акций быстрорастущих компаний;
	 \choice краткосрочных спекулятивных позиций на фондовом рынке;
	 \choice производных финансовых инструментов.
	 \end{choices}
\question Оценить риск инвестиционного портфеля при вводе в его состав актива без риска можно по формуле ($D_p$ – доля прежнего портфеля в формируемом, $\sigma$ – стандартное отклонение доходности):
	 \begin{choices}
	 \CC $\sigma_f=D_p \cdot \sigma_p$;
	 \choice $\sigma_f=\frac{D_p}{\sigma_p}$;
	 \choice риск инвестиционного портфеля не меняется при вводе в его состав актива без риска;
	 \choice для решения указанной задачи требуется решение матричных уравнений.
	 \end{choices}
\question Коэффициент Шарпа …
	 \begin{choices}
	 \choice учитывает доходность портфеля, полученную сверх ставки без риска, и весь риск:;
	 \choice $\frac{r_p - r_f}{\sigma_p}$ , где $r_p$ – средняя доходность портфеля ценных бумаг за рассматриваемый период $r_f$ – средняя ставка без риска за данный период $\sigma_p$ – стандартное отклонение доходности портфеля;
	 \choice целесообразно использовать для оценки эффективности управления менее диверсифицированных портфелей;
	 \CC всё вышеперечисленное верно.
	 \end{choices}

\end{multicols}
\end{questions}

%\vfill\null\cleardoublepage
\end{document}
