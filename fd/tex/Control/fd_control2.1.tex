% !TeX program = lualatex -synctex=1 -interaction=nonstopmode --shell-escape %.tex

\documentclass[12pt, table, twoside, a4paper]{exam}
\usepackage[rus]{borochkin}

\usepackage{borochkin_exam}

%%%%%%%%%%%%%%%%%%%%%%%%%%%%%%%%%%%%%%%%%

\professor
\iftagged{professor}{ \printanswers }

%%%%%%%%%%%%%%%%%%%%%%%%%%%%%%%%%%%%%%%%%



\begin{document}
\setcounter{section}{0\relax}%

\noindent
% Контр/р № #1, Вариант #2, Предмет #3
\studentpersonalinfo{2}{1}{МОФР}

\normalsize
\begin{questions}
\question[20] Коэффициент выборочной корреляции доходности активов равен -0.107. Объем выборки составляет 10 наблюдений. Найти доверительные интервалы для коэффициента корреляции с доверительной вероятностью 0.99. 

\begin{solution}[6em]
	
	\raggedright
	При малых объемах выборки верхнюю и нижнюю границы доверительного интервала для коэффициента корреляции можно определить по следующим формулам:
	\begin{align}
	corr_L&=\mathsf{th}(z_L); corr_U=\mathsf{th}(z_U);\\
	\mathsf{th}(z)&=\frac{e^z-e^{-z}}{e^z+e^{-z}}\\
	z_L&=0.5\ln \frac{1+corr_{XY}}{1-corr_{XY}}-\frac{corr_{XY}}{2(n-1)}-\frac{u_{1-\alpha/2}}{\sqrt{n-3}}
	\\
	z_U&=0.5\ln \frac{1+corr_{XY}}{1-corr_{XY}}-\frac{corr_{XY}}{2(n-1)}+\frac{u_{1-\alpha/2}}{\sqrt{n-3}}
	\end{align}
	где
	
	$corr_{XY}$ - точечная оценка коэффициента корреляции на основе осуществленной выборки;
	
	\textit{n} - объем выборки;
	
	$\mathsf{th}(z)$ - гиперболический тангенс;
	
	$u_{1-\alpha/2}$ - квантиль уровня $1-\alpha/2$ стандартного нормального распределения;
	
	$\alpha$ - уровень значимости, соответствующий выбранной доверительной вероятности~$p$.
	
	По таблице квантилей нормального распределения находим $u_{0.995}=2.576$.
	
	$z_L=-1.075; z_U=0.872$
	
	$th(z_L)=-0.791; th(z_U)=0.702$
	
	Таким образом, доверительный интервал для коэффициента корреляции с уровнем доверия 0,99 равен: -0.791 0.702. т.е. данный интервал с вероятностью 99\% накрывает истинное значение коэффициента корреляции.
\end{solution}

\question[10] Выборочный коэффициент корреляции равен 0.65. Он был определен на основе 5000 наблюдений. Найти доверительный интервал для коэффициента корреляции с доверительной вероятностью 0.99.

\begin{solution}[6em]
	
	\raggedright
	При объемах выборки не менее 500 верхнюю и нижнюю границы доверительного интервала для коэффициента корреляции можно определить по следующим формулам:
	\begin{align}
	corr_L&=corr_{XY}+\frac{corr_{XY}(1-corr_{XY}^2)}{2n}-u_{1-\alpha/2}\frac{1-corr_{XY}^2}{\sqrt{n}}\\
	corr_U&=corr_{XY}+\frac{corr_{XY}(1-corr_{XY}^2)}{2n}+u_{1-\alpha/2}\frac{1-corr_{XY}^2}{\sqrt{n}}
	\end{align}
	где
	
	$corr_{XY}$ - оценка коэффициента корреляции на основе осуществленной выборки;
	
	\textit{n} - объем выборки;
	
	$\alpha$ - уровень значимости, соответствующий выбранной доверительной вероятности~$p$;
	
	$u_{1-\alpha/2}$ - квантиль уровня $1-\alpha/2$ стандартного нормального распределения.
	
	$u_{1-\alpha/2}=2.5758$	
		
	Ответ: доверительный интервал для коэффициента корреляции с доверительной вероятностью~0.99 равен: 0.6290 0.6711, т.е. данный интервал с вероятностью 99\% накрывает истинное значение коэффициента корреляции.	
\end{solution}

\question[10] Российский инвестор купил акции компании \textit{А} на 0.5~M\$. Стандартное отклонение доходности акции в расчете на день составляет 2\%. Курс доллара 83~₽, стандартное отклонение валютного курса в расчете на один день 3.5\%. Коэффициент корреляции между курсом доллара и доходностью акции компании \textit{А} равен 0.8. Определить \textit{VaR }портфеля инвестора в рублях с доверительной вероятностью 95\%.

\begin{solution}[6em]
	
	\raggedright
	Риск инвестора обусловлен двумя факторами: возможным падением котировок акций компании \textit{А} и падением курса доллара. Поэтому оба фактора риска должны учитываться при расчете \textit{VaR }портфеля.
	
	Рублевый эквивалент стоимости акций составляет 41.5~M₽.
	
	Дисперсия доходности портфеля с учетом валютного риска определяется по формуле 
	\begin{align}
	\sigma_p^2=\sigma_X^2+\sigma_Y^2+2\sigma_X\sigma_Y corr_{XY}.
	\end{align}
	и равна 0.2745, стандартное отклонение составит 5.2393\%.
	
	Однодневный \textit{VaR} портфеля будет 3.576~M₽.
\end{solution}

\pagebreak
\question[10] Определить величину средних ожидаемых потерь для одного дня для портфеля стоимостью 5~M₽, в который входят акции двух компаний. Уд. вес первой акции в стоимости портфеля составляет 20\%, второй - 80\%. Стандартное отклонение доходности первой акции в расчете на один день равно 2.7\%, второй 0.6\%, коэффициент корреляции доходностей акций равен -0.5. Предполагается, что доходность акций имеет нормальное распределение. Доверительная вероятность равна 99\%.

\begin{solution}[12em]
	
	\raggedright
	Стандартное отклонение доходности портфеля 
	\begin{align}
	\label{riskXYcorr}
	\sigma_p^2=\theta_X^2\sigma_X^2+\theta_Y^2\sigma_Y^2+2\theta_X\theta_Y\sigma_X\sigma_Y corr_{XY}
	\end{align}
	
	равно 0,51\%.
	
	Стандартное отклонение дохода портфеля составляет 0.162~M₽.
	
	\textit{VaR} портфеля для уровня доверительной вероятности 99\% равен 0.026~M₽.
	
	Если мы предполагаем, что доходность портфеля имеет нормальное распределение со средним значением равным нулю и стандартным отклонением $\sigma$, тогда величина средних ожидаемых потерь вычисляется по формуле:
	\begin{align}
	\label{es_normal}
	ES=\frac{\sigma}{(p-1)\sqrt{2\pi}}e^{-\frac{VaR_{p}^2}{2\sigma^2}},
	\end{align}
	где 
	$\mbox{VaR}_{p}$ - это \textit{VaR} с доверительной вероятностью $p$,
	
	$\sigma$ - стандартное отклонение доходности портфеля в денежных единицах.
	
	Расчеты по формуле \eqref{es_normal} 
	$$ES_{p}=\frac{0.026M₽}{(0.99-1)\cdot \sqrt{2 \cdot \pi}}\cdot e^{-\frac{0.06^2~M₽^2}{2 \cdot 0.026^2~M₽^2}}=-0.068~M₽$$
\end{solution}

\question[10] Портфель состоит из акций пяти компаний, купленных на суммы   $V=[2\;3\;7\;8\;10]$~M₽. Бета акций относительно фондового индекса равны $\beta=[0.2\;0.5\;0.9\;1.5\;1.7]$. Стандартное отклонение рыночного портфеля для одного дня составляет 0.55\%. Определить однодневный  \textit{VaR }портфеля с доверительной вероятностью 99\% согласно методике, принятой в Рискметриках банка \textbf{J.P. Morgan}, на основе стандартных факторов риска. 

\begin{solution}[4em]
	
	\raggedright
	\begin{align}
	VaR_p=z_{p}\cdot \sigma_m\cdot  \beta_p\cdot  V_p,
	\end{align}
	
	Эту формулу можно представить как
	\begin{align}
	VaR_p=z_p \cdot \sigma_m \cdot \sum_{i=1}^n\beta_i \cdot V_i
	\end{align}
	где
	
	$V_i$ - стоимость акций \textit{i}-ой компании в портфеле.
	
	$z_p$ z-статистика.	
	
	\begin{align*}
		VaR_p&=2.326 \cdot 0.55\% \cdot \sum_{i=1}^n\beta_i \cdot V_i\\
		&=0.476~M₽
	\end{align*}
	
	
	\textit{VaR }портфеля равен 0.476~M₽.
\end{solution}

\end{questions}

%\vfill\null\cleardoublepage
\end{document}
