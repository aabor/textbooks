% !TeX program = lualatex -synctex=1 -interaction=nonstopmode --shell-escape %.tex

\documentclass[12pt, table, twoside, a4paper]{exam}
\usepackage[rus]{borochkin}

\usepackage{borochkin_exam}

%%%%%%%%%%%%%%%%%%%%%%%%%%%%%%%%%%%%%%%%%
\professor
\iftagged{professor}{ \printanswers }
%%%%%%%%%%%%%%%%%%%%%%%%%%%%%%%%%%%%%%%%%



\begin{document}
\setcounter{section}{0\relax}%

\noindent
% Контр/р № #1, Вариант #2, Предмет #3
\studentpersonalinfo{1}{2}{МОФР}

\normalsize
\begin{questions}

\question[9] Тест
\answerstotestShort
	
\question[20] Номинал облигации 25 000~₽, купон 2.5\%, выплачивается один раз в год. До погашения облигации 20 лет. Облигация стоит 20000~₽. 
\noaddpoints
\begin{subparts}
	\subpart[10] Определите ориентировочно доходность до погашения облигации.
	\begin{solution}[10em]
		\begin{align*}
		r^*&=\frac{(N-P)/m \cdot n + C/m }{(N+P)/2}\\
		r^*&=3,89\%
		\end{align*}
		где
		
		$r^*$ - ориентировочная доходность облигации;
		
		$N$ - номинал облигации;
		
		$P$ - рыночная цена облигации;
		
		$C$ - сумма купона;
		
		$n, m$ - срок облигации и количество выплат процентов в году, соответственно.
	\end{solution}
	
	\subpart[10] Определите доходность до погашения облигации методом линейной интерполяции.
	
	\begin{solution}[10em]
		\begin{align*}
		r_1&=9\%\\
		r_2&=10\%\\
		P_1&=\frac{C}{r_1} \cdot \left(1 - \frac{1}{(1+r_1)^n} \right) + \frac{N}{(1+r_1)^n}=19903,6~₽\\
		P_2&=\frac{C}{r_2} \cdot \left(1 - \frac{1}{(1+r_2)^n} \right) + \frac{N}{(1+r_1)^n}=23140,3~₽\\
		r^{**}&=r_1+(r_2-r_1) \cdot \frac{P_1-P}{P_1-P_2}=3,97\%
		\end{align*}	
		где
		
		$r_1, r_2$ - соответственно, округленная вверх и вниз ориентировочная доходность облигации;
		
		$P_1, P_2$ - соответственно, нижняя и верхняя границы рыночной цены облигации;
		
		$r^{**}$ - доходность до погашения облигации, определенная методом линейной интерполяции.
	\end{solution}
	
\end{subparts}
\addpoints
	
	\question[5] Средняя доходность фондового индекса равна 16\% годовых, стандартное отклонение доходности 35\%. Предполагается, что доходность имеет нормальное распределение. Инвестор формирует портфель, копирующий данный индекс. Определить вероятность того, что в следующем году портфель принесет ему убыток.
	
	\begin{solution}[6em]
		
		\raggedright
		Вероятность   попадания   доходности   актива   в   заданный интервал определяется по формуле 
		\begin{align}
		\label{prob_loss}
		P(\alpha<r<\beta)=\Phi\left(\frac{\beta-\overline{x}}{\sigma}\right)-\Phi\left(\frac{\alpha-\overline{x}}{\sigma}\right),
		\end{align}
		где 
		
		$\Phi(.) $ - кумулятивная функция нормального стандартного распределения случайной величины (доходности актива);
		
		$\alpha $ - нижняя граница рассматриваемого интервала;
		
		$\beta $ - верхняя граница рассматриваемого интервала;
		
		$\sigma $ - стандартное отклонение доходности актива.
		
		Инвестор получит убыток, если доходность портфеля окажется меньше нуля. Согласно формуле \eqref{prob_loss} вероятность того, что доходность актива окажется меньше нуля равна
		\begin{align*}
		P(-\infty<r<0)&=\Phi\left(\frac{0-16}{35}\right)-\Phi\left(\frac{-\infty-16}{35}\right)\\
		&=\Phi(-0.457)-\Phi(-\infty)\\
		&=0.3238-0\\
		&=32.38\%.
		\end{align*}
	\end{solution}
	
	\addpoints
	
	\question[10] Доходность актива имеет нормальное распределение. На основе наблюдений за 120 дней была определена ожидаемая доходность в расчете на день. Она составила 2.1\%. Пусть известно, что истинное значение стандартного отклонения доходности актива в расчете на день равно 5,5\%. В каком интервале с надежностью 0,95 располагается истинное значение ожидаемой доходности актива?
	
	\begin{solution}[8em]
		
	\raggedright
	По данным статистики было определено не истинное значение ожидаемой доходности актива, а ее точечная оценка на основе выборочных данных. Для получения ответа о надежности такой оценки определяют доверительный интервал, который с заданным уровнем вероятности накрывает точечную оценку. В результате, с заданным уровнем надежности можно быть уверенным, что действительное значение ожидаемой доходности актива лежит в границах рассчитанного интервала.
	
	В примере необходимо определить доверительный интервал для нормально распределенной случайной величины при известном значении ее истинного стандартного отклонения с коэффициентом доверия 0.95.
	
	Верхнюю и нижнюю границы доверительного интервала для математического ожидания с известной дисперсией можно определить по следующим формулам:
	\begin{align}
	\label{eq:yield_low}
	\overline{r}_L &=\overline{r}-\frac{\sigma}{\sqrt{n}}u_{1-\alpha/2},\\[8pt]
	\label{eq:yield_upper}
	\overline{r}_U &=\overline{r}+\frac{\sigma}{\sqrt{n}}u_{1-\alpha/2},
	\end{align}
	где
	
	$\overline{r}_L, \overline{r}_U$ - нижняя и верхняя  границы доверительного интервала;
	
	$\overline{r}$ - точечная оценка ожидаемой доходности на основе осуществленной выборки;
	
	$\sigma$ - истинное значение стандартного отклонения доходности актива;
	
	$n$ - объем выборки;
	
	$u_{1-\alpha/2}$ - квантиль уровня $1-\alpha/2$ стандартного нормального распределения;
	
	$\alpha$ - уровень значимости, соответствующий выбранной доверительной вероятности $\gamma$.
	
	Из соотношения $\alpha=1-\gamma$ находим значение $\alpha$, соответствующее коэффициенту доверия 95\%.
	$$\alpha=1-0.95=0.05$$.
	
	По таблице квантилей нормального распределения находим квантили $u_{1-\alpha/2}=u_{0.975}=1.96$.
	
	По формулам \eqref{eq:yield_low} \eqref{eq:yield_upper} получаем:

	\begin{align*}
	\overline{r}_L&=2.1\% - \frac{5.5\%}{\sqrt{120}} \cdot 1.96
	= 1.116\%, \\
	\overline{r}_U&=2.1\% + \frac{5.5\%}{\sqrt{120}} \cdot 1.96 =3.084\%
	\end{align*}
		
	\end{solution}
	
	
	\question[10] Доходность актива имеет нормальное распределение. Данные о его доходности за прошедшие 9 месяцев представлены в таблице
	
	\begin{tabularx}{\linewidth}[b]{@{}>{\raggedright\arraybackslash}Xrrrrrrrrr@{}}
		\toprule
		Месяцы           & 1     & 2     & 3     & 4     & 5     & 6     & 7     & 8     & 9 \\
		\midrule
		Доходность актива & 7     & 27    & 5     & -4    & -7    & 5     & -12   & -15   & -8 \\\bottomrule
	\end{tabularx}%
	
	Определите ожидаемую доходность актива. В каком интервале с надежностью 0,95 располагается истинное значение ожидаемой доходности актива?
	
	\begin{solution}[8em]
		
	\raggedright
	Ожидаемая доходность актива равна
	$$\overline{r}=-0.222\%,$$
	В данном примере не известно истинное значение стандартного отклонения актива, поэтому при определении доверительного интервала используем правило математической статистики для математического ожидания нормально распределенной случайной 
	величины при неизвестной дисперсии.
	
	Верхнюю и нижнюю границы доверительного интервала для математического ожидания с неизвестной дисперсией можно определить по следующим формулам:
	\begin{align}
	\label{yield_low_corrected}
	\overline{r}_L &=\overline{r}-\frac{s}{\sqrt{n}}t_{1-\alpha/2;n-1},\\[8pt]
	\label{yield_upper_corrected}
	\overline{r}_U &=\overline{r}+\frac{s}{\sqrt{n}}t_{1-\alpha/2;n-1},
	\end{align}
	где
	
	$s$- исправленное стандартное отклонение;
	
	$\alpha$ - уровень значимости, соответствующий выбранной доверительной вероятности $\gamma$, $\alpha=1-\gamma$.
	
	$t_{1-\alpha/2;n-1}$- квантиль уровня $1-\alpha/2$ распределения Стьюдента с $n-1$ степенями свободы.
	
	Исправленное стандартное отклонение равно:
	\begin{align}
	\label{st_dev_corrected}
	s&=\sqrt{\frac{1}{n-1}\sum_{i=1}^n \left(r_i-\overline{r}\right)^2}\\
	s&=12.872\% \nonumber
	\end{align}
	По  таблице квантилей распределения Стьюдента находим
	значение статистики $t_{1-\alpha/2;n-1}=t_{0.95;9}=2.306$ и по формулам \eqref{yield_low_corrected} и \eqref{yield_upper_corrected} определяем границы доходности актива:
	\begin{align*}
		\overline{r}_L &= -0.222\% - \frac{12.872\%}{\sqrt{9}} \cdot 2.306 =-10.117\%,\\
		\overline{r}_U &=-0.222\% + \frac{12.872\%}{\sqrt{9}} \cdot 2.306 = 9.672\%.
	\end{align*}
	
	Доверительный интервал равен: -10.117\%; 9.672\%.
	\end{solution}
	
	\pagebreak
	\question[10] Доходность актива имеет нормальное распределение. На основе наблюдений за 61 день было рассчитано исправленное стандартное отклонение в расчете на день. Оно составило 1,5\%. В каком интервале с надежностью 0.90 располагается истинное значение дисперсии и стандартного отклонения доходности актива?
	
	\begin{solution}[8em]
		\raggedright
		Верхнюю и нижнюю границы доверительного интервала для дисперсии можно определить по следующим формулам:
		
		\begin{align}
		\sigma_L^2&=\frac{s^2(n-1)}{\chi_{1-\alpha/2;n-1}^2};\\[8pt]
		\sigma_U^2&=\frac{s^2(n-1)}{\chi_{\alpha/2;n-1}^2};
		\end{align}
		где
		
		$\sigma_L^2$, $\sigma_U^2$ - нижняя и верхняя границы доверительного интервала, соответственно;
		
		$s^2$ - исправленная дисперсия доходности актива;
		
		$n$ - объем выборки;
		
		$\alpha$ - уровень значимости, соответствующий выбранной доверительной вероятности $\gamma$, $\alpha=1-\gamma$. 
		
		$\chi_{\alpha/2;n-1}^2$ - $\alpha/2$-квантиль распределения хи-квадрат с $n-1$ степенями свободы.
		
		Значение $\alpha$, соответствующее коэффициенту доверия 99\%.
		$$\alpha=1-0.90=0.1.$$
		
		Количество наблюдений случайной величины составило 61 день. Поэтому количество степеней свободы в примере равно 60. По таблице квантилей распределения $\chi^2$ находим квантили 79.082 43.188.
		
		\begin{align*}
			\sigma^2_L&=\frac{1.5\%^2 \cdot 60}{79.082}=1.707\\
			\sigma^2_U&=\frac{1.5\%^2 \cdot 60}{43.188}=3.126
		\end{align*}
		
		
		Границы доверительного интервала дисперсии доходности актива равны 1.707 3.126 и стандартного отклонения доходности актива равны 1.307\% и 1.768\%.
	\end{solution}
	
	\question[10] Доходности активов имеют нормальное распределение. На основе данных о доходности активов \textit{X} за 21 день и \textit{Y} за 101 день были рассчитаны исправленные стандартные отклонения доходности: $s_X=0.69\%$, $s_Y = 0.56\%$. Проверить гипотезу о равенстве дисперсий активов при уровне значимости 0,1.
	
	\begin{solution}[6em]
		
		\raggedright
		При проверке гипотезы о равенстве дисперсий рассматривают две гипотезы: $H_0: \sigma_X^2=\sigma_Y^2$, и $H_1: \sigma_X^2 > \sigma_Y^2$, $H_0$ - это основная (нулевая) гипотеза, $H_1$, - альтернативная гипотеза. 
		Основная гипотеза говорит о том, что дисперсии доходности активов равны. Альтернативная гипотеза говорит о том, что они не равны. Если в результате проверки гипотеза $H_0$ отклоняется в пользу гипотезы $H_1$, то это означает, что дисперсии активов отличаются. Если гипотеза $H_0$ не отклоняется, то нет оснований отрицать равенство дисперсий.
		
		В качестве критерия проверки гипотезы используют случайную величину, которая представляет собой отношение большей исправленной дисперсии к меньшей:
		\begin{align}
		\label{dev_equal_test}
		F=\frac{s_X^2}{s_Y^2}
		\end{align}
		где
		$F$ - случайная величина, имеющая распределение Фишера со степенями свободы $n-1$ и $k-1$;
		
		$n$ - объем выборки доходности актива~$X$;
		
		$k$ - объем выборки доходности актива~$Y$.
		
		Если $F<f_{1-\alpha;n-1;k-1}$, где $f_{1-\alpha;n-1;k-1}$ - квантиль распределения Фишера, то параметр $F$ попадает в область принятия гипотезы, и нулевая гипотеза принимается на уровне значимости $\alpha$.
		
		Если  $F>=f_{1-\alpha;n-1;k-1}$ параметр $F$ попадает в критическую область. Следовательно $H_0$, отклоняется в пользу гипотезы $H_1$.
		
		Рассчитаем значение параметра $F$ согласно формуле \eqref{dev_equal_test}:
		$$F=\frac{0.69^2}{0.56^2}=1.518$$
		
		По таблице квантилей распределения Фишера находим $f_{1-0.1;21-1;101-1}=f_{0.9;20;100}=1.494$. 
		
		Поскольку $1.518>1.494$, нулевая гипотеза не принимается, дисперсии активов не равны.
	\end{solution}
	
	\question[20] Доходность активов имеет нормальное распределение. Доходности активов \textit{X} и \textit{Y} за 9 периодов представлены в таблице.
	% Table generated by Excel2LaTeX from sheet 'З1_6'
	\begin{table}[htbp]
		\centering
		\begin{tabular}{rlrrrrrrrrr}
			\toprule
			\multicolumn{1}{l}{Месяцы} &       & 1     & 2     & 3     & 4     & 5     & 6     & 7     & 8     & 9 \\\midrule
			\multicolumn{1}{l}{Доходности} & X     & 7     & 27    & 5     & -4    & -7    & 5     & -12   & -15   & -8 \\
			& Y     & 14    & 0     & 32    & -11   & 5     & 8     & 4     & -3    & -2 \\\bottomrule
		\end{tabular}%
		\label{tab:addlabel}%
	\end{table}%

	\noaddpoints
	\begin{subparts}
		\subpart[10] Определить коэффициент выборочной корреляции доходности активов.
		
		\begin{solution}[6em]
			Коэффициент выборочной ковариации  определяется по формуле:
			\begin{align*}
			cov_{XY}=\frac{\sum_{i=1}^n \left(r_{X_i}-\overline{r} \right)\left(r_{Y_i}-\overline{r} \right)}{n},
			\end{align*}
			где 
			
			$r_{X_i}, r_{Y_i}$ - соответственно, доходности активов \textit{X} и \textit{Y}.
			
			Коэффициент выборочной ковариации равен 36,716.
			
			Выборочные стандартные отклонения активов составляют 12.136 11.612.
			
			Коэффициент корреляции 0.261.
		\end{solution}
		
		\subpart[10] Проверить гипотезу о значимости коэффициента корреляции при уровне значимости 0.5.
		
		\begin{solution}[6em]
			
			\raggedright
			Основная гипотеза говорит о том, что коэффициент корреляции равен нулю. Альтернативная гипотеза говорит о том, что коэффициент корреляции не равен нулю.
			
			Для проверки нулевой гипотезы на основе выборки строится статистика:
			\begin{align}
			\label{studentxy}
			T=\frac{corr_{XY}\sqrt{n-2}}{\sqrt{1-corr_{XY}^2}}
			\end{align}
			Величина \textit{Т} имеет распределение Стьюдента с \textit{n-2} степенями свободы. Если рассчитанное на основе \eqref{studentxy} значение критерия \textit{Т }принадлежит критической области, то нулевую гипотезу отклоняют. 
			При условии $\left|\frac{corr_{XY}\sqrt{n-2}}{\sqrt{1-corr_{XY}^2}}\right|<t_{1-\alpha/2;n-2}$ параметр \textit{T} попадает в область принятия нулевой гипотезы, следовательно, нулевая гипотеза принимается на уровне значимости $\alpha$.
			
			При условии $\left|\frac{corr_{XY}\sqrt{n-2}}{\sqrt{1-corr_{XY}^2}}\right|\geq t_{1-\alpha/2;n-2}$ параметр \textit{T} попадает в критическую область, следовательно нулевая гипотеза отклоняется в пользу альтернативной гипотезы на уровне значимости $\alpha$.
			
			\begin{align*}
				T=-\frac{0.261 \cdot \sqrt{7}}{\sqrt{1-0.261^2}}=0.714
			\end{align*}

			Значение параметра \textit{T} равно 0.714, по таблице квантилей распределения Стьюдента находим $t_{0.5;7}=0.711$.
			
			Значение критерия попадает в критическую область. Поэтому мы не принимаем нулевую гипотезу, т.е. коэффициент корреляции статистически значим.
			
		\end{solution}
		
	\end{subparts}
	\addpoints
	
\end{questions}

\pagebreak
\noindent\textbf{(выберите один правильный ответ)}

\begin{questions}
	\begin{multicols}{2}
		\setlength{\columnsep}{1cm}
		
		\question Чем отличается риск от неопределенности:
		\begin{choices}
			\CC В ситуации неопределенности отсутствует какаялибо информация о последствиях возможного события;
			\choice В случае риска имеются вероятностные характеристики неконтролируемых переменных, при неопределенности такие характеристики отсутствуют;
			\choice В случае рискованной ситуации существует возможность выделить возможные варианты событий и количественно оценить их вероятность;
			\choice Верны ответы Б и В.
		\end{choices}
		\question Какие риски НЕ связаны с покупательной способностью денег:
		\begin{choices}
			\choice инфляционные;
			\choice инвестиционные;
			\CC дефляционные;
			\choice валютные.
		\end{choices}
		\question К каким видам риска относится риск упущенной выгоды:
		\begin{choices}
			\choice финансовый;
			\choice инвестиционный;
			\choice валютный;
			\CC транспортный.
		\end{choices}
		\question Выделяют следующие виды неопределенностей в зависимости от времени возникновения:
		\begin{choices}
			\choice ретроспективные;
			\CC текущие;
			\choice перспективные;
			\choice всё вышеперечисленное.
		\end{choices}
		\question Что НЕ используется в качестве методов оценки экономических рисков:
		\begin{choices}
			\CC теория вероятностей;
			\choice теория игр;
			\choice экспертные подходы;
			\choice суждение на основе личного опыта.
		\end{choices}
		\question Целями инвестиционного портфеля являются:
		\begin{choices}
			\choice максимизация роста капитала;
			\CC максимизация роста дохода;
			\choice минимизация инвестиционных рисков;
			\choice любая цель из перечисленных, при условии, что она обозначена в инвестиционной стратегии.
		\end{choices}
		\question Как связаны между собой ковариация и коэффициент корреляции доходностей активов:
		\begin{choices}
			\choice эти понятия эквивалентны;
			\CC коэффициент корреляции равен корню квадратному из ковариации;
			\choice ковариация доходностей двух активов равна коэффициенту корреляции умноженному на стандартные отклонения доходностей каждого из двух активов;
			\choice коэффициент корреляции доходностей двух активов равен ковариации умноженной на стандартные отклонения доходностей каждого из двух активов.
		\end{choices}
		\question Согласно подходу Capital Asset Pricing Model, CAPM, риск инвестиций измеряется величиной $\beta$ по формуле $cov_{iM}$ – ковариация доходности акции с доходностью рыночного индекса $corr_{iM}$  коэффициент корреляции доходности акции с доходностью рыночного индекса $\sigma_i$ – стандартное отклонение доходности акции $\sigma_M$ – стандартное отклонение доходности рыночного индекса):
		\begin{choices}
			\CC $\beta_i=\frac{cov_{iM}}{\sigma_M^2}$;
			\choice $\beta_i=\frac{\sigma_i}{\sigma_M} \cdot corr_{iM}$;
			\choice $\beta$ – коэффициент публикуется поставщиком биржевых котировок;
			\choice верны два первых ответа.
		\end{choices}
		\question Коэффициент Трейнора …
		\begin{choices}
			\choice это отношение средней доходности, превышающей безрисковую процентную ставку, к систематическому риску $\beta$;
			\choice применяется для оценки эффективности управления мало диверсифицированного портфеля;
			\choice в отличие от коэффициента Шарпа, коэффициент Трейнора соотносит доходность не с общим риском, а только с риском отдельного набора финансовых инструментов;
			\CC всё вышеперечисленное верно.
		\end{choices}
		
	\end{multicols}
\end{questions}
\end{document}
