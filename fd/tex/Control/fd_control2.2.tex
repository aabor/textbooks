% !TeX program = lualatex -synctex=1 -interaction=nonstopmode --shell-escape %.tex

\documentclass[12pt, table, twoside, a4paper]{exam}
\usepackage[rus]{borochkin}

\usepackage{borochkin_exam}

%%%%%%%%%%%%%%%%%%%%%%%%%%%%%%%%%%%%%%%%%

\professor
\iftagged{professor}{ \printanswers }

%%%%%%%%%%%%%%%%%%%%%%%%%%%%%%%%%%%%%%%%%



\begin{document}
\setcounter{section}{0\relax}%

\noindent
% Контр/р № #1, Вариант #2, Предмет #3
\studentpersonalinfo{2}{2}{МОФР}

\normalsize
\begin{questions}
	\question[20] Коэффициент выборочной корреляции доходности активов равен 0.261. Объем выборки составляет 9 наблюдений. Найти доверительные интервалы для коэффициента корреляции с доверительной вероятностью 0.98. 
	
	\begin{solution}[6em]
		
		\raggedright
		При малых объемах выборки верхнюю и нижнюю границы доверительного интервала для коэффициента корреляции можно определить по следующим формулам:
		\begin{align}
		corr_L&=\mathsf{th}(z_L); corr_U=\mathsf{th}(z_U);\\
		\mathsf{th}(z)&=\frac{e^z-e^{-z}}{e^z+e^{-z}}\\
		z_L&=0.5\ln \frac{1+corr_{XY}}{1-corr_{XY}}-\frac{corr_{XY}}{2(n-1)}-\frac{u_{1-\alpha/2}}{\sqrt{n-3}}
		\\
		z_U&=0.5\ln \frac{1+corr_{XY}}{1-corr_{XY}}-\frac{corr_{XY}}{2(n-1)}+\frac{u_{1-\alpha/2}}{\sqrt{n-3}}
		\end{align}
		где
		
		$corr_{XY}$ - точечная оценка коэффициента корреляции на основе осуществленной выборки;
		
		\textit{n} - объем выборки;
		
		$\mathsf{th}(z)$ - гиперболический тангенс;
		
		$u_{1-\alpha/2}$ - квантиль уровня $1-\alpha/2$ стандартного нормального распределения;
		
		$\alpha$ - уровень значимости, соответствующий выбранной доверительной вероятности~$p$.
		
		По таблице квантилей нормального распределения находим $u_{0.995}=2.326$.
		
		$z_L=-0.699; z_U=1.2$
		
		$th(z_L)=-0.604; th(z_U)=0.834$
		
		Таким образом, доверительный интервал для коэффициента корреляции с уровнем доверия 0,98 равен: -0.604 0.834 т.е. данный интервал с вероятностью 98\% накрывает истинное значение коэффициента корреляции.
	\end{solution}
	
	\question[10] Выборочный коэффициент корреляции равен 0.9. Он был определен на основе 550 наблюдений. Найти доверительный интервал для коэффициента корреляции с доверительной вероятностью 0.98.
	
	\begin{solution}[6em]
		
		\raggedright
		При объемах выборки не менее 500 верхнюю и нижнюю границы доверительного интервала для коэффициента корреляции можно определить по следующим формулам:
		\begin{align}
		corr_L&=corr_{XY}+\frac{corr_{XY}(1-corr_{XY}^2)}{2n}-u_{1-\alpha/2}\frac{1-corr_{XY}^2}{\sqrt{n}}\\
		corr_U&=corr_{XY}+\frac{corr_{XY}(1-corr_{XY}^2)}{2n}+u_{1-\alpha/2}\frac{1-corr_{XY}^2}{\sqrt{n}}
		\end{align}
		где
		
		$corr_{XY}$ - оценка коэффициента корреляции на основе осуществленной выборки;
		
		\textit{n} - объем выборки;
		
		$\alpha$ - уровень значимости, соответствующий выбранной доверительной вероятности~$p$;
		
		$u_{1-\alpha/2}$ - квантиль уровня $1-\alpha/2$ стандартного нормального распределения.
		
		$u_{1-\alpha/2}=2.326$	
		
		Ответ: доверительный интервал для коэффициента корреляции с доверительной вероятностью~0.98 равен: 0.8813 0.9190
		, т.е. данный интервал с вероятностью 98\% накрывает истинное значение коэффициента корреляции.	
	\end{solution}
	
	\question[10] Российский инвестор купил акции компании \textit{А} на 0.125~M\$. Стандартное отклонение доходности акции в расчете на день составляет 1.5\%. Курс доллара 75~₽, стандартное отклонение валютного курса в расчете на один день 0.7\%. Коэффициент корреляции между курсом доллара и доходностью акции компании \textit{А} равен 0.4. Определить \textit{VaR }портфеля инвестора в рублях с доверительной вероятностью 90\%.
	
	\begin{solution}[6em]
		
		\raggedright
		Риск инвестора обусловлен двумя факторами: возможным падением котировок акций компании \textit{А} и падением курса доллара. Поэтому оба фактора риска должны учитываться при расчете \textit{VaR }портфеля.
		
		Рублевый эквивалент стоимости акций составляет 9.375~M₽.
		
		Дисперсия доходности портфеля с учетом валютного риска определяется по формуле 
		\begin{align}
		\sigma_p^2=\sigma_X^2+\sigma_Y^2+2\sigma_X\sigma_Y corr_{XY}.
		\end{align}
		и равна 0,0358, стандартное отклонение составит 1,8921\%.
		
		Однодневный \textit{VaR} портфеля будет 0.227~M₽.
	\end{solution}
	
	
	\pagebreak
	\question[10] Определить величину средних ожидаемых потерь для одного дня для портфеля стоимостью 15~M₽, в который входят акции двух компаний. Уд. вес первой акции в стоимости портфеля составляет 35\%, второй - 65\%. Стандартное отклонение доходности первой акции в расчете на один день равно 3.8\%, второй 1.6\%, коэффициент корреляции доходностей акций равен 0.9. Предполагается, что доходность акций имеет нормальное распределение. Доверительная вероятность равна 90\%.
	
	\begin{solution}[12em]
		
		\raggedright
		Стандартное отклонение доходности портфеля 
		\begin{align}
		\sigma_p^2=\theta_X^2\sigma_X^2+\theta_Y^2\sigma_Y^2+2\theta_X\theta_Y\sigma_X\sigma_Y corr_{XY}
		\end{align}
		
		равно 2.31\%.
		
		Стандартное отклонение дохода портфеля составляет 0.347~M₽.
		
		\textit{VaR} портфеля для уровня доверительной вероятности 90\% равен 0.444~M₽.
		
		Если мы предполагаем, что доходность портфеля имеет нормальное распределение со средним значением равным нулю и стандартным отклонением $\sigma$, тогда величина средних ожидаемых потерь вычисляется по формуле:
		\begin{align}
		\label{es_normal}
		ES=\frac{\sigma}{(p-1)\sqrt{2\pi}}e^{-\frac{VaR_{p}^2}{2\sigma^2}},
		\end{align}
		где 
		$\mbox{VaR}_{p}$ - это \textit{VaR} с доверительной вероятностью $p$,
		
		$\sigma$ - стандартное отклонение доходности портфеля в денежных единицах.
		
		Расчеты по формуле \eqref{es_normal} 
		$$ES_{p}=\frac{0.347M₽}{(0.9 - 1)\cdot \sqrt{2 \cdot \pi}}\cdot e^{-\frac{0.444^2~M₽^2}{2 \cdot 0.347^2~M₽^2}}=-0.608~M₽$$		
	\end{solution}
	
	\question[10] Портфель состоит из акций пяти компаний, купленных на суммы   $V=[20\;40\;80\;90\;120]$~M₽. Бета акций относительно фондового индекса равны $\beta=[0.8\;0.85\;0.95\;1.1\;1.4]$. Стандартное отклонение рыночного портфеля для одного дня составляет 1.20\%. Определить однодневный  \textit{VaR }портфеля с доверительной вероятностью 95\% согласно методике, принятой в Рискметриках банка \textbf{J.P. Morgan}, на основе стандартных факторов риска. 
	
	\begin{solution}[4em]
		
		\raggedright
		\begin{align}
		VaR_p=z_{p}\cdot \sigma_m\cdot  \beta_p\cdot  V_p,
		\end{align}
		
		Эту формулу можно представить как
		\begin{align}
		VaR_p=z_p \cdot \sigma_m \cdot \sum_{i=1}^n\beta_i \cdot V_i
		\end{align}
		где
		$V_i$ - стоимость акций \textit{i}-ой компании в портфеле.
	
		$z_p$ z-статистика.	
		
		\begin{align*}
			VaR_p&=1.645 \cdot 1.20\% \cdot \sum_{i=1}^n\beta_i \cdot V_i\\
			&=7,757~M₽
		\end{align*}

		\textit{VaR }портфеля равен 7,757~M₽.
	\end{solution}
	
\end{questions}

\end{document}
