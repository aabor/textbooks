% !TeX program = lualatex -synctex=1 -interaction=nonstopmode --shell-escape %.tex

\documentclass[12pt, table]{exam}
\usepackage[rus]{borochkin}

\usepackage{borochkin_exam}

%%%%%%%%%%%%%%%%%%%%%%%%%%%%%%%%%%%%%%%%%

\professor
\iftagged{professor}{ \printanswers }
%%%%%%%%%%%%%%%%%%%%%%%%%%%%%%%%%%%%%%%%%


\begin{document}
\setcounter{section}{0\relax}%
\noindent
% Контр/р № #1, Вариант #2, Предмет #3
\studentpersonalinfo{1}{2}{ОВИ}
\normalsize

\begin{questions}
\question[40] Тест
\answerstotest
	
\question[20]
Венчурный инвестор профинансировал стартап XXX, купив конвертируемые привилегированные акции серии А на сумму 1~M₽, а вклад предпринимателя состоял лишь из интеллектуальной собственности. Предприниматель владеет только обыкновенными акциями. Собственность на компанию была разделена в соотношении 50\%/50\%. Цена покупки привилегированных акций составила 1~₽ за штуку.
Для защиты интересов инвесторов в акции предыдущих серий, решением Совета Директоров компании ХХХ в проспекте эмиссии на акции серии А было указано, что в случае выпуска новой серии акции по меньшей совокупной цене, цена конвертации предыдущих серий привилегированных акций определяется методом средне-взвешенного отката.
Через год после начала инвестиций, предпринимателю потребовалось дополнительное финансирование на сумму  25~000~₽.
Для этих целей были выпущены конвертируемые привилегированные акции серии Б.
\noaddpoints
\pagebreak
\begin{subparts}
\subpart[10] Определите, при какой цене привилегированных акций Серии Б, потери инвестора в акции Серии А составят не более 35\% от первоначальных вложений.
\begin{solution}[18em]
	\begin{align}
	P_1=\frac{\frac{V_{A_1} \cdot (A + \frac{V_B}{P_0})}{A_0}-V_B}{A},
	\end{align}
	
	где
	
	$P_0$, $P_1$ - цена покупки акций Серии А и Серии Б, соответственно;
	
	$A$ - общее количество акций компании до выпуска акций Серии Б;
	
	$A_0$, $A_1$ - количество акций Серии А до и после выпуска акций Серии Б;
	
	$V_{A_1}$ - стоимость акций Серии А, после выпуска акций Серии Б;
	
	$V_B$ - стоимость акций Серии Б, или, что то же самое, сумма, вырученная от продажи акций Серии Б.
	
	Ответ: цена привилегированных акций Серии Б должна быть  0,65 ₽ 
	 за штуку.					
\end{solution}

\subpart[5] Рассчитайте также новый коэффициент конвертации привилегированных акций Серии А, который будет установлен на этом раунде финансирования.
\begin{solution}[9em]
	\begin{align}
	k=\frac{A+\frac{V_B}{P_0}}{A+\frac{V_B}{P_1}},
	\end{align}
	
	где
	
	$k$ - новый коэффициент конвертации.
	
	Ответ: новый коэффициент конвертации привилегированных акций Серии А будет  0,9933 ₽.
\end{solution}

\subpart[5] Нарисуйте таблицу, отражающую новую структуру капитала компании ХХХ.
\begin{solution}[12em]
\begin{tabular}{lrrrr}
	\toprule
	Акции & Кол-во & \%    & Цена  & Стоимость \\
	\midrule
	после финансирования &       &       &       &  \\
	Обыкновенные &                 1 000 000    & 48,99\% &               0,65 ₽  &            645 625 ₽  \\
	Прив. Серия А &                 1 006 776    & 49,32\% &               0,65 ₽  &            650 000 ₽  \\
	Прив. Серия Б &                       38 722    & 1,90\% &               0,65 ₽  &              25 000 ₽  \\
	Итого &                 2 045 499    & 100,21\% &       &         1 320 625 ₽  \\
	\bottomrule
\end{tabular}%
\end{solution}

\end{subparts}
\addpoints

\pagebreak
\question[20]
Создается компания АБВ с уставным капиталом 6 млн. руб. Учредители АБВ – компании А, Б, В, Г, Д, Е, которые вносят в уставный капитал АБВ по 0,5; 1,5; 1; 1; 1; 1 млн. руб. соответственно. Каждая компания имеет своих учредителей:

1.	Учредители компании А – компания Б с долей в 2 млн. руб. и компания Г с долей в 2 млн. руб.;

2.	Учредители компании Б – компания Ж с долей в 10 млн. руб. и компания З с долей в 5 млн. руб.;

3.	Учредители компании В – компания Б с долей в 3 млн. руб., компания Г с долей в 1 млн. руб. и компания Е с долей в 1 млн. руб.;

4.	Учредители компании Г – компания Ж с долей в 2 млн. руб. и компания З с долей в 1 млн. руб.;

5.	Учредители компании Д – компания Б с долей в 5 млн. руб.и компания З с долей в 1 млн. руб.;

6.	Учредители компании Е – компания Ж с долей в 1 млн. руб. и компания З с долей в 1 млн. руб.

\noaddpoints
\begin{subparts}
\subpart[10] Нарисуйте иерархическую схему участия в капиталах компаний учредителей и новой компании АБВ.
\pagebreak
\subpart[10] Определите конечных владельцев вновь созданной компании и их доли в капитале.
\begin{solution}[35em]
Исходя из структуры капитала компании АБВ следует, что компании А, В, Г, Е владеют по 12,5\% капитала компании АБВ, компании Б и Д владеют по 25\%.

Теперь поэтапно будем делить доли компаний между их учредителями, прибавляя их к уже обозначенным долям.

1.	Доли компании А принадлежат Б, Г и Е, следовательно:

Б получает 25 + 3,13 = 28,13\% капитала компании АБВ,

Г получает 12,5 + 3,13 = 15,63\%,

Е получает 12,5 + 6,24 = 18,74\%.

2. Доли компании В принадлежат Б и Г, следовательно:

Б получает 28,13 + 8,33 = 36,46\%,

Г получает 15,63 + 4,17 = 19,8\%.

3. Доли компании Д принадлежат Б, Е и З, следовательно:

Б получает 36,46 + 4,17 = 40,63\%,

Е получает 18,74 + 8,33 = 27,07\%,

З получает 12,5\%.

4. Доли Б принадлежат Ж и З, следовательно:

Ж получает 17,41\%,

З получает 12,5 + 23,22 = 35,72\%.

5. Доли Г принадлежат Ж и З, следовательно:

Ж получает 17,41 + 4,95 = 22,36\%,

З получает 35,72 + 14,85 = 50,57\%.

6. Доли Е принадлежат Ж и З, следовательно:

Ж получает 22,36 + 13,54 = 35,9\%,

З получает 50,53 + 13,54 = 64,07\%.

Ответ: конечными владельцами компании АБВ являются компании Ж и З с долями 35,9\% и 64,1 \% (из-за округления при расчетах возможны небольшие отклонения).
\end{solution}
\vspace{6cm}
\end{subparts}
\addpoints

\end{questions}

\pagebreak
\noindent\textbf{Тестовые вопросы (выберите один правильный ответ)}

\begin{questions}

\begin{multicols}{2}

\question Венчурный капиталист – это:
\begin{choices}
	\choice Инвестор, ищущий быстрых денег и незаинтересованный в инвестировании в компании с долгосрочным потенциалом.
	\CC Руководитель фонда прямых инвестиций, управляющий инвестиционным портфелем.
	\choice Менеджер фонда прямых инвестиций, уполномоченный управлять вложениями фонда в отдельную компанию из портфеля.
	\choice Сотрудник управляющей компании венчурными фондами, занимающийся поиском объектов для инвестиций и экономическим обоснованием целесообразности вложений
\end{choices}

\question Фонд прямых инвестиций – это:
\begin{choices}
	\choice Средство для коллективного инвестирования в акционерный капитал компаний.
	\CC Средство для коллективного инвестирования в акционерный капитал компаний или ценные бумаги, связанные с акционерным капиталом.
	\choice Средство для коллективного инвестирования в акции и облигации компаний, котирующиеся на фондовом рынке.
	\choice Средство для коллективного инвестирования в акции, облигации и производные финансовые инструменты.
\end{choices}

\question Какая из указанных стадий НЕ относится к процессу роста и развития компании:
\begin{choices}
	\choice Посевная.
	\choice Стадия запуска.
	\choice Расширение.
	\CC Ликвидация.
\end{choices}

\question Какой вариант выхода из инвестиции является наиболее доходным и считается наиболее успешным для венчурного капиталиста:
\begin{choices}
	\choice Продажа компании основателю и менеджеру.
	\choice Ликвидация компании по делу о банкротстве.
	\choice Продажа компании стратегическому инвестору.
	\CC Первичное размещение акций венчурной компании на фондовом рынке. 
\end{choices}

\question Какие из перечисленных финансовых услуг, обычно НЕ предоставляются венчурным капиталистом венчурному предприятию:
\begin{choices}
	\choice Управление группой финансовых спонсоров.
	\CC Обеспечение льгот при получении кредитов.
	\choice Поручительство по кредитам, взятым венчурным предприятием в компании АБВх.
	\choice Мониторинг финансовых результатов.
\end{choices}

\question Какова помощь венчурного капиталиста при выходе компании на биржу:
\begin{choices}
	\choice Выбор подходящего ``окна для котирования``.
	\choice Ускорение выхода  предприятия на биржу на несколько лет.
	\choice Снижение средней доходности акций при первичном размещении на фондовом рынке.
	\CC Всё вышеперечисленное.
\end{choices}

\question Квалифицированными инвесторами по умолчанию в России являются:
\begin{choices}
	\choice Брокеры, дилеры и управляющие.
	\choice Акционерные инвестиционные фонды.
	\choice Управляющие компании инвестиционных фондов, паевых инвестиционных фондов и негосударственных пенсионных фондов.
	\CC Все вышеперечисленные.
\end{choices}

\question Укажите требования, которые НЕ предъявляются к юридическим лицам при их признании квалифицированными инвесторами:
\begin{choices}
	\CC Устойчивое финансовое положение организации, отсутствие задолженности перед бюджетом по налогам и перед работниками по заработной плате.
	\choice Минимальный размер собственного капитала, установленный в нормативно-правовых документах, регламентирующих деятельность фондового рынка.
	\choice Минимальный размер годовой выручки, установленный в нормативно-правовых документах, регламентирующих деятельность фондового рынка.
	\choice Совершение сделок с ценными бумагами на фондовом рынке.
\end{choices}

\question По какой причине венчурные фонды инвестируют средства в компании в несколько раундов:
\begin{choices}
	\choice Фонд не обладает изначально суммой денег, нужной для вложений.
	\choice Инвесторы вносят средства в фонд не сразу, а предоставляют фонду обязательства выделять средства по мере необходимости в пределах заранее оговоренного объема.
	\CC Для того, чтобы быстро выявлять неудачные проекты на ранней стадии и прекращать их финансирование.
	\choice В большинстве стран мира установлены законодательные ограничения для менеджеров венчурных фондов по объему рисков, принимаемых в одной сделке.
\end{choices}

\question Укажите сотрудника венчурного фонда, который не участвует в прибыли компании:
\begin{choices}
	\choice Партнер.
	\choice Управляющий директор.
	\CC Аналитик.
	\choice Все указанные должности в разной степени могут рассчитывать на вознаграждение из прибыли компании.
\end{choices}

\question Какие виды риска считаются систематическими:
\begin{choices}
	\choice Размер пакета акций (меньшинство или контроль).
	\CC Изменение ставки рефинансирования Центрального компании АБВ РФ.
	\choice Ликвидность акций или ее отсутствие.
	\choice Размер компании.
\end{choices}

\question Укажите формулу очистки коэффициента бета (из модели оценки капитальных активов) от эффекта финансового рычага:
\begin{choices}
	\CC $\beta_{L}=\beta_{UL} \cdot \left(1+(1-T)\cdot \frac{D}{E} \right)$.
	\choice $\beta_{UL}=\beta_{L} \cdot \left(1+(1-T)\cdot \frac{D}{E} \right)$.
	\choice $\beta_{L}=\beta_{UL} \cdot \left(\frac{1}{1-T} \cdot \frac{D}{E}\right)$.
	\choice $\beta_{L}=\beta_{UL} \cdot \left(1+(1-T)\cdot \frac{D}{E} \right)$.
\end{choices}

\question Укажите формулу определения стоимости компании в период быстрого роста (справочно: FCFF - свободный денежный поток от активов; FCFE - свободный денежный поток от капитала; w - средне-взвешенная стоимость капитала; $C_E$ - стоимость акционерного капитала).
\begin{choices}
	\CC $PV_{PP}=\sum_{i=1}^{N}\frac{FCFF_i}{(1+w)^i}$. 
	\choice $PV_{PP}=\sum_{i=1}^{N}\frac{FCFF_i}{(1+w)^N}$. 
	\choice $PV_{PP}=\sum_{i=1}^{N}\frac{FCFE_i}{(1+w)^i}$. 
	\choice $PV_{PP}=\sum_{i=1}^{N}\frac{FCFE_i}{(1+C_E)^i}$. 
\end{choices}

\question Укажите формулу приведения заключительной стоимости компании к современному моменту
\begin{choices}
	\choice $PTV=\frac{TV}{(1+C_E)^N}$.
	\CC $PTV=\frac{TV}{(1+w)^N}$.
	\choice $PTV=\frac{TV}{(1+w)*N}$.
	\choice $PTV=\frac{TV}{1+w}$.
\end{choices}
\question По каким ставкам производится дисконтирование свободного денежного потока от активов, FCFF, и свободного денежного потока от капитала, FCFE?
\begin{choices}
	\CC FCFF дисконтируется по средне-взвешенной стоимости капитала, а FCFE дисконтируется по стоимости акционерного капитала. 
	\choice FCFE дисконтируется по средне-взвешенной стоимости капитала, а FCFF дисконтируется по стоимости акционерного капитала.
	\choice FCFF и FCFE дисконтируются по средне-взвешенной стоимости капитала.
	\choice FCFF и FCFE дисконтируются по стоимости акционерного капитала.
\end{choices}
\question Укажите какое из утверждений в отношении операционного денежного потока ОШИБОЧНО.
\begin{choices}
	\choice Это денежные средства, получаемые от обычной производственной деятельности фирмы. 
	\choice Это прибыль до уплаты процентов и налогов плюс амортизация и минус налоги.
	\choice Входит с положительным знаком в состав свободного денежного потока от активов.
	\CC Не включает в себя чистые капитальные затраты.
\end{choices}

\question Как определяется Увеличение чистого оборотного капитала?
\begin{choices}
	\CC это инвестиции фирмы в оборотные средства. 
	\choice это инвестиции фирмы в основные средства.
	\choice это оборотные средства за вычетом текущих обязательств.
	\choice это оборотные средства за вычетом долгосрочных обязательств.
\end{choices}

\question Как производится преобразование свободного денежного потока от активов, FCFF, в свободный денежный поток от капитала, FCFE?
\begin{choices}
	\choice FCFE = FCFF - Уменьшение долговых обязательств + Увеличение долговых обязательств.
	\CC FCFE = FCFF - $\text{Проценты} \cdot (1-T)$ - Уменьшение долговых обязательств + Увеличение долговых обязательств.
	\choice FCFE = FCFF - $\text{Проценты}$ - Уменьшение долговых обязательств + Увеличение долговых обязательств.
	\choice FCFE = FCFF + Чистые долговые обязательства.
\end{choices}

\question Укажите формулу скорректированной местной CAPM для изолированного рынка.
\begin{choices}
	\choice $C_E=R_{fG}+\beta_{LG} \cdot (R_{MG}-R_{fG})$.
	\choice $C_E=R_{fL}+\beta_{LL} \cdot (R_{ML}-R_{fL})$.
	\choice $C_E=R_{fL}+R_C+\beta_{LG} \cdot (R_{ML}-R_{fL} ) \cdot (1-R_i^2)$.
	\CC $C_E=R_{fG}+R_C+\beta_{LL} \cdot (R_{ML}-R_{fL} ) \cdot (1-R_i^2 )$.
\end{choices}

\question Какие мультипликаторы НЕ используются при оценке компаний методом сравнительного анализа?
\begin{choices}
	\choice Цена / Прибыль.
	\choice Цена / Выручка.
	\choice Цена / Балансовая стоимость.
	\CC Цена / Фонд оплаты труда.
\end{choices}

\question Инновация – это:
\begin{choices}
	\choice Новшество или нововведение.
	\choice Результат исследования.
	\CC Новое или усовершенствованное социально-экономическое решение, стремящееся к общественному признанию через использование его в практической деятельности людей.
	\choice ``A`` и ``C``.
\end{choices}

\question Какие виды инноваций создают предпосылки для экономических циклов:
\begin{choices}
	\choice Продуктные.
	\CC Технико-технологические.
	\choice Оригинальные.
	\choice Импровизированные.
\end{choices}

\question Государственная инновационная политика является предметом:
\begin{choices}
	\choice Исключительной компетенцией Российской Федерации.
	\choice Исключительной компетенцией субъектов Российской Федерации.
	\CC Совместного ведения Российской Федерации и субъектов Российской Федерации.
	\choice Компетенцией Президента Российской Федерации.
\end{choices}

\question Государственное регулирование инновационной политики дифференцируется на:
\begin{choices}
	\choice Прямое и косвенное.
	\CC Бюджетное и внебюджетное.
	\choice Основное и второстепенное.
	\choice Значимое и незначительное.
\end{choices}

\question В структуре Правительства РФ государственное регулирование инновационной политики осуществляет:
\begin{choices}
	\choice Министерство труда и социального развития.
	\CC Министерство образования и науки.
	\choice Министерство финансов.
	\choice Министерство экономического развития.
\end{choices}

\question Статус Наукограда РФ присваивается:
\begin{choices}
	\CC Правительством РФ.
	\choice Президентом РФ по представлению Правительства РФ.
	\choice Главой муниципального образования по результатам проведения референдума.
	\choice Губернатором субъекта Российской Федерации.
\end{choices}

\question Целью создания особых Экономических Зон РФ является:
\begin{choices}
	\CC Внедрение новых прогрессивных технологий на данной территории.
	\choice Создание условий для осуществления предпринимательской деятельности.
	\choice Создание технопарков и научно-исследовательских центров.
	\choice Предоставление налоговых льгот предпринимателям.
\end{choices}

\question Границы Закрытых административно-территориальных образований:
\begin{choices}
	\choice Совпадают с границами субъектов и регионов, входящих в их состав.
	\CC Не совпадают с границами субъектов и регионов, входящих в их состав. 
	\choice Создаются только на уровне города или поселка городского типа.
	\choice Могут включать в себя несколько областей.
\end{choices}

\question Предметом лицензионного договора является:
\begin{choices}
	\choice Право собственности на результат интеллектуальной деятельности или средство индивидуализации.
	\choice Право бессрочного пользования результатом интеллектуальной деятельности или средством индивидуализации.
	\CC Исключительное право на результат интеллектуальной деятельности или средство индивидуализации.
	\choice Право на получение роялти от использования результата интеллектуальной деятельности.
\end{choices}

\question Венчурные предприятия – это:
\begin{choices}
	\choice Крупные и успешные компании, акции которых котируются на фондовой бирже.
	\CC Небольшие перспективные предприятия, акции которых не котируются на фондовой бирже. 
	\choice Успешность компании и котировка акций не влияет на признание компании венчурного типа.
	\choice Акции компании, которые купил венчурный инвестор.
\end{choices}

\question Инвестиционный фонд – это:
\begin{choices}
	\choice Юридическое лицо.
	\choice Имущественный комплекс.
	\CC ``A`` и ``B``. 
	\choice Орган государственного управления.
\end{choices}

\question Объекты, не признающиеся патентоспособными:
\begin{choices}
	\CC Сорта растений, породы животных. 
	\choice Способы изготовления продуктов.
	\choice Устройство.
	\choice Вещества.
\end{choices}

\question Способом как объектом изобретения являются:
\begin{choices}
	\CC Процесс осуществления действий над материальным объектом с помощью материальных средств.
	\choice Генетическая конструкция.
	\choice Культура (линия) клеток растений или животных.
	\choice Решения, касающиеся только внешнего вида изделий.
\end{choices}

\question Что НЕ является характеристикой исключительного права на результат
интеллектуальной деятельности?
\begin{choices}
	\choice Доходы от совместного распоряжения исключительным
	правом распределяются между всеми правообладателями
	в равных долях.
	\choice Правообладатель может по своему усмотрению разрешать
	или запрещать другим лицам использование результата
	интеллектуальной деятельности.
	\choice Исключительное право (кроме исключительного права на
	фирменное наименование) может принадлежать одному
	лицу или нескольким лицам совместно.
	\CC Исключительные права действуют бессрочно.
\end{choices}

\question Если в публичном заявлении правообладателя о предоставлении любым лицам возможности безвозмездно использовать принадлежащие ему произведение науки, литературы или искусства не указан срок, то считается, что этот срок составляет...
\begin{choices}
	\choice неограничен.
	\choice один год.
	\CC пять лет. 
	\choice до момента смерти правообладателя.
\end{choices}
\question Что НЕ является объектом авторских прав?
\begin{choices}
	\choice Литературные произведения;
	\choice Драматические и музыкально-драматические произведения, сценарные произведения;
	\choice Хореографические произведения и пантомимы;
	\CC Всё вышеперечисленное относится к объектам авторских прав.
\end{choices}
\question Укажите, что НЕ является формой государственной поддержки научных и технологических парков?
\begin{choices}
	\choice Гранты.
	\choice Субсидии.
	\choice Консультации и методологическая поддержка.
	\CC Всё вышеперечисленное верно.
\end{choices}
\question Укажите, что НЕ является показателем развития науки и технологии, коммерциализации научных разработок?
\begin{choices}
	\choice количество успешных компаний, покинувших объект;
	\choice уровень загруженности (заполняемости/освоения) объекта;
	\CC плотность населения территории института развития;
	\choice уровень (коэффициент) выживаемости компаний.
\end{choices}
\question Укажите ОШИБОЧНОЕ утверждение в отношении наночастиц...
\begin{choices}
	\choice для их изучения используется сканирующий электронный микроскоп;
	\choice золотые нанооболочки широко применяются в медицине для адресной доставки лекарств в теле человека;
	\CC это частицы, одно из измерений которых меньше 1000 нм;
	\choice инъекции наночастиц железа применяются для очистки грунтовых вод.
\end{choices}
\question Укажите ВЕРНОЕ утверждение в отношении подрывных инноваций
\begin{choices}
	\choice определяют путь усовершенствований для лидера;
	\CC все ``подрывные`` продукты создаются с расчетом на такие возможности роста, которые находятся вне ядерных рынков лидера;
	\choice снабжают топливом “подрывные“ компании, которые восходят по своей траектории усовершенствований
	\choice дают лидерам возможность координировать общую работу сетей, решив проблемы совместимости и наследования.
\end{choices}

\end{multicols}
\end{questions}

\end{document}
