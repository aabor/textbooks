% !TeX program = lualatex -synctex=1 -interaction=nonstopmode --shell-escape %.tex

\documentclass[12pt, table]{exam}
\usepackage[rus]{borochkin}

\usepackage{borochkin_exam}

%%%%%%%%%%%%%%%%%%%%%%%%%%%%%%%%%%%%%%%%%

\professor
\iftagged{professor}{ \printanswers }
%%%%%%%%%%%%%%%%%%%%%%%%%%%%%%%%%%%%%%%%%



\begin{document}
\setcounter{section}{0\relax}%
\noindent
% Контр/р № #1, Вариант #2, Предмет #3
\studentpersonalinfo{1}{1}{ОВИ}
\normalsize

\begin{questions}
\question[40] Тест
\answerstotest

	
\question[20]
Венчурный инвестор профинансировал стартап XXX, купив конвертируемые привилегированные акции серии А на сумму 1~M₽, а вклад предпринимателя состоял лишь из интеллектуальной собственности. Предприниматель владеет только обыкновенными акциями. Собственность на компанию была разделена в соотношении 50\%/50\%. Цена покупки привилегированных акций составила 1~₽ за штуку.
Для защиты интересов инвесторов в акции предыдущих серий, решением Совета Директоров компании ХХХ в проспекте эмиссии на акции серии А было указано, что в случае выпуска новой серии акции по меньшей совокупной цене, цена конвертации предыдущих серий привилегированных акций определяется методом средне-взвешенного отката.
Через год после начала инвестиций, предпринимателю потребовалось дополнительное финансирование на сумму  25~000~₽.
Для этих целей были выпущены конвертируемые привилегированные акции серии Б.
\noaddpoints
\pagebreak
\begin{subparts}
\subpart[10] Определите, при какой цене привилегированных акций Серии Б, потери инвестора в акции Серии А составят не более 30\% от первоначальных вложений.
\begin{solution}[18em]
\begin{align}
P_1=\frac{\frac{V_{A_1} \cdot (A + \frac{V_B}{P_0})}{A_0}-V_B}{A},
\end{align}

где

$P_0$, $P_1$ - цена покупки акций Серии А и Серии Б, соответственно;

$A$ - общее количество акций компании до выпуска акций Серии Б;

$A_0$, $A_1$ - количество акций Серии А до и после выпуска акций Серии Б;

$V_{A_1}$ - стоимость акций Серии А, после выпуска акций Серии Б;

$V_B$ - стоимость акций Серии Б, или, что то же самое, сумма, вырученная от продажи акций Серии Б.

Ответ: цена привилегированных акций Серии Б должна быть 0,70 ₽ за штуку.					
\end{solution}

%\vspace{4cm}
\subpart[5] Рассчитайте также новый коэффициент конвертации привилегированных акций Серии А, который будет установлен на этом раунде финансирования.

\begin{solution}[9em]
\begin{align}
k=\frac{A+\frac{V_B}{P_0}}{A+\frac{V_B}{P_1}},
\end{align}

где

$k$ - новый коэффициент конвертации.

Ответ: новый коэффициент конвертации привилегированных акций Серии А будет 0,9946 ₽.
\end{solution}

\subpart[5] Нарисуйте таблицу, отражающую новую структуру капитала компании ХХХ.
\begin{solution}[12em]
	\begin{tabular}{lrrrr}
		\toprule
		Акции & Кол-во & \%    & Цена  & Стоимость \\
		\midrule
		после финансирования &       &       &       &  \\
		Обыкновенные &                 1 000 000    & 48,99\% &               0,70 ₽  &            696 250 ₽  \\
		Прив. Серия А &                 1 005 386    & 49,25\% &               0,70 ₽  &            700 000 ₽  \\
		Прив. Серия Б &                       35 907    & 1,76\% &               0,70 ₽  &              25 000 ₽  \\
		Итого &                 2 041 293    & 100,00\% &       &         1 421 250 ₽  \\
		\bottomrule
	\end{tabular}%
\end{solution}
\end{subparts}
\addpoints

\pagebreak
\question[20]
Создается компания АБВ с уставным капиталом 6 млн. руб. Учредители АБВ – компании А, Б, В, Г, Д, Е, которые вносят в уставный капитал АБВ по 0,5; 1,5; 1; 1; 1; 1 млн. руб. соответственно. Каждая компания имеет своих учредителей:

1.	Учредители компании А – компания Б с долей в 2 млн. руб. и компания Г с долей в 2 млн. руб.;

2.	Учредители компании Б – компания Ж с долей в 10 млн. руб. и компания З с долей в 5 млн. руб.;

3.	Учредители компании В – компания Б с долей в 3 млн. руб., компания Г с долей в 1 млн. руб. и компания Е с долей в 1 млн. руб.;

4.	Учредители компании Г – компания Ж с долей в 2 млн. руб. и компания З с долей в 1 млн. руб.;

5.	Учредители компании Д – компания Б с долей в 5 млн. руб.и компания З с долей в 1 млн. руб.;

6.	Учредители компании Е – компания Ж с долей в 1 млн. руб. и компания З с долей в 1 млн. руб.

\noaddpoints
\begin{subparts}
\subpart[10] Нарисуйте иерархическую схему участия в капиталах компаний учредителей и новой компании АБВ.
\pagebreak
\subpart[10] Определите конечных владельцев вновь созданной компании и их доли в капитале.

\begin{solution}[35em]
Исходя из структуры капитала компании АБВ следует, что компания А владеет 8,33\% капитала компании АБВ, компания Б владеет 25\% и фирмы В, Г, Д, Е владеют по 16,67\%.

Теперь поэтапно будем делить доли фирм между их учредителями, прибавляя их к уже обозначенным долям.

1.	Доли компании А принадлежат поровну Б и Г, следовательно:

Б получает 25 + 4,17 = 29,17\%,

Г получает 16,67 + 4,17 = 20,84\%.

2. Доли компании В принадлежат в пропорции Б, Г и Е, следовательно:

Б получает 29,17 + 10 = 39,17\%,

Г получает 20,84 + 3,33 = 24,17\%,

Е получает 16,67 + 3,33 = 20\%.

3. Доли компании Д принадлежат Б и З, следовательно:

Б получает 29,17 + 13,89 = 53,06\%,

З получает 2,78\%.

4. Доли компании Б принадлежат Ж и З, следовательно:

Ж получает 35,37\%,

З получает 2,78 + 17,69 = 20,47\%.

5. Доли компании Г принадлежат Ж и З, следовательно:

Ж получает 35,37 + 16,11 = 51,48\%,

З получает 20,47 + 8,05 = 28,52\%.

6. Доли компании Е принадлежат Ж и З, следовательно:

Ж получает 51,48 + 10 = 61,48\%,

З получает 28,52 + 10 = 38,52\%.

Ответ: конечными владельцами компании АБВ являются компании Ж и З с долями 61,48\% и 38,52\% (из-за округления при расчетах возможны небольшие отклонения).
\end{solution}
\end{subparts}
\addpoints

\end{questions}

\pagebreak
\noindent\textbf{Тестовые вопросы (выберите один правильный ответ)}

\begin{questions}
\begin{multicols}{2}
\setlength{\columnsep}{1cm}
\question Венчурным капиталом называют:
\begin{choices}
	\CC Денежные средства, вложенные в акционерный капитал рискованного предприятия.
	\choice Инвестиции в облигации инновационных компаний, котирующихся на фондовой бирже.
	\choice Инвестиции в акции инновационных компаний, котирующихся на фондовой бирже.
	\choice Денежные средства, вложенные в долговые обязательства рискованного предприятия.
\end{choices}

\question Прямые инвестиции – это:
\begin{choices}
	\choice Вложения в акционерный капитал компаний, котирующихся на фондовой бирже.
	\CC Вложения в акционерный капитал предприятий, не котирующихся на фондовом рынке.
	\choice Вложения в акционерный капитал инновационных компаний, котирующихся на фондовой бирже.
	\choice Вложения в акционерный капитал инновационных предприятий, не котирующихся на фондовом рынке.
\end{choices}

\question Активный подход к управлению венчурным фондом подразумевает:
\begin{choices}
	\choice Частый пересмотр состава инвестиционного портфеля, изменение доли вложений в капитал отдельных компаний.
	\choice Влияние менеджеров венчурного фонда на кадровую политику финансируемой компании, в том числе замена руководителя в случае низких результатов деятельности.
	\CC Активное участие менеджеров венчурного фонда в управлении финансируемой компании.
	\choice Инвестирование средств фонда на ранних стадиях развития компании, что повышает конечную доходность инвестиций, при этом риск снижается за счет более широкой диверсификации вложений.
\end{choices}

\question Какова общая продолжительность жизненного цикла компании:
\begin{choices}
	\choice 1-3 года.
	\choice 3-5 лет.
	\CC 5-10 лет.
	\choice 10 лет.
\end{choices}

\question Какая услуга венчурного капиталиста является наиболее значимой для компании, получателя средств:
\begin{choices}
	\choice Финансовые услуги.
	\choice Подбор и управление менеджментом.
	\choice Помощь в текущей деятельности.
	\CC Квалифицированный отбор предпринимательских идей.
\end{choices}

\question Какие из перечисленных услуг по подбору и управлению менеджментом, обычно НЕ предоставляются венчурным капиталистом венчурному предприятию:
\begin{choices}
	\choice Поиск новых менеджеров.
	\CC Обучение новых руководителей.
	\choice Новые мотивировки персонала.
	\choice Замена менеджеров.
\end{choices}

\question Квалифицированный инвестор, согласно российской практике – это:
\begin{choices}
	\choice Инвестор, достигший возраста 18 лет, имеющий законченное высшее образование в сфере экономики и финансов.
	\choice Физическое лицо, обладающее капиталом для инвестирования в ценные бумаги в размере более 3 млн. рублей.
	\CC Физическое или юридическое лицо, соответствующее требованиям, предъявляемым к квалифицированным инвесторам, согласно российскому законодательству в области регулирования фондового рынка.
	\choice Физическое или юридическое лицо имеющее опыт инвестиционной деятельности в течение более 3 лет.
\end{choices}

\question Укажите требования, которые НЕ предъявляются к физическим лицам при их признании квалифицированными инвесторами:
\begin{choices}
	\CC Наличие высшего экономического или финансового образования.
	\choice Владение ценными бумагами на минимальную сумму, определенную в нормативно-правовых документах, регламентирующих деятельность фондового рынка.
	\choice Опыт работы в организации, осуществляющей деятельность на рынке ценных бумаг.
	\choice Личный опыт торговли ценными бумагами на фондовом рынке.
\end{choices}

\question Какие циклы НЕ включает в себя научно-технологический цикл:
\begin{choices}
	\choice Научный.
	\choice Изобретательский.
	\choice Инновационно-инвестиционный.
	\CC Биржевой.
\end{choices}

\question Как определяется размер вознаграждения управляющей компании венчурными фондами:
\begin{choices}
	\choice Вознаграждение составляет 20-25\% от прибыли, полученной после завершения инвестиционного цикла венчурного фонда.
	\choice Ежегодные комиссионные расходы за управление в размере до 5\% от стоимости активов фонда.
	\choice Управляющая компания обеспечивает инвесторам минимальную норму доходности, оставшуюся прибыль забирая себе в качестве вознаграждения за успех.
	\CC Вознаграждение составляет 20-25\% от прибыли, при условии выплаты инвесторам предопределенной минимальной доходности на инвестиции.
\end{choices}

\question Какие методы оценки используются при определении стоимости венчурных предприятий:
\begin{choices}
	\choice Мультипликаторы прибыли и выручки.
	\choice Расчет стоимости чистых активов.
	\choice Отраслевые стандарты оценки.
	\CC Всё вышеперечисленное.
\end{choices}

\question Укажите формулу финансового рычага:
\begin{choices}
	\choice $\Delta ROE=(ROA-C_D ) \cdot \frac{D}{E}$.
	\choice $\Delta ROE=(1-T) \cdot (ROA-C_D )$.
	\CC $\Delta ROE=(1-T) \cdot (ROA-C_D ) \cdot \frac{D}{E}$.
	\choice $\Delta ROE=(1-T) \cdot ROA \cdot \frac{D}{E} $.
\end{choices}

\question Укажите формулу средневзвешенной стоимости капитала, $w$:
\begin{choices}
	\choice $w=\frac{E}{D+E} \cdot C_E + \frac{D}{D+E} \cdot C_D$.
	\CC $w=\frac{E}{D+E} \cdot C_E + \frac{D}{D+E} \cdot C_D \cdot (1-T)$.
	\choice $w=\frac{1}{D+E} \cdot C_E + \frac{1}{D+E} \cdot C_D \cdot (1-T)$.
	\choice $w=\frac{E}{1+D/E} \cdot C_E + \frac{D}{1+D/E} \cdot C_D \cdot (1-T)$.
\end{choices}

\question Укажите формулу определения заключительной стоимости компании по модели Гордона (справочно: FCFF - свободный денежный поток от активов; FCFE - свободный денежный поток от капитала; w - средне-взвешенная стоимость капитала; $C_E$ - стоимость акционерного капитала).
\begin{choices}
	\choice $TV=\frac{FCFE_{N+1}}{C_E-g}, w\neq g$.
	\choice $TV=\frac{FCFE_{N+1}}{C_E-g}, w>g$.
	\CC $TV=\frac{FCFF_{N+1}}{w-g}, w\neq g$.
	\choice $TV=\frac{FCFF_{N+1}}{w-g}, w < g$.
\end{choices}

\question Укажите, какая формула используется для определения итоговой стоимости компании методом дисконтирования денежных потоков
\begin{choices}
	\choice $V=PV_{PP}+ \frac{PTV}{1+w}$.  
	\choice $TV=\frac{FCFE_{N+1}}{C_E-g}, w>g$.
	\choice $V=V_{PP}+TV$.
	\CC $V=PV_{PP}+PTV$. 
\end{choices}

\question Денежный поток от активов включает в себя
\begin{choices}
	\choice Только денежные потоки акционерам.
	\choice Только денежные потоки кредиторам.
	\choice Денежные потоки кредиторам плюс денежные потоки акционерам минус капитальные вложения.
	\CC Денежные потоки кредиторам плюс денежные потоки акционерам.
\end{choices}

\question Укажите какое из указанных утверждений в отношении прибыли компании ОШИБОЧНО.
\begin{choices}
	\choice Операционная прибыль равна Прибыль до вычета процентов, налогов и амортизации минус Амортизация.
	\CC Чистая прибыль равна Прибыль до выплаты процентов и налогов минус Налог на прибыль.
	\choice Чистая прибыль равна Прибыль до выплаты налогов минус Налог на прибыль.
	\choice Прибыль до вычета процентов, налогов и амортизации равна Выручка минус Затраты и оперативные расходы.
\end{choices}

\question Как определяются капитальные затраты?
\begin{choices}
	\choice разница основных средств на конец и на начало года плюс амортизация.
	\choice разница чистых основных средств на конец и на начало года минус амортизация.
	\choice разница чистых основных средств на конец и на начало года.
	\CC разница чистых основных средств на конец и на начало года плюс амортизация. 
\end{choices}

\question Как производится преобразование свободного денежного потока от капитала, FCFE, в свободный денежный поток от активов, FCFF ?
\begin{choices}
	\CC FCFF = FCFE + $\text{Проценты} \cdot (1-T)$ + Уменьшение долговых обязательств - Увеличение долговых обязательств. 
	\choice FCFF = FCFE - Уменьшение долговых обязательств + Увеличение долговых обязательств.
	\choice FCFF = FCFE - Чистые долговые обязательства.
	\choice FCFF = FCFE - $\text{Проценты} \cdot (1-T)$ - Уменьшение долговых обязательств + Увеличение долговых обязательств.
\end{choices}

\question Укажите формулу глобальной CAPM.
\begin{choices}
	\choice $C_D=R_{fG}+\beta_{LG} \cdot (R_{MG}-R_{fG})$.
	\choice $C_E=R_{fL}+\beta_{LG} \cdot (R_{MG}-R_{fG})$.
	\CC $C_E=R_{fG}+\beta_{LG} \cdot (R_{MG}-R_{fG})$.
	\choice $C_E=R_{fL}+\beta_{LG} \cdot (R_{ML}-R_{fL})$.
\end{choices}

\question Укажите формулу местной CAPM для изолированного рынка
\begin{choices}
	\choice $C_E=R_{fL}+\beta_{LG} \cdot (R_{ML}-R_{fL})$.
	\choice $C_E=R_{fG}+\beta_{LG} \cdot (R_{MG}-R_{fG})$. 
	\choice $C_E=R_{fG}+\beta_{LG} \cdot (R_{ML}-R_{fL})$.
	\CC $C_E=R_{fL}+\beta_{LL} \cdot (R_{ML}-R_{fL})$.
\end{choices}

\question Основные предпосылки возникновения инноваций:
\begin{choices}
	\choice Потребность рынка.
	\choice Экономический кризис.
	\choice Изобретательство.
	\CC ``A`` и ``B``.
\end{choices}

\question Субъекты инновационной деятельности квалифицируются на:
\begin{choices}
	\CC Непосредственных и вспомогательных.
	\choice Прямых и косвенных.
	\choice Главных и второстепенных.
	\choice Значимых и незначительных.
\end{choices}

\question Эффективность государственной инновационной политики определяется взаимодействием: 
\begin{choices}
	\choice Цели политики и достигнутого результата.
	\choice Цели политики и механизма ее реализации.
	\choice Цели политики и принципов ее осуществления.
	\CC Цели политики, принципов ее осуществления и механизма ее реализации. 
\end{choices}

\question Государственная научно-техническая программа – это:
\begin{choices}
	\choice Комплекс мероприятий, взаимосвязанных по ресурсам, срокам и исполнителям, обеспечивающих эффективное решение важнейших научно-технических проблем на приоритетных направлениях развития науки и техники.
	\choice Официальный документ, утверждаемый правительством РФ.
	\choice Комплекс приоритетных направлений развития науки и техники.
	\CC ``A`` и ``B``.
\end{choices}

\question Научно-производственный комплекс наукограда – это:
\begin{choices}
	\CC Совокупность организаций, осуществляющих научную, научно-техническую и инновационную деятельность.
	\choice Совокупность технического вооружения наукограда: станки, оборудование и др.
	\choice Совокупность занятого на территории наукограда населения.
	\choice Совокупность инфраструктурных элементов городского хозяйства.
\end{choices}

\question Основаниями досрочного прекращения статуса наукограда РФ являются:
\begin{choices}
	\choice Несоответствие результатов деятельности поставленным перед ним задачам.
	\choice Мотивированное ходатайство представительного органа местного самоуправления муниципального образования.
	\choice Истечение срока, на который был установлен статус наукограда РФ.
	\CC ``A`` и ``B``. 
\end{choices}

\question Целью создания Закрытых административно-территориальных образования является:
\begin{choices}
	\CC Разработка и испытание секретного оружия.
	\choice Утилизация ядерных отходов.
	\choice Обеспечение национальной безопасности.
	\choice Организация научно-технических и научно-исследовательских центров.
\end{choices}

\question Чем отличается договор на выполнение научно-исследовательских и опытно-конструкторских работ (НИР и ОКР) от договора подряда:
\begin{choices}
	\choice По предмету и специфике.
	\CC Различные существенные условия.
	\choice По ответственности за неисполнение обязательств.
	\choice По статусу участников сделки.
\end{choices}

\question В каких случаях лицензионный договор подлежит государственной регистрации:
\begin{choices}
	\choice Во всех случаях.
	\CC В случаях, если сам объект интеллектуальной собственности подлежит регистрации.
	\choice Не подлежит государственной регистрации вообще.
	\choice В случае крупной сделки.
\end{choices}

\question Венчурные инвестиции:
\begin{choices}
	\CC Вкладываются в уставный капитал венчурной компании. 
	\choice Передаются венчурной компании по договору займа.
	\choice Передаются венчурной компании на кредитной основе.
	\choice ``A`` и ``C``.
\end{choices}

\question Имущество, передаваемое в паевой инвестиционный фонд:
\begin{choices}
	\choice Является вкладом собственника имущества в уставный капитал.
	\choice Объединяется с другим имуществом учредителей доверительного управления.
	\choice Хранится за балансом управляющей компании.
	\CC ``A`` и ``B``.
\end{choices}

\question Продуктом как объектом изобретения являются:
\begin{choices}
	\choice Устройства.
	\choice Способы изменения состояния предметов без получения конкретных продуктов.
	\CC Топологии интегральных микросхем.
	\choice Правила и методы игр.
\end{choices}

\question Авторство и имя автора 
\begin{choices}
	\choice охраняются пять лет.
	\choice охраняются пожизненно.
	\CC охраняются бессрочно.
	\choice не охраняются.
\end{choices}

\question Кто является автором результата интеллектуальной деятельности?
\begin{choices}
	\choice Граждане, оказавшие техническое, консультационное, организационное или материальное содействие автору.
	\choice Граждане, способствовавшие оформлению прав на результат интеллектуальной деятельности.
	\CC гражданин, творческим трудом которого создан
	такой результат. 
	\choice Граждане, осуществлявшие контроль за выполнением соответствующих работ.
\end{choices}
\question Укажите ОШИБОЧНОЕ утверждение относительно лицензионного договора о передачи права пользования результатами интеллектуальной деятельности.
\begin{choices}
	\CC Лицензионный договор может не предусматривать вознаграждение правообладателя.
	\choice Обладатель исключительного права (лицензиар) предоставляет или обязуется предоставить другой стороне (лицензиату) право использования такого результата или такого средства в предусмотренных договором пределах.
	\choice Письменная форма, иначе договор не 	действителен
	\choice Если не указана территория использования, то предполагается вся территория России.
\end{choices}
\question Что НЕ является объектом авторских прав?
\begin{choices}
	\choice произведения живописи, скульптуры, графики, дизайна, графические рассказы, комиксы и другие произведения изобразительного искусства;
	\choice произведения декоративно-прикладного и сценографического искусства;
	\choice произведения архитектуры, градостроительства и садово-паркового
	искусства, в том числе в виде проектов, чертежей, изображений и макетов
	\CC Всё вышеперечисленное относится к объектам авторских прав.
\end{choices}
\question Укажите ОШИБОЧНОЕ утверждение в отношении знака охраны авторского права.
\begin{choices}
	\choice Включает в себя латинскую букву ``C`` в окружности.
	\choice Включает в себя имя или наименование правообладателя.
	\choice Включает в себя год первого опубликования произведения.
	\CC Включает в себя город или страну проживания автора. 
\end{choices}
\question Какие бюджетные ограничения позволяют обойти институты развития?
\begin{choices}
	\choice синдром “конца бюджетного года“;
	\choice отсутствие в Бюджетном Кодексе РФ понятия гранта;
	\choice невозможность финансирования НИОКР;
	\CC всё вышеперечисленное.
\end{choices}
\question Какие проблемы НЕ характерны для экономики портовых экономических зон, ПОЭЗ, в России?
\begin{choices}
	\CC Российское законодательство не позволяет создавать ПОЭЗ на территории двух муниципальных образований.
	\choice Регионы лишаются части налогов, уходящих
	на льготы резидентам.
	\choice Из порта «изгоняются» региональные власти и уже сложившиеся крупные управляющие	компании.
	\choice во многих портах давно сформировалась структура собственников и управляющих компаний, для которых приход на эти территории государственного оператора означает весьма болезненный передел 	собственности.
\end{choices}

\end{multicols}
\end{questions}

\end{document}
