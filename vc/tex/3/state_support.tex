% !TeX program = lualatex -synctex=1 -interaction=nonstopmode --shell-escape %.tex
\documentclass[_Venture_p3.tex]{subfiles}

\begin{document}

\setbeamercovered{invisible}

\subsection{Определения}
\begin{frame}[shrink=5]{Государственно-частное партнерство}{при различных условиях возникновения}
амер. экономист С. Линдер, 1999 г.:
\begin{itemize}
	\item реформа системы управления;
	\item проблема конверсии;
	\item схема распределения рисков;
	\item как структурная перестройка системы предоставления государственных услуг;
	\item разделение властных полномочий.
\end{itemize}
\end{frame}

\begin{frame}[allowframebreaks]{Определения ГЧП}{разработанные международными финансовыми организациями}
\begin{block}{ГЧП}
	\quad
	– это соглашение между правительством и одним или несколькими частными партнерами, в соответствии с которым частная сторона предоставляет услуги таким образом, чтобы цели государства согласовывались с целями частного бизнеса по получению прибыли и эффективность соглашения зависела от обоснованной передачи части рисков частной стороне (Организация экономического сотрудничества и развития (ОЭСР)).
\end{block}

\pagebreak


\begin{block}{ГЧП}
	\quad
	– это соглашение, в котором частный сектор предоставляет объекты инфраструктуры и соответствующие услуги, которые традиционно предоставляло государство (Международный валютный фонд).
\end{block}

\pagebreak


\begin{block}{ГЧП}
	\quad
	- это контрактное соглашение между государственным органом (федеральным, региональным или муниципальным) и частной организацией, в рамках которого совместно используются активы и опыт каждой стороны при предоставлении продукции или услуг широкой общественности, а также распределяются риски и выгоды (Национальный совет по ГЧП США).
\end{block}
\end{frame}



\begin{frame}[allowframebreaks]{Инструменты стимулирования венчурного инвестирования}{ с участием государственных средств}
\begin{itemize}
	\item венчурные фонды со 100\% участием государства;
	\item государственные посевные фонды, ориентированные на самую рискованную стадию бизнеса;
	
	\pagebreak
	\item частно-государственные венчурные фонды, управление которыми осуществляется частными УК, а государство выступает в роли соинвестора.
	\item государственный ``фонд фондов`` - агент государства при соинвестировании в государственно-частные фонды (доля ``фонда фондов`` в венчурном фонде может составлять до 40-50\%);

	\pagebreak
	\item софинансирование инновационных компаний, привлекших частного венчурного инвестора (в виде грантов или возвратных средств);
	\item налоговые льготы на прибыль с капитала для венчурных инвесторов, а также благоприятный с налоговой точки зрения процесс вложения средств в инновационные компании.
\end{itemize}
\end{frame}



\begin{frame}[allowframebreaks]{Формы взаимодействия бизнес-структур }{с органами публичной власти в рамках ГЧП}
\begin{itemize}
	\item наличие долгосрочного договорного соглашения между органом публичной власти и хозяйствующим субъектом по реализации отдельного проекта;
	\item передача части задач и функций, относящихся к деятельности органов государственного или муниципального управления, частному бизнесу;
	
	\pagebreak
	\item привлечение частного финансирования капитальных вложений по проекту и финансовое вознаграждение частной стороне;
	\item разделение рисков между государственным и частным партнерами в проекте;

	\pagebreak
	\item результатом деятельности в рамках ГЧП всегда является некая специфическая продукция или услуга, востребованная обществом и выполняемая в строгом соответствии со стандартами.
\end{itemize}
\end{frame}



\begin{frame}[allowframebreaks]{Сравнение ГЧП с аутсорсингом и приватизацией}{}
\begin{itemize}
	\item приватизация предполагает полное отчуждение частному сектору объектов государственного имущества (включая передачу всех рисков и ответственности); 
	\item аутсорсинг — передачу выполнения части бюджетных услуг сторонней организации-исполнителю;

	\pagebreak
	\item ГЧП - обе стороны распределяют между собой все риски в надежде получить прибыль, а государство — еще и определенный социальный эффект.
\end{itemize}
\end{frame}

\subsection{Примеры}


\begin{frame}{Основные инструменты налогово-бюджетной политики  }{стимулирования инновационной деятельности бизнеса}
\begin{itemize}
	\item субсидирование, 
	\item предоставление налоговых льгот, 
	\item отсрочки по уплате налогов, 
	\item введение безналогового режима на определенный период, 
	\item государственные закупки.
\end{itemize}
\end{frame}



\begin{frame}{Направления государственной политики по поддержке инноваций }{(по рекомендациями ОЭСР)}
\begin{itemize}
	\item поддержка риска инвестирования стартапов;
	\item приобретение продукции инновационных компаний;
	\item обеспечение баланса государственного и частного финансирования инноваций.
\end{itemize}
\end{frame}

 

\begin{frame}[allowframebreaks]{Меры государственного стимулирования инноваций }{в условиях кризиса}
\begin{itemize}
	\item поддерживающие продажи, в том числе на основе общественно-частной передачи знаний и экологических технологий;

	\pagebreak
	\item предотвращение истощения оборотного капитала, в том числе посредством экспортного кредита и страхования, факторинга для дебиторской задолженности, возможности сокращения и задержки налоговых платежей;

	\pagebreak
	\item расширение доступа к финансовым ресурсам, главным образом к кредиту, через дополнительную капитализацию банка, увеличение перечня существующих кредитов и гарантий;

	\pagebreak
	\item помощь предприятиям по поддержке их инвестиционного уровня посредством инвестиционных грантов, ускоренной амортизации и финансирование R\&D.
\end{itemize}
\end{frame}


\begin{frame}[allowframebreaks]{Поддержка венчурных инвестиций в развитых странах}{}
\begin{itemize}
	\item США - ``компании по инвестированию в малый бизнес``, SBIC;
	\item Великобритания - программа Региональных Венчурных фондов (Regional Venture Capital Funds, RVCF);
	
	\pagebreak
	\item Финляндия - катализатором развития венчурных инвестиций стал фонд ``посевных`` инвестиций SITRA;
	\item Израиль государственный ``фонд фондов`` Yozma (Инициатива), соинвестировавший в частно-государственные венчурные фонды.	
\end{itemize}
\end{frame}


\subsection{Инновационная инфраструктура}

\begin{frame}[allowframebreaks]{Соцэкономический эффект }{создания сети объектов инновационной инфраструктуры}
Национальный уровень:
\begin{itemize}
	\item интеграция страны в мировую экономику;
	\item повышение рентабельности инвестиций и привлечение иностранных инвестиций в страну; увеличение доходов от экспорта;
	\pagebreak
	\item повышение благосостояния общества и качества жизни; стабилизация политической ситуации;
	\item наиболее полное и эффективное использование ресурсов; активизация научных исследований и т.д.
\end{itemize}
\end{frame}


\begin{frame}[allowframebreaks]{Соцэкономический эффект }{создания сети объектов инновационной инфраструктуры}
Региональный уровень:
\begin{itemize}
	\item повышение уровня занятости (причем главная роль отводится самозанятости);
	\item установление конкурентоспособных цен на производимую продукцию; 
	\item выравнивание межрегионального развития страны;
	
	\pagebreak
	\item рост квалификации трудовых ресурсов и производительности труда, повышение инвестиционной привлекательности региона; 
	\item диверсификация деятельности регионов;
	\item расширение налогооблагаемой базы регионов и т.д.	
\end{itemize}
\end{frame}


 

\begin{frame}[allowframebreaks]{Бизнес-инкубатор}{}
\begin{block}{Бизнес-инкубатор}
	\quad
	— это физическое пространство, на территории которого предоставляется определенный набор услуг для небольших компаний.
\end{block}

\pagebreak


\begin{block}{Бизнес-инкубатор}
	\quad
	-  является специализированным инструментом в политике регионального экономического развития и регенерации путем предоставления междисциплинарной профессиональной поддержки для малого инновационного предпринимательства в международном контексте (European Business Innovation Centre Network).
\end{block}
\end{frame}


\begin{frame}[allowframebreaks]{}{}
\begin{itemize}
	\item основная цель деятельности инкубатора состоит в содействии созданию и поддержке на ранних стадиях развития независимых и финансово устойчивых технологичных компаний;
	\item в развивающихся странах процент ``выживаемости`` компаний в инкубаторах составляет свыше 90-96\%, тогда как среди компаний вне инкубационного процесса — только 50\%;
	
	\pagebreak
	\item наиболее эффективна модель, когда он интегрирован в состав более крупного технологичного объекта (научного, технологического парка, кластера, ОЭЗ), несколько реже — в структуру вуза.
\end{itemize}
\end{frame}






\begin{frame}[allowframebreaks]{Технологический парк}{}
\begin{block}{Технологический парк}
	\quad
	– это субъект инновационной инфраструктуры, способствующий развитию предпринимательства в научнотехнической сфере путем создания благоприятных условий, включающих материальнотехническую и информационную базу.
\end{block}

Технологический парк считается наиболее эффективной формой интеграции образования, науки и производства.

\pagebreak


\begin{block}{Технологический парк}
	\quad
	– это имущественный комплекс, созданный для осуществления деятельности в сфере высоких технологий, состоящий из офисных зданий и производственных помещений, объектов инженерной, транспортной, жилой и социальной инфраструктуры общей площадью не менее 5000кв.м (Министерство экономического развития России).
\end{block}
\end{frame}





\begin{frame}[allowframebreaks]{Научный парк, исследовательский парк, индустриальный парк }{}
\begin{block}{Научный парк, исследовательский парк}
	\quad
	- это организация под управлением профессиональной компании, целью которой является рост благосостояния сообщества посредством продвижения культуры инноваций и повышения конкурентоспособности находящихся на данной территории частных компаний и научных учреждений (International Association of Scientific Parks).	
\end{block}

\pagebreak


\begin{block}{Индустриальный парк}
	\quad
	– это разновидность бизнес-парка с инженерно подготовленными земельными участками, производственными и складскими зданиями, предназначенными для размещения в основном производственно-складских комплексов и крупных производств.
\end{block}
\end{frame}




\begin{frame}[allowframebreaks]{Отличия технопарков и научных парков}{}
\begin{itemize}
	\item технопарки изначально позиционировались как бизнес-модель, в которой воплощены процессы трансфера технологий, инкубации стартапов и процесс индустриализации; 
	
	\pagebreak
	\item мотивом создания научных парков явилось в первую очередь стремление активизировать НИОКР в приоритетных отраслях экономики, обычно на стыке нескольких научных направлений.
\end{itemize}
\end{frame}



\begin{frame}{Наукограды}
\begin{block}{}
	\quad
	- успешные научные парки превращаются в научные города, наукограды, на территории которых сосредоточены по несколько исследовательских кластеров, вузы, государственные учреждения по поддержке инноваций (Франция, Испания, Швеция).
\end{block}
\end{frame}

 
\begin{frame}{Технополис}
\begin{block}{}
	\quad
	– это объект инфраструктуры (с технологическим оборудованием, предоставлением льгот для резидентов и т.д.), создаваемый совместно с университетами, научными организациями и местной властью.
\end{block}
\end{frame}

\begin{frame}{Кластер}
\begin{block}{}
	\quad
	- это сеть независимых компаний, научных и образовательных организаций, консультационных и сервисных фирм и потребителей, которые связаны между собой в единой цепочке производства добавленной стоимости (ОЭСР).
\end{block}
\end{frame}


\begin{frame}[allowframebreaks]{Кластер}{Другие варианты определений}
\begin{block}{}
	\quad
	- это группа географически соседствующих взаимосвязанных компаний и связанных с ними организаций, действующих в определенной сфере, характеризующихся общностью деятельности и взаимодополняющих друг друга (М. Портер).
\end{block}

\pagebreak


\begin{block}{}
	\quad
	- совместное размещение производителей, поставщиков услуг, образовательных и исследовательских учреждений, финансовых институтов и других государственных частных и государственных организаций, связанных между собой различными формами взаимодействия (European Commission).
\end{block}


\pagebreak


\begin{block}{}
	\quad
	географическая концентрация экономической и инновационной деятельности (Кристенсен).
\end{block}

\end{frame}



\begin{frame}[allowframebreaks]{Преимущества кластеров}{}
\begin{itemize}
	\item драйверы развития инноваций и экономического роста, научных исследований, повышении качества образования, эффективности предпринимательства, содействии более тесным связям между производственными и научными организациями, создании ``питательной среды`` для развития связей малого бизнеса с крупными компаниями и зарубежными партнерами;
	
	\pagebreak
	\item стимулируют развитие региона через специализацию на одном виде деятельности, обеспечивая приток рабочей силы, обладающей специализированными навыками, которая постоянно повышает свою квалификацию;
	\item деятельность кластера поддерживается крепкими связями с несколькими отраслями, что помогает осуществлять трансфер знаний и навыков, приспосабливаться к изменяющимся потребностям общества;

	\pagebreak
	\item эффект синергии в инновационном развитии в результате сосредоточения на территории кластера малого, среднего и крупного бизнеса.
\end{itemize}
\end{frame}

 
\begin{frame}{Особые экономические зоны}
\begin{block}{Особые экономические зоны (ОЭЗ)}
	\quad
	- это ограниченная территория с особым юридическим статусом по отношению к остальной территории и льготными экономическими условиями для национальных и /или иностранных предпринимателей.
\end{block}
\end{frame}

 

\begin{frame}{Характеристики ОЭЗ}{(Всемирный банк)}
\begin{itemize}
	\item географически обособленная территория;
	\item управляется единым органом (администрацией);
	\item предлагает преимущества для своих резидентов;
	\item имеет обособленную таможенную территорию.
\end{itemize}
\end{frame}


\begin{frame}[allowframebreaks]{Модели создания объектов инновационной инфраструктуры}{}
\begin{itemize}
	\item североамериканская модель — характеризуется минимальным вмешательством государства и высоким уровнем взаимодействия научной и производственной деятельности (яркий пример — Силиконовая долина);

	\pagebreak
	\item французско-японская модель — предполагает создание огромных технополисов, на территории которых сосредоточено несколько других объектов инновационной инфраструктуры;
	
	\pagebreak
	\item скандинавская модель — сконцентрирована преимущественно на создании небольших парков и реализации национальных программ развития, нежели на участии в международных проектах;

	\pagebreak
	\item южно-европейская модель — делает акцент в основном на модернизации производства и создании новых рабочих мест и предполагает активное участие межгосударственных европейских фондов в создании инфраструктуры.
\end{itemize}
\end{frame}



\begin{frame}[allowframebreaks]{Научные, исследовательские и технологические парки }{Особенности}
\begin{itemize}
	\item одной из общих черт научных, исследовательских и технологических парков по всему миру является то, что практически все они созданы вблизи или вокруг вузов, реже — крупного научного центра с налаженными транспортными связями, со сложившейся предпринимательской средой.

	\pagebreak	
	\item в России и Восточной Европе вузы реализовывали прежде всего свою основную функцию — обучение, а фундаментальные и прикладные исследования осуществлялись за пределами образовательных учреждений — в академиях наук, отраслевых НИИ которые имели минимальные связи с вузами.
\end{itemize}
\end{frame}

\subsection{Финансовая модель}


\begin{frame}{Эффективность инвестиций в инновационную инфраструктуру}
\begin{itemize}
	\item частный бизнес предпочитает участвовать в развитии объектов инновационной инфраструктуры в качестве резидента;
	\item совместные частные и государственные инвестиции в объекты инновационной инфраструктуры являются наиболее эффективной моделью финансирования.
\end{itemize}
\end{frame}

\begin{frame}{Формы государственной поддержки }{при создании научных и технологических парков в мире}
% Table generated by Excel2LaTeX from sheet 'Лист1'
	\begin{table}[htbp]
	\centering
	\caption{\captionf{Формы государственной поддержки научных и технологических парков}}
		\begin{tabularx}{\linewidth}[b]{@{}>{\raggedright\arraybackslash}Xr@{}}
			\setrulecolor\toprule
			\cnamef{Форма поддержки} & \cnamef{доля, \%} \\
			\midrule
			Гранты & 45.45\% \\
			Субсидии & 40.26\% \\
			\cellwithlinebreak{b}{Консультации и \linebreak методологическая поддержка} & 31.17\% \\
			Налоговые стимулы & 27.27\% \\
			Льготное кредитование & 20.78\% \\
			Нефинансовые виды поддержки &  1.30\%\\
			\bottomrule
		\end{tabularx}%
		\label{tab:addlabel}%
	\end{table}%
\end{frame}



\begin{frame}[allowframebreaks]{Способы финансирования государством}{объектов инновационной инфраструктуры}
\begin{itemize}
	\item покрытие всех издержек объекта на условиях администрирования всех его доходов в пользу бюджета (налоговые платежи, аренда, плата за консалтинговые, юридические и пр. услуги);
	
	\pagebreak
	\item покрытие дефицита баланса объекта;
	\item предоставление субсидии инкубатору в зависимости от результатов его деятельности.
\end{itemize}
\end{frame}


\subsection{Резиденты}
 
\begin{frame}{Резиденты объектов инновационной инфраструктуры}
\begin{block}{}
	\quad
	- это организации, удовлетворяющие хотя бы одному из критериев:
	 наличие подписанного соглашения, уплата членских взносов или оказание финансовой поддержки управляющей компании инфраструктурного объекта на регулярной основе, активное содействие развитию кластера на регулярной основе.
\end{block}
\end{frame}

 
\begin{frame}{}
\begin{block}{Резиденты инкубатора}
	\quad
	- становятся малые и микропредприятия, а также индивидуальные предприниматели.
\end{block}
Структура резидентов инкубаторов в Европе: 

более 80\% — только что созданные фирмы; 

около 17\% — это структурные подразделения вузов и крупного бизнеса;

1\% - давно существующие фирмы.
\end{frame}

 
\begin{frame}[allowframebreaks]{Резиденты}{научных, технологических, индустриальных технопарков, кластеров, ОЭЗ}
\begin{itemize}
	\item активно развивающиеся технологичные компании и даже транснациональные компании;
	\item государственные структуры, НИИ, вузы и образовательные центры, финансовые учреждения, венчурные инвесторы, консультанты.
\end{itemize}
\end{frame}


 


\begin{frame}[allowframebreaks]{Структура резидентов в европейских кластерах}{}
\begin{itemize}
	\item 45-60\% субъекты МСП (за исключением Финляндии — около 80\% и Исландии — 16\%); 
	\item 10-20\% — частный бизнес, кроме МСП (в Финляндии — около 5\%); 
	\item 5-20\% — консультационные компании; 

	\pagebreak
	\item 5-20\% — вузы и другие образовательные центры; 
	\item 37\% — государственные структуры, 
	\item 47\% — НИИ.
\end{itemize}

В научных парках из числа технологичных компаний 50-70\% малый бизнес, остальную долю — средние и крупные компании.
\end{frame}

 


\begin{frame}[allowframebreaks]{Меры поддержки для резидентов}{Инфраструктурная поддержка}
\begin{itemize}
	\item предоставление рабочего пространства, включая лаборатории, центрами тестирования продукции;
	\item предоставление оргтехники и специального оборудования;
	
	\pagebreak
	\item предоставление обучающих аудиторий и конференц-залов;
	\item обеспечение социальными объектами.
\end{itemize}
\end{frame}

 


\begin{frame}[allowframebreaks]{Меры поддержки для резидентов}{Техническая поддержка}
\begin{itemize}
	\item доступ к информационным источникам;
	\item программное обеспечение;
	\item услуги системы безопасности;
	
	\pagebreak
	\item обеспечение связи;
	\item услуги по транспортировке и логистике.
\end{itemize}
\end{frame}

 
\begin{frame}[allowframebreaks]{Меры поддержки для резидентов}{Финансовая поддержка}
\begin{itemize}
	\item прямые инвестиции в стартапы;
	\item сотрудничество с финансовыми институтами, венчурными фондами, бизнес-ангелами;
	\item кредитование резидентов;
	
	\pagebreak
	\item предоставление гарантий;
	\item пониженные ставки аренды земли и производственных помещений;
	\item льготы по налогам и другим обязательным платежам;
	
	\pagebreak
	\item выделение государственных грантов;
	\item заключение контрактов по госзакупкам.
\end{itemize}
\end{frame}

 


\begin{frame}[allowframebreaks]{Меры поддержки для резидентов}{Организационная и юридическая поддержка}
\begin{itemize}
	\item помощь в обеспечении прав и защиты интеллектуальной собственности;
	\item помощь в технологическом трансфере;
	
	\pagebreak
	\item помощь в бизнес-планировании, бухучете и аудите, маркетинге;
	\item сотрудничество с местными органами власти по вопросам лицензирования, получения разрешений на строительство и т.д.;
	
	\pagebreak
	\item помощь в таможенном оформлении;
	\item экспортная поддержка и помощь в международном сотрудничестве.
\end{itemize}
\end{frame}

\begin{frame}[allowframebreaks]{Меры поддержки для резидентов}{Кадровая поддержка}
\begin{itemize}
	\item учебно-методическая помощь, тренинговые программы;
	\item создание баз данных по вакансиям;
	\item сотрудничество с муниципальными или университетскими центрами занятости.
\end{itemize}
\end{frame}


\subsection{Льготы}

\begin{frame}[allowframebreaks]{Налоговые льготы }{предоставляются со стороны}
\begin{itemize}
	\item федерального правительства и в основном касаются подоходного и ряда корпоративных налогов;
	\item со стороны регионального правительства и муниципалитетов — в отношении налогов на имущество и коммунальных платежей.
\end{itemize}
\end{frame}

\begin{frame}[allowframebreaks]{Виды налоговых льгот}{}
\begin{itemize}
	\item отсрочка выплаты (предоставление налогового кредита);
	\item снижение ставок;

	\pagebreak
	\item полная отмена выплаты налога в период пребывания на территории объекта;
	\item к резидентам ОЭЗ не применяются национальные правила проведения экспортно-импортных операций.
\end{itemize}
\end{frame}

\subsection{Эффективность}
\begin{frame}[allowframebreaks]{Показатели}{развития науки и технологии, коммерциализации научных разработок}
\begin{itemize}
	\item количество новых продуктов (услуг, технологий);
	\item общий доход резидентов объекта от реализации продукции, в том числе от экспорта высокотехнологичных продуктов/услуг и от использования патентов и лицензий за рубежом;
	
	\pagebreak
	\item количество успешных компаний, покинувших объект;
	\item уровень загруженности (заполняемости/освоения) объекта;
	
	\pagebreak
	\item количество совместных проектов, реализуемых в рамках ГЧП и международного сотрудничества;
	\item расходы государства в расчете на одно рабочее место на территории объекта инновационной инфраструктуры;
	
	\pagebreak
	\item уровень (коэффициент) выживаемости компаний;
	\item количество публикаций.
\end{itemize}
\end{frame}

\subsection{Российская практика}
\begin{frame}[allowframebreaks]{Институты развития в России}{Перечень компаний, причисляемых к институтам развития}
\begin{itemize}
	\item ОАО «АвтоВАЗ»
	\item ГК «Росатом»
	\item ГК «Агентство по ипотечному  жилищному кредитованию»
	\item ОАО «Российский сельскохозяйственный банк» 
	
	\pagebreak
	\item ГК Внешэкономбанк
	\item ОАО «Российская венчурная компания»
	\item ОАО «Газпром»
	\item ОАО «Российская корпорация нанотехнологий» 
	
	\pagebreak
	\item Инвестиционный фонд Российской Федерации
	\item ОАО «Российская самолетостроительная корпорация «МиГ»
	\item ОАО «Научно-производственная корпорация «Уралвагонзавод» 
	
	\pagebreak
	\item Российский гуманитарный научный фонд 
	\item ОАО «Объединенная авиастроительная корпорация»
	\item ОАО «Российский инвестиционный фонд информационно-коммуникационных технологий»
	
	\pagebreak
	\item ОАО «Объединенная судостроительная корпорация»
	\item Российский фонд фундаментальных исследований
	\item ФГУП «Омское моторостроительное производственное объединение им. П.И. Баранова»

	\pagebreak
	\item Государственная корпорация «Ростехнологии»
	\item ОАО «Особые экономические зоны» 
	\item ОАО «Сбербанк России» 
	\item ОАО «Производственное объединение «Северное машиностроительное предприятие»

	\pagebreak
	\item Федеральный фонд содействия развитию жилищного строительства 
	\item Региональные фонды содействия развитию венчурных инвестиций в малые предприятия в научно-технической сфере
	\item Фонд «Сколково»

	\pagebreak
	\item ОАО «Российские железные дороги»
	\item Фонд содействия развитию малых форм предприятия в научно-технической сфере (фонд Бортника)

	\pagebreak
	\item ОАО «Росагролизинг»
	\item Фонд содействия реформированию жилищно-коммунального хозяйства
\end{itemize}
\end{frame}

\begin{frame}{Наукограды и особые экономические зоны в России}
% Table generated by Excel2LaTeX from sheet 'MunStatManager'
\begin{table}[htbp]
	\centering
	\tiny
	\caption{\captionf{Перечень Наукоградов РФ и ОЭЗ}}
	\begin{tabularx}{\linewidth}[b]
		{@{}>{\raggedright\arraybackslash}Xcc@{}}
	\setrulecolor\toprule
		\cnamef{Территория} & \cnamef{Статус} & \cnamef{Год присвоения} \\
		\midrule
		Бийск (Алтайский край) & НРФ   & 2005 \\
		Дубна (Московская область) & НРФ   & 2001 \\
		Жуковский (Московская область) & НРФ   & 2007 \\
		Кольцово (Новосибирская область) & НРФ   & 2003 \\
		Королёв (Московская область) & НРФ   & 2001 \\
		Мичуринск (Тамбовская область) & НРФ   & 2003 \\
		Обнинск (Калужская область) & НРФ   & 2000 \\
		Петергоф (Санкт-Петербург) & НРФ   & 2005 \\
		Пущино (Московская область) & НРФ   & 2005 \\
		Реутов (Московская область) & НРФ   & 2003 \\
		Фрязино (Московская область) & НРФ   & 2003 \\
		Троицк (Москва) & НРФ   & 2012 \\
		Черноголовка (Московская область) & НРФ   & 2008 \\
		Протвино (Московская область) & НРФ   & 2008 \\
		Верхняя Салда (Свердловская область) & ОЭЗ ППТ & 2010 \\
		Грязи (Липецкая область) & ОЭЗ ППТ & 2005 \\
		Елабуга (Республика Татарстан) & ОЭЗ ППТ & 2005 \\
		Людиново (Калужская область) & ОЭЗ ППТ & 2012 \\
		Моглино (Псковская область) & ОЭЗ ППТ & 2012 \\
		Тольятти (Подстёпки) (Самарская область) & ОЭЗ ППТ & 2010 \\
		Зеленоград (Москва) & ОЭЗ ТВТ & 2005 \\
		Иннополис (Республика Татарстан (Казань)) & ОЭЗ ТВТ & 2012 \\
		Томск (Томская область) & ОЭЗ ТВТ & 2005 \\
		\bottomrule
	\end{tabularx}%
	\label{tab:addlabel}%
\end{table}%
\end{frame}

\begin{frame}{Обход существующих процедурных ограничений бюджетного характера}{с использованием институтов развития (ИР)}
\begin{itemize}
	\item гибкость в принятии решений о расходах;
	\item отсутствие заинтересованности в результатах деятельности;	\item синдром ``конца бюджетного года``;
	
	\pagebreak
	\item отсутствие в Бюджетном Кодексе РФ понятия гранта;
	\item предоставление бюджетных кредитов;
	\item вложения в уставный капитал организаций;

	\pagebreak
	\item госзакупки;
	\item цена заказа;
	\item финансирование НИОКР.
\end{itemize}
\end{frame}



\begin{frame}{Институты развития не в форме ГК или ОАО}{}
\begin{itemize}
	\item государственные компании (ГК ``Российские автомобильные дороги``), фонд ``Сколково``;
	\item автономные некоммерческие организации (Агентство стратегических инициатив).
\end{itemize}
\end{frame}

\begin{frame}{Государственная компания}
\begin{block}{}
	\quad
	- это некоммерческая организация, не имеющая членства и созданная Российской Федерацией на основе имущественных взносов для оказания государственных услуг и выполнения иных функций с использованием государственного имущества на основе доверительного управления (ст. 7.2 Федерального закона «Об НКО»).
\end{block}
\end{frame}

\begin{frame}{Автономная некоммерческая организация}
\begin{block}{}
	\quad
	- не имеющая членства некоммерческая организация, созданная в целях предоставления услуг в сфере образования, здравоохранения, культуры, науки, права, физической культуры и спорта и иных сферах.
\end{block}
\end{frame}

\subsection{ОЭЗ}

\begin{frame}[allowframebreaks]{}{}
\begin{block}{Особая экономическая зона}
	\quad
	- часть территории страны, на которой товары рассматриваются как объекты, находящиеся за пределами национальной таможенной территории, и поэтому не подвергаются обычному таможенному контролю и налогообложению (Международная конвенция по упрощению и гармонизации таможенных процедур (Киото, 18 мая 1973 г.)).
\end{block}

\pagebreak


\begin{block}{Особая экономическая зона}
	\quad
	- определяемая Правительством РФ часть территории РФ, на которой действует особый режим осуществления предпринимательской деятельности (116-ФЗ ``Об особых экономических зонах в РФ``).
\end{block}
\end{frame}



\begin{frame}[allowframebreaks]{Основные характеристики ОЭЗ}{}
\begin{itemize}
	\item таможенные льготы; 
	\item финансовые преференции;
	\item упрощенная процедура регистрации предприятий, лицензирования, проверок, льготный визовый режим для иностранных граждан;
	
	\pagebreak
	\item обособленная система органов управления зоной, наделенных правом принимать самостоятельные решения в широком экономическом спектре;
	\item всесторонняя поддержка проекта свободной экономической зоны центральной, региональной и местной государственной властью;

	\pagebreak
	\item максимальная открытость свободной экономической зоны для мирового рынка;
	\item норма прибыли в ОЭЗ может достигать 30-35\% в год.
\end{itemize}
\end{frame}




\begin{frame}[allowframebreaks]{Предпосылки успеха ОЭЗ}{}
\begin{itemize}
	\item благоприятное географическое положение по отношению к внешнему и внутреннему рынкам и наличие развитых транспортных коммуникаций;
	\item развитая производственная и социальная инфраструктура;
	наличие относительно дешевой и в то же время квалифицированной рабочей силы;
	
	\pagebreak
	\item развитый финансовый сектор, налаженные связи с международным финансовым рынком;
	\item отсутствие административных барьеров для организации бизнеса на территории особой экономической зоны, в том числе с участием иностранного капитала;

	\pagebreak
	\item адекватный уровень инвестиционного барьера — т. е. того размера финансовых средств, которые инвестор должен вложить в развитие особой экономической зоны за право вести в ней предпринимательскую деятельность на льготных условиях;

	\pagebreak
	\item развитая и стабильная правовая основа;
	\item государственные гарантии сохранности инвестиций и иного имущества, расположенного на территории особой зоны;

	\pagebreak
	\item специальные органы управления особой экономической зоной, а также четкое разделение их полномочий и сферы ответственности с другими органами центральной, региональной и местной власти;
	\item благоприятный инвестиционный климат в стране.
\end{itemize}
\end{frame}



\begin{frame}[allowframebreaks]{Роль ОЭЗ в экономике страны и региона}{}
\begin{itemize}
	\item мощный катализатор социально-экономического, научно-технического и кадрового развития определенного региона, стимулируют создание новых рабочих мест и развитие высокотехнологичного промышленного производства;
	\item канал связи между мировой экономикой и экономикой конкретной страны (региона);
	
	\pagebreak
	\item образец наиболее прогрессивных форм производства, управления и технологий;
	\item инструмент привлечения иностранных инвестиций и мобилизации местных экономических ресурсов;

	\pagebreak
	\item стимулируют развитие новых форм бизнеса, являются своеобразным ``полем для экспериментов`` в странах с переходной экономикой.
\end{itemize}
\end{frame}



\begin{frame}[allowframebreaks]{Основные виды ОЭЗ}{}
\begin{itemize}
	\item зоны свободной торговли, или свободные таможенные зоны, транзитные зоны (Нидерланды, Латвия, Индия);
	\item промышленно-производственные зоны (США, Западная Европа, Китай);
	\item технико-внедренческие зоны, или технополисы (Япония, Китай, ``азиатские драконы`` — Тайвань, Сингапур, Южная Корея и т. д.);
	\item сервисные зоны, или зоны услуг (Кипр, Мальта, страны Карибского бассейна, другие страны, специализирующиеся на туризме и сфере сервиса);
	\item комплексные зоны, интегрирующие в себе черты двух или более свободных зон различных типов.
\end{itemize}
\end{frame}

\begin{frame}{ОЭЗ в России}{Экономические цели ОЭЗ}
\begin{itemize}
	\item активизация и расширение внешнеэкономической деятельности в целом;
	\item привлечение в экономику иностранных и национальных инвестиций;
	\item повышение конкурентоспособности национального производства и его экономической эффективности.
\end{itemize}
\end{frame}



\begin{frame}[allowframebreaks]{ОЭЗ в России}{Социальные цели ОЭЗ}
\begin{itemize}
	\item создание новых рабочих мест, рост занятости населения;
	\item обучение и подготовка квалифицированных рабочих, инженерных, хозяйственных и управленческих кадров с учетом мирового опыта;

	\pagebreak
	\item насыщение национального рынка высококачественными товарами и услугами производственного и потребительского назначения;
	\item рост благосостояния и уровня жизни населения;
	
	\pagebreak
	\item ускорение развития отсталых регионов за счет концентрации в пределах зон ограниченных национальных ресурсов.
\end{itemize}
\end{frame}



\begin{frame}[allowframebreaks]{ОЭЗ в России}{Научно-технические цели ОЭЗ}
\begin{itemize}
	\item активное использование новейших зарубежных и отечественных технологий;
	\item ускорение внедрения результатов НИОКР;
	\item концентрация научно-технических кадров, в том числе зарубежных, на приоритетных направлениях развития технологий;
	
	\pagebreak
	\item использование опыта и научно-исследовательских достижений научно-технических центров и венчурных компаний;
	\item повышение эффективности используемых производственных мощностей, в частности конверсионных.
\end{itemize}
\end{frame}



\begin{frame}{Органы управление ОЭЗ}{}
\begin{itemize}
	\item Минэкономразвития РФ (МЭРТ);
	\item Федеральное агентство по управлению особыми экономическими зонами (РосОЭЗ).
\end{itemize}
\end{frame}

 
\begin{frame}[allowframebreaks]{Система льгот и преференций для инвесторов}{(ФЗ ``Об особых экономических зонах``)}
\begin{itemize}
	\item принцип ``одного окна`` - Агентство по управлению особой экономической зоной;
	\item особый налоговый режим;.
	\item особый таможенный режим;
	
	\pagebreak
	\item право аренды земельного участка на территории особой экономической зоны для резидентов;
	\item правовые гарантии защиты прав инвесторов (неизменность законодательства, судебный порядок разрешения споров и т.д.).
\end{itemize}
\end{frame}

\begin{frame}[allowframebreaks]{Виды ОЭЗ в России}{}
\begin{itemize}
	\item промышленно-производственные;
	\item технико-внедренческие;
	\item туристско-рекреационные;
	\item портовые.
\end{itemize}
\end{frame}




\begin{frame}{Промышленно-производственные ОЭЗ (ППОЭЗ)}
\begin{block}{ППОЭЗ создаются}
	\quad
	- на неограниченном числе земельных участков, находящихся в государственной или муниципальной собственности;
	
	на участках площадью не более 20 кв. км;
	
	на участках, которые находятся на территории одного муниципального образования;
	
	на участках, не занимающих территорию муниципального образования полностью.
\end{block}
Липецкая область, Республика Татарстан.
\end{frame}

\begin{frame}{На территории ППОЭЗ не допускается}{}
\begin{itemize}
	\item размещение объектов жилищного фонда;
	\item разработка месторождений полезных ископаемых и их переработка (кроме минеральных вод и природных лечебных ресурсов);
	\item производство и переработка подакцизных товаров (за исключением легковых автомобилей и мотоциклов).
\end{itemize}
\end{frame}

\begin{frame}{}
\begin{block}{Резидент ППОЭЗ}
	\quad
	- обязан заключить соглашение о ведении промышленно-производственной деятельности и осуществить капитальные вложения не менее\\ 10 млн евро.
\end{block}
\end{frame}


 

\begin{frame}{Технико-внедренческие ОЭЗ (ТВОЭЗ)}
\begin{block}{создаются }
	\quad
	- не более чем на двух участках территории, находящихся в государственной или муниципальной собственности;
	
	на участках, общая площадь которых составляет не более 3 кв. км.;
	
	ТВОЭЗ не может находиться на территориях нескольких муниципальных образований;
	
	ТВОЭЗ не должна включать в себя полностью территорию какого-либо административно-территориального образования.
\end{block}
\end{frame}

\begin{frame}{ТВОЭЗ созданы}
Санкт-Петербург, Томск, Дубна (Московская область), Зеленоградский административный округ г. Москвы.
\end{frame}

\begin{frame}{Резиденты ТВОЭЗ}
\begin{block}{}
	\quad
	- это индивидуальные предприниматели и коммерческие организации, для них не предусмотрен финансовый барьер в виде обязательных капитальных вложений.
\end{block}
Вправе вести только технико-внедренческую деятельность: производство только опытных партий продукции либо работа в секторе информационных технологий.
\end{frame}




\begin{frame}{Туристско-рекреационные ОЭЗ (ТРОЭЗ)}
\begin{block}{}
	\quad
	- может создаваться на одном или нескольких участках, находящихся в государственной или муниципальной собственности;
	
	может размещаться на территории нескольких муниципальных образований и занимать территорию всего муниципального образования полностью;
	
	земельные участки могут находиться во владении граждан и ю.л., в составе земель лесного фонда, особо охраняемых территорий;
	
	разрешено размещать объекты жилого фонда.
\end{block}
\end{frame}

\begin{frame}{ТРОЭЗ созданы}
Калининградская область, Краснодарский край, Ставропольский край, Алтайский край, Республика Алтай, Республика Бурятия и Иркутская область.
\end{frame}

\begin{frame}{Резидент ТРОЭЗ}
\begin{block}{}
	\quad
	- в течение срока действия соглашения обязуется вести туристско-рекреационную деятельность, предусмотренную соглашением, необходимое имущество он получает по договору аренды с местными властями.
\end{block}
\end{frame}


\begin{frame}{Портовые ОЭЗ (ПОЭЗ)}
\begin{block}{}
	\quad
	- сроком на 49 лет на участках общей площадью не более 50 кв. км, имеющих общую границу и находящихся в государственной, муниципальной собственности, в собственности граждан и ю.л.;
	
	может размещаться на территории нескольких муниципальных образований и занимать территорию всего муниципального образования полностью.	
\end{block}
\end{frame}



\begin{frame}{Резидент ПОЭЗ }
\begin{block}{}
	\quad
	- обязуется осуществить капитальные вложения в рублях в сумме:
	
	не менее 100 млн евро при строительстве инфраструктуры нового морского порта;
	
	не менее 50 млн евро при строительстве объектов инфраструктуры нового речного порта или нового аэропорта;
	
	не менее трех млн евро при реконструкции объектов инфраструктуры порта.
\end{block}
\end{frame}


\begin{frame}[allowframebreaks]{Деятельность резидентов ПОЭЗ}{}
\begin{itemize}
	\item погрузочно-разгрузочные работы;
	\item складирование и хранение товаров, а также транспортно-экспедиторские услуги;
	\item снабжение и снаряжение водных и воздушных судов, в том числе судовыми и бортовыми припасами, а также оснащение водных и воздушных судов;
	
	\pagebreak
	\item ремонт, техническое обслуживание, модернизация морских и речных судов, воздушных судов, авиационной техники, в том числе авиационных двигателей и других комплектующих изделий;
	\item переработка водных биологических ресурсов;

	\pagebreak
	\item операции по подготовке товаров к продаже и транспортировке (упаковка, сортировка, переупаковка, деление партии, маркировка и тому подобные операции);
	\item простые сборочные и иные операции, осуществление которых существенно не изменяет состояние товара, в соответствии с перечнем, утверждаемым Правительством Российской Федерации;

	\pagebreak
	\item биржевая торговля товарами;
	\item оптовая торговля товарами;
	\item обеспечение функционирования объектов инфраструктуры портовой особой экономической зоны.
\end{itemize}
\end{frame}

 

\begin{frame}[allowframebreaks]{Экономика ПОЭЗ}{}
\begin{itemize}
	\item непосредственное управление ПОЭЗ осуществляет РосОЭЗ;
	\item из порта «изгоняются» региональные власти и уже сложившиеся крупные управляющие компании; 
	\item во многих портах давно сформировалась структура собственников и управляющих компаний, для которых приход на эти территории государственного оператора означает весьма болезненный передел собственности; 
	
	\pagebreak
	\item регионы лишаются части налогов, уходящих на льготы резидентам;
	\item фактически регионы не заинтересованы в создании ПОЭЗ.
\end{itemize}
\end{frame}




\end{document}
