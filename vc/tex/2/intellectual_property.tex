% !TeX program = lualatex -synctex=1 -interaction=nonstopmode --shell-escape %.tex

\documentclass[_Venture_p2.tex]{subfiles}

\begin{document}

\setbeamercovered{invisible}

\subsection{Определения}
\begin{frame}[shrink=5]{Определения}
\begin{block}{Интеллектуальная собственность}
\quad
— результаты интеллектуальной деятельности и приравненные к ним средства индивидуализации юридических лиц, товаров, работ, услуг и предприятий, которым предоставляется правовая охрана.
\end{block}
\end{frame}

\begin{frame}[shrink=5]{Определения}
\begin{block}{Интеллектуальные права}
	\quad
	— признаются на результаты интеллектуальной деятельности, которые включают исключительное право, являющееся имущественным правом, а также личные неимущественные права и иные права (право следования, право доступа и другие).
	
	Право авторства, право на имя и иные личные неимущественные права автора неотчуждаемы и непередаваемы. Отказ от этих прав ничтожен.
	
	Авторство и имя автора охраняются бессрочно. После смерти автора защиту его авторства и имени может осуществлять любое заинтересованное лицо.
\end{block}
\end{frame}

\begin{frame}[shrink=5]{Определения}
\begin{block}{Исключительное право}
	\quad
	Гражданин или юридическое лицо, обладающие исключительным правом на результат интеллектуальной деятельности или на средство индивидуализации (правообладатель), вправе использовать такой результат или такое средство по своему усмотрению любым не противоречащим закону способом.
	
	Это право может быть передано автором другому лицу по договору, а также может перейти к другим лицам по иным основаниям, установленным законом.
\end{block}
\end{frame}


\begin{frame}[shrink=20]{Характеристики исключительного права}
\begin{itemize}
\item Правообладатель может по своему усмотрению разрешать или запрещать другим лицам использование результата интеллектуальной деятельности или средства индивидуализации. Отсутствие запрета не считается согласием (разрешением).

\item Исключительное право (кроме исключительного права на фирменное наименование) может принадлежать одному лицу или нескольким лицам совместно.

\item 	Доходы от совместного распоряжения исключительным правом распределяются между всеми правообладателями в равных долях, если соглашением между ними не предусмотрено иное.

\item Исключительные права действуют в течение определенного срока.
\end{itemize}
\end{frame}

\begin{frame}[shrink=15]{Определения}
\begin{block}{Автор результата интеллектуальной деятельности }
\quad
— это гражданин, творческим трудом которого создан такой результат.

Не признаются авторами результата интеллектуальной деятельности граждане, не внесшие личного творческого вклада в создание такого результата, в том числе оказавшие его автору только техническое, консультационное, организационное или материальное содействие или помощь либо только способствовавшие оформлению прав на такой результат или его использованию, а также граждане, осуществлявшие контроль за выполнением соответствующих работ.
\end{block}
\end{frame}
\subsection{Классификация интеллекуальной собственности}
\begin{frame}{Классификация объектов интеллекуальной собственности}
\begin{figure}
	\centering
	\begin{overprint}
		\forloop{slideno}{1}{\value{slideno} < 18}{%
			\only<\value{slideno}>{
				\includegraphics[page=\value{slideno},
				scale=.7
				% trim={<left> <lower> <right> <upper>}				
				,trim={0cm 0cm 0 0cm},clip]
				{tikz/intellectual_property_classification}}}
	\end{overprint}
	\vspace*{-1.5cm}
	\caption{Виды интеллекуальной собственности}
\end{figure}
\end{frame}

\subsection{Практика правоприменения}
\begin{frame}{Результаты интеллектуальной деятельности }{Государственная регистрация }
Исключительное право на результат интеллектуальной деятельности или на средство индивидуализации признается и охраняется при условии государственной регистрации такого результата или такого средства
\end{frame}

\begin{frame}{Распоряжение исключительным правом}
Договоры о распоряжении исключительным правом:
\begin{itemize}
	\item отчуждение исключительного права;
	\item лицензионный (сублицензионный).
\end{itemize}
Заключение лицензионного договора не влечет за собой переход исключительного права к лицензиату.
\end{frame}

\begin{frame}[shrink=15]{Безвозмездное использование}
\begin{itemize}
	\item Правообладатель может сделать публичное заявление о предоставлении любым лицам возможности безвозмездно использовать принадлежащие ему произведение науки, литературы или искусства.
	
	\item Заявление делается путем размещения на официальном сайте федерального органа исполнительной власти в сети "Интернет" (не может быть отозвано).
	  	
	\item При отсутствии в заявлении правообладателя указания на срок считается, что указанный срок составляет пять лет.
\end{itemize}
\end{frame}

\begin{frame}[shrink=20]{Договор об отчуждении исключительного права}
\begin{itemize}
	\item Одна сторона (правообладатель) передает или обязуется передать принадлежащее ей исключительное право в полном объеме другой стороне (приобретателю).
	
	\item Письменная форма, иначе договор не действителен.
	
	\item Приобретатель обязуется уплатить правообладателю вознаграждение, если не предусмотрено иное.
	
	\item При отсутствии условия о размере вознаграждения или порядке его определения договор считается незаключенным. 
	
	\item Вознаграждение в форме фиксированных разовых или периодических платежей, процентных отчислений от дохода (выручки) либо в иной форме.
\end{itemize}
\end{frame}

\begin{frame}[allowframebreaks]{Лицензионный договор}
\begin{itemize}
	\item обладатель исключительного права (лицензиар) предоставляет или обязуется предоставить другой стороне (лицензиату) право использования такого результата или такого средства в предусмотренных договором пределах;
	\item письменная форма, иначе договор не действителен;
	\item территория использования, если не указано, то вся территория России; 
	
	\pagebreak
	\item срок не может превышать срок действия исключительного права;
	\item если срок в договоре не указан, то 5 лет;
	\item вознаграждение в форме фиксированных разовых или периодических платежей, процентных отчислений от дохода (выручки) либо в иной форме.
\end{itemize}
\end{frame}
\begin{frame}{Обязательные условия лицензионного договора}
\begin{itemize}
	\item предмет договора: номера документа, удостоверяющего исключительное право на такой результат или на такое средство (патент, свидетельство);
	\item способы использования;
	\item переход исключительного права к новому правообладателю не является основанием для изменения или расторжения лицензионного договора.
\end{itemize}
\end{frame}

\begin{frame}{Виды лицензионных договоров}
\begin{itemize}
	\item \textbf{\textit{Простая лицензия }}- предоставление лицензиату права использования результата интеллектуальной деятельности или средства индивидуализации с сохранением за лицензиаром права выдачи лицензий другим лицам.
	\item \textbf{\textit{Исключительная лицензия}} - предоставление лицензиату права использования результата интеллектуальной деятельности или средства индивидуализации без сохранения за лицензиаром права выдачи лицензий другим лицам.
\end{itemize}
\end{frame}

\begin{frame}{Принудительная лицензия}
суд может по требованию заинтересованного лица принять решение о предоставлении этому лицу на указанных в решении суда условиях права использования результата интеллектуальной деятельности, исключительное право на который принадлежит другому лицу (принудительная лицензия)
\end{frame}

\begin{frame}{Переход исключительного права к другим лицам без договора}
Переход исключительного права на результат интеллектуальной деятельности или на средство индивидуализации к другому лицу без заключения договора с правообладателем допускается в случаях и по основаниям, которые установлены законом, в том числе в порядке универсального правопреемства (наследование, реорганизация юридического лица) и при обращении взыскания на имущество правообладателя.
\end{frame}

\begin{frame}{Коллективное управление авторскими и смежными правами}
Авторы в случаях, когда осуществление их прав в индивидуальном порядке затруднено или когда настоящим Кодексом допускается использование объектов авторских и смежных прав без согласия обладателей соответствующих прав, но с выплатой им вознаграждения, могут создавать основанные на членстве некоммерческие организации, на которые в соответствии с полномочиями, предоставленными им правообладателями, возлагается управление соответствующими правами на коллективной основе (организации по управлению правами на коллективной основе).
\end{frame}
\subsection{Отдельные виды интеллектуальной собственности}
\begin{frame}{Авторские права}
\begin{itemize}
	\item исключительное право на произведение;
	\item право авторства;
	\item право автора на имя;
	\item право на неприкосновенность произведения;
	\item право на обнародование произведения.
\end{itemize}
\end{frame}
\begin{frame}[ allowframebreaks]{Объекты авторских прав}
\begin{itemize}
	\item литературные произведения;
	\item драматические и музыкально-драматические произведения, сценарные произведения;
	\item хореографические произведения и пантомимы;
	\item музыкальные произведения с текстом или без текста;
	\item аудиовизуальные произведения;
	\pagebreak
	\item произведения живописи, скульптуры, графики, дизайна, графические рассказы, комиксы и другие произведения изобразительного искусства;
	\item произведения декоративно-прикладного и сценографического искусства;
	\item произведения архитектуры, градостроительства и садово-паркового искусства, в том числе в виде проектов, чертежей, изображений и макетов;
	\pagebreak
	\item фотографические произведения и произведения, полученные способами, аналогичными фотографии;
	\item географические и другие карты, планы, эскизы и пластические произведения, относящиеся к географии и к другим наукам;
	программы для ЭВМ, которые охраняются как литературные произведения;
	\pagebreak
	\item производные произведения, то есть произведения, представляющие собой переработку другого произведения;
	\item составные произведения, то есть произведения, представляющие собой по подбору или расположению материалов результат творческого труда.
	\item другие произведения.
	
\end{itemize}
\end{frame}

\begin{frame}[ allowframebreaks ]{Не являются объектами авторских прав}
1) официальные документы государственных органов и органов местного самоуправления муниципальных образований, в том числе законы, другие нормативные акты, судебные решения, иные материалы законодательного, административного и судебного характера, официальные документы международных организаций, а также их официальные переводы;

2) государственные символы и знаки (флаги, гербы, ордена, денежные знаки и тому подобное), а также символы и знаки муниципальных образований;

\pagebreak
3) произведения народного творчества (фольклор), не имеющие конкретных авторов;

4) сообщения о событиях и фактах, имеющие исключительно информационный характер (сообщения о новостях дня, программы телепередач, расписания движения транспортных средств и тому подобное).
\end{frame}

\begin{frame}{Производные и составные произведения}

\begin{itemize}
	\item Переводы, обработки, экранизации, аранжировки, инсценировки или др. - авторские права на переработку.
	
	\item Сборники, антологии, энциклопедии, базы данных, интернет-сайта, атласа или др. - авторские права на подбор и расположение материалов.
	
	\item Базы данных.
\end{itemize}
\end{frame}

\begin{frame}{Государственная регистрация программ для ЭВМ и баз данных}
Правообладатель в течение срока действия исключительного права на программу для ЭВМ или на базу данных может по своему желанию зарегистрировать такую программу или такую базу данных в федеральном органе исполнительной власти по интеллектуальной собственности.
\end{frame}

\begin{frame}{Право авторства и право автора на имя}
Право авторства - право признаваться автором произведения и право автора на имя - право использовать или разрешать использование произведения под своим именем, под вымышленным именем (псевдонимом) или без указания имени, то есть анонимно, неотчуждаемы и непередаваемы, в том числе при передаче другому лицу или переходе к нему исключительного права на произведение и при предоставлении другому лицу права использования произведения. Отказ от этих прав ничтожен.
\end{frame}

\begin{frame}[allowframebreaks]{Неприкосновенность и защита произведения от искажений}
Не допускается без согласия автора внесение в его произведение изменений, сокращений и дополнений, снабжение произведения при его использовании иллюстрациями, предисловием, послесловием, комментариями или какими бы то ни было пояснениями (право на неприкосновенность произведения).

\pagebreak
При использовании произведения после смерти автора лицо, обладающее исключительным правом на произведение, вправе разрешить внесение в произведение изменений, сокращений или дополнений при условии, что этим не искажается замысел автора и не нарушается целостность восприятия произведения и это не противоречит воле автора, определенно выраженной им в завещании, письмах, дневниках или иной письменной форме.

\textbf{Авторство, имя автора и неприкосновенность произведения охраняются бессрочно.}
\end{frame}

\begin{frame}{Знак охраны авторского права}
состоит из следующих элементов:
\begin{itemize}
	\item латинской буквы "C" в окружности;
	\item имени или наименования правообладателя;
	\item года первого опубликования произведения.
\end{itemize}
\end{frame}

\begin{frame}{Свободное воспроизведение произведения в личных целях}
Допускается без согласия автора или иного правообладателя и без выплаты вознаграждения воспроизведение гражданином при необходимости и исключительно в личных целях правомерно обнародованного произведения.
\end{frame}

\begin{frame}[allowframebreaks]{\setfontsize{12pt}Свободное воспроизведение произведения в личных целях}{Исключения}
\begin{itemize}
	\item воспроизведения произведений архитектуры в форме зданий и аналогичных сооружений;
	\item воспроизведения баз данных или их существенных частей;
	\item воспроизведения программ для ЭВМ;
	\item репродуцирования книг (полностью) и нотных текстов не в целях издания;
	
	\pagebreak
	\item видеозаписи аудиовизуального произведения при его публичном исполнении в месте, открытом для свободного посещения, или в месте, где присутствует значительное число лиц, не принадлежащих к обычному кругу семьи;
	\item воспроизведения аудиовизуального произведения с помощью профессионального оборудования, не предназначенного для использования в домашних условиях.
\end{itemize}
\end{frame}

\begin{frame}[ allowframebreaks]{Свободное использование произведений}{в информационных, научных, учебных или культурных целях}
\begin{itemize}
	\item цитирование в объеме, оправданном целью цитирования;
	\item использование в качестве иллюстраций в объеме, оправданном поставленной целью;
	\item воспроизведение в периодическом печатном издании если это не было специально запрещено автором или иным правообладателем;
	\item воспроизведение публично произнесенных политических речей, обращений, докладов и аналогичных произведений в объеме, оправданном информационной целью. 
	
	\pagebreak
	\item воспроизведение произведений, которые становятся увиденными или услышанными в средствами фотографии, кинематографии, телевидения и радио, в объеме, оправданном информационной целью;
	\item публичное исполнение без цели извлечения прибыли;
	\item запись на электронном носителе, в том числе запись в память ЭВМ, и доведение до всеобщего сведения авторефератов диссертаций.
\end{itemize}
\end{frame}

\begin{frame}{Патентные права}

\begin{block}{Патентные права}
	\quad - интеллектуальные права на изобретения, полезные модели и промышленные образцы.
\end{block}
Авторам принадлежат следующие права:

1) исключительное право;

2) право авторства.
\end{frame}

\begin{frame}
В качестве изобретения охраняется техническое решение в любой области, относящееся к продукту (в частности, устройству, веществу, штамму микроорганизма, культуре клеток растений или животных) или способу (процессу осуществления действий над материальным объектом с помощью материальных средств), в том числе к применению продукта или способа по определенному назначению.
\end{frame}

\begin{frame}{Не являются изобретениями}
1) открытия;

2) научные теории и математические методы;

3) решения, касающиеся только внешнего вида изделий и направленные на удовлетворение эстетических потребностей;

4) правила и методы игр, интеллектуальной или хозяйственной деятельности;

5) программы для ЭВМ;

6) решения, заключающиеся только в представлении информации.
\end{frame}

\begin{frame}[shrink=20]{Секрет производства (ноу-хау)}
Секретом производства (ноу-хау) признаются сведения любого характера (производственные, технические, экономические, организационные и другие) о результатах интеллектуальной деятельности в научно-технической сфере и о способах осуществления профессиональной деятельности, имеющие действительную или потенциальную коммерческую ценность вследствие неизвестности их третьим лицам, если к таким сведениям у третьих лиц нет свободного доступа на законном основании и обладатель таких сведений принимает разумные меры для соблюдения их конфиденциальности, в том числе путем введения режима коммерческой тайны.
\end{frame}
\begin{frame}[allowframebreaks]{\setfontsize{12pt}Служебный секрет производства}
Исключительное право на секрет производства, созданный работником в связи с выполнением своих трудовых обязанностей или конкретного задания работодателя (служебный секрет производства), принадлежит работодателю.

\pagebreak
Гражданин, которому в связи с выполнением своих трудовых обязанностей или конкретного задания работодателя стал известен секрет производства, обязан сохранять конфиденциальность полученных сведений до прекращения действия исключительного права на секрет производства.

Нарушитель обязан возместить убытки, причиненные нарушением исключительного права на секрет производства, если иная ответственность не предусмотрена законом или договором с этим лицом.
\end{frame}

\begin{frame}{Право на технологию}
\begin{block}{Единой технологией }
	\quad признается выраженный в объективной форме результат научно-технической деятельности, который включает в том или ином сочетании результаты интеллектуальной деятельности, подлежащие правовой охране и может служить технологической основой определенной практической деятельности в гражданской или военной сфере.
\end{block}
\end{frame}

\begin{frame}{Право использования единой технологии}
принадлежит лицу, организовавшему создание единой технологии (право на технологию) на основании договоров с обладателями исключительных прав на результаты интеллектуальной деятельности, входящие в состав единой технологии.
\end{frame}
\end{document}