% !TeX program = lualatex -synctex=1 -interaction=nonstopmode --shell-escape %.tex

\documentclass[_Venture_p2.tex]{subfiles}

\begin{document}

\setbeamercovered{invisible}

\subsection{Определения}
\begin{frame}{Определения}
\begin{block}{Бизнес-ангел, англ. business angel, angel investor}
\quad
- это частные лица, вкладывающие собственные средства в небольшие начинающие компании или предпринимателей (посевное финансирование). Бизнес-ангел может совершить как одно вложение для старта бизнеса или выделять средства на постоянной основе для поддержки и содержания компании на ранних стадиях развития.
\end{block}
\end{frame}

\begin{frame}[allowframebreaks]{Бизнес ангелы}{История возникновения термина }
\begin{itemize}
	\item Нью-Йорк, нач. XX века;
	\item в Театральном квартале вокруг Бродвея «ангелами» называли обеспеченных поклонников театра, инвестировавших в новые постановки;	
	\item их привлекало покровительство искусству и близкое знакомство с именитыми актёрами и режиссёрами, ``ангел`` получал прибыль только в случае успеха постановки.
	
	\pagebreak
	\item Уильям Ветцель, профессор Университета Нью-Гэмпшира, 1978~г. впервые назвал бизнес-ангелами обеспеченных мужчин с опытом предпринимательской и управленческой деятельности, непублично вкладывающих средства в местные компании на ранних стадиях развития.	
\end{itemize}
\end{frame}

\begin{frame}[allowframebreaks]{}{}
\begin{itemize}
	\item бизнес-ангелы обычно предлагают более выгодные условия финансирования по сравнению с кредиторами, поскольку они обычно инвестируют в самого предпринимателя, не оценивая жизнеспособность его бизнеса;
	
	\pagebreak
	\item бизнес ангелы заинтересованы в запуске нового бизнеса, а не в последующей прибыли, которую они могли бы от этого получить;
	\item в связи с этим бизнес-ангелов рассматривают как противоположность венчурным капиталистам;
	
	\pagebreak
	\item инвесторами обычно выступают физические лица, но субъектом,фактически предоставляющим деньги может быть общество с ограниченной ответственностью, траст, инвестиционный фонд и т.п.;
	\item если стартап разоряется, инвесторы теряют все свои вложения;
	\item доходность инвестиций венчурных ангелов может составлять 20-30\% годовых.
\end{itemize}
\end{frame}

\subsection{Механизм финансирования}
\begin{frame}[allowframebreaks]{Механизм финансирования}{}
Бизнес-ангелы вкладывают деньги в начинающие компании: 
\begin{itemize}
	\item в обмен на долю в акционерном капитале;
	\item приобретение конвертируемых облигаций;
	\item использование онлайн платформ краудфандинга;
	\item создание сетей бизнес-ангелов.
\end{itemize}
\end{frame}


\begin{frame}[allowframebreaks]{Кто может стать бизнес-ангелом?}
В США:
\begin{itemize}
	\item Бизнес-ангелы  должны отвечать требованиям Комиссии по ценным бумагам США для аккредитованных инвесторов.
	\item Необходимо иметь чистый объем активов не менее 1 млн. долларов и годовой доход свыше 200 тыс. долл.
\end{itemize}

\pagebreak
В России:
\begin{itemize}
	\item физическое лицо должно быть официально признано квалифицированным инвестором, согласно критериям, установленным в законодательстве.
\end{itemize}
\end{frame}

\begin{frame}[allowframebreaks]{Особенности посевного финансирования}{}
\begin{itemize}
	\item очень высокие риски, превышающие таковые даже при венчурном финансировании;
	\item отсутствие четко выстроенной бизнес-модели у компании-реципиента;
	\item относительно малый объем необходимых инвестиций (до 1 млн долл.);
	\item инвесторы вкладывают собственный, а не привлеченный со стороны капитал.	
\end{itemize}
\end{frame}

\begin{frame}[allowframebreaks]{Мотивация бизнес-ангелов}{}
\begin{itemize}
	\item помощь друзьям, 
	\item желание ``ввязаться во что-нибудь интересное``, 
	\item ``рискнуть на миллион``. 
\end{itemize}
Предприниматели часто шутят, что на посевной стадии основными инвесторами выступают\\ \textbf{3F }~-~\textbf{Family, Friends, Fools }– семья, друзья и ``наивные``.
\end{frame}


\begin{frame}[allowframebreaks]{Отличия бизнес-ангелов от венчурных инвесторов}{}
\begin{itemize}
	\item инвестируют поодиночке в связи с очень высокими рисками и небольшими объемами инвестиций; 
	\item неформальный подход к оценке качества проекта, зависящий от личных отношений между инвестором и предпринимателем;
	
	\pagebreak
	\item решения бизнес-ангела об инвестировании в меньшей степени опираются на формализованную процедуру тщательной проверки, англ. due diligence;
	\item организационная структура - сети и ассоциации бизнес-ангелов, имеющие тесные связи с венчурными фондами, пример - ``Пенсионеры Калифорнии``;

	\pagebreak
	\item встречи бизнес-ангелов с компаниями и между собой происходят посредством собраний и специализированных мероприятий или онлайн. 
	\item возможно создание синдикатов бизнес-ангелов, которые управляются самими инвесторами, без привлечения УК, что экономит издержки.
\end{itemize}
\end{frame}

\begin{frame}[allowframebreaks]{Другие инвесторы на посевной стадии}{}
\begin{itemize}
	\item государственные и другие некоммерческие ``посевные`` фонды, ориентированные на поддержку начинающего инновационного бизнеса;
	\item средства из некоммерческих ``посевных`` фондов предоставляются посредством грантов, льготных беззалоговых кредитов, компенсаций процентных ставок, а также инвестиций, аналогичных венчурным.
\end{itemize}
\end{frame}

\subsection{Бизнес-ангелы в России}
\begin{frame}{Бизнес-ангелы  в России}
\begin{block}{}
	\quad
	 – это бизнесмены, владельцы различных фирм, топ-менеджеры крупных корпораций, иностранцы, просто состоятельные люди. 
	 
	 В основном это люди, обладающие серьезными средствами и имеющие опыт построения бизнеса.
\end{block}
\end{frame}

\begin{frame}[allowframebreaks]{Категории\\ бизнес-ангелов в России}{}
\begin{itemize}
	\item люди, вышедшие из науки или технологического бизнеса, эксперты в своей области, ищут проекты для интеграции в действующий бизнес;
	
	\pagebreak
	\item люди, представляющие интерес крупных корпораций и бизнес-групп, как российских, так и иностранных, неофициально ищущие объекты для предстоящих поглощений;
	\item люди из областей не связанных с техническим бизнесом и не имеющие отраслевых предпочтений, начинающие бизнес-ангелы.	
\end{itemize}
\end{frame}

\begin{frame}{Сети бизнес-ангелов в России}{}
\begin{itemize}
	\item Национальная сеть бизнес ангелов, 
	\item Московская сеть бизнес-ангелов, 
	\item Сеть бизнес-ангелов ``Частный капитал``, 
	\item Томская сеть рискового финансирования ``Бизнес-ангелы``,
	\item Нижегородская ассоциация ``Стартовые инвестиции``.
\end{itemize}
\end{frame}

\begin{frame}{Особенности российских бизнес-ангелов}{}
\begin{itemize}
	\item большинство берет контрольный пакет, 
	\item предпочитает инвестировать в своем регионе и рассматривает проект как часть своего бизнеса; 
	\item средний срок инвестиций составляет обычно 3-5 лет; 
	\item проекты, связанные с основным бизнесом, ищет около 30\% инвесторов; 
	\item средний размер синдиката бизнес-ангелов 1,5 – 7 млн. долл., состоящий из 3-4 членов.
\end{itemize}
\end{frame}

\begin{frame}[allowframebreaks]{Венчурные инвестиционные компании}{}
\begin{itemize}
	\item занимают промежуточное положение между бизнес-ангелами и венчурными фондами;
	\item регистрируются как ЗАО или ООО, а их инвесторами являются учредители или акционеры этих компаний;
	\item венчурное инвестирование для данных компаний – не единственное направление деятельности;
	
	\pagebreak
	\item эта юридическая форма выбирается компаниями в связи с высокими издержками создания и управления ЗПИФа;
	\item примеры: ОАО «НИКОР», ЗАО «АкадемПартнер», ООО «ВИСС». 
\end{itemize}
\end{frame}

\end{document}