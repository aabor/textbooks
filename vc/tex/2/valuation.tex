% !TeX program = lualatex -synctex=1 -interaction=nonstopmode --shell-escape %.tex

\documentclass[_Venture_p2.tex]{subfiles}

\begin{document}

\setbeamercovered{invisible}

\subsection{Оценка инновационных компаний}
\begin{frame}{Особенности оценки инновационных компаний}{}
\begin{itemize}
	\item текущие финансовые отчеты компании, её история, представленная в виде доходов и рыночных цен;
	\item текущие финансовые отчеты очень мало говорят об ожидаемом росте компании;
	\item зачастую оказываются первыми представителями в своем бизнесе, отсутствуют конкуренты или однородная по составу группа сопоставимых компаний.
\end{itemize}
\end{frame}

\begin{frame}[allowframebreaks]{Оценка инновационных компаний на развивающихся рынках}{Дополнительные риски и возможности}
\begin{itemize}
	\item приватизация государственных компаний и отказ от регулирования экономики;
	\item стабилизация политической системы, переход от авторитарного режима к либеральному и демократическому правлению; повышение общественного внимания к решению наиболее жизненных социальных проблем;
	
	\pagebreak
	\item быстрое разрушение препятствий для международной торговли и инвестиций;
	\item приток иностранного капитала, новых технологий и практики управления, привносимые транснациональными корпорациями;

	\pagebreak
	\item глубокие изменения в структуре промышленности, основанные на увеличении производительности труда и приближении к международным стандартам конкурентоспособности;
	\item увеличение активности в сфере слияний и поглощений, совместных предприятий и создании зарубежных подразделений компаний, что создает значительные возможности по извлечению прибыли вследствие разницы между производительностью на развитых и развивающихся рынков;
	\pagebreak
	\item ускорение роста национального фондового рынка;
	\item расширение открытости экономики соседних стран, в которых может быть также запущен процесс глобализации.
\end{itemize}
\end{frame}

\begin{frame}{Особенности стран сырьевого экспорта}
\newcolumntype{R}{>{\raggedleft\arraybackslash}p{1.5cm}}
% Table generated by Excel2LaTeX from sheet 'Лист1'
\begin{table}[htbp]
	\centering
	\tiny
	\caption{\captionf{Страны сырьевого экспорта}}
	\begin{tabularx}{\linewidth}
	[b]{@{}>{\raggedright\arraybackslash}RXRRR@{}}
	\setrulecolor\toprule
		\cnamef{Страна} & \cnamef{Основной \linebreak экспорт} & \cnamef{\% от общего объема} & \cnamef{Население,  млн. чел.} & \cnamef{ВВП по ППС, долл. / чел.} \\
		\midrule
		Алжир & Углеводороды & 97,8  & 33,4  & \$7 879 \\
		Ангола & Сырая нефть и алмазы & 98,2  & 17,4  & \$4 842 \\
		Азербайджан & Нефтепродукты и металлы & 84,4  & 8,6   & \$10 180 \\
		Иран  & Нефть и газ & 82,7  & 71,2  & \$10 781 \\
		Казахстан & Руды и металлы & 86,8  & 15,6  & \$10 771 \\
		Кувейт & Сырая нефть & 95,2  & 3,4   & \$38 338 \\
		Нигерия & Бензин и сжиженный газ & 97,3  & 143,3 & \$1 793 \\
		Россия & Нефть, газ, топливо, металлы & 78,5  & 142,3 & \$14 680 \\
		Саудовская Аравия & Сырая нефть и очищенный бензин & 90,3  & 24,3  & \$22 828 \\
		Венесуэла & Нефть и газ & 90,4  & 27,3  & \$12 273 \\
		Чили  & Медь и целлюлоза & 59,8  & 16,6  & \$13 909 \\
		Перу  & Медь и золото & 88,6  & 28,7  & \$7 636 \\
		\bottomrule
		\multicolumn{5}{@{}l}{Источник: The Economist, Country Briefings, 2009.}
	\end{tabularx}%
	\label{tab:addlabel}%
\end{table}%
\end{frame}

\subsection{Стоимость компании в венчурном бизнесе}
\begin{frame}[allowframebreaks]{Стоимость компании в венчурном бизнесе}{\scriptsize{Международные указания по оценке прямых и венчурных инвестиций}}
\begin{block}{Стоимостью компании}
	\quad
	называется стоимость финансовых инструментов, представляющих права собственности на эту организацию, плюс размер ее финансовых обязательств.
\end{block}

\pagebreak


\begin{block}{Рыночной стоимостью }
	\quad
	называется сумма, по которой актив может быть обменен между хорошо осведомленными сторонами, действующими на добровольной основе и независимо друг от друга.
\end{block}

\pagebreak

\begin{block}{Базисным бизнесом}
	\quad
	- называются те производственные подразделения, право собственности, на которые представлены финансовыми инструментами, которыми владеет инвестор, например, венчурный фонд.
\end{block}
\end{frame}

\begin{frame}[allowframebreaks]{}{}
\begin{itemize}
	\item оценка рыночной стоимости не предполагает, ни того что базисный бизнес может быть готов к продаже на отчетную дату, ни того что настоящие акционеры имеют намерение продать свои активы в ближайшем будущем; 
	
	\pagebreak
	\item целью оценки является определение гипотетической цены, по которой предполагаемые участники рынка согласились бы провести сделку;

	\item рыночной стоимостью не является сумма, которая была получена или заплачена в результате сделки по принуждению, вынужденной ликвидации или при продаже вследствие финансовых трудностей.
\end{itemize}
\end{frame}

\subsection{Методы оценки}
\begin{frame}{Оценка инновационных компаний}
\begin{figure}
	\centering
	\includegraphics[scale=.7,
	%trim={<left> <lower> <right> <upper>}				
	,trim={0.5cm 1cm 0cm 0cm},clip	
	] {tikz/valuation_methods}
	\caption{\captionf{Методы оценки инновационных компаний}}
\end{figure}
\end{frame}

\subsection{Дисконтирование денежных потоков}
\begin{frame}{Дисконтирование денежных потоков от основного бизнеса}
\begin{itemize}
	\item стоимость бизнеса определяется как современная стоимость будущих денежных потоков;
	\item суммируются дисконтированные денежные потоки за период бурного роста компании и дисконтированная заключительная стоимость;
	\item оценка котирующихся компаний на развитых рынках осуществляется по модели оценки долгосрочных активов, англ. capital asset pricing model, CAPM.
\end{itemize}
\end{frame}

\begin{frame}{Модель CAPM}
Риск инвестиций в $i$-тую акцию измеряется величиной $\beta_i$:
\begin{align}
\beta_i=\frac{cov_{iM}}{\sigma_M^2} =\frac{\sigma_i}{\sigma_M}   \cdot corr_{iM}
\end{align}

где

$cov_{iM}$ – ковариация доходности акции с доходностью рыночного индекса;

$corr_{iM}$ - коэффициент корреляции доходности акции с доходностью рыночного индекса;

$\sigma_i$ – стандартное отклонение доходности акции;

$\sigma_M$ – стандартное отклонение доходности рыночного индекса.

\end{frame}

\begin{frame}{Безрычаговый коэффициент $\beta$}
\begin{align}
\beta=\beta_{Levered}=\beta_{Unlevered} \cdot \left(1+\left(1-T \right) \cdot \frac{D}{E} \right), 
\end{align}

где 

$D$ – долг;

$E$ – собственные средства;

$T$ – ставка налога на прибыль.

\end{frame}

\begin{frame}{}
Стоимость капитала $C_E$ равна:
\begin{align}
C_E=R_f+ \beta \cdot (R_M-R_f), 
\end{align}

где 

$R_f$ – безрисковая ставка доходности;

$R_M$ - доходность фондового индекса.
\end{frame}

\begin{frame}[allowframebreaks]{Средневзвешенная стоимость капитала, WACC}
\begin{align}\label{eq:wacc}
WACC=&\frac{E}{D+E} \cdot C_E + \\\nonumber
	&\frac{D}{D+E} \cdot C_D \cdot (1-T), 
\end{align}

где
$C_D$ - стоимость долга.

\pagebreak

Если $D/E=k$, где $k$ это коэффициент долга или соотношение Долг / Капитал, то формула средневзвешенной стоимости капитала может быть преобразована следующим образом:

\begin{align}\label{eq:wacc_k}
WACC=\frac{1}{1+k} \cdot C_E + \frac{k}{1+k} \cdot C_D×(1-T).
\end{align}

Ставка дисконтирования $r$ принимается равной $WACC$.

\end{frame}

\begin{frame}
Современная стоимость прогнозных денежных потоков компании за время быстрого роста определяется методом дисконтирования:
\begin{align}\label{eq:pvpp}
PV_{PP}=\sum_{i=1}^{N} \frac{FCFF_i}{(1+r)^i}, 
\end{align}

где 

$FCFF_i$ - свободный денежный поток от активов (Free Cash Flow to the Firm) за период $i$;

$N$ – продолжительность времени быстрого роста.

\end{frame}

\begin{frame}
Заключительная стоимость компании $TV$ определяется по модели Гордона (бесконечный аннуитет с учетом стабильных темпов роста):

\begin{align}\label{eq:gordon}
TV=\frac{FCFF_{N+1}}{r-g}, r\neq g,
\end{align}

где 

$FCFF_{N+1}$ -  свободный денежный поток от активов в момент времени $N+1$;

$g$ - стабильные темпы роста свободного денежного потока от активов.
\end{frame}

\begin{frame}
	Заключительная стоимость компании приводится к современному моменту:
	\begin{align}\label{eq:ptv}
	PTV=\frac{TV}{(1+r)^N}.
	\end{align}
\end{frame}

\begin{frame}
Стоимость компании определяется, таким образом, как сумма современной стоимости прогнозных денежных потоков за время быстрого роста \ref{eq:pvpp} и заключительной стоимости \ref{eq:ptv}:

\begin{align}\label{eq:v}
V=PV_{PP}+PTV.
\end{align}

\end{frame}

\subsection{Свободные денежные потоки}
\begin{frame}[allowframebreaks]{Свободные денежные потоки }{}
Свободные денежные потоки могут быть рассчитаны:
\begin{itemize}
	\item для всей фирмы целиком (свободный денежный поток от активов – Free Cash Flow to the Firm - FCFF); 
	\item только для акционеров, т.е. свободный денежный поток от капитала (Free Cash Flow to Equity - FCFE).
\end{itemize}

\pagebreak
Свободные денежные потоки дисконтируются:
\begin{itemize}
	\item FCFF по средневзвешенной стоимости капитала, WACC;
	\item FCFE по стоимости акционерного капитала, $C_E$. 
\end{itemize}
\end{frame}


\begin{frame}[allowframebreaks]{Вычисление свободных денежных потоков}
\begin{block}{Денежные потоки от активов}
	\quad
	- это общая сумма денежных потоков кредиторам и акционерам, состоящая из операционных денежных потоков, капитальных затрат и увеличения чистого оборотного капитала. С. Росс.
\end{block}

\pagebreak

Денежный поток показывает, какую сумму денег владельцы фирмы получили от своей деятельности в течение конкретного года.

Денежные потоки от активов = 

Денежные потоки кредиторам + 

денежные потоки акционерам.

\pagebreak
Это тождество денежных потоков, оно говорит, что денежные потоки  от активов фирмы равны денежным средствам, выплаченным поставщикам капитала фирмы. 

Это отражает тот факт, что фирма зарабатывает деньги различными видами деятельности и деньги используются для уплаты кредиторам или владельцам фирмы.

\end{frame}

\begin{frame}[allowframebreaks]{Компоненты денежных потоков}{}
\begin{block}{Операционный денежный поток}
	\quad
	- это денежные средства, получаемые от обычной производственной деятельности фирмы. Они определяются как прибыль до уплаты процентов и налогов плюс амортизация и минус налоги.
\end{block}

\pagebreak


\begin{block}{Капитальные затраты }
	\quad
	- деньги, потраченные на основные средства, минус деньги, полученные от продажи основных средств. Капитальные затраты определяются как разница чистых основных средств на конец и на начало года плюс амортизация. Термин ``чистые`` означает ``за вычетом сумм, полученных от продаж основных средств``.
\end{block}

\pagebreak

\begin{block}{Увеличение чистого оборотного капитала}
	\quad
	- это инвестиции фирмы в оборотные средства. Показатель определяется как разность между суммами чистого оборотного капитала  на конец и на начало отчетного периода.
\end{block}

\pagebreak

 \begin{block}{Чистый оборотный капитал }
	\quad
	- это оборотные средства за вычетом текущих обязательств.
\end{block}

\end{frame}

\begin{frame}{Схема расчета денежных потоков от активов по С. Россу }

% Table generated by Excel2LaTeX from sheet 'Лист1'
\begin{table}[htbp]
	\centering
	\scriptsize
	\caption{\captionf{Расчет денежных потоков от активов}}
	\begin{tabularx}{\linewidth}
	[b]{@{}>{\raggedright\arraybackslash}rcX@{}}
	\setrulecolor\toprule
		\cnamef{№}     & \cnamef{Знак}  & \cnamef{Показатель} \\
		\midrule
		0.    & =     & Свободный денежный поток от активов \\
		1.    & +     & Операционный денежный поток \\
		1.1.  & +     & Прибыль до вычета процентов и налогов \\
		1.2.  & +     & Амортизация \\
		1.3.  & -     & Налоги \\
		2.    & -     & Чистые капитальные затраты \\
		2.1.  & +     & Чистые основные средства на н.г. \\
		2.2.  & -     & Чистые основные средства на к.г. \\
		2.3.  & +     & Амортизация \\
		3.    & -     & Увеличение чистого оборотного капитала \\
		3.1.  & +     & Чистый оборотный капитал на н.г. \\
		3.1.1. & +     & Оборотный капитал на н.г. \\
		3.1.2. & -     & Текущие обязательства на н.г. \\
		3.2.  & -     & Чистый оборотный капитал на к.г. \\
		3.2.1. & +     & Оборотный капитал на к.г. \\
		3.2.2. & -     & Текущие обязательства на к.г. \\
		\bottomrule
	\end{tabularx}%
	\label{tab:addlabel}%
\end{table}%
\end{frame}

\begin{frame}[allowframebreaks]{Расчет свободных денежных потоков фирмы по Перейро}
% Table generated by Excel2LaTeX from sheet 'Перейро'
\begin{table}[htbp]
	\centering
	\scriptsize
	\caption{\captionf{Определение чистой прибыли}}
	\begin{tabularx}{\linewidth}
	[b]{@{}>{\raggedright\arraybackslash}X@{}}
	\setrulecolor\toprule
		Выручка \\
		- Затраты и оперативные расходы (за исключением процентов и амортизации) \\
		= Прибыль до вычета процентов, налогов и амортизации (EBITDA) \\
		- Амортизация \\
		= Прибыль до выплаты процентов и налогов или операционная прибыль (EBIT) \\
		- Проценты \\
		= Прибыль до выплаты налогов (EBT) \\
		- Налог на прибыль \\
		\midrule
		= Чистая прибыль \\
		\bottomrule
	\end{tabularx}%
	\label{tab:addlabel}%
\end{table}%

\pagebreak
% Table generated by Excel2LaTeX from sheet 'Перейро'
\begin{table}[htbp]
	\centering
	\scriptsize
	\caption{\captionf{Свободный денежный поток\\ от активов (FCFF)}}
	\begin{tabularx}{\linewidth}
		[b]{@{}>{\raggedright\arraybackslash}X@{}}
		\setrulecolor\toprule
		EBIT \\
		- (EBIT*ставка налога на прибыль) \\
		+ Амортизация \\
		- Капитальные затраты (capital expenses 1) \\
		- Увеличение оборотного капитала (capital expenses 2) \\
		\midrule
		= Свободный денежный поток от активов (FCFF) \\
		\bottomrule
	\end{tabularx}%
	\label{tab:addlabel}%
\end{table}%

\pagebreak
% Table generated by Excel2LaTeX from sheet 'Перейро'
\begin{table}[htbp]
	\centering
	\scriptsize
	\caption{\captionf{Свободный денежный поток\\ от капитала (FCFE)}}
	\begin{tabularx}{\linewidth}
		[b]{@{}>{\raggedright\arraybackslash}X@{}}
		\setrulecolor\toprule
		Чистая прибыль \\
		+ Амортизация \\
		- Капитальные затраты (capital expenses 1) \\
		- Увеличение оборотного капитала (capital expenses 2) \\
		- Уменьшение долговых обязательств \\
		+ Увеличение долговых обязательств \\
		\midrule
		= Свободный денежный поток от капитала (FCFE) \\
		\bottomrule
	\end{tabularx}%
	\label{tab:addlabel}%
\end{table}%

\pagebreak
% Table generated by Excel2LaTeX from sheet 'Перейро'
\begin{table}[htbp]
	\centering
	\scriptsize
	\caption{\captionf{Преобразование FCFF в FCFE}}
	\begin{tabularx}{\linewidth}
		[b]{@{}>{\raggedright\arraybackslash}X@{}}
		\setrulecolor\toprule
		FCFF \\
		- $\text{Проценты} \cdot (1-T)$ \\
		- Уменьшение долговых обязательств \\
		+ Увеличение долговых обязательств \\
		\midrule
		= FCFE \\
		\bottomrule
	\end{tabularx}%
	\label{tab:addlabel}%
\end{table}%

\pagebreak
% Table generated by Excel2LaTeX from sheet 'Перейро'
\begin{table}[htbp]
	\centering
	\scriptsize
	\caption{\captionf{Преобразование FCFE в FCFF}}
	\begin{tabularx}{\linewidth}
		[b]{@{}>{\raggedright\arraybackslash}X@{}}
		\setrulecolor\toprule
		FCFE \\
		+ $\text{Проценты} \cdot (1-T)$ \\
		+ Уменьшение долговых обязательств \\
		- Увеличение долговых обязательств \\
		\midrule
		= FCFF \\
		\bottomrule
	\end{tabularx}%
	\label{tab:addlabel}%
\end{table}%

\end{frame}

\subsection{Варианты CAPM для развивающихся рынков}
\begin{frame}{Варианты CAPM для развивающихся рынков}{Глобальная CAPM}
Для рынков, находящихся в процессе прогрессивной финансовой интеграции:

\begin{align}
C_E=R_{fG}+\beta_{LG} \cdot (R_{MG}-R_{fG}),
\end{align}

где 

$R_{fG}$ - глобальная безрисковая ставка;

$R_{MG}$ - доходность глобального рынка;

$\beta_{LG}$ - это $\beta$ местной компании, вычисленная относительно глобального рыночного индекса.
\end{frame}

\begin{frame}[allowframebreaks]{Компоненты риска местного рынка}{}
\begin{itemize}
	\item риски, связанные с экономической и политической нестабильностью, которые негативно сказываются на результатах деятельности компании;
	\item возможность экспроприации частной собственности правительством;
	
	\pagebreak
	\item потенциальное возникновение ограничений на движение капитала, которые могут затруднить вывод прибыли в головной офис компании;
	\item валютный риск, связанный с возможностью девальвации национальной валюты;
	\pagebreak
	\item риск правительственного дефолта по внешним обязательствам, что приведет к резкому скачку процентных ставок;
	\item риск инфляции или гиперинфляции.
\end{itemize}
\end{frame}



\begin{frame}{Варианты CAPM для развивающихся рынков}{Местная CAPM}
Для изолированного рынка:
\begin{align}
C_E=R_{fL}+\beta_{LL} \cdot (R_{ML}-R_{fL}),
\end{align}

где 

$R_{fL}=R_{fG}+R_C$ – местная безрисковая ставка;

$R_{fG}$ – глобальная безрисковая ставка;

$R_C$ – местная премия за риск;

$R_{ML}$ - доходность местного рынка;

$\beta_{LL}$ - это $\beta$ местной компании, вычисленная относительно местного рыночного индекса.
\end{frame}

\begin{frame}[allowframebreaks]{Варианты CAPM для развивающихся рынков}{Скорректированная местная CAPM}
Исключает двойной счет риска, характерный для местной CAPM:
\begin{align}
C_E=R_{fG}+R_C+\beta_{LL} \cdot (R_{ML}-R_{fL} ) \cdot (1-R_i^2 ),
\end{align}

где 

$R_{fG}$ – глобальная безрисковая ставка;

$R_C$ – местная премия за риск;

$R_{ML}$ - доходность местного рынка;

\pagebreak

$\beta_{LL}$ - $\beta$ местной компании, вычисленная относительно местного рыночного индекса; 

$R_i^2$ – коэффициент детерминации регрессии между изменчивостью доходности местной компании и местной премией за риск. 

\pagebreak
Величину $R_i^2$ следует понимать как долю дисперсии доходности целевой компании $i$, объясненную местной премией за риск.

Корректировка рыночной премии за риск $\beta_{LL}×(R_{ML}-R_{fL} )$ применяется, чтобы исключить двойной счет риска.

\end{frame}

\subsection{Несистематический риск}
\begin{frame}{Компоненты несистематического риска}{}
\begin{itemize}
	\item размер компании;
	\item размер пакета акций (меньшинство или контроль);
	\item ликвидность акций или ее отсутствие.
\end{itemize}
\end{frame}

\subsection{Оценка компании венчурным инвестором}
\begin{frame}[allowframebreaks]{Оценка компании венчурным инвестором}
Макроэкономические условия:

ставка налога на прибыль 35,00\%;

мультипликатор Объем продаж / Капитал 2,2;

безрисковая ставка процента, 6,60\%;

премия за риск	12,00\%.

\pagebreak

Показатели деятельности компании в год оценки:

выручка 0,96 млрд. руб.;

операционная маржа -79,00\%;

ожидаемые стабильные темпы роста, начиная с 6 года 5\%.

\end{frame}

\begin{frame}{Прогноз показателей деятельности компании на 6 лет}
% Table generated by Excel2LaTeX from sheet 'Переменные'
\begin{table}[htbp]
	\centering
	\tiny
	\caption{\captionf{Прогноз показателей деятельности компании}}
	\begin{tabularx}{\linewidth}
	[b]{@{}>{\raggedright\arraybackslash}Xrrrrrr@{}}
	\setrulecolor\toprule
		\multicolumn{1}{c}{\multirow{2}[4]{*}{\cnamef{Показатель}}} & \multicolumn{6}{c}{\cnamef{Год}} \\\cmidrule{2-7}
		\multicolumn{1}{c}{} & \cnamef{1}     & \cnamef{2}     & \cnamef{3}     & \cnamef{4}     & \cnamef{5}     & \cnamef{6} \\
		\midrule
		Ожидаемые темпы роста выручки (\%) & 80\%  & 100\% & 60\%  & 40\%  & 30\%  & 5\% \\
		Операционная маржа & -24,00\% & -13,00\% & -2,50\% & 12,00\% & 13,00\% & 14,00\% \\
		Безрычаговый $\beta$ & 2,00  & 2,00  & 2,00  & 2,00  & 2,00  & 1,00 \\
		$D/E$ & 7,00\% & 8,00\% & 9,00\% & 10,00\% & 11,00\% & 12,00\% \\
		\bottomrule
	\end{tabularx}%
	\label{tab:addlabel}%
\end{table}%
\end{frame}

\begin{frame}{}
\begin{table}[htbp]
	\centering
	\tiny
	\caption{\captionf{Оценка чистой прибыли, млрд. руб.}}
	\begin{tabularx}{\linewidth}
		[b]{@{}>{\raggedright\arraybackslash}Xrrrrrr@{}}
		\setrulecolor\toprule
		\multicolumn{1}{c}{\multirow{2}[4]{*}{\cnamef{Показатель}}} & \multicolumn{6}{c}{\cnamef{Год}} \\\cmidrule{2-7}
		\multicolumn{1}{c}{} & \cnamef{1}     & \cnamef{2}     & \cnamef{3}     & \cnamef{4}     & \cnamef{5}     & \cnamef{6} \\
		\midrule
		Ожидаемые темпы роста выручки &&&&&& \\
		Выручка &&&&&& \\
		Операционная маржа &&&&&& \\
		EBIT &&&&&& \\
		Чистые операционные убытки в начале года &&&&&& \\
		Налогооблагаемая прибыль &&&&&& \\
		Выплаченные налоги  &&&&&& \\
		Эффективная ст. налога,\% &&&&&& \\
		EBIT(1-T) &&&&&& \\
		\bottomrule
	\end{tabularx}%
	\label{tab:addlabel}%
\end{table}%
\end{frame}


\iftagged{professor}
{
\begin{frame}{}
% Table generated by Excel2LaTeX from sheet 'Переменные'
\begin{table}[htbp]
	\centering
	\tiny
	\caption{\captionf{Оценка чистой прибыли, млрд. руб.}}
	\begin{tabularx}{\linewidth}
		[b]{@{}>{\raggedright\arraybackslash}Xrrrrrr@{}}
		\setrulecolor\toprule
		\multicolumn{1}{c}{\multirow{2}[4]{*}{\cnamef{Показатель}}} & \multicolumn{6}{c}{\cnamef{Год}} \\\cmidrule{2-7}
		\multicolumn{1}{c}{} & \cnamef{1}     & \cnamef{2}     & \cnamef{3}     & \cnamef{4}     & \cnamef{5}     & \cnamef{6} \\
		\midrule
		Ожидаемые темпы роста выручки (\%) & 80\%  & 100\% & 60\%  & 40\%  & 30\%  & 5\% \\
		Выручка & 1,74  & 3,47  & 5,56  & 7,78  & 10,11  & 10,62  \\
		Операционная маржа & -24,00\% & -13,00\% & -2,50\% & 12,00\% & 13,00\% & 14,00\% \\
		EBIT & -0,42  & -0,45  & -0,14  & 0,93  & 1,31  & 1,49  \\
		Чистые операционные убытки в начале года & -0,76  & -1,18  & -1,63  & -1,77  & -0,84  & 0,48  \\
		Налогооблагаемая прибыль & 0,00  & 0,00  & 0,00  & 0,00  & 0,00  & 0,48  \\
		Выплаченные налоги  & 0,00  & 0,00  & 0,00  & 0,00  & 0,00  & 0,17  \\
		Эффективная ст. налога, \% & 0,00\% & 0,00\% & 0,00\% & 0,00\% & 0,00\% & 11,28\% \\
		EBIT(1-T) & -0,42  & -0,45  & -0,14  & 0,93  & 1,31  & 1,32  \\
		\bottomrule
	\end{tabularx}%
	\label{tab:addlabel}%
\end{table}%
\end{frame}
}

\begin{frame}{}
% Table generated by Excel2LaTeX from sheet 'Переменные'
\begin{table}[htbp]
	\centering
	\tiny
	\caption{\captionf{Оценка стоимости компании, млрд. руб.}}
	\begin{tabularx}{\linewidth}
		[b]{@{}>{\raggedright\arraybackslash}Xrrrrrr@{}}
		\setrulecolor\toprule
		\multicolumn{1}{c}{\multirow{2}[4]{*}{\cnamef{Показатель}}} & \multicolumn{6}{c}{\cnamef{Год}} \\\cmidrule{2-7}
		\multicolumn{1}{c}{} & \cnamef{1}     & \cnamef{2}     & \cnamef{3}     & \cnamef{4}     & \cnamef{5}     & \cnamef{6} \\
		\midrule
		Изменения выручки &&&&&& \\
		Реинвестиции &&&&&& \\
		FCFF &&&&&& \\
		Рычаговый $\beta$ &&&&&& \\
		$C_E$ &&&&&& \\
		$C_D$ &&&&&& \\
		$T$, &&&&&& \\
		$D/E$ &&&&&& \\
		WACC  &&&&&& \\
		Кумулятивная WACC &&&&&& \\
		$PV$ &&&&&& \\
		\midrule
		$V$ &&&&&& \\
		\bottomrule
	\end{tabularx}%
	\label{tab:addlabel}%
\end{table}%
$PV$ приведенная стоимость;

$V$ стоимость компании.
\end{frame}

\iftagged{professor}
{
\begin{frame}{}
% Table generated by Excel2LaTeX from sheet 'Переменные'
\begin{table}[htbp]
	\centering
	\tiny
	\caption{\captionf{Оценка стоимости компании, млрд. руб.}}
	\begin{tabularx}{\linewidth}
		[b]{@{}>{\raggedright\arraybackslash}Xrrrrrr@{}}
		\setrulecolor\toprule
		\multicolumn{1}{c}{\multirow{2}[4]{*}{\cnamef{Показатель}}} & \multicolumn{6}{c}{\cnamef{Год}} \\\cmidrule{2-7}
		\multicolumn{1}{c}{} & \cnamef{1}     & \cnamef{2}     & \cnamef{3}     & \cnamef{4}     & \cnamef{5}     & \cnamef{6} \\
		\midrule
		Изменения выручки & 0,77  & 1,74  & 2,08  & 2,22  & 2,33  & 0,51  \\
		Реинвестиции & 0,35  & 0,79  & 0,95  & 1,01  & 1,06  & 0,23  \\
		FCFF & -0,77  & -1,24  & -1,09  & -0,08  & 0,25  & 1,09  \\
		Рычаговый $\beta$ & 2,09  & 2,10  & 2,12  & 2,13  & 2,14  & 1,08 \\
		$C_E$ & 31,69\% & 31,85\% & 32,00\% & 32,16\% & 32,32\% & 19,54\% \\
		$C_D$ & 20,00\% & 20,00\% & 20,00\% & 20,00\% & 20,00\% & 15\% \\
		$T$, \% & 0,00\% & 0,00\% & 0,00\% & 0,00\% & 0,00\% & 11,28\% \\
		$D/E$ & 7,00\% & 8,00\% & 9,00\% & 10,00\% & 11,00\% & 12,00\% \\
		WACC  & 30,93\% & 30,97\% & 31,01\% & 31,05\% & 31,10\% & 18,87\% \\
		Кумулятивная WACC & 1,3093 & 1,7148 & 2,2466 & 2,9442 & 3,8597 & 4,5880 \\
		$PV$ & -0,586  & -0,724  & -0,483  & -0,026  & 0,066  & 1,711661 \\
		\midrule
		$V$ &       &       &       &       &       & -0,04  \\
		\bottomrule
	\end{tabularx}%
	\label{tab:addlabel}%
\end{table}%
$PV$ приведенная стоимость;

$V$ стоимость компании.
\end{frame}
}


\subsection{Оценка компании перед IPO}
\begin{frame}{Оценка нефтяной компании ``X``}
% Table generated by Excel2LaTeX from sheet 'X1'
\begin{table}[htbp]
	\centering
	\small
	\caption{\captionf{Финансовая отчетность\\ компании ``X``, млрд. руб.}}
	\begin{tabularx}{\linewidth}
		[b]{@{}>{\raggedright\arraybackslash}Xrr@{}}
		\setrulecolor\toprule
		\multicolumn{1}{c}{\multirow{2}[0]{*}{\cnamef{Показатель} }} & \multicolumn{2}{c}{\cnamef{год}} \\\cmidrule{2-3}
		\multicolumn{1}{c}{} & \multicolumn{1}{c}{\cnamef{1}}     & \multicolumn{1}{c}{\cnamef{2}} \\
		\midrule
		Читая прибыль  & 5,90 ₽ & 8,80 ₽ \\
		Выручка  & 58,40 ₽ & 49,90 ₽ \\
		Балансовая стоимость & -95,80 ₽ & 46,40 ₽ \\
		Чистый денежный поток & 6,50 ₽ & 9,20 ₽ \\
		Долг / Капитал & -3,4  & 3,5 \\
		\bottomrule
	\end{tabularx}%
	\label{tab:addlabel}%
\end{table}%
\end{frame}

\begin{frame}{Определение стоимости привлечения капитала компании ``X``}
Стоимость привлечения капитала $C_E$:
\begin{align}\label{eq:ce}
C_E=C_E^{US}+R_E^{RU}
\end{align}

где  

$C_E^{US}$ - стоимость привлечения капитала для отрасли нефтедобычи компаний США; 

$R_E^{RU}$ – рисковая надбавка для инвестиций в российские компании.

Средневзвешенная стоимость капитала, $WACC$, определяется по формулам \ref{eq:wacc} или \ref{eq:wacc_k}.
\end{frame}


\begin{frame}
Оценка компании методом дисконтирования денежных потоков по модели Гордона осуществляется по формуле \ref{eq:gordon}. 

Ставка дисконтирования $r$ принимается равной средневзвешенной стоимости капитала, $WACC$.
\end{frame}

\begin{frame}{Оценка нефтяной компании ``X``}
% Table generated by Excel2LaTeX from sheet 'X1'
\begin{table}[htbp]
	\centering
	\small
	\caption{\captionf{Премии за риск и стоимость капитала, отрасль добычи нефти и газа, развивающиеся рынки}}
	\begin{tabularx}{\linewidth}
		[b]{@{}>{\raggedright\arraybackslash}Xrr@{}}
		\setrulecolor\toprule
		\multicolumn{1}{c}{\multirow{2}[0]{*}{\cnamef{Показатель, млрд. руб.} }} & \multicolumn{2}{c}{\cnamef{год}} \\\cmidrule{2-3}
		\multicolumn{1}{c}{} & \multicolumn{1}{c}{\cnamef{1}}     & \multicolumn{1}{c}{\cnamef{2}} \\
		\midrule
		Премия за риск & 10,17\% & 9,24\% \\
		Стоимость акционерного капитала & 22,22\% & 17,84\% \\
		Ставка налога на прибыль & 20,00\% & 20,00\% \\
		Ожидаемые темпы роста отрасли & -1,00\% & -1,35\% \\
		\bottomrule
	\end{tabularx}%
	\label{tab:addlabel}%
\end{table}%
\end{frame}


\begin{frame}{Оценка нефтяной компании ``X``}
% Table generated by Excel2LaTeX from sheet 'X1'
\begin{table}[htbp]
	\centering
	\small
	\caption{\captionf{Мультипликаторы стоимости, отрасль добычи нефти и газа, развивающиеся рынки}}
	\begin{tabularx}{\linewidth}
		[b]{@{}>{\raggedright\arraybackslash}Xrr@{}}
		\setrulecolor\toprule
		\multicolumn{1}{c}{\multirow{2}[0]{*}{\cnamef{Показатель} }} & \multicolumn{2}{c}{\cnamef{год}} \\\cmidrule{2-3}
		\multicolumn{1}{c}{} & \multicolumn{1}{c}{\cnamef{1}}     & \multicolumn{1}{c}{\cnamef{2}} \\
		\midrule
	    Цена / Прибыль & 46,93 & 129,48 \\
		Цена / Выручка & 0,84  & 0,84 \\
		Цена / Балансовая стоимость & 2,2   & 2,2 \\
		\bottomrule
	\end{tabularx}%
	\label{tab:addlabel}%
\end{table}%
\end{frame}


\begin{frame}{Оценка нефтяной компании ``X``}
% Table generated by Excel2LaTeX from sheet 'X1'
\begin{table}[htbp]
	\centering
	\small
	\caption{\captionf{Оценка компании ``X``\\ по мультипликаторам, млрд. руб.}}
	\begin{tabularx}{\linewidth}
		[b]{@{}>{\raggedright\arraybackslash}Xrr@{}}
		\setrulecolor\toprule
		\multicolumn{1}{c}{\multirow{2}[0]{*}{\cnamef{Показатель}}} & \multicolumn{2}{c}{\cnamef{год}} \\\cmidrule{2-3}
		\multicolumn{1}{c}{} & \multicolumn{1}{c}{\cnamef{1}}     & \multicolumn{1}{c}{\cnamef{2}} \\
		\midrule
		Цена / Прибыль &&\\
		Цена / Выручка &&\\
		Цена / Балансовая стоимость &&\\
		\midrule
		Стоимость компании &&\\
		\bottomrule
	\end{tabularx}%
	\label{tab:addlabel}%
\end{table}%
\end{frame}

\iftagged{professor}
{
\begin{frame}{Оценка нефтяной компании ``X``}
% Table generated by Excel2LaTeX from sheet 'X1'
\begin{table}[htbp]
	\centering
	\small
	\caption{\captionf{Оценка компании ``X``\\ по мультипликаторам, млрд. руб.}}
	\begin{tabularx}{\linewidth}
		[b]{@{}>{\raggedright\arraybackslash}Xrr@{}}
		\setrulecolor\toprule
		\multicolumn{1}{c}{\multirow{2}[0]{*}{\cnamef{Показатель}}} & \multicolumn{2}{c}{\cnamef{год}} \\\cmidrule{2-3}
		\multicolumn{1}{c}{} & \multicolumn{1}{c}{\cnamef{1}}     & \multicolumn{1}{c}{\cnamef{2}} \\
		\midrule
		Цена / Прибыль & 278,12 ₽ & 1 147,19 ₽ \\
		Цена / Выручка & 48,81 ₽ & 41,75 ₽ \\
		Цена / Балансовая стоимость & -210,57 ₽ & 102,07 ₽ \\
		\midrule
		Стоимость компании & 163,47 ₽ & 430,34 ₽ \\
		\bottomrule
	\end{tabularx}%
	\label{tab:addlabel}%
\end{table}%
\end{frame}
}

\begin{frame}{Оценка нефтяной компании ``X``}
% Table generated by Excel2LaTeX from sheet 'X1'
\begin{table}[htbp]
	\centering
	\small
	\caption{\captionf{Оценка компании ``X`` методом дисконтирования денежных потоков, млрд. руб.}}
	\begin{tabularx}{\linewidth}
		[b]{@{}>{\raggedright\arraybackslash}Xrr@{}}
		\setrulecolor\toprule
		\multicolumn{1}{c}{\multirow{2}[0]{*}{\cnamef{Показатель }}} & \multicolumn{2}{c}{\cnamef{год}} \\\cmidrule{2-3}
		\multicolumn{1}{c}{} & \multicolumn{1}{c}{\cnamef{1}}     & \multicolumn{1}{c}{\cnamef{2}} \\
		\midrule
		Стоимость акционерного капитала &&\\
		Стоимость долга &&\\
		Налог на прибыль &&\\
		Средневзвешенная стоимость капитала &&\\
		\midrule
		Стоимость компании &&\\
		\bottomrule
	\end{tabularx}%
	\label{tab:addlabel}%
\end{table}%
\end{frame}

\iftagged{professor}
{
\begin{frame}{Оценка нефтяной компании ``X``}
% Table generated by Excel2LaTeX from sheet 'X1'
\begin{table}[htbp]
	\centering
	\small
	\caption{\captionf{Оценка компании ``X`` методом дисконтирования денежных потоков, млрд. руб.}}
	\begin{tabularx}{\linewidth}
		[b]{@{}>{\raggedright\arraybackslash}Xrr@{}}
		\setrulecolor\toprule
		\multicolumn{1}{c}{\multirow{2}[0]{*}{\cnamef{Показатель }}} & \multicolumn{2}{c}{\cnamef{год}} \\\cmidrule{2-3}
		\multicolumn{1}{c}{} & \multicolumn{1}{c}{\cnamef{1}}     & \multicolumn{1}{c}{\cnamef{2}} \\
		\midrule
		Стоимость акционерного капитала & 22,22\% & 17,84\% \\
		Стоимость долга & 3,24\% & 4,65\% \\
		Налог на прибыль & 20,00\% & 20,00\% \\
		Средневзвешенная стоимость капитала & -5,36\% & 6,82\% \\
		\midrule
		Стоимость компании & -151,17 ₽ & 113,07 ₽ \\
		\bottomrule
	\end{tabularx}%
	\label{tab:addlabel}%
\end{table}%
\end{frame}
}

\begin{frame}{Оценка нефтяной компании ``X``}
% Table generated by Excel2LaTeX from sheet 'X1'
\begin{table}[htbp]
	\centering
	\small
	\caption{\captionf{Синтетическая оценка\\ компании ``X``, млрд. руб.}}
	\begin{tabularx}{\linewidth}
		[b]{@{}>{\raggedright\arraybackslash}Xrr@{}}
		\setrulecolor\toprule
		\multicolumn{1}{c}{\multirow{2}[0]{*}{\cnamef{Показатель} }} & \multicolumn{2}{c}{\cnamef{год}} \\\cmidrule{2-3}
		\multicolumn{1}{c}{} & \multicolumn{1}{c}{\cnamef{1}}     & \multicolumn{1}{c}{\cnamef{2}} \\
		\midrule
		Средняя оценка &&\\
		Дисконт за размер &&\\
		Премия за контроль &&\\
		Дисконт за неликвидность &&\\
		\midrule
		Стоимость компании &&\\
		\bottomrule
	\end{tabularx}%
	\label{tab:addlabel}%
\end{table}%
\end{frame}

\iftagged{professor}
{
\begin{frame}{Оценка нефтяной компании ``X``}
% Table generated by Excel2LaTeX from sheet 'X1'
\begin{table}[htbp]
	\centering
	\small
	\caption{\captionf{Синтетическая оценка\\ компании ``X``, млрд. руб.}}
	\begin{tabularx}{\linewidth}
		[b]{@{}>{\raggedright\arraybackslash}Xrr@{}}
		\setrulecolor\toprule
		\multicolumn{1}{c}{\multirow{2}[0]{*}{\cnamef{Показатель} }} & \multicolumn{2}{c}{\cnamef{год}} \\\cmidrule{2-3}
		\multicolumn{1}{c}{} & \multicolumn{1}{c}{\cnamef{1}}     & \multicolumn{1}{c}{\cnamef{2}} \\
		\midrule
		Средняя оценка & 163,47 ₽ & 271,70 ₽ \\
		Дисконт за размер & 0,00\% & 0,00\% \\
		Премия за контроль & 0,00\% & 0,00\% \\
		Дисконт за неликвидность & 0,00\% & 0,00\% \\
		\midrule
		Стоимость компании & 163,47 ₽ & 271,70 ₽ \\
		\bottomrule
	\end{tabularx}%
	\label{tab:addlabel}%
\end{table}%
\end{frame}
}

\begin{frame}{Оценка нефтяной компании ``X``}
% Table generated by Excel2LaTeX from sheet 'X2'
\begin{table}[htbp]
	\centering
	\small
	\caption{\captionf{Биржевая оценка компании ``X``}}
	\begin{tabularx}{\linewidth}
		[b]{@{}>{\raggedright\arraybackslash}Xrr@{}}
		\setrulecolor\toprule
		\cnamef{Показатель} & \cnamef{Ед. изм.} & \cnamef{Значение} \\
		\midrule
		Количество акций & шт.   &          294 120 000    \\
		Номинал & руб.  & 500,00 ₽ \\
		Бухгалтерская стоимость & млрд. руб. & 147,1 \\
		Размер лота & шт.   & 1 \\
		Цена размещения & руб.  & 550,00 ₽ \\
		Размещено акций & шт.   &            58 822 000    \\
		Капитализация & млрд. руб. & 161,77 ₽ \\
		Собрано средств & млрд. руб. & 32,35 ₽ \\
		Доля акций в обращении  & \%    & 20,00\% \\
		\bottomrule
	\end{tabularx}%
	\label{tab:addlabel}%
\end{table}%
\end{frame}

\end{document}