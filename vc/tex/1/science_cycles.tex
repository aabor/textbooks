% !TeX program = lualatex -synctex=1 -interaction=nonstopmode --shell-escape %.tex

\documentclass[_Venture_p1.tex]{subfiles}

\begin{document}

\setbeamercovered{invisible}

\subsection{Классификация}
\begin{frame}{Научно-технологические циклы}
\begin{figure}
	\centering
	\includegraphics[scale=.8,
	%trim={<left> <lower> <right> <upper>}				
	,trim={0.5cm 2.8cm 0cm 0cm},clip	
	] {tikz/science_cycles_classification}
	\caption{Виды, фазы и структура научно-технологических циклов}
\end{figure}
\end{frame}
\subsection{Хронология долгосрочных циклов}
\begin{frame}{Хронология долгосрочных научно-технологических циклов}
	1-ый технологический уклад 1735—1830 гг.
	
	2-ой технологический уклад 1830—1890 гг.
	
	3-ий технологический уклад 1890—1935 гг.
	
	4-ый технологический уклад 1935-1980 гг.
	
	5-ый технологический уклад 1981-2020 гг.
	
	6-ый технологический уклад 2021-2060 гг.
\end{frame}

\begin{frame}{1-ый технологический уклад 1735—1830 гг.}
\begin{itemize}
	\item появлением новых технологий в текстильной промышленности; 
	\item использование энергии падающей воды.
\end{itemize}
\end{frame}

\begin{frame}{2-ой технологический уклад 1830—1890 гг.}
\begin{itemize}
	\item изобретение парового двигателя; 
	\item появление железнодорожного транспорта;
	\item развитие машиностроения.
\end{itemize}
\end{frame}

\begin{frame}{3-ий технологический уклад 1890—1935 гг.}
\begin{itemize}
	\item активное использование электроэнергии; 
	\item развитие тяжелого машиностроения, электротехнической промышленности, сталелитейного и сталепрокатного производства, цветной металлургии; 
	\item радиосвязи, автоперевозок. 
\end{itemize}
\end{frame}

\begin{frame}{4-ый технологический уклад 1935-1980 гг.}
\begin{itemize}
	\item создание новых поколений танков и самолетов, атомного оружия и ракетных средств его доставки; 
	\item с 50-х гг применение ЭВМ первых поколений, атомная энергетика, мирное освоение космоса, реактивная авиация, пластмассы и синтетические смолы, первая «зеленая революция», телевидение и т.п.; 
	\item лидерами технологического прорыва были США и СССР, затем к ним подключились Япония и Западная Европа.
\end{itemize}
\end{frame}

\begin{frame}{5-ый технологический уклад 1981-2020 гг.}
\begin{itemize}
	\small
	\item ведущие направления техники: микроэлектроника, биотехнология микроорганизмов, информатика, композиты, нефтегазовое топливо, космические технологии; 
	\item ведущие отрасли экономики: информационная техника и связь, телекоммуникации, нефтегазовая промышленность, мобильная связь;
	\item лидирующие страны: США, Япония, Западная Европа, к ним подключились новые индустриальные страны; 
	\item СССР стал терять технологическое лидерство, а с 90-х годов Россия оказалась в состоянии технологической деградации.
\end{itemize}
\end{frame}

\begin{frame}[allowframebreaks]{6-ый технологический уклад 2021-2060 гг.}
\begin{itemize}
	\small
	\item нанотехнологии, биотехнология растений, животных на базе достижений генной инженерии, глобальные информационные сети, водородная и иная экологически безопасная энергетика, принципиально новые виды транспорта, экологически чистые технологии, высокотехнологичные агропроизводственные системы; 
	
	\pagebreak
	\item помимо прибыли для инвесторов и предпринимателей, эти технологии повышают качество жизни, экологическую безопасность, геополитическую стабильность для общества в целом;
	\item лидирующие страны: США, Западной Европе, Японии, к ним скоро присоединятся Индия и Китай.
\end{itemize}
\end{frame}
\subsection{Критические технологии для России}
\begin{frame}[allowframebreaks]{Критические технологии для России}
\begin{itemize}
	\item Базовые и критические военные и промышленные технологии для создания перспективных видов вооружения, военной и специальной техники;
	\item Базовые технологии силовой электротехники;
	\item Биокаталитические, биосинтетические и биосенсорные технологии;
	\item Биомедицинские и ветеринарные технологии;
	
	\pagebreak
	\item Геномные, протеомные и постгеномные технологии;
	\item Клеточные технологии;
	\item Компьютерное моделирование наноматериалов, наноустройств и нанотехнологий;
	\item Нано-, био-, информационные, когнитивные технологии;
	\pagebreak
	\item Технологии атомной энергетики, ядерного топливного цикла, безопасного обращения с радиоактивными отходами и отработавшим ядерным топливом;
	\item Технологии биоинженерии;
	\item Технологии диагностики наноматериалов и наноустройств;
	
	\pagebreak
	\item Технологии доступа к широкополосным мультимедийным услугам;
	\item Технологии информационных, управляющих, навигационных систем;
	\item Технологии мониторинга и прогнозирования состояния окружающей среды, предотвращения и ликвидации ее загрязнения;
	\pagebreak
	\item Технологии наноустройств и микросистемной техники;
	\item Технологии новых и возобновляемых источников энергии, включая водородную энергетику;
	\item и др., согласно распоряжению Правительства РФ от 14 июля 2012 г. N 1273-р. 
\end{itemize}
\end{frame}
\end{document}