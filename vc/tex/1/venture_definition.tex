% !TeX program = lualatex -synctex=1 -interaction=nonstopmode --shell-escape %.tex
\documentclass[_Venture_p1.tex]{subfiles}



\begin{document}

\setbeamercovered{invisible}

\subsection{Определения}
\begin{frame}{Понятие вечурного капитала}{согласно международным ассоициациям венчурного инвестирования}
\begin{itemize}
	\item Европейская ассоциация прямого и венчурного инвестирования (ЕАПВИ), охватывает все европейские страны, кроме Великобритании; 
	\item Национальная ассоциация прямого и венчурного инвестирования (НАПВИ), США;
	\item Британская ассоциация прямого и венчурного инвестирования (БАПВИ). 
\end{itemize}
\end{frame}
\begin{frame}[allowframebreaks]{Определение венчурного капитала}
\begin{block}{Венчурный капитал }
\quad
— это денежные средства, вложенные квалифицированными инвесторами совместно с предпринимателем в акционерный капитал рискованного предприятия, для финансирования ранней стадии развития (посевной и запуска) или стадии расширения. Компенсацией высокого риска, принимаемого инвестором, является более высокая доходность инвестиций, чем в среднем по рынку. Венчурный капитал является подкатегорией прямых инвестиций. ЕАПВИ.
\end{block}

\pagebreak

\begin{block}{Венчурный капиталист}
	\quad
	– это менеджер фонда прямых инвестиций, уполномоченный управлять вложениями фонда в отдельную компанию из портфеля. Активный подход (обычная модель прямых инвестиций) предполагает, что венчурный капиталист привносит не только денежный капитал, но также чрезвычайно ценные собственные знания, деловые контакты, торговые марки, стратегические консультации и пр.
\end{block}
\pagebreak

\begin{block}{Венчурная компания (венчур)}
	\quad 
	- это инновационная компания, долю в капитале которой купил венчурный инвестор.
\end{block}


\begin{block}{Хищный капиталист }
	\quad
	– это негативное определение инвестора, ищущего быстрых денег и не заинтересованного в инвестировании в компании с долгосрочным потенциалом.
\end{block}

\pagebreak

\begin{block}{Прямые инвестиции  }
	\quad
	– это вложения в акционерный капитал предприятий, не котирующихся на фондовом рынке.
\end{block}
\end{frame}

\begin{frame}{Прямые инвестиции могут использоваться}
\begin{itemize}
	\small
	\item для развития новых продуктов и технологий (также называют венчурным капиталом), 
	\item для увеличения оборотных средств, 
	\item для осуществления поглощений, 
	\item для улучшения финансовых показателей деятельности компании
	\item для решения конфликтов среди собственников и руководителей предприятия. 
	\item при выкупе компании менеджментом, 
	\item в случае ухода от дел основателя компании, руководящего семейным бизнесом.
\end{itemize}
\end{frame}

\begin{frame}
\begin{block}{Фонд прямых инвестиций }
	\quad 
	- это средство для коллективного инвестирования в акционерный капитал компаний или ценные бумаги, связанные с акционерным капиталом. Фонд может иметь юридическую форму акционерного общества или товарищества с ограниченной ответственностью.
\end{block}
\end{frame}

\begin{frame}{Другие определения венчурного капитала}
\begin{block}{Прямые инвестиции }
	\quad 
	- это финансирование, предоставляемое в обмен на долю в акционерном капитале компаний, имеющих высокий потенциал роста. Фирмы прямых инвестиций собирают капитал не за счет продажи акций на бирже, а из других источников, таких как пенсионные фонды, фонды недвижимости и богатые люди. Собранные средства используются, наряду с заемными, для инвестиций в начинающие компании, имеющие высокий потенциал роста. БАПВИ.
\end{block}
\end{frame}

\begin{frame}
\begin{itemize}
	\item В Европе прямые инвестиции представляют собой весь спектр инвестиционного сектора, от венчурного капитала до выкупа или продажи компаний менеджментом. 
	\item В США прямые инвестиции и венчурный капитал рассматриваются как отдельные типы инвестиций.
\end{itemize}
\end{frame}

\begin{frame}[allowframebreaks]{Европейская практика}
\begin{itemize}
	\item венчурный капитал является особой частью отрасли прямых инвестиций и относится к вложениям в компании на посевной, пусковой и ранних стадиях развития. 
	\item прямые инвестиции означают выкуп или продажу компании менеджментом, независимо от суммы сделок. 
	
	\pagebreak
	\item фонды венчурного капитала инвестируют в компании на ранних стадиях развития, не имеющих значительных достижений и требующих больших денежных вливаний.  
	\item фонды прямых инвестиций вкладываются в более зрелые компании с целью устранения их разного рода недостатков и стимулирования роста.
\end{itemize}
\end{frame}

\begin{frame}{Определение венчурного капитала в США}
\begin{block}{Венчурный капитал }
	\quad
	- денежные средства, предоставляемые профессионалами, которые инвестируют, наряду с менеджментом, в молодые, быстро растущие компании, которые имеют потенциал развития до крупных экономических субъектов. Венчурный капитал является важным источником акционерного капитала для начинающих компаний.НАПВИ.
\end{block}
\end{frame}

\begin{frame}[allowframebreaks]{Различия в толковании венчурного капитала}{США, Европа, Великобритания, Россия}
\begin{itemize}
	\small
	\item \textbf{США:} подчеркивается предпринимательская составляющая деятельности венчурного капиталиста (активное участие в управлении финансируемой компании), а также отмечается самая широкая сфера деятельности венчурных капиталистов. 
	\item \textbf{Европа: }венчурный капиталист часто является активным инвестором, венчурные инвестиции осуществляются на ранних стадиях развития компании, а прямые инвестиции на более поздних. 
	
	\pagebreak
	\item \textbf{Великобритания: }предпринимательская составляющая венчурных инвестиций не выделяется совсем, участие инвестора предполагается только в денежной форме.
	\item \textbf{Россия:} финансировании проектов с невысокой степенью новизны, но обладающих большим потенциалом коммерческого успеха
\end{itemize}
\end{frame}

\begin{frame}{Определение понятия венчурный капитал в России}
\small
\begin{block}{Венчурный капитал}
	\quad
	– это денежные средства, вложенные квалифицированными инвесторами совместно с предпринимателем и менеджерами в акционерный капитал инновационных предприятий с высоким потенциалом роста на ранних стадиях (посевной и запуска) и стадии расширения. Венчурный капиталист активно участвует в создании стоимости компании, предоставляя предприятию свои административные и деловые контакты, опыт ведения бизнеса и управления финансами компании.	
\end{block}
\end{frame}
\begin{frame}
\begin{itemize}
	\item Понятие квалифицированного инвестора определяется в ФЗ РФ «О рынке ценных бумаг». 
	\item Инновационными называют те предприятия, которые создают или используют в своей деятельности критические технологии, перечень которых утверждается высшими органами государственной власти. 
\end{itemize}
\end{frame}
\subsection{Классификация венчурных инвесторов}
\begin{frame}{Венчурные капиталисты}
	\begin{figure}
		\center
		% trim={<left> <lower> <right> <upper>}
		\includegraphics[trim={0.5cm 3.5cm 0 0cm},clip, scale=0.8]{tikz/venture_capitalits_classification}
	\caption{Классификация венчурных инвесторов}
	\end{figure}
\end{frame}

\subsection{Венчурный бизнес}
\begin{frame}[allowframebreaks]{Цели венчурного бизнеса}
Венчурный фонд зарабатывает прибыль от финансовой посреднической деятельности между инвесторами, готовыми пойти на высокий риск вложения денег в компании, создающие инновационные продукты, и предпринимателями, являющимися собственниками этих компаний.

\pagebreak
\begin{itemize}
	\item Венчурный капиталист инвестирует свои ресурсы в капитал начинающих предприятий и помогает им достигнуть экономической самостоятельности. 
	\item Обычно этот процесс занимает 3-5 лет. 
	\item После этого компания может быть продана другим инвесторам, нацеленным на прибыльное размещение своих сбережений. 
	\item Продажа наиболее успешных компаний происходит на бирже, менее успешные компании поглощаются стратегическими инвесторами.
\end{itemize}
\end{frame}

\begin{frame}
\textbf{Производственным процессом венчурного бизнеса }является создание новых коммерчески состоятельных компаний, преимущественно из инновационных секторов экономики, в ответ на запросы биржевых и стратегических инвесторов.
\end{frame}

\subsection{Венчурные компании}
\begin{frame}{Рост и развитие венчурных компаний}
\begin{figure}
	\centering
	\includegraphics[scale=1,
				%trim={<left> <lower> <right> <upper>}				
				,trim={0cm 4.5cm 3cm 0cm},clip	
		] {tikz/venture_company_growth_steps}
	\caption{Стадии процесса роста и развития венчурных компаний}
\end{figure}
\end{frame}

\begin{frame}[allowframebreaks]{Стадии процесса роста и развития венчурных компаний}
\begin{itemize}
	\item[1.] Посевная стадия – компания находится в стадии формирования, имеется лишь проект или бизнес-идея, идет процесс создания управленческой команды, проводятся НИОКР (Научно-исследовательские и опытно-конструкторские разработки) и маркетинговые исследования.
	\item[2.] Стадия запуска – компания недавно образована, обладает опытными образцами, пытается организовать производство и выпуск продукции на рынок.
	
	\pagebreak
	\item[3.] Ранний рост – компания осуществляет выпуск и коммерческую реализацию готовой продукции, хотя пока не имеет устойчивой прибыли. На эту стадию приходится «точка безубыточности».
	\item[4.] Расширение – компания занимает существенную долю рынка, становится прибыльной, ей требуются расширение производства и сбыта, проведение дополнительных маркетинговых исследований, увеличение основных фондов и капитала.
	
	\pagebreak
	\item[5.] Промежуточная «мезонинная» стадия, на которой привлекаются дополнительные инвестиции  для улучшения финансовых показателей компании и повышения ее рыночной капитализации.
	\item[6.] Выход – продажа доли инвестора другому стратегическому инвестору, либо первичное размещение акций компании на фондовом рынке (IPO), либо выкуп менеджментом (MBO). 
\end{itemize}
\end{frame}

\subsection{Услуги венчурных капиталистов}
\begin{frame}{Услуги венчурных капиталистов}
\begin{figure}
	\centering
	\begin{overprint}
		\forloop{slideno}{1}{\value{slideno} < 27}{%
			\only<\value{slideno}>{
				\includegraphics[page=\value{slideno},
				scale=.65
				% trim={<left> <lower> <right> <upper>}				
				,trim={0cm 0cm 0 0cm},clip]
				{tikz/venture_capitalist_service}}}
	\end{overprint}
	\vspace*{-1.5cm}
	\caption{\captionf{Услуги венчурного капиталиста для предпринимателя}}
\end{figure}
\end{frame}

\end{document}