% !TeX program = lualatex -synctex=1 -interaction=nonstopmode --shell-escape %.tex

\documentclass[_Venture_p1.tex]{subfiles}

\begin{document}

\setbeamercovered{invisible}

\subsection{Организация работы венчурного фонда}
\begin{frame}{Функционирование венчурного фонда}
\begin{itemize}
	\small
	\item Венчурный фонд аккумулирует средства нескольких инвесторов с целью диверсификации рисков. 
	\item Этими средствами управляет профессиональная управляющая компания (УК). 
	\item Как правило, управляющая компания занимается сбором средств в фонд и выступает посредником между инвесторами и компаниями-реципиентами. 
	\item В венчурной индустрии руководитель или ведущий менеджер УК часто называется «венчурный капиталист».
\end{itemize}
\end{frame}

\begin{frame}{}
\begin{overprint}
\begin{figure}
	\centering
		\forloop{slideno}{1}{\value{slideno} < 10}{%
			\only<\value{slideno}>{
				\includegraphics[page=\value{slideno},
				scale=.9
				% trim={<left> <lower> <right> <upper>}				
				,trim={1cm 3cm 0 0cm},clip]
				{tikz/venture_fund_organization}}}
	%\vspace*{-1cm}
	\caption{Схема работы венчурного фонда}
\end{figure}
%\vspace*{-.5cm}
\end{overprint}
\end{frame}

\begin{frame}{Практика работы венчурного фонда}
\begin{itemize}
	\item Венчурный фонд формируется на срок 5-10 лет. 
	\item Основной объект вложения венчурного фонда – доли в компаниях на стадии стартап. 
	\item Цель фонда – рост капитализации портфельных компаний и получение прибыли от продажи долей в компаниях на «выходе».
\end{itemize}
\end{frame}

\begin{frame}{Способы выхода инвестора из венчурой компании}
\begin{itemize}
	\item через фондовый рынок посредством первоначального публичного предложения;
	\item через продажу доли венчурного инвестора другому инвестору (фонду прямых инвестиций или стратегическому инвестору);
	\item через выкуп доли инвестора менеджментом, в т.ч. через привлечение заемных средств.
\end{itemize}
Продажа происходит на пике стоимости компании через 5-7 лет после начала инвестиций.
\end{frame}

\subsection{\setfontsize{12pt} Процесс венчурных инвестиций}
\begin{frame}[allowframebreaks]{Процесс венчурных инвестиций}
\begin{itemize}
	\item венчурные фонды инвестируют собранные средства в большое число проектов (10-30). 
	\item действует принцип «3-3-3-1», т.е. из десяти компаний три оказываются убыточными, три приносят умеренную прибыль, три являются успешными и только одна сверх доходной (эта компания может быть выведена на биржу);

	\pagebreak
	\item процесс отбора и изучения компаний для инвестиций является в значительной степени формализованным и предполагает привлечение экспертов со стороны.
	\item инвестиции в одну компанию происходят в несколько раундов, что помогает закрыть неудачные проекты на ранней стадии. 
	\item это стимулирует предпринимателей повышать эффективность, перед просьбой об очередном финансировании.
	
	\pagebreak
	\item венчурные фонды расходуют свои средства постепенно, в течение нескольких лет. 
	\item инвесторы фонда не вносят все средства сразу, а предоставляют фонду обязательства выделять средства по мере необходимости в пределах заранее оговоренного объема. 
	\item это избавляет УК от задачи обеспечивать доходность еще не вложенных средств и позволяет сосредоточиться на поиске перспективных компаний и управлении уже имеющимися проектами.
\end{itemize}
\end{frame}

\subsection{Вознаграждение управляющих}
\begin{frame}[allowframebreaks]{Вознаграждение управляющих венчурным фондом}
\begin{itemize}
	\item Первоначальной задачей УК фонда становится поиск инвесторов и сбор средств в венчурный фонд. 
	\item Сама УК может как иметь долю в венчурном фонде, так и не иметь ее. 
	\item Широкой практикой в США стало участие самой УК в капитале фонда с миноритарной долей порядка 1\%. 
	
	\pagebreak
	\item В начальный период функционирования фонда под руководством УК происходит выбор объектов инвестиций.
	\item Затем менеджеры УК участвуют в работе совета директоров венчурных компаний и всесторонне способствуют  их росту и развитию. 
	\item По окончании деятельности фонда, после «выхода» из всех проектов, вознаграждение УК составляет 20-25\% от прибыли, что является компенсацией за эффективное управление. 
	\pagebreak
	\item Вознаграждение за успех определяется от суммы, которая останется после того, как инвесторам будет выплачена первоначальная сумма  их вложений плюс заранее оговоренный доход с нормой доходности.
	\item Расходы на деятельность УК в процессе работы фонда составляют 2-4\% от суммы активов под управлением ежегодно.
\end{itemize}
\end{frame}

\subsection{Иерархия сотрудников}
\begin{frame}{Иерархия сотрудников венчурного фонда }
\begin{itemize}
	\item[1.] Партнер – это венчурный капиталист, который руководит процессом сбора средств, готовит решения по сделкам, участвует в управлении портфельными компаниями и в прибыли УК.
	\item[2.] Управляющий директор – наемный сотрудник, не участвующий в прибыли, готовит документы по сделкам и взаимодействует с портфельными компаниями.
	\item[3.] Аналитик – наемный сотрудник, не участвующий в прибыли, анализирует секторы и компании, иногда участвует в подготовке сделок и управлении компаниями.
	
\end{itemize}
\end{frame}

\subsection{Организационно-правовая форма венчурных фондов}
\begin{frame}[allowframebreaks]{Товарищество с ограниченной ответственностью}{Ограниченное партнерство (коммандитное товарищество)}
\begin{itemize}
	\item Юридическая форма позволяет разделение партнеров на ограниченных и генеральных. 
	\item Ограниченный партнер не несет ответственности за результаты деятельности фонда на сумму, большую, чем его вклад в фонд.
	\item Генеральный партнер несет неограниченную ответственность по всем обязательствам фонда. 

	\pagebreak
	\item УК является генеральным партнером, а инвесторы – ограниченными.
	\item Преимущества – отсутствие двойного налогообложения, вариативность договоров с инвесторами.
	\item Для контроля за УК инвесторы могут создать инвестиционный комитет фонда, решения которого обязательны для УК. 
	
	\pagebreak
	\item Инвестиционный комитет утверждает все основные решения УК: выбор объектов инвестиций, объем выделяемых средств, время и порядок выхода, раздел прибыли. 
	\item В процессе работы венчурного фонда при нем может создаваться консультативный совет, состоящий из экспертов по отраслям, интересующим фонд, или же специалистов по венчурному инвестированию. В совете могут участвовать известные в венчурном бизнесе лица.
	
	\pagebreak
	\item После определения круга инвесторов, составляется инвестиционный меморандум, в котором четко прописываются порядок создания и функционирования фонда, права и обязанности каждой стороны, механизм разрешения споров. 
	\item Меморандум не является юридически обязывающим документом, часто носит конфиденциальный характер.
	
\end{itemize}
\end{frame}

\begin{frame}{Акционерный инвестиционный фонд }
В России наиболее распространенной юридической формой венчурных инвестиций стали закрытые паевые инвестиционные фонды (ЗПИФы).

Большинство венчурных фондов, имеющих в той или иной форме правительственную поддержку созданы в этой юридической оболочке. 
\end{frame}

\begin{frame}
\begin{block}{Акционерный инвестиционный фонд }
	\quad
	- открытое акционерное общество, исключительным предметом деятельности которого является инвестирование имущества в ценные бумаги и иные объекты, предусмотренные ФЗ ``Об инвестиционных фондах``, и фирменное наименование которого содержит слова ``акционерный инвестиционный фонд`` или ``инвестиционный фонд``.
\end{block}
\end{frame}

\begin{frame}
\begin{itemize}
	\item Акции акционерного инвестиционного фонда, предназначенные для квалифицированных инвесторов (акции, ограниченные в обороте), могут принадлежать только квалифицированным инвесторам. 
	\item Указанное ограничение должно содержаться в соответствующем решении о выпуске акций такого акционерного инвестиционного фонда.
\end{itemize}
\end{frame}

\begin{frame}[allowframebreaks]{Закрытый паевой инвестиционный фонд }
\begin{block}{Паевой инвестиционный фонд (ПИФ) }
	\quad
	- обособленный имущественный комплекс, состоящий из имущества, переданного в доверительное управление управляющей компании учредителем (учредителями) доверительного управления с условием объединения этого имущества с имуществом иных учредителей доверительного управления, и из имущества, полученного в процессе такого управления, доля в праве собственности на которое удостоверяется ценной бумагой, выдаваемой управляющей компанией.
\end{block}

\pagebreak
\begin{block}{Закрытый паевой инвестиционный фонд (ЗПИФ)}
	\quad
	- это ПИФы, владельцы инвестиционных паев которых не имеют права требовать от управляющей компании прекращения договора доверительного управления паевым инвестиционным фондом до истечения срока его действия иначе, как в случаях, предусмотренных ФЗ ``Об инвестиционных фондах``.
\end{block}
\end{frame}

\begin{frame}[allowframebreaks]{Инвестирование в компании ранних стадий предполагает}
\begin{itemize}
	\item небольшое число инвесторов;
	\item неформальные отношения между инвестором и реципиентами;
	\item возможность организации внутреннего контроля инвесторов за действиями  УК через инвестиционный комитет;
	
	\pagebreak
	\item большую свободу в выборе объектов вложений;
	\item отсутствие государственного регулирования со стороны органа, ответственного за рынок ценных бумаг;
	
	\pagebreak
	\item взаимное доверие между инвесторами в фонде;
	\item конфиденциальный характер и высокую степень закрытости фонда по отношению к третьим лицам (на ранних стадиях размеры вложений, стратегия выхода, а иногда и данные о компаниях реципиентах являются сугубо конфиденциальными).
\end{itemize}
\end{frame}

\begin{frame}[allowframebreaks]{Недостатки ЗПИФов }{как юридической формы для венчурыных инвестиций}
\begin{itemize}
	\small
	\item[1.] Существенный объем ежегодных расходов на пользование услугами депозитария, регистратора, оценщика и аудитора, что составляет в среднем 3-5\% от размера фонда ежегодно. Вместе с затратами на управление, расходы фонда за финансовый год могут составлять до 6-9\% от его размера.
	
	\pagebreak
	\item[2.] Отсутствие возможности внесения средств по мере необходимости, поэтому венчурный ЗПИФ инвестирует свободные средства на биржевом рынке.
	\item[2.] ЗПИФы ориентированы на крупные объемы активов, большое число пайщиков и консервативный тип вложения, для которых требования открытости, внешнего контроля за УК со стороны регулятора имеет первостепенное значение.
\end{itemize}
\end{frame}

\begin{frame}[allowframebreaks]{Фонды особо рисковых (венчурных) инвестиций}{Требования российского законодательства}
\begin{itemize}
	\item Могут быть только акционерными инвестиционными фондами и закрытыми паевыми инвестиционными фондами. 
	\item Акции (инвестиционные паи) предназначены для квалифицированных инвесторов. 
	
	\pagebreak
	\item Могут входить активы, приобретенные по договору займа или кредитному договору, если заимодавцем (кредитором) по такому договору является квалифицированный инвестор в силу федерального закона.
\end{itemize}
\end{frame}

\begin{frame}[allowframebreaks]{Требования к составу и структуре активов}
\begin{itemize}
	\item денежные средства, в том числе иностранная валюта, на счетах и во вкладах в кредитных организациях;
	\item долговые инструменты, в том числе выпущенные российскими хозяйственными обществами, в которых фонд приобрел более 25\% капитала;
	
	\pagebreak
	\item акции российских акционерных обществ;
	\item доли в уставных капиталах российских обществ с ограниченной ответственностью.	
\end{itemize}
\end{frame}

\begin{frame}{\setfontsize{12pt}Акции частных компаний в активах инвестиционных фондов}
\begin{itemize}
	\item Могут входить только акции и доли в уставных капиталах хозяйственных обществ, представивших управляющей компании этого фонда бизнес-план развития. 
	\item Требования не распространяются на акции акционерных обществ, допущенные к торгам у организаторов торговли на рынке ценных бумаг.
\end{itemize}
\end{frame}

\begin{frame}[allowframebreaks]{\setfontsize{12pt}Требования к фондам особо рисковых (венчурных) инвестиций}
\begin{itemize}
	\item денежные средства, находящиеся во вкладах в одной кредитной организации, могут составлять не более 25 процентов стоимости активов;
	\item оценочная стоимость ценных бумаг, включенных в котировальные списки фондовых бирж (за исключением котировального списка ``И`` - низколиквидные ценные бумаги), может составлять не более 30 процентов стоимости активов;
	
	\pagebreak
	\item оценочная стоимость долей в уставных капиталах российских обществ с ограниченной ответственностью или компаний из списка ``И`` в сумме должны составлять >10\% через 1 год, >30\% через 3 года, >50\% через 6 лет (требование не применяется, если до окончания срока договора доверительного управления осталось менее 1 года).	
\end{itemize}
\end{frame}
\end{document}