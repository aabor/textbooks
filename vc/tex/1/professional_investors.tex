% !TeX program = lualatex -synctex=1 -interaction=nonstopmode --shell-escape %.tex

\documentclass[_Venture_p1.tex]{subfiles}

\begin{document}

\setbeamercovered{invisible}

\subsection{Определения}
\begin{frame}[allowframebreaks]{Определения}
Квалифицированными инвесторами являются, согласно ст. 51.2 ФЗ РФ «О рынке ценных бумаг» [1]:

1.	брокеры, дилеры и управляющие; 

2.	Центральный Банк РФ и кредитные организации;

3.	акционерные инвестиционные фонды;

4.	управляющие компании инвестиционных фондов, паевых инвестиционных фондов и негосударственных пенсионных фондов;

\pagebreak
5.	страховые организации;

6.	негосударственные пенсионные фонды;

7.	некоммерческие фонды, приобретающие паи закрытых фондов инвестирующих в малые и средние предприятия;

\pagebreak
8.	государственные корпорации "Внешэкономбанк" и "Российская корпорация нанотехнологий";

9.	Агентство по страхованию вкладов;

10.	международные финансовые организации, в том числе Мировой банк, Международный валютный фонд, Европейский центральный банк, Европейский инвестиционный банк, Европейский банк реконструкции и развития.

\end{frame}

\begin{frame}[allowframebreaks]{\setfontsize{12pt}Признание ф.л. квалифицированным инвестором }{отвечает любым двум требованиям из указанных}
1.	владеет ценными бумагами, общая стоимость которых составляет не менее 3 миллионов рублей. 

\pagebreak
2.	имеет опыт работы в организации, которая осуществляла сделки с ценными бумагами:

a.	не менее 1 года, если такая организация является квалифицированным инвестором; или

b.	не менее 3 месяцев, если такая организация является квалифицированным инвестором и на дату признания лица квалифицированным инвестором это лицо является работником указанной организации; или

c.	не менее 2 лет в иных случаях;

\pagebreak
3.	совершало ежеквартально не менее чем по 10 сделок с ценными бумагами и (или) иными финансовыми инструментами в течение последних 
4 кварталов, совокупная цена которых за указанные 4 квартала составила не менее 300 тысяч рублей, 

или совершало не менее 5 сделок с ценными бумагами и (или) иными финансовыми инструментами в течение последних 3 лет, совокупная цена которых составила не менее 3 миллионов рублей.

\end{frame}

\begin{frame}[allowframebreaks]{\setfontsize{12pt}Признание ю.л. квалифицированным инвестором}{Коммерческая организация + любые 2 из 4 требований}
1.	имеет собственный капитал не менее 100 миллионов рублей;

2.	совершало ежеквартально не менее чем по 5 сделок с ценными бумагами и (или) иными финансовыми инструментами в течение последнего года, совокупной стоимостью не менее 3 миллионов рублей;

\pagebreak
3.	выручка за последний отчетный год составляет не менее 1 миллиарда рублей;

4.	имеет сумму активов за последний отчетный год не менее 2 миллиардов рублей.

\end{frame}
\subsection{Квалифицированные инвесторы в США}
\begin{frame}{Квалифицированные инвесторы в США}
В США применяется понятие аккредитованных инвесторов, означающее для его обладателей возможность совершать финансовые сделки недоступные для других участников рынка. 
\end{frame}

\begin{frame}{Применительно к венчурным инвестициям}
1.	``Закон об инвестиционных компаниях 1940 г.``, регулирующий оборот ценных бумаг, предназначенных для квалифицированных приобретателей;


2.	``Акт Комиссии по биржам (Commission Final Rule)``, принятый в соответствии с ``Законом о ценных бумагах 1933 г.``  и регулирующий приобретение высокорисковых и необеспеченных финансовых инструментов квалифицированными институциональными покупателями.

\end{frame}

\begin{frame}{Квалифицированные приобретатели в США}
\begin{itemize}
	\item физические или юридические лица, являющиеся инвестором в том или ином институте на сумму не менее 5 млн. долл.;
	\item трасты, учрежденные указанными лицами;
	\item лицо, вкладывающее свои и привлеченные средства на общую сумму не менее 25 млн. долл.
\end{itemize}
\end{frame}

\begin{frame}{\setfontsize{12pt}Квалифицированные институциональные покупатели }
\begin{itemize}
	\item юридическое лицо или иное образование вкладывающее свои и привлеченные средства на сумму не менее 100 млн. долл.;
	\item некоторые трасты;
	\item компании прямого инвестирования;
	\item дилеры;
	\item банки;
	\item объединения квалифицированных институциональных покупателей.
\end{itemize}
\end{frame}
\subsection{Квалифицированные инвесторы в Еврозоне}
\begin{frame}{Квалифицированные инвесторы в Еврозоне}
1) институт профессионального клиента для некоторых финансовых инструментов европейского рынка ценных бумаг и коллективных инвестиций;

2) институт квалифицированного инвестора для ценных бумаг, выпускаемых по упрощенной процедуре.
\end{frame}

\begin{frame}
Ключевым отличием европейской методологии от американской является разделение всех квалифицируемых инвесторов на 2 группы:

1) квалифицированные инвесторы по умолчанию (de jure) (в список входят разного рода финансовые организации – банки, инвестиционные компании и пр.);

2) квалифицированные инвесторы по признанию (ad hoc).
\end{frame}

\begin{frame}{Признание квалификации инвестора в Еврозоне}{соответствие хотя бы 2 из 3 критериев}
\begin{itemize}
	\small
	\item [1)] инвестор проводил крупные сделки (сделки существенного объема) на соответствующем рынке (ценных бумаг или соответствующих финансовых инструментов) со средней частотой, по \item [2)] крайней мере, 10 сделок в квартал в течение последних 4 кварталов;
	\item [3)] объем портфеля финансовых активов инвестора превышает 500 тыс. евро;
	\item [4)] инвестор работает или работал не менее 1 года в финансовом секторе на профессиональной должности, требующей знаний инвестирования в ценные бумаги.
	
\end{itemize}
\end{frame}
\end{document}