% !TeX program = lualatex -synctex=1 -interaction=nonstopmode --shell-escape %.tex
% Use of ’standalone’ class with a beamer overlay:
\documentclass[beamer]{standalone}
% Load packages needed for this TeX file:
\RequirePackage[rus]{borochkin_tikz}

\tikzset{
	basic/.style  = {draw, drop shadow, font=\scriptsize, rectangle},
	root/.style   = {basic, rounded corners=2pt, thin, align=center,
		fill=beamer@nngudarkblue, text=white},
	level 2/.style = {basic, rounded corners=6pt, thin,align=center, fill=beamer@nngudarkblue!60,
		text=white,
	text width=2.5cm},
	level 3/.style = {basic, thin, align=left, fill=beamer@nngudarkblue!60, text=white,
		text width=1.7cm}
}

% Surround TeX code with ’document’ environment:
\begin{document}
\begin{standaloneframe}[shrink=23] % e.g. ’fragile’
\begin{tikzpicture}[
level 1/.style={sibling distance=30mm},
edge from parent/.style={->,draw},
>=latex]

% root of the the initial tree, level 1
\node[root] {Венчурные инвесторы}
% The first level, as children of the initial tree
  child {node[level 2] (c1) {тип}}
  child {node[level 2] (c2) {специализация}}
  child {node[level 2] (c3) {отрасли}}
  child {node[level 2] (c4) {размер}}
  child {node[level 2] (c5) {стратегия}};

% The second level, relatively positioned nodes
\begin{scope}[every node/.style={level 3}]
\node [below of = c1, xshift=5pt] (c11) {Частные};
\node [below of = c11] (c12) {Коллективные};

\node [below of = c2, xshift=5pt] (c21) {Поглощения};
\node [below of = c21] (c22) {Финансовое оздоровление};
\node [below of = c22, yshift=-.3cm] (c23) {Изменение структуры капитала};

\node [below of = c3, xshift=5pt] (c31) {полупроводники};
\node [below of = c31] (c32) {нано-технологии};
\node [below of = c32, yshift=-.3cm] (c33) {информа-ционные технологии};
\node [below of = c33, yshift=-.3cm] (c34) {торговля};
\node [below of = c34] (c35) {прочие};

\node [below of = c4, xshift=5pt] (c41) {мелкие};
\node [below of = c41] (c42) {средние};
\node [below of = c42, yshift=-.3cm] (c43) {крупные};

\node [below of = c5, xshift=5pt] (c51) {универсальные};
\node [below of = c51] (c52) {специализированные};

\end{scope}

% lines from each level 1 node to every one of its "children"
\foreach \value in {1,2}
  \draw[->] (c1.190) |- (c1\value.west);

\foreach \value in {1,...,3}
  \draw[->] (c2.191) |- (c2\value.west);

\foreach \value in {1,...,5}
  \draw[->] (c3.191) |- (c3\value.west);

\foreach \value in {1,...,3}
\draw[->] (c4.191) |- (c4\value.west);

\foreach \value in {1,2}
\draw[->] (c5.191) |- (c5\value.west);
\end{tikzpicture}
\end{standaloneframe}
\end{document}